\paragraph{Existence et unicité de la solution}
Tout d'abord on gardera les notations introduites dans
(\ref{eq:stokes-fourier-strain-tensor-3d}) et
(\ref{eq:stokes-fourier-strain-tensor-2d}) pour $E(u)$ et $\tilde E(u)$
respectivement. D'autre part on note toujours $\abs{E(u)}^2 =
\sum_{i,j = 0}^3E_{i,j}^2(u)$ et $\abs{\tilde E(u)}^2 = \sum_{i,j =
  0}^2\tilde E_{i,j}^2(u)$. Pour simplifier, on supposera que le
domaine $\Lambda$ est un rectangle de dimensions $L_1,L_2$,
c'est-à-dire $\Lambda = \cparent{(x_1, x_2) \setsuchthat
  0<x_1<L_1,\ 0<x_2<L_2}$.

\begin{lemme}\label{lem:lemme-1}
  Sous l'hypothèse $\frac{\partial u_1}{\partial x_1} +
  \frac{\partial u_2}{\partial x_2} + \beta u_3 = 0$, on a la
  relation:
  \begin{equation}
    \int_\Lambda \abs{E(u)}^2\,\mathrm dx_1\mathrm dx_2 %
    = \int_\Lambda \parent{\abs{\tilde E(u)}^2 %
      + \frac{1}{2}\parent{  \parent{\frac{\partial u_3}{\partial x_1}}^2 %
                           + \parent{\frac{\partial u_3}{\partial x_2}}^2 %
                           + \beta^2 u_1^2 %
                           + \beta^2 u_2^2}}\,\mathrm dx_1\mathrm dx_2.
  \end{equation}
\end{lemme}

\begin{proof}
  Le calcul de $\abs{E(u)}^2$ donne:
  \begin{equation}\label{eq:proof-details}
    \begin{split}
      \abs{E(u)}^2 = \abs{\tilde E(u)}^2 %
      + \frac{1}{2}\parent{  \parent{\frac{\partial u_3}{\partial x_1}}^2 %
                           + \parent{\frac{\partial u_3}{\partial x_2} %
                                     + \beta^2u_1^2 %
                                     + \beta^2 u_2^2}}\\
      %
      - \beta u_1 \frac{\partial u_3}{\partial x_1} %
      - \beta u_2 \frac{\partial u_3}{\partial x_2} %
      + \beta^2 u_3.
    \end{split}
  \end{equation}
  En intégrant par partie le terme $u_1 \frac{\partial u_3}{\partial x_1}$ on obtient:
  \begin{equation}
    \begin{split}
      \int_\Lambda u_1 \frac{\partial u_3}{\partial x_1}\,\mathrm dx_1\mathrm dx_2 %
      &= \int_0^{L_2}\mathrm dx_2\int_0^{L_1} %
           u_1 \frac{\partial u_3}{\partial x_1}\,\mathrm dx_1\\
      &= - \int_0^{L_2}\mathrm dx_2 \int_0^{L_1} %
           u_3 \frac{\partial u_1}{\partial x_1}\,\mathrm dx_1\\
      &= - \int_\Lambda %
           u_3 \frac{\partial u_1}{\partial x_1}\,\mathrm dx_1\mathrm dx_2.
    \end{split}
  \end{equation}
  De même on aura:
  \begin{equation}
    \int_\Lambda u_2 \frac{\partial u_3}{\partial x_2}\,\mathrm dx_1\mathrm dx_2 %
    = - \int_\Lambda u_3 \frac{\partial u_2}{\partial x_2}\,\mathrm dx_1\mathrm dx_2.
  \end{equation}
  En reprenant (\ref{eq:proof-details}), en intégrant sur $\Lambda$
  et en utilisant l'hypothèse $\frac{\partial u_1}{\partial x_1} +
  \frac{\partial u_2}{\partial x_2} + \beta u_3 = 0$, on obtient le
  résultat annoncé.
\end{proof}

\begin{lemme}\label{lem:lemme-2}
  Il existe une constante positive $\chi > 0$ telle que:
  \begin{equation}
    \chi \norm{\nabla u}_{L^2(\Lambda)} \leq \norm{E(u)}_{L^2(\Lambda)},
  \end{equation}
  pour tout $u\in H_0^1(\Lambda)^3$ qui satisfait $\frac{\partial
    u_1}{\partial x_1} + \frac{\partial u_2}{\partial x_2} + \beta u_3
  = 0$. Ici $\norm{\nabla u}^2_{L^2(\Lambda)} =
  \sum_{i,j=1}^3\norm{\frac{\partial u_i}{\partial
      x_j}}^2_{L^2(\Lambda)}$ et $\norm{E(u)}^2_{L^2(\Lambda)} =
  \int_\Lambda\abs{E(u)}^2\,\mathrm dx_1\mathrm dx_2$.
\end{lemme}

\begin{proof}
  Il est connu que l'inégalité de Korn en dimension 2 est vraie,
  i. e. l'existance d'une constante $\chi > 0$ qui satisfait:
  \begin{equation}
    \chi \sum_{i,j = 1}^2\norm{\frac{\partial u_i}{\partial x_j}}^2_{L^2(\Lambda)} %
    \leq \int_\Lambda \abs{\tilde E(u)}^2\,\mathrm dx_1\mathrm dx_2.
  \end{equation}
  Le lemme (\ref{lem:lemme-1}) permet de conclure.
\end{proof}

\begin{proposition}\label{prop:proposition-2}
  Si le tenseur $\mu$ satisfait:
  \begin{equation}\label{eq:prop-2-hypothesis}
    \mu_{i,j}(x_1, x_2) \geq \mu_0 \quad \forall (x_1, x_2)\in \Lambda,\ 1\leq i,j\leq 3,
  \end{equation}
  où $\mu_0 > 0$ est une constante positive indépendante de $(x_1,
  x_2)\in\Lambda$, alors le problème
  (\ref{eq:stokes-fourier-visc-weak-1})-(\ref{eq:stokes-fourier-visc-weak-3})
  admet une et une seule solution.
\end{proposition}

\begin{proof}
  Définissont l'espace $V = H_0^1(\Lambda)^2\times
  H_{0,0}^1(\Lambda)$ où $H_{0,0}^1(\Lambda) = H_0^1(\Lambda) \cap
  L_0^2(\Lambda)$. Clairement l'espace $H_{0,0}^1(\Lambda)$ est un
  sous-espace fermé de $H_0^1(\Lambda)$ de codimension 1. Soit
  encore $a:V\times V\to \mathbb R$ la forme bilinéaire continue
  définie par:
  \begin{equation}
    a(u,v) = \int_\Lambda 2\mu\otimes E(u):E(v)\,\mathrm dx_1\mathrm dx_2.
  \end{equation}
  Définissons encore la forme bilinéaire continue $b:V\times Y\to\mathbb R$, par:
  \begin{equation}
    b(u, q) = \int_\Lambda \parent{\frac{\partial u_1}{\partial x_1} %
      + \frac{\partial u_2}{\partial x_2} %
      + \beta u_3}q\,\mathrm dx_1\mathrm dx_2,
  \end{equation}
  où ici $Y = L_0^2(\Lambda)$.

  Il est facile de voir que le problème
  (\ref{eq:stokes-fourier-visc-weak-1})-(\ref{eq:stokes-fourier-visc-weak-3})
  est équivalent au problème de chercher $(u,p)\in V\times Y$ tel
  que:
  \begin{equation}
    \begin{split}
      a(u,v) - b(v, p) = \int_\Lambda f\cdot v\,\mathrm dx_1\mathrm dx_2,%
        \quad \forall v\in V,\\
      b(u,q) = 0,%
        \quad \forall q\in Y.
    \end{split}
  \end{equation}
  La constante $C$ peut s'obtenir a posteriori en considérant
  (\ref{eq:stokes-fourier-visc-weak-1})-(\ref{eq:stokes-fourier-visc-weak-3})
  avec les fonctions tests $v = (0,0,s)$ où $s\in H_0^1(\Lambda)$
  est dans l'orthogonal de $H_{0,0}^1(\Lambda)$.

  Pour démontrer la proposition \ref{prop:proposition-2}, il suffit
  de vérifier que la forme $a(.,.)$ est coercive sur $V_0$, où
  $V_0 = \cparent{v\in V\setsuchthat b(v,q) = 0\ \forall q\in Y}$, et
  que la condition classique inf--sup sur la forme bilinéaire $b$
  est satisfaite.

  Le lemme \ref{lem:lemme-2} avec l'hypothèse
  (\ref{eq:prop-2-hypothesis}) montrent bien que $a$ est coercive sur
  $V_0$. D'autre part en utilisant l'inégalité concernant la
  condition inf--sup dans $\mathbb R^2$ on a si $q\in L_0^2(\Lambda)$:
  \begin{equation}
    \begin{split}
      \sup_{\norm{v}_{H_0^1} = 1}b(v,q) %
      &\geq \sup_{\norm{(v_1, v_2, 0)}_{H_0^1}}b(v,q)\\
      &= \sup_{\norm{(v_1, v_2, 0)}_{H_0^1}}\int_\Lambda \parent{\frac{\partial v_1}{\partial x_1} %
        + \frac{\partial v_2}{\partial x_2}}q\,\mathrm dx_1\mathrm dx_2\\
      &\geq \delta \norm{q}_{L_0^2},
    \end{split}
  \end{equation}
  où $\delta > 0$. Ainsi on a prouvé la proposition \ref{prop:proposition-2}.
\end{proof}
