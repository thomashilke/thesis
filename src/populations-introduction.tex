Une cuve d'électrolyse d'aluminium industrielle est caractérisée par
le fait que plusieurs phénomènes physiques entrent en jeu et
interagissent à des échelles similaires. Tout d'abord, le courant
électrique continu qui traverse une cuve d'électrolyse industrielle
moderne est de l'ordre de \num{500} \si{\kilo\ampere} à \num{1}
\si{\mega\ampere}. Une telle intensité de courant génère d'une part un
champ d'induction magnétique qui affecte l'opération de la cuve, mais
également les cuves voisines. D'autre part, la chute de
potentiel d'environ \num{4} \si{\volt} à travers les électrodes et les
fluides \cite{Haupin1995} provoque la dissipation de grandes quantités
d'énergie sous forme thermique. Cette production de chaleur est
nécessaire car elle permet de maintenir le bain électrolytique sous
forme liquide. Cependant, une surchauffe trop importante est néfaste
pour une cuve, qui risque alors de subir une usure et une fin de vie
prématurée en plus d'occasionner des pertes d'énergie.

Au niveau de la cathode en carbone qui forme le fond de la cuve, la
réaction d'électrolyse
\begin{equation}
\cee{Al^{3+} + 3e^- -> Al}
\end{equation}
produit de l'aluminium sous forme métallique. L'aluminium métallique,
qui est peu soluble dans l'électrolyte, est liquide à la température
d'opération d'une cuve et de densité légèrement supérieure à celle de
l'électrolyte. Il forme une couche de métal en fusion au fond de la
cuve. C'est alors la surface du métal en fusion qui joue véritablement
le rôle de cathode sur laquelle a lieu la réaction d'électrolyse.

Le bain électrolytique et le métal liquide sont traversés par
l'ensemble du courant électrique et, en présence du champ d'induction
magnétique, subissent une force de Lorentz qui les met en
mouvement. L'écoulement dans les fluides est bénéfique. Il permet de
transporter la chaleur à l'extérieur du système plus rapidement que
par conduction pure, homogénéise la distribution de la température et
la composition chimique dans l'ensemble du bain et facilite le
transport et la dispersion les particules d'alumine injectées à
intervalles réguliers dans le bain.

Cependant, il est crucial que l'interface bain-métal soit aussi stable
que possible. Pour des raisons d'économie d'énergie, la distance
moyenne entre le fond des anodes\footnote{Le fond des anodes est
  défini comme la surface des anodes orientée vers
  le bas, face à la cathode.} et l'interface est réduite
autant que possible. Dans une cuve moderne, cette distance est de
l'ordre de \num{2} à \num{4} \si{\centi\meter}. Si les écoulements
dans les fluides deviennent trop rapides, des turbulences se forment
et viennent perturber la forme de l'interface. De plus, le système
magnétohydrodynamique formé par le circuit électrique, le champ
d'induction magnétique et les deux fluides peut devenir, sous
certaines conditions, physiquement instable, c'est-à-dire que de
petites perturbations du système sont amplifiées par elles-mêmes. De
telles instabilités affectent la forme de l'interface bain-métal, et
leur occurrence présente un défi qui a déjà fait l'objet de nombreux
travaux de recherche. Le lecteur intéressé se référera par exemple à
\cite{Descloux1998}, \cite{Sneyd1985} ou encore \cite{Maillard1996}.

Le métal liquide ayant une conductivité électrique bien supérieure à
celle du bain électrolytique \cite{Wang1992,Apfelbaum2003},
le contact entre les anodes et la nappe de métal pourrait créer un
court-circuit. Les court-circuits diminuent le rendement du procédé et
peuvent créer des défauts à la surface des anodes qui affectent
l'opération de la cuve à long terme.

De l'alumine doit être injectée régulièrement dans le bain
électrolytique, sous forme de poudre, pour compléter l'alumine
dissoute consommée par la réaction d'électrolyse. Puisque la majeur
partie du bain est recouverte par les blocs anodiques et en raison de
la présence de la croûte qui protège sa surface libre, l'injection de
poudre d'alumine ne peut avoir lieu que en quelques points d'injection
distribués le long du canal central. Dans une cuve industrielle, le
nombre de points d'injection est typiquement compris entre 4 et 6. La
dispersion et le transport des particules d'alumine et de la
concentration d'alumine qui résulte de leur dissolution dépend
crucialement de la présence et de la force de l'écoulement dans le
bain électrolytique. De nombreux travaux de recherche ont pour
objectif la modélisation et l'approximation numérique des écoulement
dans le bain et le métal d'une cuve d'électrolyse industrielle. Dans
ce travail, nous utiliserons les résultats de S. Pain \cite{Pain2006},
G. Steiner \cite{Steiner2009} et J. Rochat \cite{Rochat2016} à cette
fin. Le modèle d'écoulement proposé dans \cite{Steiner2009} consiste à
considérer un système couplé formé par les équations du potentiel
électrique dans les conducteurs, par les équations de Maxwell dans le
vide, dans le caisson ferromagnétique, dans le bain électrolytique et
dans le métal, et finalement par des équations de Navier-Stokes dans
chaque fluide. Dans ce problème, l'interface bain-métal est une
inconnue. Les écoulements dans les fluides sont turbulents. Les
structures des écoulements dont la taille est inférieure à la
résolution de la grille de discrétisation sont modélisés par un modèle
de longueur de mélange de Smagorinski \cite{Rochat2016}. Dans ce
chapitre nous utiliserons ce modèle, implémenté dans le logiciel
\citealucell{}, pour obtenir le domaine $\Omega$ occupé par le bain d'une
cuve d'électrolyse industrielle, une approximation de la vitesse
d'écoulement stationnaire $u_h$ et une approximation de la densité de
courant électrique stationnaire $j_h$ dans $\Omega$.

Le logiciel \alucell{}, actuellement développé conjointement par
l'EPFL et la société YcoorSystems SA à Sierre, est le fruit de
nombreux travaux de recherche. Cette thèse se base sur les méthodes
numériques de calculs d'écoulement dans les cuves d'électrolyse
d'aluminium développées à l'EPFL, en particulier par S. Pain
\cite{Pain2006}, G. Steiner \cite{Steiner2009}, J. Rochat
\cite{Rochat2016} et T. Hofer \cite{Hofer2011}, et qui sont toutes
implémentées avec \alucell{}. Un historique détaillé du logiciel se
trouve dans la thèse de G. Steiner \cite{Steiner2009}. Par conséquent
utiliserons également \alucell{} pour l'implémentation des méthodes
numériques dans ce chapitre.

Ce travail porte sur la modélisation des particules d'alumine
injectées dans le bain et leur dissolution. On suivra l'approche
adoptée par T. Hofer \cite{Hofer2011} et on représentera l'évolution
d'une famille de particules sous la forme d'une distribution en
taille et en espace $\population$. Cette population évolue d'une part
dans le bain par le biais de la vitesse d'écoulement $u$ de celui-ci,
et d'autre part par la dissolution des particules au cours du temps,
en fonction de la concentration d'alumine locale dans le bain et de sa
température telle que décrite par la vitesse de dissolution $f$
introduite dans la section \ref{sec:particle-dissolution}. En plus de
la densité de particule $\population$, nous modéliserons l'évolution
de la concentration d'alumine dissoute $c$ et la température
$\temperature$ dans le bain, puisque la dissolution des particules en
dépend.

Dans la section \ref{sec:populations-model}, nous introduisons les
équations qui décrivent l'évolution de la densité de particules $n_p$,
la concentration d'alumine dissoute $c$ et la température du bain
$\temperature$. Dans la section \ref{sec:populations-discretisation}
nous traitons la discrétisation en temps du système formé par les
équations pour $\population$, $\concentration$ et $\temperature$
introduites dans la section \ref{sec:populations-model}. Finalement,
dans la section \ref{sec:populations-industriel} nous considérons et
discutons l'application du modèle de transport et dissolution de
particules d'alumine au cas d'une cuve d'électrolyse industrielle
AP32.
