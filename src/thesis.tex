\documentclass[a4paper,11pt]{book}

\usepackage[utf8]{inputenc}
\usepackage[T1]{fontenc}
\usepackage[french]{babel}
\usepackage{lmodern}
\usepackage{amsmath,amsfonts,amssymb,amsopn,amsthm}
\usepackage{graphicx,fancyhdr,xcolor}
\usepackage[a4paper,top=22mm,bottom=28mm,inner=35mm,outer=25mm]{geometry}
%\usepackage{tikz}
\usepackage{epstopdf}
\usepackage{tabularx}
\usepackage{booktabs}
\usepackage{subcaption}
%\usepackage{url}
%\usepackage{listings}
\usepackage[version=3]{mhchem}
\usepackage[toc,page]{appendix}
\usepackage{siunitx}
\usepackage{multicol}
\usepackage{enumitem}

\graphicspath{
  {../media/introduction/}
  {../media/particles/}
  {../media/introduction/electrolysis-pot/}
  {../media/particles/solidfraction/}
  {../media/particles/enthalpy/}
  {../media/particles/beta/}
  {../media/particles/temperature/}
  {../media/particles/remelt/}
  {../media/particles/diss-rate/}
  {../media/particles/dissolution/}
  {../media/particles/trajectories/}
  {../media/populations/ap32-mesh-components/print/}
  {../media/populations/ap32-fluid-flow/print/}
  {../media/populations/cadence/}
  {../media/populations/anode-configuration/}
  {../media/populations/variances/}
  {../media/fourier/convergence/}
  {../media/fourier/anode-numerotations/}
}

\newlength{\rasterimagewidth}
\setlength{\rasterimagewidth}{1.0\textwidth}

\usepackage{adjustbox,xcolor}
%\usepackage{showframe}

%\usepackage[right]{showlabels,rotating}
\usepackage[final,right]{showlabels,rotating}

\renewcommand{\showlabelfont}{\tiny\ttfamily\color{magenta}}
\renewcommand{\showlabelsetlabel}[1]
             {\adjustbox{right=0cm}{%
                 \raisebox{1.2\normalbaselineskip}[0pt][0pt]{\framebox{\showlabelfont #1 }}}}

\setlength{\parskip}{0.6\baselineskip}
\setlength{\parindent}{1.3em}

\renewcommand{\arraystretch}{1.2}

\newcommand{\fig}[1]{\ref{fig:#1}}
\newcommand{\eq}[1]{(\ref{eq:#1})}
\newcommand{\needcite}{\footnote{Citation needed}}

\newcolumntype{Y}{>{\centering\arraybackslash}X}
%\newcolumntype{Z}{>{\centering\arraybackslash}X}
\newcolumntype{Z}{>{\raggedleft\arraybackslash}X}


\sisetup{list-final-separator = { et }}


\DeclareSIUnit\octet{o}


\usepackage[para]{footmisc}

\usepackage[backend=biber,
  style=alphabetic,
  sortlocale=fr_CH,
  natbib=true,
  url=false,
  doi=true,
  eprint=false]{biblatex}
\renewcommand\multicitedelim{\addcomma\space}
\addbibresource{../src/references.bib}

%\usepackage{pgfplots}
\pgfplotsset{compat=1.9}
\pgfplotsset{
  colormap={parula}{
    rgb255=(53,42,135)
    rgb255=(15,92,221)
    rgb255=(18,125,216)
    rgb255=(7,156,207)
    rgb255=(21,177,180)
    rgb255=(89,189,140)
    rgb255=(165,190,107)
    rgb255=(225,185,82)
    rgb255=(252,206,46)
    rgb255=(249,251,14)},
}


\DeclareMathOperator{\erf}{erf}
\DeclareMathOperator{\diam}{diam}
\DeclareMathOperator{\supp}{supp}
\DeclareMathOperator{\Div}{Div}

\title{Méthodes numériques liées à la distribution d'alumine dans une cuve d'électrolyse d'aluminium}
\date{\today}
\author{Thomas Hilke}

\begin{document}
\maketitle

\tableofcontents


%%% Math typography
\newcommand{\ie}{i.e.}
\renewcommand{\div}{\mathrm{div}}
\newcommand{\norm}[1]{\left\lVert #1 \right\rVert}
\newcommand{\parent}[1]{\left( #1 \right)}
\newcommand{\jump}[1]{\left[ #1 \right]}
\newcommand{\abs}[1]{\left\lvert #1 \right\rvert}
\newcommand{\cparent}[1]{\left\{ #1 \right\}}
\newcommand{\setsuchthat}{\ \mid|\ }
\newcommand{\rplus}{\mathbb R_+}
\newcommand{\nstar}{\mathbb N^*}
\newcommand{\intd}[1]{\,\mathrm d#1}
\newcommand{\dt}{\Delta t}
\newcommand{\dr}{\Delta r}
\newcommand{\heaviside}{H}

%%% Theorems
\theoremstyle{definition}
\newtheorem{proposition}{Proposition}
\newtheorem{lemme}{Lemme}
\newtheorem{remarque}{Remarque}

%%% Quantities
\newcommand{\temperature}{\Theta} % temperature du bain en fonction de
% x et t.
\newcommand{\temperaturer}{\tilde{\temperature}} % temperature du bain en fonction de x et t.
\newcommand{\tinj}{\temperature_\text{Inj}} % temperature d'injection des particules
\newcommand{\tliq}{\temperature_\text{Liq}} % temperature du liquidus du bain
\newcommand{\tinit}{\temperature_\text{Init}} % temperature initiale du bain
\newcommand{\tinitr}{\tilde\temperature_\text{Init}} % temperature initiale du bain en coordonnee spheriques
% (constant ou fonction de x)
\newcommand{\telectrolyte}{\temperature_\mathrm{e}} % temperature de
                                % l'electrolyte (liquide)

\newcommand{\aluminahc}{C_{\mathrm{p,Al}}}
\newcommand{\aluminadensity}{\rho_\mathrm{Al}}

\newcommand{\electrolytedensity}{\rho_\mathrm{e}}
\newcommand{\electrolytehc}{C_{\mathrm{p,e}}}
\newcommand{\electrolyteshc}{C_{\mathrm{p,e,1}}}
\newcommand{\electrolytelhc}{C_{\mathrm{p,e,2}}}
\newcommand{\fusionenthalpy}{\Delta H_\mathrm{sl}}
\newcommand{\electrolytetdiff}{D_\temperature}
\newcommand{\electrolytestdiff}{D_{\temperature,\mathrm{1}}}
\newcommand{\electrolyteltdiff}{D_{\temperature,\mathrm{2}}}

\newcommand{\concentration}{c}
\newcommand{\enthalpy}{H}
\newcommand{\enthalpydensity}{H}
\newcommand{\enthalpydensityr}{\tilde{\enthalpydensity}}

\newcommand{\tlat}{\temperature_\text{Lat}}

\newcommand{\Omegae}{{\Omega_\text{e}}}
\newcommand{\Omegam}{{\Omega_\text{m}}}
\newcommand{\rhoal}{{\rho_\text{Al}}}
\newcommand{\rhoe}{{\rho_\text{e}}}
\newcommand{\rhom}{{\rho_\text{m}}}
\newcommand{\rmax}{R_\mathrm{Max}}

\newcommand{\dissolutionrate}{\kappa}
\newcommand{\csat}{\concentration_\text{Sat}}
\newcommand{\tcrit}{\temperature_\text{Crit}}

\newcommand{\population}{n_p}
\newcommand{\dragforce}{F_{\mathrm D}}
\newcommand{\gravityforce}{F_{\mathrm g}}
\newcommand{\buoyancyforce}{F_{\mathrm A}}

\newcommand{\reynolds}{R_{\mathrm e}}
\newcommand{\electrolyteviscosity}{\mu}
\newcommand{\electrolyteturbviscosity}{\mu}

\newcommand{\tend}{T}
\newcommand{\faraday}{F}
\newcommand{\cdiffusivity}{D_\concentration}
\newcommand{\dirac}{\delta}
\newcommand{\kronecker}{\delta}
\newcommand{\aluminadissolutionenthalpy}{\Delta H_\text{Diss}}
\newcommand{\conductivity}{\sigma}
\newcommand{\electrolyteturbulentviscosity}{\mu_\mathrm{T}}
\newcommand{\electrolytelaminarviscosity}{\mu_\mathrm{L}}

\chapter*{Notations}
\addcontentsline{toc}{chapter}{Notations}

\begin{tabularx}{\textwidth}{@{}ll@{}}
  \toprule
  Symbol & Description \\
  \midrule

  \bottomrule
\end{tabularx}


\chapter{Introduction}
\label{chap:introduction}
L'aluminium est un métal léger qui figure parmi les éléments
métalliques les plus abondant sur Terre, avec une proportion d'environ
\num{8}\% de la masse totale de la croûte terrestre. Malgré tout il
n'apparaît jamais sous forme métallique en raison de sa forte affinité
avec l'oxygène, et se trouve sous forme d'oxyde d'aluminium \ce{Al2O3}
(appelé également alumine). L'alumine est un constituant de différents
minerais, mais seul la bauxite est utilisée pour la production de
l'aluminium.

La bauxite est constituée principalement par de l'oxyde d'aluminium
hydraté, de silicate, d'oxyde de fer et d'impuretés en plus faibles
proportions. Elle provient essentiellement d'Australie, de Chine et du
Brésil. Par exemple, ces trois pays combinés ont extrait plus de
\num{179} millions de tonnes de bauxite au cours de l'année
\num{2014}. Le procédé de Bayer permet ensuite d'isoler et purifier
l'alumine de la bauxite. Après concassage et broyage du minerais, la
soude caustique chaude dissout l'alumine:
\begin{align*}
  \cee{NaOH &-> Na+ + OH-},\\
  \cee{Al2O3.(H2O) + 2OH- &-> 2AlO2- + 2H2O},\\
  \cee{Al2O3.(H2O)3 + 2OH- &-> 2AlO2- + 4H2O}.
\end{align*}

Les autres composés restent insolubles, et sont éliminés par
filtration. En refroidissant le filtrat, de l'hydrate d'alumine est
récupéré par précipitation. Finalement, cet hydrate d'alumine est
calciné pour éliminer les molécules d'eau:
\begin{equation*}
  \cee{Al2O3.(H2O)3 -> Al2O3}.
\end{equation*}

La deuxième partie du processus consiste à inverser la réaction
d'oxydation pour finalement obtenir de l'aluminium sous forme
métallique. C'est cet aspect du procédé, qui porte les noms de ses
deux inventeurs Paul Héroult et Mark Hall, qui nous intéresse dans
ce travail et que nous décrivons plus en détail dans le paragraphe
qui suit.


\part{Transport et dissolution de particules d'alumine}
\label{part:alumina}

\chapter{Particules d'alumine dans un bain électrolytique}
\label{chap:particles}
\section{Introduction}
\label{sec:particle-introduction}
% Plan de l'introduction:
% * Introduction du chapitre: aspects de l'operation d'une cuve lies
%   aux particules d'alumine dans un bain
%   conclusion: les technologies de cuve moderne repose de plus en
%   plus sur la dissolution efficace et rapide de l'alumine dans le bain.
% * histoire d'une particule d'alumine de l'injection a la dissolution
%   * proprietes des grains, structures cristallines, taille, etc
%   * technologie d'injection, piqueurs, injecteurs, doses, etc
%   * contact avec le bain, aggregation, dispersion, transport
% * aspect importants de la dissolution de l'alumine que l'on retient
%   et etudie dans ce chapitre
% * Organisation du chapitre

Au cours de l'opération d'une cuve d'électrolyse, de l'oxyde
d'aluminium doit être injecté dans le bain électrolytique afin
de compenser l'alumine dissoute qui à été consommée par la
réaction d'électrolyse.

En raison de la faible surchauffe\footnote{La surchauffe du bain
  électrolytique est d\'finie comme la différence entre la température
  du bain et la température du liquidus $\tliq$.} du bain, la surface
de celui-ci est recouverte, en tout temps, par une croûte
solide. Cette croûte est principalement constituée par de
l'électrolyte solidifié. Sa présence est
désirable. Elle joue le rôle d'isolant thermique, protège la structure
supérieure de la cuve des éclaboussures et facilite la canalisation
des émanations gazeuse.

Cependant, la présence de la croûte limite l'accès à la surface du
bain, et en particulier, l'injection d'alumine nécessite la mise en
place d'un mécanisme qui permette de la percer. Ces dispositifs
appelés piqueurs, percent mécaniquement la croûte en plusieurs
endroits et à intervalle régulier, et maintiennent des accès libres à
la surface du bain.

Ces accès permettent à des injecteurs de déposer, à la surface du
bain, des doses de poudre d'oxyde d'aluminium à intervalles
réguliers. Cette poudre est constituée de particules grossièrement
sphériques, sous forme cristalline et dont la température $tinj$ se
situe entre \num{100}\si{\celsius} et \num{150}\si{\celsius}.
% approche pour le paragraphe: Decrire comment les choses se
% produisent dans le cas ideal, et donc simple. Dire que cette
% situation ne se realise jamais en pratique, pour les raisons
% suivantes: ...
Dans une situation idéale, après leur injection dans le bain, les
particules se dispersent dans celui-ci et se dissolvent peu-à-peu tout
en étant transportées par l'écoulement des fluides. Cette alumine
dissoute vient contribuer à la concentration d'alumine
dissoute. Malheureusement, dans un système industriel réel, de
nombreux phénomènes viennent entraver la bonne marche de ce processus.


Tout d'abord, commençons par noter que la taille des particules est
contrainte pour différentes raisons.
diamètre des particules est pour la plupart compris entre
\num{20}\si{\micro\meter} et \num{100}\si{\micro\meter}.
Des particules trop fines présentent des
difficulté de manipulation au niveau mécanique, et la faible
capacité du bain à les mouiller rend leur dissolution difficile. De
plus, leur présence en trop grand nombre dans le bain peu avoir un
impact négatif sur la conductivité électrique de celui-ci. A
l'inverse, des particule trop grande mettent trop de temps à se
dissoudre \cite{Fini2017}.

Contrairement à l'intuition, réchauffer les particules d'alumine avant
leur injection dans le bain diminue leur capacité de dissolution. Ceci
est dû à la désorption de gaz à la surface des particule.

Ceci est du aux gaz adsorbés à la surface des particules et au
taux d'humidité qu'elle contiennent.







Ce chapitre est dédié au traitement d'une série de phénomènes
physiques qui entre en jeu lorsque les particules d'une dose d'alumine
entre en contact avec le bain électrolytique et commencent à se
dissoudre dans celui-ci.

Ce chapitre est dédié à la discussion de différents phénomènes qui
déterminent la capacité des particules d'alumine à se dissoudre dans
ledit bain électrolytique, et à contribuer à la concentration d'oxyde
d'aluminium dissoute. On discutera d'une part des phénomènes
liés à la température du bain électrolytique au voisinage des
particules, et d'autre part à l'effet de la gravité sur la
trajectoire des particules dans le bain électrolytique.

La température du bain électrolytique influe sur la capacité des
particules d'alumine à se dissoudre pour plusieurs raisons. Faisons
remarquer tout d'abord qu'on cherche à maintenir la température du
bain électrolytique au-dessus, mais aussi proche que possible de la
température de liquidus $\tliq$. En effet, plus la température de
l'électrolyte est basse, plus les pertes d'énergie par diffusion
thermiques sont minimisées. Cependant, lorsque la température du
bain se rapproche de $\tliq$, la cuve d'électrolyse
devient de plus en plus sensible à des écarts de température. On
cherche alors un équilibre délicat en cherchant à diminuer la
dissipation thermique tout en maintenant la cuve d'électrolyse dans
un état stable.

En raison de cette faible surchauffe du bain, la surface de celui-ci
est systématiquement recouverte par une croûte solide.

Lors de l'injection d'une dose d'alumine, la différence de température
entre les particules et le bain provoque la solidification de celui-ci
à la surface des particules. Tant que cette couche de bain solidifiées
persiste, les particules sont transportées par l'écoulement sans
qu'elles se dissolvent. Progressivement, l'équilibre thermique
entre la dose de particules et l'électrolyte se rétablit et le bain
solidifié autour des particules se résorbe. C'est seulement à partir
de cet instant que celle-ci peuvent se dissoudre et contribuer à
l'oxyde d'aluminium dissout dans le bain électrolytique. Ce laps de
temps pendant lequel la particule ne peut pas se dissoudre est le
temps de latence.

Le temps de latence d'une particule dépend essentiellement de la
surchauffe du bain au voisinage de celle-ci, de sa température
initiale ainsi que de son rayon. Plus le volume d'une
particule est grand, plus la couche de bain gelé est importante, et
plus long est le temps nécessaire à refondre celle-ci. Il en va de
même pour sa température initiale: plus celle-ci est
froide, plus la couche de bain gelé est importante, et plus le temps
de refonte est long. Finalement, une température de surchauffe
importante diminue le temps de latence, puisque les flux de chaleur
qui établissent l'équilibre thermique au voisinage de la particule
sont plus importants.

% parler ici de la dependance de la dissolution par rapport a la
% temperature.

Il y a bien entendu d'autres effets qui jouent un rôle dans la
dissolution des particules d'alumine, mais que l'on négligera dans
le présent travail. En particulier, on notera que les particules,
bien qu'approximativement sphériques, sont en réalité des
structures cristallines complexes partiellement poseuses
\cite{Ostbo2002}. La dissolution individuelle de chaque grain est un
processus complexe qui fait intervenir des transformations entre différentes phases
cristallines et une désintégration partielle avant que ceux-ci soit
entièrement dissout. La composition chimique du bain électrolytique
joue également un rôle sur la réaction de dissolution de l'alumine,
par l'intermédiaire de l'enthalpie de dissolution. Au moment de
l'injection d'une dose de poudre d'alumine, une partie des particules
peuvent se retrouver densément compactées. La présence du bain
électrolytique et les haute température peuvent provoquer un
phénomène de frittage entre les particules. Les particules ainsi
solidarisées forment des agrégats de tailles macroscopiques
\cite{Ostbo2002}.
Ces agrégats sont nuisible au bon fonctionnement des cuves
d'électrolyse industrielles.



%%- non-sphericite des particules,
%%- agglomeration,
%%- changement de structure cristalline,
%%- structure de l'écoulement du bain au voisinage des particules, turbulences,


\section{Formation de bain gelé}
\label{sec:particle-freeze}
% introduction: dans cette section, on introduit le problème de Stefan
%  pour une particule isolée dans un bain électrolytique infini
Dans cette section nous détaillons et étudions numériquement les
phénomènes physiques qui sont responsables du temps de latence
préalable au début de la dissolution des grains d'alumine dans le bain
électrolytique. Dans ce but, on considère un modèle de Stefan pour
traiter le problème de la transition de phase dans le bain
électrolytique environnant une région de bain solidifiée. Pour
approcher numériquement l'évolution de la région de transition de
phase autour de la particule de bain, on fera l'hypothèse d'une
symétrique sphérique de la solution.

Commençons par remarquer que la chaleur spécifique $\aluminahc$ de
l'alumine solide et sa densité $\aluminadensity$ conduit à une
capacité thermique massique $\aluminadensity\aluminahc$ presque égale
à celle du bain électrolytique
$\electrolytedensity\electrolytehc$. L'écart entre ces deux capacités
thermiques étant de \num{32}\% seulement, nous remplacerons les
caractéristiques thermiques de l'alumine solide par celle du bain
électrolytique pour simplifier le modèle de solidification et
liquéfaction dudit bain autour d'une particule d'alumine solide
injectée dans ce bain.

% Choix des parametre thermiques constants dans chaque phase
Ce choix se justifie par le fait que ces valeurs varient peu sur
l'intervalle de température qui nous intéresse, et on peut se
contenter de considérer une valeur moyenne sur cet intervalle.

% description du système: particule et bain, symétrie sphérique,
%  caractérisation des matériaux.
\paragraph{Un modèle thermique avec transition de phase}
On considère une particule sphérique, placée à l'origine du système de
coordonnées, de température initiale $\tinj<\tliq$, où $\tliq$ est la
température du liquidus de l'électrolyte. Cette sphère est donc gelée,
c'est-à-dire sous forme solide. Le reste de l'espace est occupé par le
bain électrolytique à l'état liquide, au repos, et de température
initiale $\tinit > \tliq$.

On s'intéresse à l'évolution de la température au cours du temps dans
le système formé par la particule et le bain liquide environnant. En
particulier, la température de la particule étant inférieure à la
température du liquidus $\tliq$, le bain au voisinage de la surface de
la particule va commencer par geler. Puis, la température du système
s'équilibrant par diffusion thermique, cette couche de bain va peu à
peu fondre. On cherche à déterminer l'évolution au cours du temps de
l'épaisseur de cette couche de bain gelée.

% modèle mathématique: définitions des domaines, front de transition
%  de phase, problème de la chaleur dans chaque phase, condition de Stefan, formulation
%  forte, conditions aux limites.
On introduit maintenant le cadre nécessaire à l'écriture d'un modèle
mathématique qui décrive l'évolution thermique de la particule et du
bain.

Soit $\Omega\subset \mathbb R^3$ un domaine ouvert occupé par du bain
électrolytique. On note $\temperature(t, x)$ la température de
l'électrolyte au point $x \in \Omega$ et à l'instant $t > 0$.

Soient $\Omega_1(t)\subset \Omega$ le domaine occupé par l'électrolyte
dans la phase solide, $\Omega_2(t)\subset \Omega$ le domaine occupé
par l'électrolyte dans la phase liquide et $\Gamma(t) \subset \Omega$
la région de transition de phase définis par:
\begin{align}\label{eq:phase-domains}
  &\Omega_1(t) = \cparent{x\in\Omega \mid %
                         \temperature(t, x) < \tliq},\\
  &\Omega_2(t) = \cparent{x\in\Omega \mid \temperature(t, x) %
    > \tliq},\\
  &\Gamma(t) = \cparent{x\in\Omega\mid \temperature(t, x) = \tliq}
\end{align}

Soit $\electrolytedensity$ la densité du bain que l'on suppose
constante. Soient $\electrolytehc$, $\electrolytetdiff$ la chaleur
spécifique le coefficient de diffusion thermique du bain,
respectivement. Dans la suite, on fera l'hypothèse que les paramètres
$\electrolytehc$ et $\electrolytetdiff$ sont des fonctions de la
température $\temperature$, mais constants dans chaque phase,
c'est-à-dire que
\begin{equation}
  \electrolytehc(\temperature) = \left\{
  \begin{array}{ll}
    \electrolyteshc, & \text{ si } \temperature < \tliq,\\
    \electrolytelhc, & \text{ si } \temperature > \tliq,
  \end{array}
  \right.
  \quad\text{et}\quad
  \electrolytetdiff(\temperature) = \left\{
  \begin{array}{ll}
    \electrolytestdiff, & \text{ si } \temperature < \tliq,\\
    \electrolyteltdiff, & \text{ si } \temperature > \tliq,
  \end{array}
  \right.
\end{equation}
où $\electrolyteshc$, $\electrolytelhc$, $\electrolytestdiff$,
$\electrolyteltdiff$ sont des réels positifs donnés.

La température de l'électrolyte $\temperature$ satisfait une équation
de la chaleur dans chaque phase:
\begin{align}
  &\electrolytedensity\electrolyteshc \frac{\partial
    \temperature}{\partial t} - \electrolytestdiff \Delta\temperature
  = 0, & \forall x \in \Omega_1(t),\ \forall t\in [0,\infty),\label{eq:heat-solid-phase}\\
    %
    &\electrolytedensity\electrolytelhc \frac{\partial
    \temperature}{\partial t} - \electrolyteltdiff \Delta\temperature
  = 0, & \forall x \in \Omega_2(t),\ \forall t\in [0,\infty).\label{eq:heat-liquid-phase}
\end{align}
Soit $\nu$ la normale unité à l'interface $\Gamma$ dirigée vers
$\Omega_2$. Pour toute fonction $g:\Omega_1\cap\Omega_2\to\mathbb R$,
on note $\jump{g}:\Gamma\to\mathbb R$ le saut de la fonction $g$ sur
l'interface $\Gamma$ défini par:
\begin{equation}
  \jump{g}(x) = \lim_{\epsilon\to 0}g(x + \epsilon\nu) - g(x -
  \epsilon\nu) \quad \text{ sur } \Gamma.
\end{equation}

On suppose que l'interface entre les phases liquide est solide
satisfait la condition de Stefan. On note $v_\Gamma(t, x)$ la vitesse
de l'interface $\Gamma(t)$. La condition de Stefan pour l'interface
$\Gamma(t)$ s'écrit alors:
\begin{equation}
  \jump{\electrolytetdiff\nabla\temperature \cdot \nu} %
  = - \fusionenthalpy v_\Gamma,
  \label{eq:stefan-condition}
\end{equation}
où $\fusionenthalpy$ est la est l'enthalpie par unité de volume
libérée lors de la transition de la phase solide à liquide du
bain électrolytique.

La condition (\ref{eq:stefan-condition}) correspond à un bilan
d'énergie thermique au niveau de l'interface entre les phases. Le
membre de gauche de la relation (\ref{eq:stefan-condition}) correspond
à la quantité d'énergie absorbée ou libérée par le déplacement du
front de solidification, tandis que le membre de droite correspond à
la somme des flux d'énergie thermique au niveau du front de
solidification.

% subdivision du bord, conditions aux limites
On suppose donnée une subdivision $\Gamma_\mathrm{N}$,
$\Gamma_\mathrm{D}\subset \partial \Omega$ telle que
$\Gamma_\mathrm{N}\cap \Gamma_\mathrm{D} = \emptyset$ et
$\Gamma_\mathrm{N}\cup \Gamma_\mathrm{D} = \partial \Omega$, et les
fonctions $\temperature_\mathrm{D}:\Gamma_\mathrm{D}\to\mathbb R$ et
$\temperature_\mathrm{N}:\Gamma_\mathrm{N}\to\mathbb R$.

La partie du bord $\Gamma_\mathrm{N}$ correspond aux conditions au
limites de Neumann, tandis que sur la partie du bord
$\Gamma_\mathrm{D}$ correspond aux conditions aux limites de
Dirichlet. Les équations (\ref{eq:heat-solid-phase}),
(\ref{eq:stefan-condition}) sont complétées par les conditions aux
limites:
\begin{align}
  &\temperature(t, x) = \tliq, %
  &\forall x \in \Gamma(t),\ \forall t \in [0,\infty),\label{eq:heat-transition}\\
    %
  &\temperature(t, x) = \temperature_\mathrm{D}(x), %
  &\forall x \in \Gamma_\mathrm{D}(t),\ \forall t \in [0,\infty),\label{eq:heat-dirichlet}\\
    %
  &\frac{\partial \temperature}{\partial \nu}(t, x) = \temperature_\mathrm{N}(x), %
  &\forall x \in \Gamma_\mathrm{N}(t),\ \forall t \in [0,\infty),\label{eq:heat-neumann}
\end{align}
ainsi que par une condition initiale $\tinit$ appropriée à $t = 0$ pour
$\temperature$.

Le problème de Stefan classique consiste à chercher une fonction
$\temperature:[0,\infty)\times\Omega\to\mathbb R$ et deux
  sous-domaines $\Omega_1,\Omega_2$ qui satisfasse les équations
  (\ref{eq:phase-domains}) et
  (\ref{eq:heat-solid-phase})-(\ref{eq:stefan-condition}).

Le lecteur intéressé trouvera une discussion détaillée des différentes
formulations des problèmes de Stefan et de leurs analyses
mathématiques dans l'ouvrage de J. Hill \cite{HillStefanProblems},
ainsi que celui de L. I. Rubenstein \cite{Rubenstein1971}, par
exemple.

\paragraph{Formulation faible du problème de Stefan}
On donne maintenant une forme alternative du problème de Stefan
classique formé par les équations
(\ref{eq:heat-solid-phase})-(\ref{eq:heat-neumann}). Cette
formulation, qu'on qualifie d'{\em enthalpique},
remplace le problème de frontière libre du problème de Stefan
classique par un problème parabolique non-linéaire dégénéré. L'un des
avantages et de ne pas avoir à suivre explicitement l'interface
$\Gamma$.L'interface $\Gamma$ est obtenue après calcul par un
``post-processing''.

Soit $f_s(\temperature)$ la fraction solide du bain électrolytique. La
fraction solide est définie par:
\begin{equation}
  f_s(\temperature) = \left\{
  \begin{array}{ll}
    0           & \text{ si } \temperature \geq \tliq,\\
    1           & \text{ si } \temperature < \tliq.
  \end{array}
  \right.
\end{equation}
Le comportement de la fonction $f_s$ au voisinage de $\tliq$ est
représenté sur le graphique de gauche de la figure
\ref{fig:solid-fraction-enthalpy}.

\begin{figure}
  \begin{center}
    \input{../media/particles/solidfraction/solidfraction.tex}
    \input{../media/particles/enthalpy/enthalpy.tex}
    \caption{Gauche: fraction solide du bain $f_s$ en fonction de la
      température. Droite: Enthalpie du bain électrolytique en
      fonction de la température.}
    \label{fig:solid-fraction-enthalpy}
  \end{center}
\end{figure}

Les propriétés thermiques du bain électrolytique sont
caractérisées par la relation entre la température $\temperature$ et
l'enthalpie par unité de masse $\enthalpy$ qui s'exprime de la manière
suivante:
\begin{equation}
  \enthalpy(\temperature) = %
    \int_0^\temperature
      \electrolytedensity\electrolytehc\intd{s} %
    + \fusionenthalpy(1 - f_s(\temperature)),
\end{equation}
où $\electrolytedensity$ est la densité du bain, $\fusionenthalpy$ est
l'enthalpie de transition de phase par unité de volume, et $f_s$ la
fraction solide du bain en fonction de la température.

Le graphique de droite sur la figure \ref{fig:solid-fraction-enthalpy}
représente le comportement de la fonction $\enthalpy(\temperature)$ au
voisinage de la température de transition $\tliq$. La densité de
l'électrolyte $\electrolytedensity$ est supposée constante pour tout
$\temperature$. En particulier, on suppose que la transition entre les
phases solide et liquide à $\electrolytedensity$ constant.

Bien que la fonction $\enthalpy$ présente une discontinuité en
$\temperature = \tliq$, elle est strictement monotone, et on peut
définir une fonction $\beta(\enthalpy)$ telle que
$\beta(\enthalpy(\temperature)) = \temperature$ pour tout
$\temperature$ de la manière suivante:
\begin{equation}\label{eq:beta}
  \beta(\enthalpy) = \left\{
  \begin{array}{ll}
    \frac{\enthalpy}{\electrolytehc\electrolytedensity} %
      & \enthalpy < \tliq \electrolytehc\electrolytedensity,\\
    \tliq %
      & \tliq \electrolytehc\electrolytedensity < \enthalpy < \tliq \electrolytehc\electrolytedensity + \fusionenthalpy,\\
    \frac{\enthalpy - \fusionenthalpy}{\electrolytehc\electrolytedensity} %
      & \tliq\electrolytehc\electrolytedensity + \fusionenthalpy < \enthalpy.\\
  \end{array}
  \right.
\end{equation}
La fonction $\beta$ est représentée sur la figure \ref{fig:beta}. Pour
les besoins de la représentation, on a choisi ici $\tliq = 1218\si{\kelvin}$.
\begin{figure}
  \begin{center}
    \input{../media/particles/beta/beta.tex}
    \caption{Fonction $\beta(\enthalpy)$. On a noté $H_1 =
      \tliq\electrolyteshc\electrolytedensity$ et $H_2 =
      \tliq\electrolyteshc\electrolytedensity + \fusionenthalpy$, et
      $\tliq = 1218\si{\kelvin}$.}
    \label{fig:beta}
  \end{center}
\end{figure}

La formulation enthalpique du problème de Stefan consiste à chercher
une paire de fonctions
$\temperature,\enthalpydensity:[0,\infty)\times\Omega\to\mathbb R$
telles que:
\begin{align}
  & \frac{\partial \enthalpydensity}{\partial t} %
  - \div\parent{ \nabla \temperature} = 0,%
  & \forall (t, x) \in [0,\infty)\times
    \Omega,\label{eq:heat-one-phase}\\
  & \temperature = \beta(\enthalpydensity),
    &\forall (t, x) \in [0,\infty)\times,\label{eq:heat-temperature-enthalpy}\\
  & \temperature = \temperature_\mathrm{D}, %
  & \forall (t, x) \in [0,\infty)\times \Gamma_\mathrm{D},\label{eq:heat-one-phase-dirichlet}\\
  & \frac{\partial \temperature}{\partial \nu} = \temperature_\mathrm{N}, %
  & \forall (t, x) \in [0,\infty)\times \Gamma_\mathrm{N},\label{eq:heat-one-phase-neumann}\\
  & \temperature(0, x) = \tinit(x),
  & \forall x\in\Gamma,\label{eq:heat-initial-condition}
\end{align}
Les fonctions $\tinit$, $\temperature_\mathrm{D}$ et
$\temperature_\mathrm{N}$ ont été définies au paragraphe précédent, et
$\beta$ est définie par la relation (\ref{eq:beta}). L'équation
(\ref{eq:heat-one-phase}) est à comprendre au sens faible.

%%Étant donnée une fonction $\enthalpydensity$ qui satisfait les
%%équations (\ref{eq:heat-one-phase})-(\ref{eq:heat-one-phase-neumann}),
%%la température $\temperature$ est définie par la relation
%%$\temperature(t, x) = \beta(\enthalpydensity(t, x))$ $\forall (t, x)
%%\in [0,\infty)\times\Omega$. Finalement les domaines $\Omega_1(t)$,
%%  $\Omega_2(t)$ occupés par les phases solide et liquides et la région
%%  de transition de phase $\Gamma(t)$ sont définis par les relations
%%  (\ref{eq:phase-domains}) et (\ref{eq:phase-frontier}).

%%Étant données les fonctions $\enthalpydensity$, $\temperature$
%%solutions du problème
%%(\ref{eq:heat-one-phase})-(\ref{eq:heat-initial-condition}), les
%%phases solide $\Omega_1(t)$ et liquide $\Omega_2(t)$ sont définies par
%%la relation (\ref{eq:phase-domains}). La région de transition de phase
%%est définie par la relation (\ref{eq:phase-frontier}).

Plus précisément, on peut montrer que si $\enthalpydensity$ et
$\temperature$ sont solutions faibles de
(\ref{eq:heat-one-phase})-(\ref{eq:heat-temperature-enthalpy}) et si
$\enthalpydensity$ et $\temperature$ sont suffisamment régulières
dans $\Omega_1(t)$ et $\Omega_2(t)$ définis par
(\ref{eq:phase-domains}), alors $\temperature$ satisfait
(\ref{eq:heat-solid-phase}), (\ref{eq:heat-liquid-phase}) et (\ref{eq:stefan-condition}).

%%Si $\temperature$ est suffisamment régulière dans $\Omega$, alors la
%%condition (\ref{eq:temperature-condition}) ainsi que les équations
%%(\ref{eq:heat-solid-phase}), (\ref{eq:heat-liquid-phase}) sont
%%satisfaites par $\temperature$. On montre finalement que la
%%température $\temperature$ satisfait la condition de Stefan
%%(\ref{eq:stefan-condition}).

En effet, soit $\Lambda = [0, \infty)\times \Omega$, soit
\begin{align}
  & \Lambda_1 = \cparent{(t, x)\in\Lambda\mid x \in \Omega_1(t)},\\
  & \Lambda_2 = \cparent{(t, x)\in\Lambda\mid x \in \Omega_2(t)},\\
\end{align}
et soit $\Gamma_{tx}$ dans l'espace-temps:
\begin{equation}
  \Gamma_{tx} = \cparent{(t, x) \in \Lambda\mid x \in \Gamma(t)}.
\end{equation}
Soit $\nu_{tx}$ la normale unité à $\Gamma_{tx}$, on note $\nu_t$,
$\nu_x$ les composantes temporelles et spatiales de $\nu_{tx}$, \ie,
$\nu_{tx} = (\nu_t, \nu_x)^t$. Soit $\enthalpydensity$, $\temperature$
solutions faibles du problème de Stefan
(\ref{eq:heat-one-phase})-(\ref{eq:heat-temperature-enthalpy}). On
note le $v = (\enthalpydensity, -\electrolytetdiff\nabla
\temperature)^t$. L'équation (\ref{eq:heat-one-phase}) s'écrit donc:
\begin{equation}\label{eq:space-time-divergence}
  \Div v = 0 \quad\text{ dans }\Lambda,
\end{equation}
où $\Div = (\frac{\partial}{\partial t}, \nabla)$.



%%Au sens faible,
%%l'équation (\ref{eq:space-time-divergence}) s'écrit:
%%\begin{equation}\label{eq:one-phase-weak-space-time}
%%  \int_\Lambda v\cdot \nabla g\,\intd{x}\intd{t} = 0, \quad \forall g\in C^\infty_0(\Lambda),
%%\end{equation}
%%où $C^\infty_0(\Lambda)$ et l'ensemble des fonctions infiniment
%%dérivables à support compact et nulles sur le bord $\partial\Lambda$.
%%En supposant que $\enthalpydensity$ et $\temperature$ sont suffisamment
%%régulière sur $\Lambda_1$ et $\Lambda_2$ et en utilisant le
%%théorème de la divergence dans l'espace-temps, l'intégrale
%%(\ref{eq:one-phase-weak-space-time}) se réécrit:
%%\begin{equation}\label{eq:one-phase-space-time}
%%  -\int_{\Lambda_1} \div v g\,\intd{x}\intd{t} + \int_{\partial \Lambda_1}gv\cdot \nu_1\intd{\sigma}
%%  - \int_{\Lambda_2}\div v g\,\intd{x}\intd{t} + \int_{\partial \Lambda_2}gv\cdot \nu_2\intd{\sigma}
%%  = 0, \quad \forall g\in C^\infty_0(\Lambda).
%%\end{equation}
%%
%%En choisissant la fonction test $g$ telle que $\supp(g) \subset
%%\Lambda_1$ ou $\supp(g)\subset \Lambda_2$, on déduit que les restrictions de
%%$\temperature$ aux domaines $\Lambda_1$ et $\Lambda_2$ satisfont les
%%équations (\ref{eq:heat-solid-phase}), (\ref{eq:heat-liquid-phase}) au
%%sens fort.
%%
%%On choisit maintenant $g$ telle que $\supp(g)\subset \Lambda$, $\supp
%%g\cap\Gamma_{xy} \neq\emptyset$. Puisque seule l'intégrale sur
%%l'interface $\Gamma_{tx}$ subsiste, et que $\nu_1 = -\nu_2$ sur $\Gamma_{tx}$, l'expression
%%(\ref{eq:one-phase-space-time}) se réduit à:
%%\begin{equation}
%%\int_{\Gamma_{tx}} g\jump{v}\cdot \nu_1\intd{\sigma} = 0.
%%\end{equation}
%%On a donc $\jump{v}\cdot \nu_1 = 0$ sur tout $\Gamma_{tx}$, c'est-à-dire
%%que:
%%\begin{align}
%%  &\jump{\enthalpydensity}\nu_t = - \jump{\nabla \beta(\enthalpydensity)}\nu_x,\\
%%  \Rightarrow &\jump{\enthalpydensity}\frac{\nu_t}{\abs{\nu_x}} = - \jump{\nabla \beta(\enthalpydensity)}\frac{\nu_x}{\abs{\nu_x}}.\\
%%\end{align}
%%Puisque $\frac{\nu_t}{\abs{\nu_x}} = v_\Gamma$,
%%$\frac{\nu_x}{\abs{\nu_x}}$ est la normale unité à l'interface
%%$\Gamma(t)$ et $\jump{\enthalpydensity} = \fusionenthalpy$, on retrouve
%%bien la condition de Stefan (\ref{eq:stefan-condition}).

Le lecteur intéressé par les aspects fonctionnels liés à cette
équivalence peut consulter le travail de travail de thèse de
P.-A. Gremaud \cite{Gremaud1991}, ou l'ouvrage de J. Hill
\cite{HillStefanProblems}.\footnote{Ou plutôt le livre rouge, sur
l'étagère en bas au fond à droite de la bibliothèque de
mathématique...}

% discrétisation: symétrie sphérique ou plane, discrétisation en temps,
%  discrétisation en espace.
\paragraph{Schéma de discrétisation en temps}
Pour obtenir un schéma de discrétisation en temps du problème de
Stefan, on part de la formulation enthalpique
(\ref{eq:heat-one-phase})-(\ref{eq:heat-one-phase-neumann}), et on
suit le travail de M. Paolini et al. \cite{Paolini1988}.

Soit un temps final $T > 0$, $N\in \mathbb \nstar$ le nombre de pas de
temps. Soit $\tau = T / N$ le pas de temps et $t^n = \tau n$. On note
$L_\beta$ la constante de Lipschitz de la fonction $\beta$:
\begin{equation}
  L_\beta = \sup_{\enthalpy}\abs{\beta'(\enthalpy)}.
\end{equation}
Soit $\mu$ un paramètre de relaxation fixé, tel que $0 < \mu \leq
1/L_\beta$.

On note $\temperature^n(x)$ et $\enthalpydensity^n(x)$ les approximation
de $\temperature(t^n, x)$ et $\enthalpydensity(t^n, x)\ \forall x\in
\Omega$.  On suppose donnée la densité d'enthalpie initiale $h^0$, et
on pose $\temperature^0 = \tinit$, $\enthalpydensity^0 = \enthalpy(\tinit)$,
avec $\enthalpy$ donnée par (\ref{eq:}). Pour $0 < n \leq N$, on résout
successivement les équations:
\begin{align}
  &\temperature^{n+1} - \frac{\tau}{\mu} \div\parent{
    \electrolytetdiff \nabla\temperature^{n+1}} = %
  \beta\parent{\enthalpydensity^n}, & \text{dans } \Omega,\label{eq:chernoff-heat}\\
  &\temperature^{n+1}(x) = \temperature_{\mathrm D}(t^{n+1}, x), &
  \forall x\in%
  \Gamma_\mathrm{D},\label{eq:chernoff-dirichlet}\\
  &\frac{\partial\temperature^{n+1}}{\partial \nu}(x) = %
  \temperature_\mathrm{N}(t^{n+1}, x), %
  & \forall x \in \Gamma_\mathrm{N},\label{eq:chernoff-neumann}
\end{align}
avec la correction pour la densité d'enthalpie à chaque pas de temps:
\begin{equation}\label{eq:chernoff-update}
\enthalpydensity^{n+1} = \enthalpydensity^{n} +
\mu\parent{\temperature^{n+1} - \beta(\enthalpydensity^{n})}, \quad
\text{dans } \Omega.
\end{equation}

Ce schéma numérique basé sur la formule de Chernoff
(\ref{eq:chernoff-update}) a été proposé en premier par M. Paolini
\cite{Paolini1988}. On décrit maintenant la discrétisation en
espace des équations (\ref{eq:chernoff-heat})-(\ref{eq:chernoff-neumann}).


\paragraph{Formulation du schéma de Chernoff en coordonnées sphériques}
La formulation faible du système d'équations
(\ref{eq:chernoff-heat})-(\ref{eq:chernoff-neumann}) consiste $\forall
n\leq N$ à chercher une fonction $\temperature^{n+1}:\Omega\to\mathbb R$
à telle que la relation:
\begin{equation}\label{eq:weak-form-cart-coord}
  \int_\Omega \temperature^{n+1} v\,\intd{x} + \frac{\tau}{\mu} %
  \int_\Omega \nabla \temperature^{n+1} \nabla v\,\intd{x} - %
  \int_{\Gamma_\mathrm{N}} \temperature^{n+1}_\mathrm{N} v\,\intd{\sigma} %
  = \int_\Omega \beta(\enthalpydensity^n)v\,\intd{x}
\end{equation}
soit vérifiée pour toute fonction test $v:\Omega\to\mathbb R$ à
symétrie sphérique.

On tire parti de la symétrie sphérique du système formé par la
particule et le bain environnant en reformulant maintenant le
problème (\ref{eq:weak-form-cart-coord}) en coordonnées sphériques.

Soit $\Phi:[0,\infty)\times[0,2\pi)\times[0,\pi)\to\mathbb R^3$ le
changement de variable de coordonnées sphériques en coordonnées
cartésiennes défini par:
\begin{equation}
  \Phi(r, \psi, \phi) = \begin{pmatrix}
    r\cos\psi\sin\phi\\
    r\sin\psi\sin\phi\\
    r\cos\phi
  \end{pmatrix},
\end{equation}
dont le jacobien de $\Phi$ est donné par:
\begin{equation}\label{eq:spherical-jacobian}
  \det J_\Phi(r, \psi,\phi) = -r^2\sin\phi.
\end{equation}
On note $\hat e_r$, $\hat e_\psi$ et $\hat e_\phi$ les vecteurs de la
base standard orthonormée associée au système de coordonnée sphérique.

On dit qu'une fonction $f:\mathbb R^3\to\mathbb R$ est à symétrie
sphérique s'il existe une fonction $\tilde f:\mathbb
\rplus\to\mathbb R$ telle que
\begin{equation}\label{eq:spherical-symmetry}
  \tilde f\parent{\sqrt{x_1^2 + x_2^2 + x_3^2}} = f(x_1, x_2,
  x_3)\quad \forall(x_1, x_2,x_3)\in\mathbb R^3.
\end{equation}
Le gradient de la fonctions $f$ en coordonnées sphérique s'écrit selon
$\tilde f$:
\begin{equation}\label{eq:spherical-gradient}
  \tilde\nabla\tilde f = \frac{\mathrm d\tilde f}{\mathrm dr}\hat e_r.
\end{equation}

On se donne un réel $\rmax > 0$ et le domaine $\Omega = \mathcal B(0,
\rmax) \subset \mathbb R^3$ la boule ouverte de centre $0$ de rayon
$\rmax$. On suppose maintenant que les données du problème de Stefan
(\ref{eq:heat-one-phase})-(\ref{eq:heat-one-phase-neumann}), \ie, les
fonctions $\temperature_\mathrm{D}$, $\temperature_\mathrm{N}$ ainsi
que la condition initiale $\tinit$ pour la température, sont à
symétrie sphérique, c'est-à-dire qu'il existe des fonctions $\tilde
\temperature_\mathrm{D}$, $\tilde\temperature_\mathrm{N}$ et $\tinitr$
telles que définies par la relation (\ref{eq:spherical-symmetry}).

Sous ces hypothèses, on cherche une solution au problème de Stefan
(\ref{eq:heat-one-phase})-(\ref{eq:heat-one-phase-neumann}) en
utilisant le schéma de Chernoff
(\ref{eq:chernoff-heat})-(\ref{eq:chernoff-update}) qui soit également
à symétrie sphérique.

La subdivision du bord $\Gamma_\mathrm{D}$, $\Gamma_\mathrm{N}$ doit
également satisfaire la condition de symétrie sphérique. On distingue alors les
deux seuls cas possibles:
soit le bord de Dirichlet $\Gamma_\mathrm{D} = \partial\mathcal
B(0,\rmax)$, soit le bord de Neumann $\Gamma_\mathrm{N} =
\partial\mathcal B(0,\rmax)$.

Dans le premier cas, la formulation faible de l'équation
(\ref{eq:weak-form-cart-coord}) en coordonnées sphériques consiste à
chercher une fonction $\tilde \temperature^{n+1}:[0,\rmax] \to \mathbb
R$, $\tilde\temperature^{n+1}(\rmax) =
\tilde\temperature_{\mathrm{D}}^{n+1}$ telle que:
\begin{multline}\label{eq:weak-form-spher-coord-dirichlet}
  4\pi\int_0^{\rmax} \tilde \temperature^{n+1}\tilde v r^2\,\intd{r}
  + \frac{4\pi\tau}{\mu}\int_0^{\rmax} \frac{\mathrm d %
    \tilde \temperature^{n+1}}{\mathrm d r} \frac{\mathrm d \tilde v}{\mathrm d %
    r}r^2\,\intd{r}  %
  = 4\pi\int_0^{\rmax} \beta(\tilde \enthalpydensity^n)\tilde v r^2\,\intd{r}
\end{multline}
pour toute fonction $\tilde v:[0,\rmax]\to\mathbb R$, $\tilde v(\rmax)
= 0$.

Dans le deuxième cas, la formulation faible de l'équation
(\ref{eq:weak-form-cart-coord}) en coordonnées sphériques consiste à
chercher une fonction $\tilde \temperature^{n+1}:[0,\rmax] \to \mathbb
R$, telle que:
\begin{multline}\label{eq:weak-form-spher-coord-neumann}
  4\pi\int_0^{\rmax} \tilde \temperature^{n+1}\tilde v r^2\,\intd{r}
  + \frac{4\pi\tau}{\mu}\int_0^{\rmax} \frac{\mathrm d %
    \tilde \temperature^{n+1}}{\mathrm d r} \frac{\mathrm d \tilde v}{\mathrm d %
    r}r^2\,\intd{r} \\
  - 4\pi\tilde\temperature^{n+1}_\mathrm{N}(\rmax)\tilde v(\max) %
  = 4\pi\int_0^{\rmax} \beta(\tilde \enthalpydensity^n)\tilde v r^2\,\intd{r},
\end{multline}
pour toute fonction $\tilde v:[0,\rmax]\to\mathbb R$.

\paragraph{Discrétisation en espace}
On considère une subdivision uniforme de l'intervalle $[0, \rmax]$, et
on note $\delta$ taille des éléments de la subdivision. Pour
discrétiser les problèmes faibles
(\ref{eq:weak-form-spher-coord-dirichlet}),
(\ref{eq:weak-form-spher-coord-neumann}) on utilise une méthode
éléments finis Lagrange, continue et linéaire par morceau. Les
intégrales qui interviennent dans les expressions
(\ref{eq:weak-form-spher-coord-dirichlet}),
(\ref{eq:weak-form-spher-coord-neumann}) sont approchées numériquement
par une formule de quadrature de Gauss à 3 points.


%%On donne maintenant la discrétisation du problème faible
%%(\ref{eq:weak-form-spher-coord}) par une méthode de Galerkin. Soit un
%%réel $\delta$ tel que $0 < \delta < \rmax$, et une subdivision
%%$\mathcal T_\delta$ de l'intervalle $[0, \rmax]$, telle que pour tout
%%élément $K\in\mathcal T_\delta$ son diamètre $\diam(K)$ soit borné par
%%$\delta$.
%%
%%On définit l'espace éléments finis
%%\begin{equation}
%%V_\delta = \cparent{v \in C^0(0, \rmax)\mid v\vert_K \in \mathbb
%%  P_1(K)\ \forall K\in\mathcal T_h},
%%\end{equation}
%%où $\mathbb P_1(K)$ est l'espace des polynômes de degré 1 sur
%%l'intervalle $K$.
%%
%%Pour tout $n \leq N$, on note $\tilde \temperature_\delta^{n+1} \in
%%V_\delta$ une approximation de la fonction $\tilde \temperature_\delta^{n+1}$.
%%L'approximation de Galerkin du problème faible (\ref{eq:weak-form-spher-coord})
%%consiste à chercher une fonction $\tilde \temperature_\delta^{n+1} \in
%%V_\delta$ telle que la relation
%%\begin{multline}\label{eq:galerkin-weak-form-spher-coord}
%%  4\pi\int_0^{\rmax} \tilde \temperature_\delta^{n+1}\tilde v_\delta r^2\,\intd{r}
%%  + \frac{4\pi\tau}{\mu}\int_0^{\rmax} \frac{\partial %
%%    \tilde \temperature_\delta^{n+1}}{\partial r} \frac{\partial \tilde v_\delta}{\partial %
%%    r}r^2\,\intd{r} \\
%%  - \frac{\abs{\Gamma_N}}{\rmax^2}\tilde\temperature^{n+1}_\mathrm{N}(\rmax)\tilde v_\delta(\rmax) %
%%  = 4\pi\int_0^{\rmax} \beta(\tilde \enthalpydensity_\delta^n)\tilde v_\delta r^2\,\intd{r},
%%\end{multline}
%%soit vérifiée pour tout $v_\delta \in V_\delta$.

\paragraph{Un test exacte du problème de Stefan à symétrie plane}
Un certain nombre de solutions exactes au problème de Stefan
classique sont connues. On décrit à présent l'une d'elles, découverte en
premier par J. Neumann. Cette solution a l'avantage de correspondre à
une situation physique que l'on peut facilement interpréter.

On considère le problème de Stefan classique formé par les équations
(\ref{eq:heat-solid-phase})-(\ref{eq:heat-liquid-phase}), et on fixe
les données de la manière suivante. On fixe
$\electrolytehc(\temperature) = 1$ et
$\electrolytetdiff(\temperature)=1$ $\forall\temperature \in \mathbb
R$, $\electrolytedensity = 1$, $\fusionenthalpy = 1$ et $\tliq =
0$. On s'intéresse à la température dans un matériau qui occupe la
demi-droite positive, c'est-à-dire que $\Omega = \rplus$.

On choisit la condition initiale suivante pour la température
$\temperature$:
\begin{equation}
  \temperature(0, x) = 0, \quad \forall x\in\rplus,
\end{equation}
et la condition aux limites de Dirichlet sur le bord $\Gamma_\mathrm{D} =
\cparent{0}$:
\begin{equation}
  \temperature(t, 0) = -1, \quad \forall t > 0.
\end{equation}

On note $X(t) > 0$ la position du front de solidification à l'instant
$t$. Dans ces conditions, on peut montrer \cite{HillStefanProblems}
par un calcul algébrique que la solution des équations
(\ref{eq:heat-solid-phase})-(\ref{eq:stefan-condition}) s'écrit:
\begin{align}
  & X(t) = \sqrt{2\gamma t},& \forall t > 0,\label{eq:neumann-sol-frontier}\\
  & \temperature(t, x) = -\gamma\int_{x / \sqrt{2\gamma
      t}}^1\exp\parent{\frac{\gamma(1 - \xi^2)}{2}}\, \intd{\xi},%
  & \forall (t, x) \in [0,\infty)\times[0,X(t)),\label{eq:neumann-sol-solid-phase}\\
    & \temperature(t, x) = 0,
    & \forall (t, x) \in [0,\infty)\times[X(t), \infty),\label{eq:neumann-sol-liquid-phase}
\end{align}
où on a noté $\gamma$ la constante réelle définie comme la solution de
l'équation transcendante:
\begin{equation}
  \gamma \int_0^1\exp\parent{\frac{\gamma(1 - \xi)^2}{2}}\,\intd{\xi}
  = 1.
\end{equation}

On trouvera les détails de la dérivation de cette solution, entre
autres, dans l'ouvrage de J. Hill \cite{HillStefanProblems}.

On calcul une approximation numérique de $\gamma$ à l'aide d'une
itération de Newton, implémentée avec le logiciel MatLAB. On obtient
\begin{equation}\label{eq:gamma}
 %\gamma =\num{0.768955338463582}.
  \gamma =\num{0.768955}.
\end{equation}

La solution
(\ref{eq:neumann-sol-solid-phase})-(\ref{eq:neumann-sol-liquid-phase})
pour la température $\temperature$ s'écrit de manière explicite
en vue de son évaluation numérique sous la forme suivante:
\begin{equation}\label{eq:neumann-sol}
  \temperature(t, x) = \left\{
  \begin{array}{ll}
    -\exp\parent{\frac{\gamma}{2}}\sqrt{\frac{\pi\gamma}{2}} %
    \parent{\erf{\frac{\gamma}{2}} - \erf{\frac{x}{2\sqrt{t}}}}, %
    & \text{si } x \leq \sqrt{2\gamma t},\\
    0 %
    & \text{sinon,}
  \end{array}
  \right.
\end{equation}
où la fonction $\erf:\mathbb R\to(-1,1)$ est la fonction d'erreur
standard définie par:
\begin{equation}
  \erf(x) = \frac{1}{\sqrt \pi}\int_0^x \exp\parent{-s^2}\,\intd{s}.
\end{equation}

La solution exacte (\ref{eq:neumann-sol}) est représentée sur la
figure \ref{fig:neumann-sol} à l'instant $t = 0.5$, avec $\gamma$
donné par la relation (\ref{eq:gamma}).

\begin{figure}
  \begin{center}
    \input{../media/particles/remelt/neumann-exact.tex}
    \caption{Solution exacte de Neumann du problème de Stefan classique
      (\ref{eq:heat-solid-phase})-(\ref{eq:stefan-condition}), évaluée
      à $t = 0.5$ sur l'intervalle $[0,1]\subset \rplus$.}
    \label{fig:neumann-sol}
  \end{center}
\end{figure}

D'un point de vue physique, la solution exacte de Neumann correspond à
la situation suivante. Un matériau, qui présente une transition entre
les phases solide et liquide à $\tliq = 0$, occupe le demi espace
$\cparent{(x_1, x_2, x_3)\in \mathbb R^3 \mid x_1 > 0}$. Ce matériau
est initialement dans l'état liquide à la température $\temperature =
\tliq$. Au temps initial, on le met en contact sur le bord
$\cparent{(x_1, x_2, x_3)\in\mathbb R^3 \mid x_1 = 0}$ avec un
réservoir thermique à température constante $\temperature = -1$. Au
cours de l'évolution temporelle, un front de transition de phase, de
liquide à solide, se propage dans le matériau. Pour des raisons de
symétrie, le front de transition reste en tout temps parallèle au
bord $\cparent{(x_1, x_2, x_3)\in\mathbb R^3 \mid x_1 = 0}$.

On utilisera cette solution exacte dans la suite pour valider le schéma de
discrétisation des équations
(\ref{eq:heat-one-phase})-(\ref{eq:heat-one-phase-neumann}), que l'on
décrit maintenant.

\paragraph{Validation numérique}
Pour valider l'implémentation du schéma numérique
(\ref{eq:chernoff-heat})-(\ref{eq:chernoff-update}), on évalue la
convergence de l'erreur entre la solution exacte définie par la
relation (\ref{eq:neumann-sol}) et $\temperature^N_\delta$,
l'approximation de la température à l'instant $T$. On fixe $\mu =
L_\beta = 1$. On choisit $\Omega = [0, 1] \subset \rplus$ et $T =
0.5$. On impose des conditions aux limites de Dirichlet sur le bord de
$\Omega$, \ie, $\Gamma_\mathrm{D} = \cparent{0, 1}$. On se donne
$\temperature_\mathrm{D}(0) = -1$ et $\temperature_\mathrm{D}(1) =
0$. On choisit le pas de temps $\tau$ dans l'intervalle
$[\num{3.9e-5}, \num{5e-3}]$, et $\delta = O(\tau)$.

\begin{figure}
  \begin{center}
    \input{../media/particles/remelt/neumann-convergence-h.tex}
    \caption{Erreur $L^2$ entre la solution exacte de Neumann et
      l'approximation numérique $\temperature^N_\delta$}
    \label{fig:neumann-convergence}
  \end{center}
\end{figure}

On constate sur la figure \ref{fig:neumann-convergence} que l'erreur
$L^2$ entre la solution exacte et l'approximation numérique converge
vers 0, avec un ordre approximatif $O(h^{3/4})$.




% application: formation et refonte de bain gele autour d'une
% particule.
\paragraph{Formation de gel autour d'une particule}
Dans cette partie, on applique la méthode numérique
(\ref{eq:chernoff-heat})-(\ref{eq:chernoff-neumann}) au calcul de
la formation de gel autour d'une particule de bain gelé.

On donne dans le tableau \ref{tab:freeze-physical-parameters} la
valeur des différents paramètres physique liés aux propriétés
thermiques du bain électrolytique.

\begin{table}
  \begin{center}
    \caption{Paramètres physiques qui interviennent dans le
      phénomène de formation de gel autour de particules.}
    \label{tab:freeze-physical-parameters}
    \begin{tabularx}{\textwidth}{@{}lllX@{}}
      \toprule
      Quantité              & Valeur       & Unité                                       & Description \\
      \midrule
      $\electrolytedensity$ & \num{2130}    & \si{\kg\per\cubic\meter}                    & Masse volumique du bain électrolytique \\
      $\electrolyteshc$     & \num{1403}    & \si{\joule\per\kilo\gram\per\kelvin}        & Chaleur spécifique de l'électrolyte dans la phase solide \\
      $\electrolytelhc$     & \num{1861.3}  & \si{\joule\per\kilo\gram\per\kelvin}        & Chaleur spécifique de l'électrolyte dans la phase liquide \\
      $\fusionenthalpy$     & \num{5.508e5} & \si{\joule\per\kilo\gram}                   & Chaleur latente de transition de phase solide-liquide \\
      $\tinj$               & \num{423.15}  & \si{\kelvin}                                & Température de la particule de bain gelé au moment de l'injection \\
      $\tliq$               & \num{1223.15}  & \si{\kelvin}                                & Température du liquidus de l'électrolyte \\
      $\electrolytetdiff$   & \num{2}     & \si{\joule\per\second\per\meter\per\kelvin} & Conductivité thermique de l'électrolyte \\
      \bottomrule
    \end{tabularx}
\end{center}
\end{table}

Puisqu'on s'intéresse à une particule sphérique, on fait l'hypothèse
que les données du problème présentent une symétrie sphérique. Ainsi,
on décrite l'ensemble des données en fonction de la distance à
l'origine $r$. De plus, on cherche la température $\temperature$ dans
l'électrolyte en fonction de la coordonnée $r$. Par abus de notation
et pour éviter de surcharger les notation, on réutilise les mêmes
symboles, mais explicitant la coordonnée $r$.

\begin{figure}[h]
  \begin{center}
    \input{../media/particles/temperature/temperature.tex}
    \caption{Température initiale du système formé par la particule de
      bain gelé placée à l'origine du système de coordonnées, et du
      bain électrolytique environnant.}
    \label{fig:particle-initial-temperature}
  \end{center}
\end{figure}

Soit $R_p$ le rayon de la particule placée à l'origine du système de
coordonnées et soit $\tinj\in\mathbb R$ la température à laquelle se
trouve les particules d'alumine au moment de leur injection dans la
cuve. Soit $\telectrolyte$ la température du bain électrolytique. La
condition initiale du système formé par la particule et le bain
environnant est définie par:
\begin{equation}
  \tinit(r) = \left\{
  \begin{array}{ll}
    \tinj & \text{ si } r < R_p,\\
    \telectrolyte & \text{ si } r \geq R_p,\\
  \end{array}
  \right.\quad \forall r\in \rplus .\label{eq:initial-temperature}
\end{equation}

La figure \ref{fig:particle-initial-temperature} représente la
température initiale au voisinage de la particule. En dehors de la
particule, la température du bain est suffisante pour que celui-ci
soit dans la phase liquide. Cependant, on suppose que le bain liquide
est au repos en tout temps.

On note $R_f$ la distance entre l'origine et la position du front de
solidification, définie par la relation:
\begin{equation}
\beta(\enthalpydensityr(t, R_f(t))) = \tliq.
\end{equation}
L'épaisseur de la couche de bain solidifiée $R_g$ est définie par la
relation:
\begin{equation}
R_g(t) = R_f(t) - R_p.
\end{equation}
On définit alors le temps de latence $\tlat(R_p) > 0$, le temps
nécessaire à ce que l'épaisseur de bain solidifié atteigne à nouveau
zéro:
\begin{equation}
  R_g(\tlat).
\end{equation}

\begin{figure}
\begin{center}
  \input{../media/particles/remelt/frozen-layer-sur2-5.tex}
  \input{../media/particles/remelt/frozen-layer-sur5.tex}
  \input{../media/particles/remelt/frozen-layer-sur10.tex}
  \caption{Évolution du rayon de la sphère de bain gelé pour des
    température de surchauffe de \num{2.5}\si{\kelvin},
    \num{5}\si{\kelvin} et \num{10}\si{\kelvin} et des particules de
    rayons initiaux $r_0 = $\num{40}\si{\micro\meter},
    \num{60}\si{\micro\meter} et \num{80}\si{\micro\meter}.}
  \label{fig:freeze-radius}
\end{center}
\end{figure}
On considère une particule de bain solidifié de rayon initial $R_0 =$
\num{40}\si{\micro\meter}, \num{60}\si{\micro\meter} et
\num{80}\si{\micro\meter}. La surchauffe du bain électrolytique est
définie ici comme la différence entre la température initiale du bain
$\telectrolyte$ et la température de liquidus $\tliq$. La figure
\ref{fig:freeze-radius} présente l'évolution du rayon de la particule
pour trois surchauffes différentes du bain électrolytique environnant.

On constate que le temps nécessaire à refondre le bain gelé
diminue avec la surchauffe. De même, le temps de refonte diminue avec
la taille de la particule.

La température initiale de la particule est $T_\text{inj} =
423.15\si{\kelvin}$. Dans le cas le plus défavorable où $T -
T_\text{Liq} = 2.5\si{\kelvin}$ et $r_0 = 80\si{\micro\meter}$, le
temps nécessaire pour refondre entièrement la couche de gel est de
l'ordre de 120\si{\milli\second}.

Dans les calculs présentés ici, on a fait plusieurs hypothèses
simplificatrices. On a supposé que le fluide environnant est au
repos, ce qui n'est certainement pas le cas dans une cuve
d'électrolyse industrielle. Les forces de Lorentz et la formation de
bulles de gaz agitent les fluides, et créent des turbulences qui sont,
entre autres, de l'échelle des particules \cite{Rochat2016}. Une
agitation de l'électrolyte au voisinage d'une particule accélère le
transport de l'énergie thermique, et tend à accélérer la refonte du
bain gelé.

Pour simplifier le modèle, nous avons étudié la formation de gel
autour d'une particule sphérique de bain solidifiée. Bien entendu,
dans une cuve d'électrolyse industrielle on injecte des particules
particules d'alumine à basse température, plutôt que des particule de
bain gelé. Cependant, les propriétés thermiques des deux systèmes sont
proches, ce qui justifie cette approximation.

Lors de l'injection d'une dose de particules d'alumine, l'hypothèse
d'une particule isolée dans un bain infini au repos n'est plus
vérifiée. Les différentes particules qui constituent la dose
interagissent thermiquement entre elles par l'intermédiaire du bain
environnant. Ce comportement explique la raison pour laquelle
l'hypothèse de particules isolée mène à des temps typique de
formation et refonte de gel largement inférieurs aux valeurs
recommandées par exemple dans le travail de V. Dassylva-Raymond \cite{Dassylva2015}.

Selon la recommandation de V. Dassylva-Raymond \cite{Dassylva2015}, on
choisira, dans la suite, pour le temps de latence d'une dose de
particules injectées dans le bain une valeur uniforme $\tlat =
1\si{\second}$, indépendante du rayon.


\section{Dissolution d'une particule dans le bain}
\label{sec-particle-dissolution}
Dans cette section, nous nous intéressons à la modélisation de
la dissolution d'une particules d'alumine individuelle dans un bain
électrolytique. On suppose que cette particule, après avoir été
injectée à température $\tinj$, s'est maintenant réchauffée et se
trouve à la même température $\temperature$ que le bain environnant.

On donne ci-dessous un aperçu de la compléxité du phénomène de
dissolution des particules d'alumine dans le bain
électrolytique. C'est un sujet qui a déjà fait l'objet de nombreux
travaux de recherche, voir par exemple \cite{Dassylva2015},
\cite{Kvande1986}, \cite{Gerlach1975}, \cite{Solheim1995}. La
taille initiale des particules qui forme une dose est comprise pour la
plupart entre $\num{20}\si{\micro\meter}$ et
$\num{100}\si{\micro\meter}$. Ces particules sont seulement
approximativement sphériques. Elles sont le résultat de
l'agglomération de micro-cristaux et peuvent présenter une porosité
importante, jusqu'à \num{75}\%. Cette porosité joue un rôle sur la
capacité du bain à mouiller la surface de la particule, et à permettre
sa dissolution. De même, si les particules sont suffisamment fines
celle-ci deviennent volatiles et sont facilement soufflées loin du
trou d'injection par les gaz s'en échappant. Si ces particules fines
atteignent malgré tout l'électrolyte, celui-ci a du mal à les mouiller
et a permettre leur dissolution. A l'inverse, des particules trop
grandes sont trop longues à dissoudre et risquent de sédimenter au
fond de la cuve.

L'alumine a une forte affinité avec l'eau
\cite{Patterson2001}. Lorsque les particules entrent en contact avec
le l'électrolyte en fusion, le gaz qui résulte de la désorption des
molécule d'eau crée une forte agitation qui favorise la dispersion et
la dissolution des particules. Par exemple, Haverkamp et
al. \cite{Haverkamp1994} ont montré que le temps de dissolution de
particules d'alumine hydratées peut être jusqu'à \num{40}\% plus
faible par rapport à des particules sèches.

La dissolution de l'oxyde d'aluminium l'électrolyte est une réaction
endothermique. La vitesse de cette réaction dépend de la chimie du
bain, \ie, des concentrations respectives des différentes espèces qui
le constitue, et de la surchauffe du bain à proximité et de la
particule. La température du liquidus $\tliq$, dont dépend la
surchauffe, est elle-même fonction de la concentration d'alumine
dissoute et des autres espèces chimiques. La concentration de
saturation de l'alumine $\csat$ dépend aussi de la température de
l'électrolyte.

Finalement, l'écoulement des fluide à proximité de la surface de la
particule peut influencer sa dissolution. Au niveau microscopique,
deux mécanismes entrent en compétition au niveau de la surface de la
particule d'alumine. Il y a d'une part la réaction de dissolution
elle-même qui consiste à détacher une molécule d'\ce{Al2O3} et à
former différents complexes avec les ions \ce{AlF6^{3-}}
\cite{Haupin1995}, \cite{Kvande1986}, et d'autre part le mécanisme de
diffusion moléculaire qui transporte l'alumine dissoute loin de la
particule. La diffusion moléculaire est essentielle pour maintenir la
concentration à la surface de la particule inférieure à la
concentration de saturation $\csat$ et permettre la réaction de
dissolution. Clairement, si la concentration est saturée dans
l'ensemble du bain, la dissolution ne peut pas avoir lieu. La réaction
de dissolution étant elle-même endothermique, elle est contrôlée par
la disponibilité d'énergie sous forme thermique à proximité de la
particule, \ie, de la surchauffe du bain.

Dans cette section, on propose un modèle qui décrive la vitesse de
dissolution d'une particule dans le bain d'une cuve d'électrolyse
d'aluminium, basé sur le travail de thèse de T. Hofer
\cite{Hofer2011}.

On fait l'hypothèse que les particules d'alumine sont des sphères
parfaites et non poreuses. On négligera la possible hydratation de
l'alumine qui les constituent et son effet sur leur dissolution. Dans
le travail de T. Hofer \cite{Hofer2011}, la vitesse de dissolution
d'une particule est une fonction $f$ qui décrit la variation de son
rayon et qui qui dépend d'une part de son rayon actuel $r$, et d'autre
part de la concentration locale $c$. Dans ce travail, nous
considèrerons en plus l'influence de la température locale de
l'électrolyte $\temperature$ sur sa vitesse de dissolution,
c'est-à-dire que
\begin{equation}\label{eq:radius-edo}
  \frac{\mathrm dr}{\mathrm dt} = f(r, \concentration, \temperature).
\end{equation}

On fait l'hypothèse que la dissolution des particules est
contrôlée par la diffusion de la concentration dans l'électrolyte et
que la concentration de saturation $\csat$ est indépendante de la
température $\temperature$ de l'électrolyte.

En suivant le travail de T. Hofer \cite{Hofer2011}, on propose
d'écrire la fonction $f$ comme
\begin{equation}\label{eq:dissolution-velocity}
  f(r, c, \temperature) = -\frac{\dissolutionrate(c, \temperature)}{r},
\end{equation}
où $\dissolutionrate$ est le taux de dissolution. Le facteur $1/r$ est
lié à l'hypothèse que la diffusion de la concentration contrôle
dissolution, comme l'ont montré Wang et Flanagan \cite{Wang1999}.

Clairement, le taux de dissolution $\dissolutionrate$ doit être
maximal lorsque $c = 0$, et s'annuler lorsque l'électrolyte est saturé
en alumine, \ie, lorsque $c = \csat$. De même, lorsque le bain se
trouve à la température du liquidus $\tliq$, il n'y a pas d'énergie
thermique à disposition pour la réaction de dissolution, et donc
$\dissolutionrate$ doit s'annuler si $\temperature = \tliq$. A
l'inverse, $\dissolutionrate$ doit être maximal lorsque
$\temperature\gg\tliq$.

Motivé par ces dernières considérations, on propose la forme suivante
pour l'expression du taux de dissolution:
\begin{equation}\label{eq:dissolution-rate}
  \dissolutionrate(\concentration, \temperature) =
  \left\{
  \begin{array}{ll}
    K\displaystyle\frac{\csat - \concentration}{\csat}\parent{1 -
      \exp\parent{-\displaystyle\frac{\temperature - \tliq}{\tcrit -
          \tliq}}} & \text{si }\temperature \geq \tcrit\text{ et }
    0\leq c
    \leq \csat,\\
    0                                     & \text{sinon,}
  \end{array}
  \right.
\end{equation}
où $K > 0$ et $\tcrit > \tliq$ sont deux paramètres du modèle. Le taux
de dissolution limite $K$ est tel que
\begin{equation}
  \lim_{\temperature \to\infty}\dissolutionrate(0, \temperature) = K.
\end{equation}
La figure \ref{fig:diss-rate} illustre la relation entre le taux de
dissolution et la température dans le cas où la concentration locale
$\concentration = 0$ et le taux de dissolution limite $K = 1$.

\begin{figure}[h]
  \begin{center}
    \input{../media/particles/diss-rate/diss-rate.tex}
    \caption{Relation entre la température $\temperature$ et le taux de
      dissolution $\dissolutionrate$ lorsque $c = 0$ et que le taux
      de dissolution limite $K = 1$. En vert sont les traits de
      construction, et donnent une interprétation du paramètre
      $\tcrit$.}
    \label{fig:diss-rate}
  \end{center}
\end{figure}

La vitesse de dissolution $f$ décrite ci-dessus constitue la base qui
permet la description de l'évolution d'une population de particules
dans un bain électrolytique qui fait l'objet de la section suivante.


\section{Dissolution d'une population de particules}
\label{sec:particle-population-dissolution}
Selon la distribution de tailles des particules d'alumine, une dose de
\num{1}\si{\kilo\gram} contient typiquement entre \num{1.e+9} et
\num{1.e+12} particules. Il n'est donc pas envisageable de suivre
individuellement l'évolution de chacune d'elles dans le bain. En
suivant le travail de T. Hofer \cite{Hofer2011} on adopte une approche
statistique et on décrit l'ensemble des particules dans le bain par
l'intermédiaire d'une distribution continue. On note $n_p(t, r)\intd{r}$ le
nombre de particules dont le rayon est compris dans l'intervalle $[r,
  r + \intd{r}]$ à l'instant $t$. En suivant le travail de T. Hofer \cite{Hofer2011}, on suppose que
cette population $\population$ se dissout selon l'équation
\begin{align}
&\frac{\partial \population}{\partial t} - \frac{\partial
}{\partial r}\parent{f(r,c,\temperature)n_p} = 0\quad \text{si } r > 0,\ t > 0,\label{eq:population-pde}\\
&n_p(0, r) = n_{p,0}(r) \quad \text{si } r > 0,\label{eq:population-ic}
\end{align}
où la vitesse de dissolution $f$ est donnée par l'expression
(\ref{eq:dissolution-velocity}) et $n_{p,0}$ est une distribution de
particules initiale donnée.

Supposons maintenant que $\concentration$ et $\temperature$ sont des
constantes données. Alors
$\dissolutionrate(\concentration,\temperature)$ est un réel fixé, et
on omettra d'indiquer ses arguments dans la suite, lorsqu'il n'y a pas
de risque d'ambiguïté. Dans ces conditions, on peut trouver une
solution du système d'équations l'équation
(\ref{eq:population-pde})-(\ref{eq:population-ic}) sous forme
analytique par la méthode des caractéristiques.

Considérons tout d'abord une particule individuelle de rayon initiale
$r_0$. Son rayon varie au cours du temps selon l'équation
(\ref{eq:radius-edo}) que l'on réécrit en explicitant le membre de
droite:
\begin{align*}
  &\frac{\mathrm d r}{\mathrm d t} = -\frac{\dissolutionrate}{r} \quad \text{si } t
  > 0,\\
  & r(0) = r_0.
\end{align*}
Ainsi, en notant $\dot r = \frac{\mathrm d}{\mathrm dt}$ on obtient:
\begin{align}
  &r\dot r = -\dissolutionrate,\nonumber\\
  &\frac{1}{2}\frac{\mathrm d}{\mathrm dt}r^2 = -\dissolutionrate,\nonumber\\
  &r^2 = -2\dissolutionrate t + r_0^2,\nonumber\\
  &r(t) = \sqrt{r_0^2-2\kappa t}.\label{eq:particle-exact}
\end{align}
Si $\bar t = \frac{r_0^2}{2\dissolutionrate}$ on obtient $r(\bar
t) = 0$. La particule de rayon $r_0$ sera dissoute après un temps $\bar
t = \frac{r_0^2}{2\dissolutionrate}$.

Revenons maintenant à l'équation pour la population de particule
$\population$. On a:
\begin{equation}\label{eq:population-pde-2}
  \frac{\partial n_p}{\partial t} -
  \frac{\dissolutionrate}{r}\frac{\partial n_p}{\partial r} +
  \frac{\dissolutionrate}{r^2}n_p = 0.
\end{equation}
\newcommand{\rcharacteristic}{R_{(\bar t, \bar r)}}
Si, dans le plan $(t, r)$ on fixe $t = \bar t > 0$, $r = \bar r > 0$,
l'équation de la courbe caractéristique $\rcharacteristic$ qui passe par $(\bar t,
\bar r)$ est donnée par
\begin{equation*}
\frac{\mathrm d}{\mathrm dt}R_{(\bar t, \bar r)}(t) =
-\frac{\dissolutionrate}{R_{(\bar t, \bar r)}(t)}.
\end{equation*}
Ainsi $R_{(\bar t, \bar r)}(t) = \sqrt{\bar r^2 - 2\dissolutionrate (t
  - \bar t)}$, en vertu de (\ref{eq:particle-exact}). Notons $N_p(t) =
n_p(t, R_{(\bar t, \bar r)}(t))$ la valeur de $n_p$ sur cette
caractéristique. En dérivant par rapport à $t$ et en utilisant
(\ref{eq:population-pde-2}) on a
\begin{align*}
  \frac{\mathrm d}{\mathrm dt}N_p(t) %
  &= \frac{\partial n_p}{\partial t}(t, \rcharacteristic(t)) %
  + \frac{\partial}{\partial r}n_p(t,\rcharacteristic(t)) %
    \frac{\mathrm d}{\mathrm dt}\rcharacteristic(t)\\
  &= \frac{\partial n_p}{\partial t}(r, \rcharacteristic) - \frac{\dissolutionrate}{\rcharacteristic}\frac{\partial}{\partial r} n_p(t, \rcharacteristic(t))\\
  &= -\frac{\dissolutionrate}{R_{(\bar t, \bar r)}^2(t)}N_p(t). %
\end{align*}
Ainsi
\begin{equation*}
  \frac{1}{N_p(t)}\frac{\mathrm d}{\mathrm dt}N_p(t) =
  -\frac{\dissolutionrate}{R^2_{(\bar t, \bar r)}(t)}
\end{equation*}
qui implique que
\begin{equation*}
  \frac{\mathrm d}{\mathrm dt}\ln(N_p(t)) =
  -\frac{\dissolutionrate}{\parent{\bar r^2 - 2\dissolutionrate (t - \bar t)}}.
\end{equation*}
En intégrant on obtient
\begin{equation*}
  \ln(N_p(t)) = \frac{1}{2}\ln\parent{\bar r^2 -
    2\dissolutionrate\parent{t - \bar t}} + D
\end{equation*}
et donc
$N_p(t) = \parent{\bar r^2 - 2\dissolutionrate\parent{t - \bar t}}^{1/2} \exp(D)$. Si $t = 0$, on a:
\begin{equation*}
  N_p(0) = \exp(D)\sqrt{\bar r^2 +
    2\dissolutionrate \bar t} = n_{p,0}(R(0)) = n_{p,0}(\sqrt{\bar r^2 +
    2\dissolutionrate \bar t})
\end{equation*}
Ainsi
$\exp(D) = \frac{n_{p,0}\parent{\sqrt{\bar r^2 + 2\dissolutionrate \bar t\,}}}{\sqrt{\bar r^2 + 2\dissolutionrate \bar t\,}}$
et
\begin{equation*}
  N_p(t) = \frac{n_{p,0}(\sqrt{\bar r^2 + 2\dissolutionrate \bar
      t})}{\sqrt{\bar r^2 + 2\dissolutionrate \bar t\,}}\parent{\bar r^2 -
    2\dissolutionrate \parent{t - \bar t}}^{1/2}.
\end{equation*}
Clairement
$N_p(\bar t) = n_p(\bar t, \bar r)$
et donc
\begin{equation}\label{eq:population-exact-solution}
  n_p(\bar t, \bar r) = \bar r\frac{n_{p,0}(\sqrt{\bar r^2 +
      2\dissolutionrate \bar t})}{\sqrt{\bar r^2+2\dissolutionrate \bar t}}.
\end{equation}

\begin{remarque}\label{rem:population-exact-step}
  Supposons connu $n_p(\bar t - \Delta t, r)$. Nous avons ainsi
  \begin{equation*}
    R_{(\bar t, \bar r)}(\bar t - \Delta t) = \sqrt{\bar r^2 +
      2\dissolutionrate \Delta t\,}.
  \end{equation*}
  Puisque $N_p(t) = C\parent{\bar r^2 - 2\dissolutionrate\parent{t -
      \bar t}}^{1/2}$, alors
  \begin{equation*}
N_p(\bar t - \Delta t) = C\parent{\bar r^2 + 2\dissolutionrate \Delta
  t} = n_p\parent{t - \Delta t, \sqrt{\bar r^2 + 2\dissolutionrate
    \Delta t\,}}.
  \end{equation*}
  Ainsi
  \begin{equation*}
    C = \frac{n_p(t - \Delta t, \sqrt{\bar r^2 + 2\dissolutionrate
        \Delta t})}{\bar r^2 + 2\dissolutionrate\Delta t}
  \end{equation*}
  et donc
  \begin{equation*}
    N(t) = \frac{n_p(t - \Delta t, \sqrt{\bar r^2 + 2\dissolutionrate
        \Delta t})}{\bar r^2 + 2\dissolutionrate\Delta t}\parent{\bar
      r^2 - 2\dissolutionrate\parent{t - \bar t}}^{1/2}.
  \end{equation*}
  Puisque $N(\bar t) = n_p(\bar t, \bar r)$, on obtient
  \begin{equation*}
    n_p(\bar t, \bar r) = \bar r\frac{n_p(t - \Delta t, \sqrt{\bar
        r^2 + 2\dissolutionrate \Delta t})}{\sqrt{\bar r^2 +
        2\dissolutionrate \Delta t}}.
  \end{equation*}
\end{remarque}

Dans cette partie nous avons présenté le modèle décrit l'évolution d'une
distribution de particules qui se dissout dans un bain électrolytique
selon la fonction (\ref{eq:dissolution-velocity}). On a montré que
l'évolution de la population $n_p$ s'exprime sous forme exacte,
donnée par (\ref{eq:population-exact-solution}), pour
autant que la concentration $c$ et la température $\temperature$
sont maintenue constante.

Nous utiliserons le résultat de la remarque
\ref{rem:population-exact-step} lors de la discusssion de la
discrétisation du modèle de transport et dissolution complet dans le
chapitre \ref{chap:populations}.

La section suivante tient lieu de conclusion ce chapitre, et discute
de la sédimentation des particules d'alumine dans le bain sous
l'action de la force de gravité.

%%\paragraph{Approximation numérique}
%%Soit $T>0$ le temps final, $N \in \mathbb N^*$ le nombre de pas de
%%temps, $\tau = \frac{T}{N}$, $\Delta r = \frac{\rmax}{M}$,
%%$t_i = i \tau$, $r_j = j\Delta r$ constantes


\section{Chute de particules dans le bain}
\label{sec:particle-fall}
La densité de l'alumine $\aluminadensity$ qui constitue les particules
est de \num{3960}\si{\kilo\gram\per\cubic\meter} soit environ deux
fois supérieure à celle du bain électrolytique $\electrolytedensity$,
qui est de \num{2130}\si{\kilo\gram\per\cubic\meter}. Une particule
d'alumine placée dans un bain électrolytique animé par un écoulement
stationnaire subit ainsi l'effet de trois forces
distinctes. D'une part, cette particule est freinée dans le fluide
par l'intermédiaire d'une force de traînée $\dragforce$. Cette force
est opposée à la vitesse relative de la particule par rapport au
fluide. D'autre part, cette particule est entraînée vers le fond de
la cuve par la force de gravité $\gravityforce$, à laquelle s'oppose
la force d'Archimède $\buoyancyforce$.

Dans cette section, nous nous proposons d'évaluer l'importance des
forces de gravité et d'Archimède devant la force de traînée. Dans ce
but, on considère une particule d'alumine sphérique de rayon initial
$r_0$ placée dans un bain électrolytique au repos. On suppose que la
température du bain $\temperature$ et la concentration d'alumine
$\concentration$ sont maintenue constantes au cours du temps. On
suppose de plus que le mouvement du fluide autour de la particule
n'influence pas sa dissolution. Alors, la variation du rayon $r$ de la
particule au cours du temps est décrite par l'équation
(\ref{eq:radius-edo-cst-diss-rate}). Comme on l'a vu dans la section
\ref{sec:particle-population-dissolution}, le rayon à l'instant $t$
est donné par
\begin{equation}\label{eq:radius}
  r(t) = \sqrt{r_0^2 - 2\dissolutionrate t},
\end{equation}
où $\dissolutionrate$ est le taux de dissolution, constant au cours du
temps. Clairement pour
\begin{equation}\label{eq:final-time}
  T = \frac{r_0^2}{2\dissolutionrate}
\end{equation}
on obtient $r(T) = 0$, \ie, la particule est complètement dissoute
au temps $T$.

On modélise la particule d'alumine, supposée non poreuse, par un point
matériel de volume $V(t) = \frac{4}{3}\pi r^3(t)$. On travaille dans
le référentiel de la Terre, et on suppose qu'elle se déplace
verticalement dans ce référentiel. On note $x(t)\in \mathbb R$ sa
position selon un système de coordonnées vertical et dirigé vers le
bas à l'instant $t\in [0,T]$ et $\dot x(t) = \frac{\mathrm dx}{\mathrm
  dt}(t)$ sa vitesse. La particule étant sphérique, on propose
d'approximer la force de traînée par la loi de Stokes. Si $v$ est la
vitesse de la particule par rapport au fluide, lui-même au repos dans
le référentiel, on a
\begin{equation*}
\dragforce(r, v) = -6\pi\mu r v.
\end{equation*}
Ici on a noté $\mu$ la viscosité dynamique de l'électrolyte. Il est
connu que cette approximation est valide pour autant que le nombre de
Reynolds de l'écoulement $\reynolds$ soit suffisamment faible, \ie,
$\reynolds \lesssim 1$. On montrera a posteriori que cette condition
est en général satisfaite pour des particules d'alumine qui sont
typiquement injectées dans le bain électrolytique.

Les forces de gravité $\gravityforce$ et d'Archimède $\buoyancyforce$
en fonction du rayon $r$ de la particule sont données respectivement
par
\begin{equation*}
\gravityforce(r) = \frac{4}{3}\pi r^3 \aluminadensity g
\quad\text{et}\quad
\buoyancyforce(r) = -\frac{4}{3}\pi r^3 \electrolytedensity g
\end{equation*}
où $g = 9.81$ \si{\meter\per\squared\second} est l'accélération de la gravité.

L'équation du mouvement de la particule s'obtient à l'aide de la
deuxième loi de Newton qui lie la somme des forces à l'impulsion $p =
m\dot x$:
\begin{equation*}
\frac{\mathrm dp}{\mathrm dt}(t) = \dragforce(r(t), \dot x(t)) + \gravityforce(r(t)) + \buoyancyforce(r(t)),
\end{equation*}
soit
\begin{equation}\label{eq:mvt}
  \frac{\mathrm d}{\mathrm dt}\parent{\aluminadensity V(t)\dot x(t)} =
  \frac{4}{3}\pi r^3(t)\parent{\aluminadensity - \electrolytedensity}g -
  6\pi\electrolyteviscosity r(t)\dot x(t).
\end{equation}
On suppose que la particule se trouve initialement en $x(0) = 0$ avec
une vitesse $\dot x(0) = v_0$.

En utilisant la relation (\ref{eq:radius}), nous avons
\begin{equation}\label{eq:volume-variation}
  \frac{\mathrm d}{\mathrm dt}V(t) = -4\pi \dissolutionrate r(t).
\end{equation}
En remplaçant (\ref{eq:volume-variation}) dans l'égalité
(\ref{eq:mvt}), en tenant compte du fait que $V(t) = \frac{4}{3}\pi
r^3(t)$, puis en divisant par $\pi r(t)$ nous obtenons
\begin{equation}\label{eq:simple-mvt}
\aluminadensity \frac{4}{3}r^2(t)\frac{\mathrm d^2}{\mathrm dx^2}x(t)
= g\parent{\aluminadensity - \electrolytedensity}\frac{4}{3}r^2(t) -
\parent{6\electrolyteviscosity - 4\aluminadensity
  \dissolutionrate}\frac{\mathrm dx}{\mathrm dt}.
\end{equation}
En utilisant (\ref{eq:radius}), (\ref{eq:final-time}) et en
intégrant l'équation (\ref{eq:simple-mvt}) sur l'intervalle de temps $[0, T]$, nous
obtenons
\begin{equation}\label{eq:mvt-integration-1}
\parent{6\electrolyteviscosity - 4\aluminadensity\dissolutionrate}x(T)
= g\parent{\aluminadensity -
  \electrolytedensity}\frac{r_0^4}{3\dissolutionrate}-\aluminadensity\frac{4}{3}\int_0^T\parent{r_0^2
- 2\dissolutionrate t}\frac{\mathrm d^2}{\mathrm dt^2}x(t)\,\intd{t}.
\end{equation}
En intégrant le dernier terme de l'équation
(\ref{eq:mvt-integration-1}) par partie, nous obtenons finalement la
profondeur terminale de la particule au moment de sa dissolution complète
\begin{equation}\label{eq:particle-terminal-position}
x(T) = \frac{g\parent{\aluminadensity - \electrolytedensity
    }}{\dissolutionrate
  \parent{18\electrolyteviscosity - 4\aluminadensity
    \dissolutionrate}}r_0^4 + \frac{2 \aluminadensity
  v_0}{9\electrolyteviscosity - 2\aluminadensity \dissolutionrate} r_0^2.
\end{equation}

\begin{table}
  \begin{center}
    \caption{Paramètres physiques qui interviennent dans la chute
      d'une particule d'alumine dans un bain électrolytique.}
    \label{tab:fall-physical-parameters}
    \begin{tabularx}{\textwidth}{@{}lllX@{}}
      \toprule
      Quantité                & Valeur       & Unités                                      & Description \\
      \midrule
      $\electrolytedensity$   & \num{2130}   & \si{\kg\per\cubic\meter}                    & Masse volumique du bain électrolytique \\
      $\aluminadensity$       & \num{3960}   & \si{\kg\per\cubic\meter}                    & Masse volumique de l'oxyde d'aluminium \\
      $g$                     & \num{9.81}   & \si{\meter\per\square\second}               & Accélération de la gravité terrestre\\
      $\dissolutionrate$      & \num{0.5e-9} & \si{\square\meter\per\second}               & Taux de dissolution de l'alumine \\
      $\electrolyteviscosity$ & \num{2e-3}   & \si{\kilo\gram\per\meter\per\second}        & Viscosité dynamique du bain électrolytique \\
      \bottomrule
    \end{tabularx}
  \end{center}
\end{table}

Les valeurs des paramètres physiques qui correspondent à une particule
d'alumine dans un bain et qui interviennent dans l'équation
(\ref{eq:particle-terminal-position}) sont synthétisées dans le
tableau \ref{tab:fall-physical-parameters}. Contrairement à
l'hypothèse admise ici, le bain d'une cuve d'électrolyse est animé par
une forte agitation. Ces turbulences correspondent à une viscosité
équivalente $\electrolyteturbviscosity$ qui caractérise l'écoulement
moyenné au cours du temps. Dans le bain d'une installation
industrielle, cette viscosité turbulente est typiquement de l'ordre de
\num{1} \si{\kilo\gram\per\meter\per\second} ou plus, et domine donc
largement la viscosité physique du fluide. Pour cette raison nous
considérons des viscosités dans l'intervalle $[\num{2e-3}, \num{1}]$.

Les doses de particules d'alumine ne sont en général pas délicatement
déposées à la surface du bain; les particules sont lâchées depuis une
hauteur qui varie entre \num{20}\si{\centi\meter} et
\num{40}\si{\centi\meter} par rapport à la surface du bain. On
modélise cette condition à l'aide de la vitesse initiale $v_0$. La
vitesse verticale atteinte par une masse en chute libre dans le champ
de pesanteur terrestre sur une hauteur $L$ et initialement au repos
est donnée par l'expression $\sqrt{2gL}$. Par conséquent nous
considérerons des vitesses initiales $v_0$ entre \num{0} et
\num{3} \si{\meter\per\second}.

\begin{table}
  \begin{center}
    \caption{Profondeur terminale de la particule dans le bain
      électrolytique en fonction des conditions initiales $r_0$
      [\si{\micro\meter}], $v_0$ [\si{\meter\per\second}] et de la
      viscosité dynamique $\electrolyteviscosity$
      [\si{\kilo\gram\per\meter\per\second}].}
    \label{tab:fall-results}
    \begin{tabularx}{\textwidth}{@{}lXXXX@{}}
      \toprule
      & $\electrolyteviscosity = \num{2e-3}$ & $\electrolyteviscosity = \num{1e-2}$ & $\electrolyteviscosity = \num{1e-1}$ & $\electrolyteviscosity = \num{1}$ \\
      \midrule
      \input{../media/particles/fall/results.tex}
      \bottomrule
    \end{tabularx}
  \end{center}
\end{table}

Le tableau \ref{tab:fall-results} présente la profondeur
terminale de la particule d'alumine donnée par la relation
(\ref{eq:particle-terminal-position}) en fonction des conditions
initiales $r_0$, $v_0$ et la viscosité dynamique du fluide
$\electrolyteviscosity$. Lorsque $\electrolyteviscosity =
\num{2e-3}$, c'est-à-dire que la particule est placée dans un fluide
immobile, une profondeur maximale de \num{4} \si{\centi\meter}
environ est atteinte lorsque $r_0 = \num{80}$ \si{\micro\meter} et
avec une vitesse initiale $v_0 = \num{3}$ \si{\meter\per\second}. La
profondeur maximale lorsque le rayon initial est inférieur à
\num{80} \si{\micro\meter} est systématiquement inférieure à
\num{1} \si{\centi\meter}.

Si on suppose maintenant que le fluide est turbulent, la viscosité
effective de l'écoulement moyen est supérieure à la viscosité
laminaire du bain. Cette situation correspond aux trois dernières
colonnes à droite du tableau \ref{tab:fall-results}. Dans ce cas, on
constate que la profondeur maximale est systématiquement de l'ordre de
\num{1}\si{\milli\meter}, ou inférieure.

Le bain électrolytique d'une cuve n'est jamais au repos, en
particulier dans les canaux où le fluide est agité par les bulles de
gaz qui résultent de l'électrolyse et qui remontent à la surface du
bain. Dans la suite de ce travail, nous ferons l'hypothèse que
l'injection des doses de particules d'alumine ne perturbe pas
l'écoulement du bain. Dans ce cadre, nous négligerons l'effet de la
force de gravité sur la trajectoire des particules.

Dans cette section on a donné une expression exacte pour la position
de la particule lorsque celle-ci est complètement dissoute,
c'est-à-dire lorsque $t = T$. La position de la particule
$x(t)$ pour tout $t \in(0, T)$ nécessite l'intégration de l'équation
différentielle linéaire non-homogène et à coefficients variables
(\ref{eq:simple-mvt}). Nous ne chercherons pas ici à expliciter la
solution exacte pour $x(t)$. Cependant, il est facile d'obtenir une
approximation numérique de $x$ à l'aide des intégrateur offerts par
MatLAB\textregistered, \texttt{ode45} par exemple. La figure \ref{fig:}

Nous concluons cette section par trois remarques qui traitent
premièrement de la validité de la loi de Stokes pour la force de traînée,
deuxièmement de l'importance de la vitesse initiale de la particule
sur la profondeur terminale de la particule, et troisièmement du temps
caractéristique pour qu'une particule soit emportée par un fluide en
mouvement laminaire.

%%\begin{remarque}
%%  (Sur le fait qu'on peut negliger le terme de reaction dans
%%  l'equation du mouvement)
%%\end{remarque}

%%\begin{remarque}
%%  Le temps caracteristique du transitoire pour atteindre la vitesse
%%  du fluide doit etre faible devant les autres temps
%%  caracteristiques pour que l'approximation de dire que les
%%  particules sont transportee par les lignes de courant du fluide
%%  est une bonne approximation.
%%\end{remarque}

\begin{remarque}
  En considérant la relation (\ref{eq:particle-terminal-position}), on
  peut évaluer l'effet de la vitesse initiale de la particule sur sa
  position terminale. Les contributions des deux termes à droite de
  l'égalité (\ref{eq:particle-terminal-position}) sont identiques
  lorsque la vitesse initiale est telle que
  \begin{equation*}
    v_0 = \frac{g}{4\dissolutionrate}\frac{\aluminadensity -
      \electrolytedensity}{\aluminadensity} r_0^2.
  \end{equation*}
  Pour une particule de rayon initial $r_0 = \num{80}$
  \si{\micro\meter}, cette vitesse initiales est de $v_0 = \num{14.5}$
  \si{\meter\per\second}, ce qui correspond à une hauteur de chute
  libre d'environ \num{10} \si{\meter}. On conclut que, dans une cuve
  d'électrolyse industrielle typique, la vitesse initiales des
  particules suffisamment dispersées qui pénètrent dans le bain est
  négligeable sur leur profondeur de pénétration.
\end{remarque}

\begin{remarque}
  Le nombre de Reynolds d'une sphère de rayon $r$ en mouvement
  rectiligne uniforme dans un fluide au repos est donné par
  \begin{equation*}
    \reynolds = \frac{2 \electrolytedensity v r}{\electrolyteviscosity}
  \end{equation*}
  où $v$ est sa vitesse relative au fluide.

  En supposant $r$ fixé, l'équation du mouvement d'une particule
  dans le fluide s'écrit
  \begin{equation}\label{eq:mvt-fixed-r}
    \aluminadensity \frac{4}{3}\pi r^3 \ddot x(t) = \frac{4}{3}\pi
    r^3\parent{\aluminadensity - \electrolytedensity}g -
  6\pi\electrolyteviscosity r\dot x(t).
  \end{equation}
  On obtient la vitesse limite de chute de la particule $v_L$ en
  remplaçant dans (\ref{eq:mvt-fixed-r}) $\dot x$ par $v_L$ et en posant $\ddot x
  = 0$ (accélération nulle). En résolvant pour $v_L$ on obtient
  \begin{equation}\label{eq:terminal-velocity}
    v_L = \frac{2}{9}\frac{g\parent{\aluminadensity - \electrolytedensity}}{\electrolyteviscosity}r^2.
  \end{equation}
  Le nombre de Reynolds associé à cet écoulement s'écrit comme
  \begin{equation*}
    \reynolds =
    \frac{4}{9}\frac{g\electrolytedensity\parent{\aluminadensity - \electrolytedensity}}{\electrolyteviscosity^2}r^3.
  \end{equation*}
  En reprenant les conditions considérées dans le tableau
  \ref{tab:fall-results} et les paramètres du tableau
  \ref{tab:fall-physical-parameters}, on obtient un nombre de Reynolds
  maximal pour $\electrolyteviscosity = \num{2e-3}$
  \si{\kilo\gram\per\meter\per\second} et $r_0 = 80$
  \si{\micro\meter}, soit $\reynolds \simeq 2.2$. Ce résultat justifie
  l'utilisation de la loi de Stokes pour modéliser la force de traînée
  de la particule.
\end{remarque}

\begin{remarque}
  En négligeant la force de gravité ($g = 0$), l'équation du mouvement
  d'une particule d'alumine de rayon $r_0$ supposé constant au cours
  du temps s'écrit
  \begin{equation*}
    \aluminadensity \frac{4}{3}\pi r_0^3 \ddot x(t) = -
    6\pi\electrolyteviscosity r_0 \dot x(t),
  \end{equation*}
  et on considère les conditions initiales $x(0) = 0$ et $\dot x(0) =
  v_0$. On peut intégrer cette équation exactement, et on obtient la
  vitesse de la particule au cours du temps
  \begin{equation*}
    \dot x(t) = v_0\exp\parent{-\frac{9\electrolyteviscosity}{2\aluminadensity r_0^2}t}.
  \end{equation*}
  La vitesse approche zéro sur une échelle de temps données par
  $\frac{2\aluminadensity r_0^2}{9\electrolyteviscosity}$,
  c'est-à-dire environ \num{2.8e-3} \si{\second} pour une particule de
  rayon $r_0 = \num{80}$ \si{\micro\meter} et dans un fluide au repos
  de même densité, \ie, avec $\mu = $ \num{2e-3}. Autrement dit, le
  temps nécessaire à une particule pour être entraînée et transportée
  par un fluide à la même vitesse que celui-ci est négligeable devant
  les temps caractéristiques des autres phénomènes qui prennent place
  dans la cuve et que l'on s'intéresse à modéliser. Cette observation
  permet de justifier l'approximation qui sera faite dans le chapitre
  \ref{chap:populations}, où on supposera que les particules suivent
  exactement les lignes de courant de l'écoulement.
\end{remarque}

\begin{figure}
 \begin{center}
    \begin{subfigure}[b]{0.49\textwidth}
      \input{../media/particles/trajectories/particle_fall.tex}
      \caption{Trajectoire des particules dans le fluide.}
      \label{fig:particle-trajectories-a}
    \end{subfigure}
    \begin{subfigure}[b]{0.49\textwidth}
      \input{../media/particles/trajectories/particle_fall_zoom.tex}
      \caption{Zoom sur la phase d'accélération initiale.}
      \label{fig:particle-trajectories-b}
    \end{subfigure}

    \caption{Trajectoires de particules de rayon initial
      $r_0 = $ \numlist{40;60;80} \si{\micro\meter}. Chaque
    particule est placée à l'origine avec une vitesse nulle
    au temps $t = 0$. Les trajectoires sont intégrées jusqu'à ce que
    les particules soient complètement dissoutes, c'est-à dire sur l'intervalle
    de temps $\left[0, \frac{r_0^2}{2\dissolutionrate}\right]$. Notez
    les différentes échelles sur les deux graphiques.}
    \label{fig:particle-trajectories}
 \end{center}
\end{figure}



\chapter{Populations de particules}
\label{chap:populations}
\section{Introduction}
\label{sec:populations-introduction}
Une cuve d'électrolyse d'aluminium industrielle est caractérisée par
le fait que plusieurs phénomènes physiques entrent en jeu et
interagissent sur elle à des échelles similaires. Tout d'abord, le courant
électrique continu qui traverse une cuve d'électrolyse industrielle
moderne est de l'ordre de \num{500} \si{\kilo\ampere} à \num{1}
\si{\mega\ampere}. Une telle intensité de courant génère d'une part un
champs d'induction magnétique qui affecte l'opération de la cuve, mais
également de ses voisines de série. D'autre part, la chute de
potentiel d'environ \num{4} \si{\volt} à travers les électrodes et les
fluides \cite{Haupin1995} provoque la dissipation de grandes quantités
d'énergie sous forme thermique. Cette production de chaleur est
nécessaire car elle permet de maintenir le bain électrolytique sous
forme liquide. Cependant, une surchauffe trop importante est
néfaste pour une cuve, qui risque alors de subir une usure et une
fin de vie prématurée en plus d'occasionner des pertes d'énergie.

Au niveau de la cathode en carbone qui forme le fond de la cuve, la
réaction d'électrolyse
\begin{equation}
\cee{Al^{3+} + 3e^- -> Al}
\end{equation}
produit de l'aluminium sous forme métallique. L'aluminium métallique,
qui est peu soluble dans l'électrolyte, est liquide à la température
d'opération d'une cuve et de densité légèrement supérieure à celle de
l'électrolyte. Il forme une couche de métal en fusion au fond de la
cuve. C'est alors la surface du métal en fusion qui joue véritablement
le rôle de cathode sur laquelle a lieu la réaction d'électrolyse.

Le bain électrolytique et le métal liquide sont traversés par
l'ensemble du courant électrique et, en présence du champs d'induction
magnétique, subissent une force de Lorentz qui les met en
mouvement. L'écoulement dans les fluides est bénéfique. Il permet de
transporter la chaleur à l'extérieur du système plus rapidement que
par conduction pure, homogénéise la distribution de la température et
la composition chimique dans l'ensemble du bain et facilite le
transport et la dispersion les particules d'alumine injectées à
intervalles réguliers dans le bain.

Cependant, il est crucial que l'interface bain-métal soit aussi stable
que possible. Pour des raisons d'économie d'énergie, la distance
moyenne entre le fond des anodes\footnote{Le fond des anodes est
  défini comme la surface des anodes orientée vers
  le bas, face à la cathode.} et l'interface est réduite
autant que possible. Dans une cuve moderne, cette distance est de
l'ordre de \num{2} à \num{4} \si{\centi\meter}. Si les écoulements
dans les fluides deviennent trop rapides, des turbulences se forment
et viennent perturber la forme de l'interface. De plus, le système
magnétohydrodynamique formé par le circuit électrique, le champ
d'induction magnétique et les deux fluides peut devenir, sous
certaines conditions, physiquement instable, c'est-à-dire que de
petites perturbations du système sont amplifiées par elles-mêmes. De
telles instabilités affectent la forme de l'interface bain-métal, et
leur occurrence présente un défi qui a déjà fait l'objet de nombreux
travaux de recherche. Le lecteur intéressé se référera par exemple à
\cite{Descloux1998}, \cite{Sneyd1985} ou encore \cite{Maillard1996}.

Le métal liquide ayant une conductivité électrique bien supérieure à
celle du bain électrolytique (\cite{Wang1992}, \cite{Apfelbaum2003}),
le contact entre les anodes et la nappe de métal pourrait créer un
court-circuit. Les court-circuits diminuent le rendement du procédé et
peuvent créer des défauts à la surface des anodes qui affectent
l'opération de la cuve à long terme.

De l'alumine doit être injectée régulièrement dans le bain
électrolytique, sous forme de poudre, pour compléter l'alumine
dissoute consommée par la réaction d'électrolyse. Puisque la majeur
partie du bain est recouverte par les blocs anodiques et en raison de
la présence de la croûte qui protège sa surface libre, l'injection de
poudre d'alumine ne peut avoir lieu que en quelques points d'injection
distribués le long du canal central. Dans une cuve industrielle, le
nombre de points d'injection est typiquement compris entre 4 et 6. La
dispersion et le transport des particules d'alumine et de la
concentration d'alumine qui résulte de leur dissolution dépend
crucialement de la présence et de la force de l'écoulement dans le
bain électrolytique. De nombreux travaux de recherche ont pour
objectif la modélisation et l'approximation numérique des écoulement
dans le bain et le métal d'une cuve d'électrolyse industrielle. Dans
ce travail, nous utiliserons les résultats de S. Pain \cite{Pain2006},
G. Steiner \cite{Steiner2009} et J. Rochat \cite{Rochat2016} à cette
fin. Le modèle d'écoulement proposé dans \cite{Steiner2009} consiste à
considérer un système couplé formé par le problème du potentiel
électrique dans les conducteurs, par les équations de Maxwell dans le
vide, dans le caisson ferromagnétique, dans le bain électrolytique et
dans le métal, et finalement par des équations de Navier-Stokes dans
chaque fluide. Dans ce problème, l'interface bain-métal est une
inconnue. Les écoulements dans les fluides sont turbulents. Les
structures des écoulements dont la taille est inférieure à la
résolution de la grille de discrétisation sont modélisés par un modèle
de longueur de mélange de Smagorinski \cite{Rochat2016}. Dans ce
chapitre nous utiliserons ce modèle, implémenté dans le logiciel
Alucell, pour obtenir le domaine $\Omega$ occupé par le bain d'une
cuve d'électrolyse industrielle, une approximation de la vitesse
d'écoulement stationnaire $u_h$ et une approximation de la densité de
courant électrique stationnaire $j_h$ dans $\Omega$.

Ce travail porte sur la modélisation des particules d'alumine
injectées dans le bain et leur dissolution. On suivra l'approche
adoptée par T. Hofer \cite{Hofer2011} et on représentera l'évolution
d'une famille de les particules sous la forme d'une distribution en
taille et en espace $\population$. Cette population évolue d'une part
dans le bain par le biais de la vitesse d'écoulement $u$ de celui-ci,
et d'autre part par la dissolution des particules au cours du temps,
en fonction de la concentration d'alumine locale dans le bain et de sa
température telle que décrite par la vitesse de dissolution $f$
introduite dans la section \ref{sec:particle-dissolution}. En plus de
la densité de particule $\population$, nous modéliserons l'évolution
de la concentration d'alumine dissoute $c$ et la température
$\temperature$ dans le bain, puisque la dissolution des particules en
dépend.

Dans la section \ref{sec:populations-model}, nous introduisons les
équations qui décrivent l'évolution de la densité de particules $n_p$,
la concentration d'alumine dissoute $c$ et la température du bain
$\temperature$. Dans la section \ref{sec:populations-discretisation}
nous traitons la discrétisation en temps du système formé par les
équations pour $\population$, $\concentration$ et $\temperature$
introduites dans \ref{sec:populations-model}. Finalement, dans la
section \ref{sec:populations-industriel} nous considérons et discutons
l'application du modèle de transport et dissolution de particules
d'alumine au cas d'une cuve d'électrolyse industrielle AP32.


\section{Modèle de transport et dissolution d'alumine}
\label{sec:populations-model}
Soit $\Omega\subset \mathbb R^3$ le domaine ouvert donné, occupé par
le bain électrolytique. Ce bain est animé par une vitesse d'écoulement
$u:\Omega\to\mathbb R^3$ stationnaire donnée et telle que $\div u = 0$
dans $\Omega$ et $u\cdot \nu = 0$ sur le bord $\partial \Omega$ de
$\Omega$, où ici $\nu$ est la normale unité. On note $\tend$ le temps
auquel nous souhaitons connaître la solution. Dans ce chapitre on note
$\concentration(t, x)$ la concentration d'alumine dissoute en
\si{\mol\per\cubic\meter} et $\temperature(t, x)$ la température dans
le bain en Kelvin à l'instant $t \in [0, T]$ et à l'endroit $x \in
\Omega$.

\paragraph{Vitesse de transport}
La force de gravité a pour effet d'entraîner les particules d'alumine
vers le fond de la cuve. Nous avons vu dans la section
\ref{sec:particle-fall} que le temps caractéristique nécessaire pour
qu'une particule typique atteigne d'une part sa vitesse stationnaire de
chute et d'autre part la vitesse de l'écoulement du bain est de
l'ordre de quelques millisecondes. La vitesse maximale de l'écoulement
stationnaire dans une cuve d'électrolyse d'aluminium étant de l'ordre
de \num{0.1} \si{\meter\per\second}, cette période transitoire s'étend
sur des distances d'environ \num{1e-4} \si{\meter}, ce qui est
largement inférieure à la résolution spatiale des grilles que l'on
utilise dans des calculs industriels. Par conséquent, nous ferons deux
hypothèses simplificatrices dans le cadre du modèle de transport et
dissolution de particules d'alumine. Premièrement, nous supposerons
que le champ de gravité a pour effet de transporter les particules
vers le fond de la cuve. On prendra comme vitesse de transport la
vitesse stationnaire de chute (\ref{eq:terminal-velocity}) d'une
particule soumise à la force de traînée de Stokes. Cette force dépend
du rayon des particules. Ce transport gravitationnel des particules, qui
a lieu dans l'ensemble du bain, est modélisé par un champ $w(r):
\Omega\times[0, \infty)\to\mathbb R^3$ défini par
\begin{equation}\label{eq:fall-velocity}
  w(r) =
  -\frac{2g\parent{\aluminadensity -
      \electrolytedensity}}{9\electrolyteviscosity} r^2 \hat e_3
\end{equation}
en vertu de (\ref{eq:terminal-velocity}). On a noté $\hat e_3$ le
vecteur unitaire vertical dirigé vers le haut. Clairement, le champ
$w$ est tel que $\div w = 0$ dans $\Omega$. Deuxièmement, nous
supposerons que les particules suivent exactement les lignes de
courant du champ de transport $u + w$ dans le domaine $\Omega$. Par
ailleurs, la concentration d'alumine dissoute $\concentration$ et la
température du bain électrolytique $\temperature$ seront transportés
par la vitesse d'écoulement $u$, et diffusés.

\paragraph{Populations de particules}
Dans une cuve industrielle, de l'alumine en poudre doit être injectée
à intervalles réguliers dans le bain. Nous modélisons ces injections
sous la forme d'une série d'évènements instantanés, successifs dans le
temps. Soit $K$ le nombre total d'injections, et soit un ensemble de
nombres réels $\tau^k$, $k = 1, 2, \dots, K$ tels que $0\leq \tau^k <
T$ $\forall k$, où les $\tau^k$ représentent les instants auxquels
surviennent chaque injection. Pour toute injection $k = 1,2, \dots,
K$, la distribution initiale de particule est notée $S^k:(x,
r)\in\Omega\times\rplus\to S^k(x, r)\in\rplus$ et est supposée
donnée. Les distributions initiales de particules $S^k$ s'expriment en
\si{\kilo\gram\per\cubic\meter\per\meter}.

On note $n_p^k(t, x, r)$, $k = 1, 2, \dots, K$ la densité en espace et
en taille de particules issues de l'injection $k$ dans le bain
électrolytique à l'instant $t \in [0, T]$. Autrement dit, la quantité
$n_p^k(t, x, r)\intd{x}\intd{r}$ représente le nombre de particules, à
l'instant $t$, dans le volume infinitésimal $\intd{x}$ autour du point
$x\in\Omega$ (entre $x$ et $x + \mathrm dx$) et dont la taille est comprise dans l'intervalle $[r,
  r+\intd{r}]$. Soit $\tlat \geq 0$ le temps de latence avant que les
particules ne commencent à se dissoudre suite à leur injection dans le
bain. Ce temps de latence a fait l'objet d'une discussion dans la
section \ref{sec:particle-freeze}. Chaque population de particules
$n_p^k$, $k = 1, 2,\dots, K$ satisfait les
équations lorsque $x\in\Omega,\ r\in\rplus$:
\begin{align}
  &n_p^k(t, x, r) = 0, & 0\leq t < \tau^k,\label{eq:population-pre-injection}\\
  &n_p^k(\tau^k, x, r) = S^k(x, r),\label{eq:population-injection}\\
  &\frac{\partial n_p^k}{\partial t} + (u(x) + w(r))\cdot \nabla n_p^k
  = 0, &\tau^k < t \leq \tau^k +
  \tlat,\label{eq:population-transport}\\
  &\frac{\partial n_p^k}{\partial t} + (u(x) + w(r))\cdot \nabla n_p^k
  + \frac{\partial}{\partial r} \parent{f(r, \concentration,
    \temperature) n_p^k} = 0, &\tau^k + \tlat < t \leq \tend,\label{eq:population-dissolution}
\end{align}
où l'opérateur $\nabla = (\partial_{x_1}, \partial_{x_2},
\partial_{x_3})^t$. La vitesse de dissolution $f$ qui intervient dans
l'équation (\ref{eq:population-dissolution}) a déjà été définie dans
la section \ref{sec:particle-dissolution} par l'expression
(\ref{eq:dissolution-velocity}). Les équations
(\ref{eq:population-transport}) et (\ref{eq:population-dissolution})
sont complétées par une condition aux limites sur le bord entrant. Plus
précisément, soit $\Gamma^-$ la partie du bord $\partial \Omega$ de $\Omega$
définie par
\begin{equation}\label{eq:np-bc}
  \Gamma^- = \cparent{x\in\partial\Omega \mid \nu \cdot  (w + u)
  < 0}.
\end{equation}
Ici, $\nu$ est le vecteur normal unitaire extérieur à la surface
$\partial \Omega$. La condition aux limites sur le bord entrant s'écrit
alors pour tout $k = 1, 2, \dots, K$ et $t > \tau^k$
\begin{equation}
n_p^k(t, x) = 0, \quad x\in\Gamma^-.
\end{equation}

La densité totale de particules au temps $t \in [0, T]$, à l'endroit
$x\in\Omega$ et fonction de $r$ est donnée par
\begin{equation}\label{eq:populations-sum}
  n_p(t, x, r) = \sum_{k=1}^K n_p^k(t, x, r).
\end{equation}

\begin{remarque}
Les quantités $n_p^k$ introduites ici sont similaires au champ $n_p$
décrit dans la section \ref{sec:particle-population-dissolution}, à
ceci près que les champs $n_p^k$ sont des fonctions du temps $t$ et du
rayon $r$ des particules, mais aussi du lieu $x$ dans le domaine
$\Omega$ occupé par le bain électrolytique.
\end{remarque}

\paragraph{Consommation de l'alumine dissoute}
La concentration d'alumine dissoute est consommée\footnote{On parle de
consommation de l'alumine dissoute lorsqu'une paire de ions \ce{Al^{3+}} issus de
la dissociation d'une molécule d'\ce{Al2O3} est réduite au niveau de
la cathode.} par la réaction
d'électrolyse. Si $I$ est le courant électrique total imposé
traversant la cuve, le débit total d'alumine dissoute consommée en
\si{\mol\per\second} est proportionnel à $I$ et s'écrit
\begin{equation}\label{eq:mass-consumption}
  \frac{I}{6\faraday}
\end{equation}
où $F = \num{96485.33}$ \si{\coulomb\per\mol} est la constante de
Faraday. Le facteur \num{6} provient du fait qu'il
faut \num{6} électrons pour réduire une molécule d'\ce{Al2O3}, et
produire deux molécules d'aluminium métallique \ce{Al}.

On suppose que la consommation d'alumine dissoute qui a lieu dans le
bain électrolytique est proportionnelle à la densité de courant
électrique locale $j:\Omega\to\mathbb R^3$. On définit alors le terme
de disparition de la concentration d'alumine $q_1$ associé à la
consommation par la réaction d'électrolyse
\begin{equation}
  q_1(x) = -\frac{I}{6\faraday} %
  \frac{\abs{j(x)}}{\displaystyle\int_\Omega\abs{j(x)}\intd{x}}
\end{equation}
de sorte à avoir la consommation totale sur $\Omega$
\begin{equation}
  \int_\Omega q_1(x)\,\intd{x} = -\frac{I}{6\faraday}.
\end{equation}

\paragraph{Dissolution des particules d'alumine}
La masse perdue par la population de particules d'alumine
$\population$ vient contribuer à la concentration d'alumine
dissoute. On définit $q_2$ le terme source de la concentration
qui représente l'apport dû à la dissolution des particules de la
manière suivante. Si $t \in [0, T]$, on note $\bar k(t)$ l'entier tel
que $\bar k(t) = \max_{1\leq k\leq K}\cparent{k \mid \tau^k + \tlat < t}$ . Alors on définit
\begin{equation}
  q_2(t, x) = -\sum_{k = 1}^{\bar
    k(t)}\frac{4\pi\aluminadensity}{[\ce{Al2O3}]}\int_{\rplus}n_p^k(t,
  x, r)f(r, \concentration(t, x), \temperature(t, x))r^2\,\intd{r}.
\end{equation}
où [\ce{Al2O3}] $ = \num{0.102}$ \si{\kilo\gram\per\mol} est la masse
molaire de l'alumine. L'ensemble des indices $k$ tels que $1\leq k
\leq \bar k(t)$ représente l'ensemble des populations de particules
$n_p^k$ qui se dissolvent dans le bain à l'instant $t\in[0, T]$.

\paragraph{Concentration d'alumine dissoute}
La concentration d'alumine dissoute $\concentration$ est transportée
dans le bain par la vitesse d'écoulement $u$, mais est de plus sujette
à une diffusion liée d'une part à l'agitation moléculaire, et d'autre
part aux turbulences de l'écoulement. Soit
$\cdiffusivity:\Omega\to\mathbb R^*_+$ la diffusivité de la
concentration d'alumine dissoute dans le bain, supposée connue. Alors
la concentration $\concentration$ doit satisfaire l'équation
\begin{equation}\label{eq:concentration}
  \frac{\partial \concentration}{\partial t}(t, x) + u(x)\cdot\nabla \concentration(t, x) - \div\parent{\cdiffusivity(x)
  \nabla \concentration(t, x)} = q_1(x) + q_2(t, x),\quad \forall
  x\in\Omega,\ t\in (0, T).
\end{equation}
Puisque qu'il ne peut y avoir de flux de masse d'alumine à travers le
bord du domaine $\Omega$, la concentration doit satisfaire la
condition aux limites de Neumann homogène
\begin{equation}\label{eq:c-bc}
  \electrolytecdiff(x)\frac{\partial \concentration}{\partial \nu}(t, x) = 0 \quad  \quad
  \forall x\in\partial \Omega,\ t\in[0, T].
\end{equation}

\paragraph{Termes sources de la température du bain}
Pour simplifier le modèle, on fait l'hypothèse que le bain
électrolytique est isolé thermiquement, c'est-à-dire qu'il n'y a pas
de flux d'énergie thermique à travers le bord du domaine occupé par le
bain.

L'énergie thermique du bain provient de trois sources distinctes.

Premièrement, les particules d'alumine sont injectées avec
une température $\tinj < \temperature$, la température locale du
bain. De l'énergie thermique est prélevée dans le bain pour rétablir
l'équilibre thermique entre celui-ci et les particules. On note $p_1(t,
x)$ la densité de puissance thermique extraite du bain pour réchauffer
les particules, que l'on définit par
\begin{equation}\label{eq:power-heating}
  p_1(t, x) = -\parent{\tinit-\tinj}\sum_{k = 1}^{K}\dirac(t -
  \tau^k)\int_{\rplus}\aluminadensity\aluminahc\frac{4}{3}\pi r^3 S^k(x, r)\,\intd{r}
\end{equation}
où $\dirac$ est une masse de Dirac et $\tinit$ est la température
initiale du bain. Dans (\ref{eq:power-heating}) et pour rappel, $\tinit$ est la
température initiale du bain électrolytique, $\aluminahc$ et
$\aluminadensity$ sont la chaleur spécifique et la densité de
l'alumine, tandis que les $S^k$ sont les distributions en espace et en
rayon des doses de poudre d'alumine qui interviennent dans le membre
de droite de (\ref{eq:population-injection}).

\begin{remarque}
  En réalité, la puissance nécessaire à réchauffer les particules
  $p_1(t, x)$ est proportionnelle à ${\temperature(t, x) -
    \tinj}$. Cependant, dans (\ref{eq:power-heating}) on utilise la
  température initiale du bain $\tinit$ à la place de
  $\temperature(t, x)$, ce qui revient à négliger l'écart entre
  $\temperature(t, x)$ et $\tinit$, qui est faible devant ${\tinit -
  \tinj}$.
\end{remarque}

Deuxièmement, la réaction de dissolution de la poudre d'alumine est
endothermique. On note $p_2(t, x)$ la densité de puissance thermique
utilisée par la réaction, que l'on définit pour tout $x\in\Omega$ et
$t\in[0, T]$ par
\begin{equation}\label{eq:p2}
p_2(t, x) = - \aluminadissolutionenthalpy q_2(t, x),
\end{equation}
où $\aluminadissolutionenthalpy$ est l'enthalpie molaire nécessaire à
la dissolution de l'alumine. On rappelle que $q_2(t, .)$ est le
débit molaire par unité de volume d'alumine dissoute à l'instant $t$.

Et troisièmement, la résistivité électrique de l'électrolyte engendre
la conversion d'une partie de l'énergie électrique en énergie
thermique par effet Joule. Cette source d'énergie thermique $p_3$ par
effet Joule dépend de la densité de courant stationnaire $j$ et
s'écrit
\begin{equation}\label{eq:p3}
p_3(x) = \frac{j\cdot j}{\conductivity}.
\end{equation}
où le nombre réel $\sigma > 0$ est la conductivité électrique du bain que l'on
suppose constante dans tout $\Omega$.

\paragraph{Température du bain}
Tout comme la concentration d'alumine, la température du bain
électrolytique $\temperature$ est transportée par la vitesse
d'écoulement $u$, et diffusée dans le bain. Cette diffusion est due à
la l'agitation moléculaire d'une part, et à aux turbulences de
l'écoulement d'autre part. Soit $\electrolytetdiff:\Omega\to\mathbb
R^*_+$ la diffusivité de la température dans le bain, supposée
donnée. Comme précédemment on a noté $\electrolytedensity$,
$\electrolytehc$ la densité et la chaleur spécifique du bain
électrolytique. Alors la température $\temperature$ doit satisfaire
l'équation
\begin{equation}\label{eq:temperature}
\frac{\partial \temperature}{\partial t}(t, x) +
u(x)\cdot\nabla\temperature - \div\parent{\electrolytetdiff(x)\nabla
\temperature(t, x)} =
\frac{1}{\electrolytedensity\electrolytehc}\sum_{i = 1}^{3}p_i(t, x),
\quad \forall x\in\Omega,\ t\in(0, T).
\end{equation}
De plus, et conformément à l'hypothèse d'isolation thermique, la température
$\temperature$ doit satisfaire une condition de Neumann homogène
\begin{equation}\label{eq:t-bc}
  \electrolytehc\frac{\partial \temperature}{\partial \nu}(t, x) = 0, \quad
  \forall\ x\in\partial \Omega,\ t\in[0, T].
\end{equation}

\paragraph{Formulation du problème}
Le problème de transport et dissolution d'alumine en fonction de la
température consiste à chercher des fonctions $n_p^k:(0,
T)\times\Omega\times\rplus\to \mathbb R$, $c:(0, T)\times\Omega\to
\mathbb R$ et $\temperature:(0, T)\times\Omega\to\mathbb R$ qui
satisfont les équations couplées (\ref{eq:population-pre-injection}) à
(\ref{eq:population-dissolution}), (\ref{eq:concentration}) et
(\ref{eq:temperature}) ainsi que les conditions aux limites
(\ref{eq:np-bc}), (\ref{eq:c-bc}) et (\ref{eq:t-bc}), auxquelles on
ajoutera des conditions initiales appropriées.

La section suivante traite de la discrétisation en temps de ce
système d'équations.


\section{Discrétisation en temps}
\label{sec:populations-discretisation}
On répète ici, par soucis de clarté, le système d'équations aux
dérivées partielles qui correspondent au modèle de transport et
dissolution d'alumine en fonction de la température. On a $\forall k =
1, 2, \dots, K$
\begin{align}
  & n_p^k(t, x, r) = 0,
  & 0\leq t < \tau^k,\label{eq:mdl-eq-1}\\
%
  & \frac{\partial n_p^k}{\partial t} + \parent{u(x) + w(r)}\cdot \nabla  n_p^k = 0,
  & \tau^k < t \leq \tau^k + \tlat,\label{eq:mdl-eq-2}\\
%
  & \frac{\partial n_p^k}{\partial t} + \parent{u(x) + w(r)}\cdot \nabla  n_p^k + \frac{\partial }{\partial r}\parent{f(r, \concentration, \temperature)n_p^k} = 0,
  & \tau^k + \tliq < t \leq \tend,\label{eq:mdl-eq-3}
\end{align}
et
\begin{align}
  & \frac{\partial \concentration}{\partial t} + u(x)\cdot\nabla \concentration - \cdiffusivity(x) \Delta \concentration = q_1 + q_2,
  & \forall t\in (0, T),\label{eq:mdl-eq-4}\\
%
  & \frac{\partial \temperature}{\partial t} + u(x)\cdot\nabla\temperature - \electrolytetdiff(x)\Delta \temperature = \frac{1}{\electrolytedensity\electrolytehc}\sum_{i = 1}^{3}p_i,
  & \forall t\in(0, T)\label{eq:mdl-eq-5}
\end{align}
dans $\Omega$.

Ces équations forment un système couplé pour les inconnues $n_p^k$,
$\concentration$ et $\temperature$. En effet, la densité de
particules $n_p$ dépend de la concentration $\concentration$ et de
la température $\temperature$ par l'intermédiaire de la vitesse de
dissolution $f$, tandis que la concentration et la température
dépendent de la densité de particule $n_p$ à travers leurs termes
sources respectifs $q_2$, $p_1$ et $p_2$.

En suivant l'approche adoptée dans \cite{Hofer2011}, on propose de
discrétiser les équations (\ref{eq:mdl-eq-1}) à (\ref{eq:mdl-eq-5})
par une méthode de splitting en temps de la façon suivante. Soit
l'entier $N$, le nombre de pas de temps et $\dt = T/N$ un pas de temps
uniforme. Soient $t_n = n\dt$, $n = 0, 1, \dots, N$, une subdivision de
l'intervalle de temps $[0, T]$. On note $n_{p,n}^k$ une approximation
de $n_p^k(t_n, ., .)$, $n_{p,n}$ de $n_p(t_n, ., .)$, $\concentration_n$ une approximation de
$\concentration(t_n, .)$ et $\temperature_n$ une approximation de
$\temperature(t_n, .)$. Si $k$ et l'indice de l'injection d'une
population de particules, on définit $p^k$ et $q^k$ les plus grands
entiers tels que
\begin{equation}
  t_{p^k} < \tau^k \quad\text{et}\quad t_{q^k} < \tau^k + \tlat.
\end{equation}
En d'autres termes, $p_k$ est le dernier pas de temps qui précède
l'injection de la population $k$, et $q_k$ est le dernier pas de temps
qui précède le début de la dissolution de la population $k$. Bien
entendu, si $\tlat = 0$ alors $p_k = q_k$.

Etant donnés $n_{p,n}^k$ $\forall k=1, 2, \dots, K$,
$\concentration_n$ et $\temperature_n$, on pose
\begin{align}
  & n_{p,n+1}^k = 0, &&\text{si } n \leq p^k,\\
  & n_{p,n+1}^k = S^k, &&\text{si } n = p^k + 1,\\
  & \frac{n_{p,n+1}^k - n_{p,n}^k}{\dt} + u\cdot\nabla n_{p,n+1}^k = 0, && \text{si } p^k + 1 < n \leq q^k.\label{eq:splitting-np1-u}
\end{align}
dans $\Omega$ et pour tout $r > 0$. Puis, si $n$ est tel que $q^k + 1 \leq n < N$, on pose
\begin{align}
  &\displaystyle\frac{\bar n_{p,n+1}^k - n_{p,n}^k}{\dt} +
  u\cdot\nabla \bar n_{p,n+1}^k = 0,\label{eq:splitting-np2-u}\\
  &\displaystyle\frac{\bar{\bar n}_{p,n+1}^k - \bar n_{p,n}^k}{\dt} +
  w\cdot\nabla \bar{\bar n}_{p,n+1}^k = 0,\label{eq:splitting-np2-w}\\
    &\displaystyle\frac{n_{p,n+1}^k - \bar{\bar n}_{p,n+1}^k}{\dt} +
    \displaystyle\frac{\partial}{\partial r}\parent{f(r,
      \concentration_n, \temperature_n)n_{p,n+1}^k} =
    0\label{eq:splitting-dissolution}
\end{align}
dans $\Omega$ et pour tout $r > 0$. Et finalement
\begin{align}
&\frac{\concentration_{n+1} - \concentration_{n}}{\dt} + u\cdot\nabla
\concentration_n - \cdiffusivity \Delta \concentration_n = q_1
+ q_{2,n},\label{eq:splitting-c} \\
&\frac{\temperature_{n+1} - \temperature_{n}}{\dt} + u\cdot\nabla
\temperature_n - \electrolytetdiff \Delta \temperature_n =
\frac{1}{\electrolytedensity \electrolytehc}\parent{\sum_{i =
    1}^2p_{i,n} + p_3}\label{eq:splitting-t}
\end{align}
dans $\Omega$. On précise maintenant la forme des termes sources
discrétisés $q_{2,n}$, $p_{i,n}$, $i = 1,2$.

\paragraph{Discrétisation de la source d'alumine $q_{2}$}
Le terme source $q_{2}$ qui apparaît dans l'équation
(\ref{eq:mdl-eq-4}) correspond à la masse d'alumine qui est transférée
entre les particules qui se dissolvent et l'alumine dissoute par unité
de temps. Afin de permettre une conservation de la masse d'alumine
exacte par le schéma numérique entre les champs $n_p$ et $c$, nous
tirons parti du splitting en temps des équations (\ref{eq:mdl-eq-3})
et (\ref{eq:mdl-eq-4}) \cite{Hofer2011}. Plus précisement, grâce au
splitting en temps, les quantités $n_{p,n+1}^k$, $k = 1, 2, \dots, K$
sont indépendentes de $c_{n+1}$. On pose alors
\begin{equation}
  q_{2,n}(x) = -\frac{1}{\dt} \sum%_\substack{1\leq k\leq K\\ q^k < n}
  \int_{0}^\infty
  \frac{\aluminadensity}{[\ce{Al2O3}]} \frac{4}{3}\pi r^3
  \parent{n_{p,n+1}^k(x) - \bar n_{p,n+1}^k(x)} \intd{r}, \quad x\in\Omega.
\end{equation}
Ici, la somme porte sur toutes les populations de particules qui se
dissolvent à l'instant $t_n$, c'est-à-dire les populations $k$
telles que $t_{q^k} < t_n$.


\paragraph{Discrétisation des source de puissance thermique $p_1$ et
$p_2$} Le terme source de puissance thermique $p_1$ est discrétisé
en régularisant la masse de Dirac sur les intervalles $[t_{p^k},
  t_{p^k} + \dt]$, $k = 1, 2, \dots, K$. On pose
\begin{equation}
p_{1,n}(x) = -\parent{\tinit - \tinj}\sum_{k = 1}^K
\frac{1}{\dt}\kronecker_{n,p^k}\int_{\rplus} \aluminadensity\aluminahc
\frac{4}{3}\pi r^3S^k(x, r)\,\intd{r}
\end{equation}
où $\kronecker_{ij}$ est le symbol de Kronecker définit par
\begin{equation}
  \kronecker_{ij} = \left\{
  \begin{array}{ll}
    1&\quad\text{si } i = j,\\
    0&\quad\text{si } i\neq j.
  \end{array}\right.
\end{equation}
Le terme source $p_2$ qui correspond à la puissance thermique
nécessaire à la dissolution des particules est discrétisé en
utilisant $q_{2,n}$:
\begin{equation}
p_{2,n} = -\aluminadissolutionenthalpy q_{2,n}.
\end{equation}

L'équation (\ref{eq:splitting-dissolution}) est discrétisé selon $r$
à l'aide du schéma de caractéristiques présenté dans la section
\ref{sec:particle-population-dissolution}. Les équations
(\ref{eq:splitting-np1-u}), (\ref{eq:splitting-np2-u}),
(\ref{eq:splitting-np2-w}), (\ref{eq:splitting-c}) et
(\ref{eq:splitting-t}) sont des équations d'advection ou
d'advection-diffusion et sont discrétisées en espace selon la
méthode adoptée dans \cite{Hofer2011}. La discrétisation est basée
sur des éléments finis stabilisée par la méthode SUPG
\cite{Quarteroni2008}. La section suivant traite de l'application de
ce modèle numérique à une cuve d'électrolyse d'aluminium industrielle.


\section{Dissolution de poudre d'alumine dans une cuve industrielle}
\label{sec:populations-industriel}
Dans cette section nous appliquons le modèle de transport et
dissolution d'alumine en fonction de la température proposé dans la
section \ref{sec:populations-model} dans le cadre d'une cuve
d'électrolyse industrielle pour déterminer la répartition de l'alumine
dissoute dans le bain de celle-ci. Nous utilisons la cuve AP32, qui
exploite la technologie de cuve d'électrolyse AP
Technology\texttrademark\ développée par RioTinto. Les premières cuves
basées sur la technologie AP ont été mises en production au début des
années 1990, et plus de \num{4000} d'entre elles fonctionnent encore
actuellement dans les halles de productions à travers le monde
\cite{RiotintoAP30}.

Nous commençons par présenter le design et le mode d'opération de la
cuve AP32. Nous détaillerons ensuite le choix des différentes données
qui interviennent dans modèle numérique proposé dans la section
\ref{sec:populations-discretisation} et dans le cadre de la cuve
AP32. Finalement, nous présenterons une sélection de résultats
numériques obtenus.

\paragraph{Géométrie de la cuve AP32} La structure de la cuve AP32
occupe au sol une longueur d'environ \num{17} \si{\meter} et une
largeur d'environ \num{7} \si{\meter}. L'ensemble de la structure
s'élève sur environ \num{5} \si{\meter}. La figure
\ref{fig:ap32-geometry} montre la disposition des différents éléments
à l'intérieur de la cuve. Les fluides s'étendent horizontalement sur
environ \num{14} \si{\meter} par \num{3.5} \si{\meter}. L'épaisseur de
la couche d'aluminium liquide (en jaune sur la figure
\ref{fig:ap32-geometry-elements}) en contact avec la cathode est
d'environ \num{17} \si{\centi\meter}, tandis que l'épaisseur maximale
du bain électrolytique, au niveau des canaux entre les blocs
anodiques, est d'environ \num{20} \si{\centi\meter}. La figure
\ref{fig:ap32-geometry-electrolyte} illustre le volume occupé par le
bain dans lequel nous nous intéressons à déterminer la concentration
d'alumine, en orange. Les indentation rectangulaire à la surface de
celui-ci correspondent au volume occupé par les anodes partiellement
immergées. L'ACD est typiquement de l'ordre de \num{3}
\si{\centi\meter}. Ces différentes épaisseurs des fluides varient d'un
point à l'autre de la cuve à cause des écoulements dans les fluides,
de la déformation de l'interface bain-métal et des irrégularités à la
surface des anodes. De plus, le volume de métal liquide varie
constamment, d'une part à cause du produit de la réaction
d'électrolyse, et d'autre part à cause des opérations de siphonnage du
métal, qui interviennent environ une fois par jour.

\begin{figure}[t]
  \begin{center}
    \begin{subfigure}[b]{0.49\textwidth}
      \includegraphics[width=\textwidth]{../media/populations/ap32-mesh-components/print/metal-bath-anodes-cathode-bus-bars.png}
      \caption{Éléments à proximité des fluides}
      \label{fig:ap32-geometry-elements}
    \end{subfigure}
%
    \begin{subfigure}[b]{0.49\textwidth}
      \includegraphics[width=\textwidth]{../media/populations/ap32-mesh-components/print/bath.png}
      \caption{Bain électrolytique}
      \label{fig:ap32-geometry-electrolyte}
    \end{subfigure}
%
    \caption{Géométrie des éléments importants à proximité
      du bain électrolytique dans une cuve AP32
      (fig. \ref{fig:ap32-geometry-elements}), et détail du volume
      occupé par le bain électrolytique dans cette même cuve
      (fig. \ref{fig:ap32-geometry-electrolyte}). On distingue les
      les anodes en haut et la cathode
      en bas (\textbf{noir}), le bain électrolytique
      (\textbf{orange}), le métal liquide (\textbf{jaune}) et les
      bus bar (\textbf{gris clair}).}
    \label{fig:ap32-geometry}
  \end{center}
\end{figure}

Le plan anodique est composée de deux rangées de 10 anodes chacune,
représentées en noir sur la figure \ref{fig:ap32-geometry-elements}. La
surface du plan anodique est d'environ \num{40.3} \si{\square\meter}
et seulement \num{25}\% de la surface du bain est libre, le reste
étant recouvert par les anodes. La cuve est conçue pour que
l'électrolyte soit traversé par un courant électrique total $I = $
\num{320000} \si{\ampere}, ce qui correspond à une densité de courant
d'environ \num{0.8} \si{\ampere\per\square\centi\meter} à la surface
des anodes. En supposant un rendement de réaction de \num{100}\%, ce
courant électrique permet de réduire par électrolyse \num{29.8}
\si{\gram\per\second} ou \num{2577.1} \si{\kilo\gram} par jour
d'aluminium métallique, \ie, un peu plus qu'\num{1} \si{\cubic\meter}
de métal par jour. Cet accroissement de volume de métal correspond à
une variation de l'épaisseur du métal liquide d'environ \num{2}
\si{\centi\meter}.

Du coté des anodes, la réaction d'électrolyse produit environ
\num{0.8} \si{\mol\per\second} d'oxygène \ce{O2}. Cet oxygène réagit
immédiatement avec le carbone de l'anode pour former du \ce{CO2}. Dans
l'ensemble de la cuve, l'électrolyse produit au total environ \num{80}
\si{\liter} par seconde de gaz, qui remonte vers la surface du bain
par le canal central et les canaux latéraux. La réaction de l'oxygène
avec le carbone des anodes provoque l'érosion de celles-ci à une
vitesse d'environ \num{1100} \si{\kilo\gram} par jour. Étant donné le
nombre total d'anodes et leur taille respectives, chaque anode d'une
cuve AP32 a une durée de vie d'environ 30 jours, après quoi elle doit
être remplacée par une anode neuve.

Pour compenser l'alumine dissoute qui est consommée par la réaction
d'électrolyse, il faut injecter en moyenne au cours du temps $56.3$
\si{\gram\per\second} de poudre d'alumine. Comme déjà mentionné dans
la section \ref{sec:introduction-hall-heroult}, la poudre d'alumine
est déposée à la surface du bain dans le canal central par une série
d'injecteurs dont la position est fixe. Un piqueur vient percer
mécaniquement un trou dans la croûte et créer un accès à la surface
libre du bain avant chaque injection. Ce trou se rebouche rapidement,
et pour cette raison l'injection de la poudre d'alumine ne peut pas
avoir lieu continûment.

Pour maintenir un rendement énergétique maximum, éviter l'émission de
gaz fluorés et éviter l'occurence des effets d'anodes, il est crucial
que la concentration d'oxyde d'aluminium dissout dans le bain soit
maintenue dans un intervalle très précis. Malheureusement, pour de
nombreuses raisons il est impossible de maintenir un bilan précis de
la quantité d'alumine dans le bain en fonction de ce qui est injecté
et de ce qui est consommé. En effet, l'environnement rend difficile la
pesée précise des quantités déposées, une partie des particules
volatiles ne parviennent jamais dans le bain, des agrégats se forme,
dont une partie s'accumule au fond de la cuve sur la cathode, et des
réactions chimiques parasites viennent, entre autres, grever ce bilan.

Pour contourner cette difficulté, les opérateurs exploitent le fait
que la resistivité du bain électrolytique dépend de la concentration
d'alumine dissoute, et atteint un minimum à la concentration optimale
$\concentration \approx$ \num{3}\% masse. En mesurant la chute de
potentiel électrique à travers le bain électrolytique, on maintient la
concentration d'alumine dissoute au voisinage de la concentration
optimale en alternant une phase de sur-alimentation en alumine et une
phase de sous-alimentation. Durant la phase de sur-alimentation, la
concentration d'alumine va passer au-delà de la concentration optimale
par conséquent accroître la resistivité du bain. Passé un certain
seuil, on débute une phase de sous-alimentation, durant laquelle la
résistivité commence par chuter, puis croît à nouveau. Passé un
certain seuil, on amorce une phase de sur-alimentation, et ainsi de
suite.

Dans chacune des phase de sur-alimentation ou sous-alimentation, les
injecteurs déposent les doses d'alumine selon une cadence préétablie
et périodique. La période et une taille des doses peut être spécifiée
indépendemment pour chaque injecteur.

\begin{figure}[t]
  \begin{center}
    \begin{tikzpicture}
      \begin{axis}[
          hide axis,
          colorbar,
          scale only axis,
          height=0.41\rasterimagewidth,,
          width=\rasterimagewidth,
          colorbar horizontal,
          point meta min=0.00,
          point meta max=0.05,
          colorbar style={
            title=Vitesse $u$ [\si{\meter\per\second}],
            width=7.4cm,
            height=0.3cm,
            xtick={0.00, 0.01, 0.02, 0.03, 0.04, 0.05},
            xticklabel style={
              /pgf/number format/fixed,
              /pgf/number format/fixed zerofill,
              /pgf/number format/precision=2
            },
            scaled x ticks = false,
            at={(0.5\rasterimagewidth,0.4cm)},
            anchor=north
          }
        ]
        \addplot [] coordinates {(0,0)};
        \node (myfirstpic) at (0,0) {\includegraphics[width=\rasterimagewidth]{{../media/populations/ap32-fluid-flow/print/acd-all-anodes-velocity-0.00-0.05}.png}};
      \end{axis}
    \end{tikzpicture}
    \caption{Champ de vitesse $u$ dans le bain électrolytique d'une
      cuve AP32 restreint sur une surface placée à mi-hauteur de
      l'ACD, vue depuis dessus. Cette situation correspond à un état
      d'opération standard.}
    \label{fig:ap32-flow-acd}
  \end{center}
\end{figure}

\begin{figure}[t]
  \begin{center}
    \includegraphics[width=\rasterimagewidth]{../media/populations/ap32-fluid-flow/print/chanel-velocity-streamlines.png}
    \caption{Lignes de courant correspondant au champ de vitesse
      représenté sur la figure \ref{fig:ap32-flow-acd}. Les lignes de
      courant prennent leur origine le long du canal central.}
    \label{fig:ap32-flow-streamlines}
  \end{center}
\end{figure}

\paragraph{Calcul de l'écoulement dans le bain} Une approximation de vitesse
d'écoulement du bain $u$ et de la densité de courant $j$ dans la cuve
AP32 est obtenue par l'intermédiaire du modèle multi-physique
stationnaire proposé par S. Steiner \cite{Steiner2009}, J. Rochat
\cite{Rochat2016} déjà introduit dans la section
\ref{sec:populations-introduction}. La figure \ref{fig:ap32-flow-acd}
représente la vitesse d'écoulement ainsi calculée par le logiciel
Alucell dans le bain électrolytique de la cuve AP32, dans
l'ACD. Lorsque la densité de courant électrique est répartie
uniformément sur toutes les anodes, l'écoulement dans les fluides
forme deux tourbillons principaux qui tournent en sens opposés. Deux
petits tourbillons se forment dans les coins avals. Les vitesse
maximales de l'écoulement (\num{5} \si{\centi\meter\per\second}
environ) sont atteinte dans le canal central au niveau des extrémités
de la cuve, ainsi que le long de la paroi amont, de part et d'autre de
la cuve. Dans le reste du bain et en particulier sous les anodes la
vitesse d'écoulement dépasse rarement \num{2}
\si{\centi\meter\per\second}. La figure
\ref{fig:ap32-flow-streamlines} illustre les lignes de courant de
l'écoulement dans le bain. On remarque les lignes de courant
s'engagent volontiers dans les canaux latéraux et dans le bain en
pourtour des rangées d'anodes.

\paragraph{Conditions sur l'injection et l'effet Joule}
Le schéma numérique proposé dans la section
\ref{sec:populations-discretisation} est conçu de manière à conserver
exactement d'une part la masse d'alumine dans les populations de
particules $\population$ et concentration d'alumine dissoute
$\concentration$, et d'autre part la quantité d'énergie thermique liée
à la température du bain $\temperature$. Pour des raisons déjà
évoquées, ces bilans ne sont pas exactement respectés dans une cuve
industrielle réelle, et il faut en général injecter un peu plus d'alumine
que ce qui est consommé par la réaction d'électrolyse. Quand à
l'énergie thermique, nous avons supposé que le bain est isolé
thermiquement, alors que dans une cuve réelle une quantité non
négligeable d'énergie s'échappe par le métal, les parois latérales de
la cuve, les anodes, la surface du bain et par le \ce{CO2} qui
s'échappe dans l'atmosphère.

Pour éviter que la masse totale d'alumine dans la cuve croisse sans
limite au cours du temps, il faut s'assurer que dans un état pseudo
stationnaire, la masse d'alumine reste proche de la masse d'alumine
initialement présente dans le bain. En d'autres termes, si $M_n$ est
la masse totale d'alumine dans le bain à l'instant $t_n$ telle que
définie dans la section (\ref{sec:populations-discretisation}), alors
on demande à ce que
\begin{equation*}
\lim_{n\to\infty} \frac{M_{n+1} - M_{0}}{t_n} = 0,
\end{equation*}
c'est-à-dire que
\begin{equation}\label{eq:injection-mass-condition}
  \frac{I[\cee{Al2O3}]}{6F}
  =\lim_{n\to \infty}\frac{1}{t_{n+1}}\ \sum_{\mathclap{\substack{k\\ p^k + 1\leq n +
        1}}}\ \int_\Omega\int_0^\infty\aluminadensity S^k \frac{4}{3}\pi r^3\intd{r}\intd{x}.
\end{equation}
Il faut donc choisir la masse des doses d'alumine injectées $S^k$, $k
= 1, 2,\dots$ et les temps d'injection $\tau^k$, $k = 1,2,\dots$ de
sorte à ce que le débit de masse de poudre d'alumine moyen au cours du
temps soit égal à $\frac{I[\cee{Al2O3}]}{6F}$, c'est-à-dire
\num{56.382e-3}\si{\kilo\gram\per\second}.

\begin{figure}
  \begin{center}
    \input{../media/populations/anode-configuration/anode-configuration.pdf_tex}
    \caption{Vue schématique de la partie supérieure du bain
      électrolytique. Les blocs rectangulaires représentent
      l'emplacement des anodes, tandis que les cercles marquent
      l'emplacement des injecteurs disposés le long du canal central.}
    \label{fig:anode-configuration}
  \end{center}
\end{figure}

La cuve AP32 possède 4 injecteurs placés au-dessus du canal central,
numérotés de 1 à 4, dans le sens de la coordonnée $x$ croissante (voir
la figure \ref{fig:anode-configuration}). Les paramètres qui
définissent chaque injecteurs sont regroupé dans la table
\ref{tab:injectors}.

\begin{table}
  \begin{center}
    \caption{Paramètres caractérisant les 4 injecteurs de la cuve AP32.}
    \label{tab:injectors}
    \begin{tabularx}{\textwidth}{@{}rrrrZ@{}}
      \toprule
      Injecteur & Position & Première injection & Intervalle d'injection & Masse de dose\\
      \midrule
      \#1         & \num{-4.4}\si\meter & \num{16}\si\second & \num{16}\si\second  & \num{0.225}\si{\kilo\gram} \\
      \#2         & \num{-1.6}\si\meter & \num{32}\si\second & \num{32}\si\second  & \num{0.451}\si{\kilo\gram} \\
      \#3         & \num{ 1.6}\si\meter & \num{48}\si\second & \num{48}\si\second  & \num{0.676}\si{\kilo\gram} \\
      \#4         & \num{ 4.4}\si\meter & \num{64}\si\second & \num{24}\si\second  & \num{0.338}\si{\kilo\gram} \\
      \bottomrule
    \end{tabularx}
  \end{center}
\end{table}
La masse des doses dans le tableau \ref{tab:injectors} est choisie de
telle sorte à ce que l'ensemble des 4 injecteurs injecte 25\% de la
masse d'alumine en moyenne. La forme des densité de particules
initiales $S^k$ est décrite dans \cite{Hofer2011}. Plus précisément,
la forme spatiale de chaque $S^k$ est une fonction lisse à support
compact centrée autour du point d'injection de l'injecteur
correspondant. La distribution en rayon initiale est approximée par
par une loi log-normale basée sur des mesures expérimentales. Puisque
le rapport des périodes d'injection des 4 injecteurs sont des nombres
rationnels, on peut définir une périodes globale liée à l'ensemble des
4 injecteurs. Étant données les périodes d'injections reportées dans le tableau
\ref{tab:injectors}, le cycle d'injection global est périodique après une
transition initiale de \num{64}\si\second, avec une période
de \num{192}\si\second. La figure \ref{fig:injections} représente les
injections qui ont lieu durant les \num{192} premières secondes de la
simulation.

\begin{figure}
  \begin{center}
    \input{../media/populations/cadence/cadence.tex}
    \caption{Temps d'injections des différents injecteurs de la cuve
      AP32. Chaque cercle représente une injection. Chaque ligne
      correspond à l'un des 4 injecteurs.}
    \label{fig:injections}
  \end{center}
\end{figure}

On détermine maintenant une condition sur la conductivité
électrique $\conductivity$, issue d'un argument similaire sur
l'énergie thermique du bain. Pour que l'énergie thermique du bain
reste proche de l'énergie thermique initiale, on demande à ce que
\begin{equation}
  \lim_{n\to\infty}\frac{\electrolytedensity\electrolytehc\displaystyle\int_\Omega\parent{\temperature_{n+1}
    - \temperature_0}\intd{x}}{t_{n+1}} = 0,
\end{equation}
ce qui correspond à la condition
\begin{eqnarray}
  \conductivity \int_\Omega j\cdot j\intd{x} &=&
  \aluminahc\parent{\tinit - \tinj}\lim_{n\to\infty}\frac{1}{t_{n+1}}\sum_{\mathclap{\substack{k\\ p^k +
  1 \leq n + 1}}}\int_\Omega\int_0^\infty\aluminadensity
  \frac{4}{3}\pi r^3 S^k\intd{r}\intd{x}\nonumber \\
  &&+ {\aluminadissolutionenthalpy}\lim_{n\to\infty}\frac{1}{t_{n+1}}\sum_{m = 0}^{n}\sum_{\mathclap{\substack{k\\ q^k < m + 1}}} \int_\Omega\int_0^\infty \aluminadensity \frac{4}{3}\pi r^3\parent{n_{p,m+1}^k-n_{p,m}^k}\intd{r}\intd{x}\label{eq:energy-condition}
\end{eqnarray}
en considérant (\ref{eq:energy-mass-balance}). Le premier terme du
membre de droite de (\ref{eq:energy-condition}) est égal à
\begin{equation*}
\aluminahc\parent{\tinit-\tinj}  \frac{I[\cee{Al2O3}]}{6F}
\end{equation*}
en vertu de (\ref{eq:injection-mass-condition}). Pour que le deuxième
terme du membre de droite de (\ref{eq:energy-condition}) converge, il
est suffisant que chaque population $k\geq 1$ se dissolve en un temps
fini donné quelque soit $k$. Plus précisément, les populations $n_p^k$
se dissolvent en un temps fini s'il existe un entier positif $q'$ tel
que $n_{p,n}^k = 0$ pour tout $n > q^k + q'$.

Le cas échéant on peut télescoper la somme sur $k$ et en remarquant que
\begin{equation*}
  \int_\Omega n_{p,q^k}^k\intd{x} = \int_\Omega S^k\intd{x}
\end{equation*}
pour tout $k \leq 1$ et pour tout $r > 0$ lorsque la vitesse de chute
$w = 0$, le deuxième terme du membre de droite de
(\ref{eq:energy-condition}) se réduit à
\begin{equation*}
  {\aluminadissolutionenthalpy}\lim_{n\to\infty}\frac{1}{t_{n+1}}\sum_{\mathclap{\substack{k\\ p^k +
  1 \leq n + 1}}}\int_\Omega\int_0^\infty\aluminadensity
  \frac{4}{3}\pi r^3 S^k\intd{r}\intd{x}.
\end{equation*}
Ainsi finalement,
\begin{equation*}
  \frac{1}{\sigma}\int_\Omega j\cdot j\intd{x}
  =\aluminahc\parent{\tinit-\tinj}\frac{I[\cee{Al2O3}]}{6F} + {\aluminadissolutionenthalpy}\frac{I[\cee{Al2O3}]}{6F}
\end{equation*}
et on fixe $\conductivity$ de sorte à satisfaire cette dernière
relation, c'est-à-dire que
\begin{equation}\label{eq:conductivity-condition}
  \conductivity = \frac{\displaystyle\int_\Omega j\cdot j\intd{x}}{\parent{\aluminahc\parent{\tinit-\tinj} + {\aluminadissolutionenthalpy} }
    \displaystyle\frac{I[\cee{Al2O3}]}{6F}}.
\end{equation}

\paragraph{Diffusivité du bain électrolytique} La vitesse d'écoulement
stationnaire du bain $u$ obtenue par la méthode introduite dans
\cite{Steiner2009}, \cite{Rochat2016} est la vitesse moyenne d'un
écoulement turbulent. Les structures turbulentes du fluide sont
décrites par un modèle de longueur de mélange de Smagorinsky
\cite{Rochat2016}. Dans le modèle de Smagorinsky, les structures
turbulentes de l'écoulement se traduise par une viscosité de
l'écoulement moyen $u$ proportionnelle au tenseur des déformation de
$u$. Dans le présent travail, nous caractérisons la diffusivité
$\electrolytecdiff$ de la concentration $\concentration$ et la
diffusivité $\electrolytetdiff$ de la température $\temperature$ par
deux réels $\electrolytemoldiff$, $\electrolyteturbdiff$ pour tout
$x\in\Omega$ de la manière suivante:
\begin{align}
  \electrolytetdiff(x) =  \electrolytecdiff(x) = \electrolytemoldiff +
  \electrolyteturbdiff\parent{2\sum_{i,j}\mathcal E_{ij}(x)\mathcal E_{ij}(x)}^{1/2},
\end{align}
où $\mathcal E_{ij}$, $i, j = 1,2,3$ est le tenseur des déformation
de l'écoulement défini par
\begin{equation}
  \mathcal E_{ij} = \frac{1}{2}\parent{\frac{\partial u_i}{\partial
      x_j} + \frac{\partial u_j}{\partial x_i}}.
\end{equation}
Les valeurs du paramètre $\electrolytemoldiff$, associé à la diffusion
moléculaire et de $\electrolyteturbdiff$, associé à la diffusion
induite par les turbulences de l'écoulement, sont rapporté dans le
tableau \ref{tab:dissolution-physical-parameters}.


\begin{table}
  \begin{center}
    \caption{Paramètres physiques et paramètres liés à la cuve AP32
      qui interviennent dans le transport et la dissolution de poudre
      d'alumine.}
    \label{tab:dissolution-physical-parameters}
    \begin{tabularx}{\textwidth}{@{}lllX@{}}
      \toprule
      Quantité                         & Valeur           & Unités                                      & Description \\
      \midrule
      $\electrolytedensity$            & \num{2130}       & \si{\kg\per\cubic\meter}                    & Masse volumique du bain électrolytique                          \\
      $\aluminadensity$                & \num{3960}       & \si{\kg\per\cubic\meter}                    & Masse volumique de l'oxyde d'aluminium                          \\
      $\electrolytehc$                 & \num{2945}       & \si{\joule\per\kilo\gram\per\kelvin}        & Chaleur spécifique du bain électrolytique liquide               \\
      $\aluminahc$                     & \num{1200}       & \si{\joule\per\kilo\gram\per\kelvin}        & Chaleur spécifique de l'oxyde d'aluminium                       \\
      $g$                              & \num{9.81}       & \si{\meter\per\square\second}               & Accélération de la gravité terrestre                            \\
      $I$                              & \num{320000}     & \si{\ampere}                                & Courant électrique total                                        \\
      $\faraday$                       & \num{96485.33}   & \si{\coulomb\per\mol}                       & Constante de Faraday                                            \\
      $\faradayyield$                  & \num{0.945}       & []                                          & Rendement de Faraday de l'électrolyse                           \\\relax
      [\ce{Al2O3}]                     & \num{0.102}      & \si{\kilo\gram\per\mol}                     & Masse molaire de l'oxyde d'aluminium                            \\
      $\tinit$                         & \num{1223}       & \si{\kelvin}                                & Température initiale du bain électrolytique                     \\
      $\tinj$                          & \num{423}        & \si{\kelvin}                                & Température d'injection des particules d'alumine                \\
      $\tliq$                          & \num{1218}       & \si{\kelvin}                                & Température du liquidus du bain électrolytique                  \\
      $\tcrit$                         & \num{1218.86}    & \si{\kelvin}                                & Température critique de dissolution dans le bain électrolytique \\
      $K$                              & \num{0.5e-09}    & \si{\square\meter\per\second}               & Taux de dissolution des particules d'alumine                    \\
      $\aluminadissolutionenthalpy$    & \num{5.3e5}      & \si{\joule\per\kilo\gram}                   & Enthalpie de dissolution de l'oxyde d'aluminium                 \\
      $\conductivity$                  & $\approx$ \num{900} & \si{\siemens\per\meter}                     & Conductivité électrique du bain électrolytique                  \\
%     $\electrolytelaminarviscosity$   & \num{2e-3}       & \si{\kilo\gram\per\meter\per\second}        & Viscosité laminaire du bain électrolytique                      \\
%     $\electrolyteturbulentviscosity$ & \num{2e-3}       & \si{\kilo\gram\per\meter\per\second}        & Viscosité turbulente du bain électrolytique                     \\
      $\electrolytemoldiff$            & \num{5e-4}       & \si{\squared\meter\per\second}              & Diffusivité moléculaire dans le bain                            \\
      $\electrolyteturbdiff$           & \num{5e-4}       & \si{\squared\meter\per\second}              & Diffusivité turbulente dans le bain                             \\
      $\electrolyteviscosity$          & \num{2e-3}       & \si{\kilo\gram\per\meter\per\second}        & Viscosité laminaire du bain électrolytique                      \\
      $\csat$                          & \num{1689.7}     & \si{\mol\per\cubic\meter}                   & Concentration de saturation de l'alumine dissoute               \\
      $\csatwp$                        & \num{7.8}        & w\%                                         & Concentration de saturation de l'alumine dissoute               \\
      $\cinit$                         & \num{635.3}      & \si{\mol\per\cubic\meter}                   & Concentration initiale de l'alumine dissoute                    \\
      $\cinitwp$                       & \num{3.0}        & w\%                                         & Concentration initiale de l'alumine dissoute                    \\
      \bottomrule
    \end{tabularx}
  \end{center}
\end{table}

\paragraph{Conditions initiales de la concentration et de la
  température}
Les conditions initiales des population de particules $n_p^k$ sont
données implicitement dans la description du modèle par la
relation (\ref{eq:splitting-np1-init}). Plus précisément, aucune
particule n'est présente dans le bain électrolytique à $t = 0$, et
donc $n_p(0, x,r) = 0$ pour $r>0$ et $x\in\Omega$.

On observe expérimentalement sur des cuves d'électrolyse industrielles
que celle-ci atteigne un état stationnaire périodique lié à la période
du cycle d'injection après un temps caractéristique de transition. On
observe typiquement l'établissement de cet état périodique par
d'intermédiaire de mesures indirectes de la résistivité électrique de
la cuve. Par analogie, dans le cadre du modèle de transport et
dissolution d'alumine nous nous attendons à ce que la densité de
particules $n_p$, la concentration $\concentration$ et la température
$\temperature$ atteigne asymptotiquement lorsque $t\to\infty$ un état
périodique stationnaire de période $P$ identique à la période du
cycle d'injection global, soit \num{192}\si{\second} dans notre cas.

Il est clair que plus les conditions initiales pour $c$ et
$\temperature$ sont éloignées de l'état périodique stationnaire
asymptotique, plus le temps nécessaire pour atteindre un tel état est
long. Par conséquent, nous choisissons des conditions initiales
uniformes en espace pour la concentration et la température, et qui
soient égales aux conditions optimales d'opération de la cuve AP32:
\begin{align*}
  & \concentration(0, x) = \cinit,\quad\forall x\in\Omega,\\
  & \temperature(0, x) = \tinit,\quad\forall x\in\Omega.
\end{align*}
Le paramètre $\tinit$ est reporté dans la table
\ref{tab:dissolution-physical-parameters}. La concentration initiale
$\cinit$ en \si{\mol\per\cubic\meter} s'exprime en fonction de la
concentration initiale $\concentration_{\text{init},\%\mathrm w}$ en \% masse à
l'aide de la formule:
\begin{equation*}
  \cinit = \frac{\cinitwp \cdot
    100^{-1}}{\electrolytedensity[\cee{Al2O3}]\parent{1 - \cinitwp\cdot
      100^{-1}\parent{1 - \displaystyle\frac{\electrolytedensity}{\aluminadensity}}}}.
\end{equation*}
Nous exploiterons également cette dernière formule pour convertir les
valeurs du champ de concentration en \% masse lors de la visualisation
des résultats. Les deux valeurs de la concentration initiale selon les
unités physique sont rapportées dans la table
\ref{tab:dissolution-physical-parameters}.
% paramètres numérique. (deltat, maillage, deltar, N_r)

\begin{figure}[t]
  \begin{center}
    \includegraphics[width=\rasterimagewidth]{../media/populations/ap32-mesh-components/print/bath-mesh.png}
    \caption{Aperçu du maillage du domaine occupé par le bain
      électrolytique dans la cuve AP32.}
    \label{fig:bath-mesh}
  \end{center}
\end{figure}

\paragraph{Paramètres de discrétisation}
Le maillage du domaine $\Omega$ est obtenu en même temps que les
champs $u$ et $j$ par la méthode déjà évoquée plus haut proposée par
\cite{Steiner2009}, \cite{Rochat2016}. On peut voir sur la figure
\ref{fig:bath-mesh} un aperçu du maillage de $\Omega$ qui correspond à
l'une des extrémités de la cuve. Le maillage est fortement anisotrope,
avec des rapports d'aspect d'environ \num{25} dans l'ACD. Le diamètre
des mailles est compris entre \num{38}\si{\centi\meter} (aux
extrémités de la cuve) et \num{9}\si{\centi\meter} (dans l'ACD et les
canaux). Le maillage comporte \num{282240} éléments tetraédriques. On
choisit le nombre de discrétisation des rayons des particules $M = 5$
et $\dr = \num{40}$ \si{\micro\meter}. Enfin on fixe $\dt = 1$
\si{\second} et $T = \num{10000}$ \si{\second}, le temps auquel on
veut évaluer la concentration.


\paragraph{Solution de référence} Afin d'évaluer l'effet de la
température de l'électrolyte sur la dissolution de la poudre
d'alumine, nous commençons par présenter un calcul pour lequel les
particule d'alumine ne chute pas dans le bain, \ie, $w = 0$, pour
lequel les particules commencent à se dissoudre instantanément,
c'est-à-dire que $\tlat = 0$, et pour lequel la
vitesse de dissolution ne dépend pas de température du bain
$\temperature$. Plus précisément, la vitesse de dissolution définie
par (\ref{eq:dissolution-rate}) est remplacée par
\begin{equation}
  \dissolutionrate(c, \temperature)= \left\{
  \begin{array}{ll}
    K\displaystyle\frac{\csat - \concentration}{\csat} &\quad \text{si } 0\leq
    \concentration \leq \csat,\\
    0& \quad \text{sinon}.
  \end{array}
  \right.
\end{equation}
Cette situation est équivalente à supposer que la température du
bain $\temperature$ reste largement supérieure à la température
critique $\tcrit$ au cours du calcul, de sorte à ce que le terme
exponentiel dans (\ref{eq:dissolution-rate}) soit proche de zéro.

Un des objectifs recherchés par les opérateurs de cuve
industrielles est de minimiser les écarts de concentration autour de
la concentration d'alumine dissoute optimale. Pour évaluer le champ
de concentration par rapport à cet objectif, nous proposons de
représenter la variance  du champ de concentration
$c$ dans $\Omega$ au cours du temps définie par:
\begin{equation}
  \Var_c(t) = \sqrt{\int_\Omega\parent{\concentration(t, x) - \bar \concentration(t)}^2\intd{x}},
\end{equation}
où $\bar \concentration$ est la moyenne de $\concentration$ sur
$\Omega$:
\begin{equation}
  \bar \concentration(t) = \frac{1}{\abs{\Omega}}\int_\Omega c(t, x)\intd{x}
\end{equation}
et $\abs{\Omega}$ est le volume de $\Omega$. Clairement, la situation
idéale où la concentration d'alumine est uniforme dans tout
l'électrolyte correspond à une variance de $\concentration$ nulle. A
l'inverse, plus $\concentration$ s'écarte de sa valeur moyenne, plus
sa variance est importante.

La valeur de $\Var_\concentration$ au cours du temps offre un moyen
d'évaluer si le système a atteint l'état stationnaire périodique
recherché. Bien entendu ce n'est qu'une condition nécessaire, la
variance d'une fonction stationnaire est elle-même stationnaire, mais
l'inverse n'est pas nécessairement vrai. En pratique on prend soins de
vérifier que la différence en norme $L^2$ à deux instants successifs
séparés par la période du cycle d'injection global $P$ est inférieure
à \num{1}\%. Pour tous les calculs présenté dans cette partie, la
solution satisfait ce critère pour $T = \num{10000}$ \si{\second}.

\begin{figure}
  \begin{center}
    \input{../media/populations/variances/variance-control.tex}
    \caption{Évolution de $\Var_\concentration$ au cours du temps sur
      l'intervalle $[0, T]$ dans le bain électrolytique de la cuve
      AP32 lorsque la température n'est pas prise en compte dans la
      vitesse de dissolution des particules.}
    \label{fig:alumin-control-var}
  \end{center}
\end{figure}

La figure \ref{fig:alumin-control-var} représente l'évolution de
la variance de la concentration au cours du temps. On remarque une
phase initiale lorsque $t\in[0,500]$, au cours de laquelle la variance
croît rapidement. Cette croissance ralenti quand la concentration
approche de l'état stationnaire, et à partir de $t \approx
\num{4500}$ \si{\second} nous pouvons considérer que la
concentration est dans un état stationnaire et périodique. Les
fluctuation de $\Var_\concentration$ que l'on peut observer sur la
figure \ref{fig:alumin-control-var} sont dues aux injections de poudre
d'alumine dans le bain. Immédiatement après l'injection, la
dissolution des particules provoque un accroissement rapide de la
concentration d'alumine dissoute localement autour du point
d'injection. Lorsque la dose est totalement dissoute, ce pique de
concentration est atténué par la diffusion de la concentration
dans l'électrolyte. Cet effet se traduit par un accroissement de la
variance de la concentration immédiatement après l'injection d'une
dose, puis par une décroissance de la variance.

Cependant, ces variations de la concentration dans l'état
stationnaire sont très localisées autour des points d'injection. Dans
le reste du bain électrolytique, les variations de la concentration
restent de l'ordre de \num{1}\%.

Puisque, lorsque l'état stationnaire périodique est atteint, la
concentration varie peu au cours du cycle d'injection global, nous
pouvons nous permettre de visualiser la distribution de la
concentration dans le bain électrolytique à un instant arbitraire du
cycle d'injection global. Le temps $T = \num{10000}$ \si{\second}
auquel la solution est évaluée correspond donc à environ \num{51}
périodes du cycle d'injection global, sans compter la phase
transitoire initiale de \num{64} \si{\second}.

\begin{figure}[h]
  \begin{center}
    \begin{tikzpicture}
      \begin{axis}[
          hide axis,
          colorbar,
          scale only axis,
          height=0.41\rasterimagewidth,,
          width=\rasterimagewidth,
          colorbar horizontal,
          point meta min=2.38,
          point meta max=4.21,
          colorbar style={
            title=Concentration $c$ [\%w],
            width=7.4cm,
            height=0.3cm,
            xtick={2.38, 2.50, 3.00, 3.50, 4.00, 4.21},
            xticklabel style={
              /pgf/number format/fixed,
              /pgf/number format/fixed zerofill,
              /pgf/number format/precision=2
            },
            scaled x ticks = false,
            at={(0.5\rasterimagewidth,0.4cm)},
            anchor=north
          }
        ]
        \addplot [] coordinates {(0,0)};
        \node (myfirstpic) at (0,0) {\includegraphics[width=\rasterimagewidth]{{../media/populations/application/print/alumina-control-2.38-4.21}.png}};
      \end{axis}
    \end{tikzpicture}
    \caption{Concentration d'alumine dissoute dans à $t =
      \num{10000}$ \si{\second} dans l'ACD de la cuve AP32, lorsque la
    température n'est pas prise en compte dans la vitesse de
    dissolution des particules.}
    \label{fig:ap32-alumina-wo-t}
  \end{center}
\end{figure}

On s'intéresse à la distribution de la concentration d'alumine là où a
lieu la réaction d'électrolyse, c'est-à-dire essentiellement dans
l'ACD. On se contente donc de visualiser la distribution de la
concentration d'alumine dans cette zone. Dans le domaine $\Omega$
occupé par l'électrolyte, l'ACD est maintenue constante avec $3.2$
\si{\centi\meter} d'épaisseur sur l'ensemble de l'interface. Pour
visualiser la concentration d'alumine dissoute dans l'ACD, on évalue
$\concentration$ sur une surface fictive placée dans l'électrolyte,
parallèle à l'interface et à une distance égale à la moitié de
l'ACD. La figure \ref{fig:ap32-alumina-wo-t} présente la distribution de la
concentration d'alumine dans le bain électrolytique.

On remarque sur la figure \ref{fig:ap32-alumina-wo-t} que la
concentration atteint des maximums locaux aux voisinages des points
d'injection, ce qui montre que l'essentiel de la poudre d'alumine se
dissout dans ces régions. Cette alumine dissoute est ensuite
transportée par l'écoulement. Les deux injecteurs de gauche alimentent
le tourbillon de gauche, tandis que les deux injecteurs de droite
alimentent essentiellement le tourbillon de droite. Les régions du bain
sous-alimentées sont les coins en aval, où la concentration descend
en dessous de \num{3} \%w et où l'écoulement est
caractérisé par la présence de petits tourbillons isolés. La région
centrale en amont des points d'injections est remarquablement uniforme
avec une concentration proche de \num{3} \%w.


Nous consacrons maintenant le reste de cette partie au
modèle de transport et dissolution d'alumine qui dépend de la
température du bain dans le bain de la cuve AP32, et l'on étudie
l'influence des nouveaux paramètres que ce modèle introduit,
en particulier le temps de latence $\tlat$, la température initial du
bain $\tinit$, la température critique $\tcrit$ et la vitesse de
chute des particules dans le bain $w$.


% Sensibilité par rapport a la latence de dissolution
\paragraph{Sensibilité par rapport au temps de latence}
Nous avons montré dans le paragraphe \ref{sec:particle-freeze} que le
temps de latence qui précède le début de la dissolution d'une
particule lâchée dans le bain d'une cuve d'électrolyse est de l'ordre
de \num{1e-1} \si{\second}. Ce temps caractéristique est
négligeable devant le temps de dissolution qui est au minimum de
\num{10} \si{\second} pour le choix de paramètres reportés dans la
table \ref{tab:dissolution-physical-parameters}.

Cependant, lors de l'injection d'une dose d'alumine typique, les
particules ne peuvent plus être suffisamment dispersées pour que
les hypothèses du modèle introduit dans la section
\ref{sec:particle-freeze} soient satisfaites, et c'est l'effet
collectif de l'ensemble des particules qui prédomine. Selon
\cite{Dassylva2015}, toutes les particules d'une dose subissent un
temps de latence de l'ordre de \num{1} \si{\second} quel que soit leur
taille.

Nous proposons maintenant de déterminer si la distribution de
concentration dans le bain électrolytique est sensible au temps de
latence de dissolution des particule lors de leur injection. Dans ce
but, nous présentons les résultats de 4 calculs du champ de
concentration d'alumine dissoute $c$ dans le bain électrolytique de
la cuve AP32 avec le modèle de transport et dissolution de
particules en fonction de la température du bain, avec les
paramètres reportés dans la table
\ref{tab:dissolution-physical-parameters} à l'exception du temps de
latence $\tlat$ pour lequel nous fixons successivement $\tlat = $
\numlist{1;2;5;10} secondes.

\begin{figure}[!hp]
  \begin{center}
      \begin{tikzpicture}
        \begin{axis}[
            %colorbar,
            hide axis,
            scale only axis,
            height=0.26\rasterimagewidth,,
            width=\rasterimagewidth,
            %colorbar horizontal,
            point meta min=2.54,
            point meta max=3.08,
            colorbar style={
              title=Concentration [\%w],
              width=7.4cm,
              height=0.3cm,
              xtick={2.54, 2.75 3, 3.08, 3.5, 4, 4.5, 5, 5.5, 6},
              at={(0.5\rasterimagewidth,0.4cm)},
              anchor=north
            }
          ]
          \addplot [] coordinates {(0,0)};
          \node (myfirstpic) at (0,0) {\framebox{\includegraphics[width=\rasterimagewidth]{{../media/populations/application/print/alumina-influance-th1-2.54-3.08}.png}}};
        \end{axis}
      \end{tikzpicture}
      \begin{tikzpicture}
        \begin{axis}[
            %colorbar,
            hide axis,
            scale only axis,
            height=0.26\rasterimagewidth,,
            width=\rasterimagewidth,
            %colorbar horizontal,
            point meta min=2.54,
            point meta max=3.09,
            colorbar style={
              title=Concentration [\%w],
              width=7.4cm,
              height=0.3cm,
              xtick={2.54,2.75, 3,3.09, 3.5, 4, 4.5, 5, 5.5, 6},
              at={(0.5\rasterimagewidth,0.4cm)},
              anchor=north
            }
          ]
          \addplot [] coordinates {(0,0)};
          \node (myfirstpic) at (0,0) {\framebox{\includegraphics[width=\rasterimagewidth]{{../media/populations/application/print/alumina-influance-th2-2.54-3.09}.png}}};
        \end{axis}
      \end{tikzpicture}
      \begin{tikzpicture}
        \begin{axis}[
            %colorbar,
            hide axis,
            scale only axis,
            height=0.26\rasterimagewidth,,
            width=\rasterimagewidth,
            %colorbar horizontal,
            point meta min=2.54,
            point meta max=3.09,
            colorbar style={
              title=Concentration [\%w],
              width=7.4cm,
              height=0.3cm,
              xtick={2.54, 2.75, 3, 3.09, 3.5, 4, 4.5, 5, 5.5, 6},
              at={(0.5\rasterimagewidth,0.4cm)},
              anchor=north
            }
          ]
          \addplot [] coordinates {(0,0)};
          \node (myfirstpic) at (0,0) {\framebox{\includegraphics[width=\rasterimagewidth]{{../media/populations/application/print/alumina-influance-th5-2.54-3.09}.png}}};
        \end{axis}
      \end{tikzpicture}
      \begin{tikzpicture}
        \begin{axis}[
            colorbar,
            hide axis,
            scale only axis,
            height=0.52\rasterimagewidth,,
            width=\rasterimagewidth,
            colorbar horizontal,
            point meta min=2.54,
            point meta max=3.09,
            colorbar style={
              title=Concentration [\%w],
              width=7.4cm,
              height=0.3cm,
              xtick={2.54,2.75, 3, 3.09, 3.5, 4, 4.5, 5, 5.5, 6},
              at={(0.5\rasterimagewidth,3.0cm)},
              anchor=north
            }
          ]
          \addplot [] coordinates {(0,0)};
          \node (myfirstpic) at (0,50) {{\includegraphics[width=\rasterimagewidth]{{../media/populations/application/print/alumina-influance-th510-2.54-3.09}.png}}};
        \end{axis}
      \end{tikzpicture}
    \caption{Champ de concentration $c$ dans l'ACD de la cuve AP32 à
      $t = \num{10000}$ \si{\second}. De haut en bas, le temps de
      latence de dissolution $T_\text{Lat} = 1\si\second$,
      $2\si\second$, $5\si\second$ et $10\si\second$.}
    \label{fig:dissolution-alumin-influence-tlat}
  \end{center}
\end{figure}


La figure \ref{fig:dissolution-alumin-influence-tlat} présente la
distribution de concentration d'alumine dissoute dans l'ACD de la cuve
AP32 à $T = $ \num{10000} \si{\second} pour les différentes valeurs
de $\tlat$ croissantes de haut en bas.

Les quatre champs de concentration présentés sur la figure
\ref{fig:dissolution-alumin-influence-tlat} sont très similaires. Les
maximums et les minimum de la concentration d'alumine sont atteint
sont les mêmes et apparaissent aux mêmes endroits. On remarque
malgré tout de petites modification de la distribution d'alumine
dissoute au voisinage des points d'injection. On observe par exemple
autour du point d'injection \#2 que la concentration est plus diffuse
lorsque $\tlat$ croît.

Nous nous intéressons maintenant à l'effet de la température
initiale du bain sur la distribution de la concentration d'alumine dissoute.

% Sensibilité par rapport a la surchauffe du bain
\paragraph{Sensibilité par rapport à la surchauffe initiale du bain}
La vitesse de dissolution des particules dans l'électrolyte
(\ref{eq:dissolution-velocity}) dépend de la température locale de
celui-ci. Comme précisé plus haut dans cette section, sous
l'hypothèse que chaque dose de particules se dissout entièrement
en un temps fini, le terme source d'énergie par effet Joule est
construit de sorte à maintenir une quantité d'énergie thermique
constante en moyenne dans le temps. En l'absence de transition de
phase dans l'électrolyte, cela signifie que la température moyenne au
cours du temps de celui-ci est maintenue constante. Par conséquent,
une température initiale du bain $\tinit$ plus élevée résulte en une
température $\temperature$ dans l'état stationnaire périodique plus
élevée en moyenne au cours d'une période du cycle d'injection global.

\begin{figure}[!hp]
  \begin{center}
      \begin{tikzpicture}
        \begin{axis}[
            %colorbar,
            hide axis,
            scale only axis,
            height=0.26\rasterimagewidth,,
            width=\rasterimagewidth,
            %colorbar horizontal,
            point meta min=2.43,
            point meta max=3.05,
            colorbar style={
              title=Concentration [\%w],
              width=7.4cm,
              height=0.3cm,
              xtick={2.43,2.5,2.75, 3, 3.05, 3.5, 4, 4.5, 5, 5.5, 6},
              at={(0.5\rasterimagewidth,0.4cm)},
              anchor=north
            }
          ]
          \addplot [] coordinates {(0,0)};
          \node (myfirstpic) at (0,0) {\framebox{\includegraphics[width=\rasterimagewidth]{{../media/populations/application/print/alumina-influance-sur1-2.43-3.05}.png}}};
        \end{axis}
      \end{tikzpicture}
      \begin{tikzpicture}
        \begin{axis}[
            %colorbar,
            hide axis,
            scale only axis,
            height=0.26\rasterimagewidth,,
            width=\rasterimagewidth,
            %colorbar horizontal,
            point meta min=2.51,
            point meta max=3.02,
            colorbar style={
              title=Concentration [\%w],
              width=7.4cm,
              height=0.3cm,
              xtick={2.51,2.75, 3,3.02, 3.5, 4, 4.5, 5, 5.5, 6},
              at={(0.5\rasterimagewidth,0.4cm)},
              anchor=north
            }
          ]
          \addplot [] coordinates {(0,0)};
          \node (myfirstpic) at (0,0) {\framebox{\includegraphics[width=\rasterimagewidth]{{../media/populations/application/print/alumina-influance-sur2-2.51-3.02}.png}}};
        \end{axis}
      \end{tikzpicture}
      \begin{tikzpicture}
        \begin{axis}[
            %colorbar,
            hide axis,
            scale only axis,
            height=0.26\rasterimagewidth,,
            width=\rasterimagewidth,
            %colorbar horizontal,
            point meta min=2.51,
            point meta max=3.13,
            colorbar style={
              title=Concentration [\%w],
              width=7.4cm,
              height=0.3cm,
              xtick={2.51,2.75, 3,3.13, 3.5, 4, 4.5, 5, 5.5, 6},
              at={(0.5\rasterimagewidth,0.4cm)},
              anchor=north
            }
          ]
          \addplot [] coordinates {(0,0)};
          \node (myfirstpic) at (0,0) {\framebox{\includegraphics[width=\rasterimagewidth]{{../media/populations/application/print/alumina-influance-sur5-2.51-3.13}.png}}};
        \end{axis}
      \end{tikzpicture}
      \begin{tikzpicture}
        \begin{axis}[
            colorbar,
            hide axis,
            scale only axis,
            height=0.52\rasterimagewidth,,
            width=\rasterimagewidth,
            colorbar horizontal,
            point meta min=2.51,
            point meta max=3.13,
            colorbar style={
              title=Concentration [\%w],
              width=7.4cm,
              height=0.3cm,
              xtick={2.51, 2.75, 3, 3.13, 3.5, 4, 4.5, 5, 5.5, 6},
              at={(0.5\rasterimagewidth,3.0cm)},
              anchor=north
            }
          ]
          \addplot [] coordinates {(0,0)};
          \node (myfirstpic) at (0,50) {{\includegraphics[width=\rasterimagewidth]{{../media/populations/application/print/alumina-influance-sur10-2.51-3.13}.png}}};
        \end{axis}
      \end{tikzpicture}
      \caption{Champ de concentration $c$ dans l'ACD de la cuve AP32 à
        $t = \num{10000}$ \si{\second}. De haut en bas, la température
        de surchauffe initiale est $\tsur = $ \numlist{1;2;5;10}.}
    \label{fig:dissolution-alumin-influence-sur}
  \end{center}
\end{figure}


Nous proposons d'évaluer la sensibilité de la dissolution d'alumine
dans le bain électrolytique en fonction de la température de celui-ci
en faisant varier la température initiale du bain. On note $\tsur$ la
température de surchauffe initiale du bain au-dessus de sa température de
liquidus:
\begin{equation}
  \tsur = \tinit - \tliq.
\end{equation}
La figure \ref{fig:dissolution-alumin-influence-sur} présente la distribution
de la concentration d'alumine dissoute $c$ dans l'ACD de la cuve AP32
issue de quatre calculs, pour lesquels les paramètres sont reportés
dans la table \ref{tab:dissolution-physical-parameters}, à
l'exception de la température initiale du bain $\tinit$. La
température initiale du bain est fixée à $\tinit = \tliq + \tsur$ avec $\tsur = $ \numlist{1;2;5;10} kelvin.

Les champs de concentration sur la figure
\ref{fig:dissolution-alumin-influence-sur} correspondent de haut en
bas à des température de surchauffe $\tsur$ croissante. On remarque
en particulier sur le dernier champ de concentration de cette figure,
qui correspond à $\tsur = $ \num{10} \si{\kelvin}, la similarité avec la
solution de référence illustrée sur la figure
\ref{fig:ap32-alumina-wo-t} pour laquelle la température ne joue pas
de rôle dans la dissolution des particules. Les valeurs maximales
atteinte par la concentration d'alumine dissoute sont cependant plus faibles.

Le premier champ de concentration de la figure
\ref{fig:dissolution-alumin-influence-sur} correspond à une
température de surchauffe $\tsur =$ \num{1} \si{\kelvin} et s'écarte
significativement de la solution de référence. La concentration
d'alumine dissoute est homogène au point de ne plus pouvoir
distinguer la position des injecteurs.

La dissolution des particules d'alumine dépend d'une part du fait que
la température de l'électrolyte dans leur voisinage doit être
supérieur à la température du liquidus $\tliq$, et d'autre part de la
taille de la région de transition entre le régime de dissolution
contrôlé par la diffusion de l'énergie thermique et le régime de
dissolution contrôlé par la diffusion de l'alumine dissoute à
proximité de la surface des particules. La transition entre ces deux
régimes est contrôlée par le paramètre $\tcrit$. Dans le paragraphe
suivant nous proposons d'évaluer la distribution de concentration dans
le bain électrolytique en fonction du paramètre $\tcrit$.


\clearpage
% Sensibilité par rapport au modèle de dépendence (tcrit, exp vs heaviside)
\paragraph{Sensibilité par rapport à la température critique de transition}
\begin{figure}[!hp]
  \begin{center}
    \begin{tikzpicture}
      \begin{axis}[
          %colorbar,
          hide axis,
          scale only axis,
          height=0.26\rasterimagewidth,
          width=\rasterimagewidth,
          %colorbar horizontal,
          point meta min=2.55,
          point meta max=3.08,
          colorbar style={
            title=Concentration [\%w],
            width=7.4cm,
            height=0.3cm,
            xtick={2.5,2.55, 2.75, 3, 3.08, 3.5, 4, 4.5, 5, 5.5, 6},
            at={(0.5\rasterimagewidth,0.4cm)},
            anchor=north
          }
        ]
        \addplot [] coordinates {(0,0)};
        \node (myfirstpic) at (0,0) {\framebox{\includegraphics[width=\rasterimagewidth]{{../media/populations/application/print/alumina-influance-heaviside-2.55-3.08}.png}}};
      \end{axis}
    \end{tikzpicture}
    \begin{tikzpicture}
      \begin{axis}[
          colorbar,
          hide axis,
          scale only axis,
          height=0.52\rasterimagewidth,,
          width=\rasterimagewidth,
          colorbar horizontal,
          point meta min=2.55,
          point meta max=3.08,
          colorbar style={
            title=Concentration [\%w],
            width=7.4cm,
            height=0.3cm,
            xtick={2.5,2.55,2.75, 3,3.08, 3.5, 4, 4.5, 5, 5.5, 6},
            at={(0.5\rasterimagewidth,3.0cm)},
            anchor=north
          }
        ]
        \addplot [] coordinates {(0,0)};
        \node (myfirstpic) at (0,50) {{\includegraphics[width=\rasterimagewidth]{{../media/populations/application/print/alumina-influance-exp-2.55-3.08}.png}}};
      \end{axis}
    \end{tikzpicture}
    \caption{Champ de concentration $c$ dans l'ACD de la cuve AP32 à
      $t = $ \num{10000} \si{\second}. En haut, $T_\text{Crit} =
      T_\text{Liq}$. En bas, $T_\text{Crit} = T_\text{Liq} + 0.86$
      \si{\kelvin}.}
    \label{fig:dissolution-influence-exp-heaviside}
  \end{center}
\end{figure}


Dans la limite où $\tcrit - \tliq$ devient grand, on voit facilement
dans l'expression (\ref{eq:dissolution-rate}) que le taux de
dissolution des particules, et donc leur vitesse de dissolution,
devient nulle dans l'intervalle de température que le bain
électrolytique peut raisonnablement admettre. Par conséquent, si la
valeur de $\tcrit - \tliq$ est trop élevée, le temps dissolution des
particules peut devenir arbitrairement long, ce qui n'est pas réaliste
dans le cas d'une cuve d'électrolyse d'aluminium industrielle.

A l'inverse, en prenant la limite $\tcrit \to  \tliq$, le taux de
dissolution (\ref{eq:dissolution-rate}) s'écrit:
\begin{equation}
  \dissolutionrate(\concentration,\temperature) = \left\{
  \begin{array}{ll}
  K \displaystyle\frac{\csat - \concentration}{\csat}& \text{ si } \temperature
  \geq \tliq\text{ et } 0\leq \concentration \leq \csat,\\
  0 &  \text{ sinon.}
  \end{array}
  \right.
\end{equation}
En d'autres termes, la dissolution des particules est contrôlée
uniquement par la concentration locale $\concentration$ si
$\temperature \geq \tliq$, et ne se dissolvent pas sinon.

Nous présentons maintenant la distribution de concentration d'alumine
dissoute dans le bain de la cuve AP32 dans l'état stationnaire
périodique résultant de deux calculs. Comme précédemment, l'ensemble
des paramètres du modèle de transport et dissolution est reporté dans
la table \ref{tab:dissolution-physical-parameters}. Pour le premier
calcul on fixe $\tcrit = \tliq$, et le taux de dissolution est donné par
(\ref{eq:dissolution-rate-zero-tcrit}). Pour le deuxième calcul on
fixe $\tcrit = \tliq + 0.86$ \si{\kelvin}.

La figure \ref{fig:dissolution-alumin-influence-exp-heaviside}
présente la concentration d'alumine dans l'ACD de la cuve AP32 à $t =
\num{10000}$ \si{\second}, lorsque l'état stationnaire périodique est
atteint. Le premier champ de concentration correspond au cas où
$\tcrit = \tliq$, tandis que le deuxième champ de concentration
correspond au cas où $\tcrit = \tliq + 0.86$ \si{\kelvin}. Même si
l'on parvient à observer de petites variations entre ces deux
solution, en particulier dans le coin aval droite et autour du point
d'injection \#4, ces deux champs de concentration sont indistinguables
l'un de l'autre.

Finalement, nous nous penchons sur l'effet de la chute
gravitationnelle des particules d'alumine sur le champ de
concentration d'alumine dissoute.

% Sensibilité par rapport a la vitesse de chute dans le bain
\paragraph{Effet de la vitesse de chute des particules sur leur
  dissolution}
\begin{figure}[!hp]
  \begin{center}
      \begin{tikzpicture}
        \begin{axis}[
            hide axis,
            colorbar,
             scale only axis,
             height=0.41\rasterimagewidth,,
             width=\rasterimagewidth,
             colorbar horizontal,
             point meta min=2.38,
             point meta max=4.21,
             colorbar style={
               title=Concentration [\%w],
               width=7.4cm,
               height=0.3cm,
               xtick={2.38, 3, 3.5, 4,4.21, 4.5, 5, 5.5, 6},
               at={(0.5\rasterimagewidth,0.4cm)},
               anchor=north
            }
          ]
          \addplot [] coordinates {(0,0)};
          \node (myfirstpic) at (0,0) {\includegraphics[width=\rasterimagewidth]{{../media/populations/application/print/alumina-control-2.38-4.21}.png}};
        \end{axis}
      \end{tikzpicture}
      \begin{tikzpicture}
        \begin{axis}[
            colorbar,
            hide axis,
            scale only axis,
            height=0.41\rasterimagewidth,
            width=\rasterimagewidth,
            colorbar horizontal,
            point meta min=2.38,
            point meta max=4.21,
            colorbar style={
              title=Concentration [\%w],
              width=7.4cm,
              height=0.3cm,
              xtick={2.38, 3, 3.5, 4,4.21, 4.5, 5, 5.5, 6},
              at={(0.5\rasterimagewidth,0.4cm)},
              anchor=north
            }
          ]
          \addplot [] coordinates {(0,0)};
          \node (myfirstpic) at (0,0) {\includegraphics[width=\rasterimagewidth]{{../media/populations/application/print/dummy-tmp-result-c-2.38-4.21}.png}};
        \end{axis}
      \end{tikzpicture}
      \caption{Champ de concentration d'alumine dissoute dans l'ACD de
        la cuve AP32 à $t = \num{10000}\si\second$. En haut,
        dissolution des particules sans chute gravitationnelle dans le
        bain. En bas, dissolution des particules avec chute
        gravitationnelle dans le bain.}
      \label{fig:sedimentation-comparaison}
  \end{center}
\end{figure}


Nous avons déterminé, dans la section \ref{sec:particle-fall}, vitesse
de chute et la profondeur maximale atteinte dans le fluide de
particules soumisent à la gravité et à une force de traînée
de Stokes. Dans le cas le plus favorable, certaine particules peuvent
chuter de plusieurs centimètre dans le bain électrolytique avant de se
dissoudre complètement.

Dans ce dernier paragraphe, nous proposons de comparer deux calculs
qui illustrent l'effet de la chute des particules dans le bain sur la
concentration d'alumine dissoute dans celui-ci.

Pour le premier calcul, qui joue le rôle de point de référence, les
paramètres sont reporté dans la table
\ref{tab:dissolution-physical-parameters} et nous annulons
l'accélération de la gravité $g = 0$, ce qui donne lieu à une vitesse
de chute strictement nulle, soit $w(r) = 0$ pour tout $r>0$. De plus,
l'injection des dose d'alumine a lieu dans le canal central comme
indiqué sur la figure \ref{fig:injections}, au niveau de l'ACD.

Pour le deuxième calcul la valeur de $g$ est restaurée, telle que
donnée dans la table \ref{tab:dissolution-physical-parameters} et la
vitesse de chute des particules est donnée par l'expression
\ref{eq:fall-velocity}. Cependant, les points d'injections sont
déplacé verticalement vers le haut du canal.

Nous avons montré dans la section \ref{sec:particle-fall} que les
profondeurs maximales atteinte par le particules lorsque la viscosité
du fluide est supérieure ou égale à \num{1e-2}
\si{\kilo\gram\per\meter\per\second} sont de l'ordre du
millimètre. Or, la taille verticale des mailles dans les canaux ne
permette pas de capturer des effets de cette amplitude. Pour cette raison,
nous choisissons pour le paramètre de viscosité $\electrolyteviscosity$, qui
intervient dans la définition (\ref{eq:fall-velocity}), la viscosité
laminaire du fluide, soit $\electrolyteviscosity = $ \num{2e-3}
\si{\kilo\gram\per\meter\per\second}. Cette valeur de la viscosité
donne lieu à des profondeurs de chute maximale de l'ordre de
quelques centimètres.


La figure \ref{fig:sedimentation-comparaison} représente les champs de
concentration dans l'ACD de la cuve AP32 obtenus par ces deux calculs,
évalué à $t = \num{10000}$ \si{\second}, lorsque l'état stationnaire
périodique est atteint.

On remarque quelques différence entre les deux solutions,
essentiellement au niveau des points d'injection. Les valeur maximales
de la concentration d'alumine dissoute à proximité des points
d'injection \#1, \#3 et \#4 sont plus grandes lorsque lorsque la chute
des particules est négligée. On constate l'effet inverse pour
l'injecteur \#2: la concentration d'alumine dissoute sous cet
injecteur atteint un maximum local plus marqué lorsque la chute des
particules est pris en compte.

Nous concluons cette partie par une discussion de ces résultats.



\part{Approximation d'écoulement dans le bain électrolytique}
\label{part:fluid}

\chapter{Une méthode numérique pour le calcul de l'écoulement de Stokes}
\label{chap:stokes-fourier}

\printbibliography[heading=bibintoc]

\end{document}
