Dans cette partie, nous avons proposé un modèle de transport et
dissolution des particules d'alumine dans le bain électrolytique d'une
cuve d'électrolyse d'aluminium. La dissolution des particules dépend
de leur taille et de la concentration d'alumine dissoute dans leur
voisinage, et de la température du bain électrolytique d'autre
part. Nous avons proposé un schéma numérique pour approcher la densité
de particules $n_p$, concentration d'alumine dissoute $c$ et la
température de l'électrolyte $\temperature$ dans le bain
électrolytique basé sur un splitting en temps des différentes
équations, et une méthode élément fini Lagrange continue, linéaire par
morceau décrite dans \cite{Hofer2011} pour la discrétisation en
espace. Nous avons ensuite appliqué ce modèle au cas de la cuve
industrielle AP32 et calculé la distribution du champ de concentration
$c$ dans le bain électrolytique jusqu'à ce que le système atteigne un
état stationnaire périodique. Nous avons évalué le comportement de
cette distribution de concentration stationnaire en fonction des
nouveaux paramètres introduits par ce modèle par rapport au travail de
T. Hofer \cite{Hofer2011}.

Tout d'abord, nous avons montré que le temps de latence de dissolution
des particules a un impact négligeable sur la distribution de
concentration. La vitesse de l'écoulement dans le bain aux points
d'injection est de l'ordre de 2 à 3 \si{\centi\meter\per\second}. Dans
le cas le plus extrême, c'est-à-dire lorsque $\tlat = 10$
\si{\second}, les particules sont transportées sur une distance
maximale de 20 à 30 \si{\centi\meter} avant de commencer à se
dissoudre, ce qui correspond à environ 2\% de la longueur de la
cuve. Dans la mesure où un tel temps de latence surestime certainement
celui des particules dispersées dans le bain d'une cuve d'électrolyse,
on peut raisonnablement négliger l'effet d'un tel phénomène dans le
cadre du modèle proposé dans cette section et poser $\tlat = 0$, ce
qui réduit la complexité de l'implémentation du modèle et le temps de
calcul nécessaire.

Remarquons enfin qu'ici nous avons supposé que l'ensemble des
particules qui constituent une dose d'alumine pénètrent et se
dispersent dans le bain électrolytique. Dans une cuve industrielle, il
peut arriver qu'une partie des particules s'agglomèrent après être
entrée en contact avec le bain, et forme ce qu'on appelle des agrégats
\cite{Dassylva2015}. Ces agrégats mettent un temps, assimilé à un
temps de latence, nettement plus long à se dissoudre et par conséquent
peuvent être transportées par écoulement du bain sur des distance bien
plus grandes. La description formation de ces agrégats et leur
déplacement dans le bain, la modélisation des phénomènes
thermochimiques qui on lieu en leur sein et la dissolution des
particules qui les constituent est un sujet complexe qui sort du cadre
de ce travail, et nous laissons à d'autres le loisir de l'aborder.

Nous nous sommes ensuite intéressé à la dissolution des particules
dans le bain électrolytique en fonction de sa température de
surchauffe $\tsur$. C'est très certainement le paramètre le plus
important du modèle de transport et dissolution d'alumine, et qu'il
faut choisir avec soin. Il est clair que si la température du bain
$\temperature$ est proche de la température de liquidus $\tliq$, une
petite perturbation est suffisante pour $\temperature$ soit
inférieure à $\tliq$. Une telle perturbation intervient lorsque des
particules d'alumine extraient de l'énergie thermique du bain pour
se dissoudre. Dans une telle situation, la dissolution est fortement
ralentie, et les particules sont transportées sur de longues
distance, typiquement plusieurs mètres avant de se dissoudre
complètement. Les particules peuvent alors déposer de la masse sous
forme d'alumine dissoute dans une plus grande région, ce qui contribue
à influencer et modifier significativement la distribution de
l'alumine dissoute dans le bain électrolytique. A l'inverse, plus la
température de surchauffe du bain est élevée, \ie, plus la
température du bain s'écarte de la température du liquidus, plus la
perturbation nécessaire à ce que la température du bain passe
en-dessous de $\tliq$ est importante. A partir d'une certaine limite,
l'énergie nécessaire à réchauffer et dissoudre une dose
d'alumine ne constitue plus une perturbation suffisante. Dans ce cas,
la température ne joue essentiellement plus de rôle dans la
dissolution des particules d'alumine. Cette constatation rejoint les
observation faites sur des cuves industrielles. En effet, lorsque la
température de surchauffe est grande, les doses d'alumine se dissolvent
plus facilement que lorsque la surchauffe est faible. La difficulté
est de trouver un équilibre pour que l'alumine se dissolve
suffisamment vite pour que la concentration dans le bain soit
contrôlable, tout en maintenant une surchauffe aussi faible que
possible pour minimiser les pertes énergétiques.

Au niveau microscopique, la réaction de dissolution des particules est
contrôlé par deux mécanismes. Pour que la dissolution puisse avoir
lieu, il faut d'une part que la concentration d'alumine dissoute à
proximité de la particule soit inférieure à la concentration de
saturation. Il faut d'autre part fournir de l'énergie à la réaction
qui est endothermique. Le paramètre $\tcrit$ contrôle la taille de la
région de transition entre ces deux régimes: lorsque la température
est largement supérieure à $\tcrit$, la vitesse de dissolution est
essentiellement contrôlée par la valeur de la concentration $c$ à
proximité de la particule. Nous avons évalué la sensibilité de la
distribution de concentration d'alumine dissoute dans le bain par
rapport au paramètre $\tcrit$. Clairement, une valeur arbitrairement
grande pour $\tcrit$ bloque complètement la dissolution des
particules. En revanche, lorsque $\tcrit$ est compris dans
l'intervalle $[\tliq, \tliq + 0.86]$, la distribution de concentration
est insensible à la valeur précise de $\tcrit$. Le rôle essentiel de
la température dans ce cas-là est de bloquer la dissolution des
particules lorsque la température $\temperature$ est inférieure à
$\tliq$.

Finalement, nous avons étudié l'effet de la chute gravitationnelle des
particule dans le bain sur la distribution de la concentration
d'alumine dissoute. La chute des particules apportent des variations
notable de la concentration uniquement autour des points d'injection,
qui sont de toute manière soumis à des fluctuations importantes, due
aux injections qui interviennent à intervalles régulier au cours du
cycle global d'injection. De plus, les vitesses de chute des
particules sont très certainement surestimées dans notre
modèle. L'écoulement dans le canal central, dans lequel ont lieu les
injection de particules, est turbulent. Cet écoulement turbulent se
traduit par une viscosité effective ressenties par les particules qui
est certainement supérieure à la viscosité laminaire de
l'électrolyte. Les profondeurs maximales atteintes par les particules
dans ce cas sont alors de l'ordre du millimètre ou inférieur. En
suivant les mêmes arguments qui nous ont conduit à ne pas considérer
de temps de latence, nous pouvons raisonnablement négliger la chute
des particules dans le bain électrolytique et sont effet sur la
concentration de l'alumine.

Dans cette partie, nous avons supposé que la température de liquidus de
l'électrolyte $\tliq$ ne dépend pas de la chimie du bain, et en
particulier de la concentration d'alumine dissoute, que la
concentration de saturation de l'alumine dissoute ne dépend ni de la
chimie du bain ni de la température de celui-ci, et finalement que
l'enthalpie de dissolution $\aluminadissolutionenthalpy$ ainsi que la vitesse
de dissolution $K$ de dépendent pas de la chimie du bain. Ces hypothèses
sont valides pour autant que le système s'écarte peu des conditions
d'exploitations idéales. Ce n'est bien entendu pas le cas lorsqu'on
considère une cuve d'électrolyse industrielle qui subit de nombreuses
perturbations et peut traverser des phases d'instabilité. Un sujet de
recherche futur consiste à prendre en comptes ces différentes
dépendance dans le modèle de transport et dissolution d'alumine dans
le bain électrolytique.


% future work:
% - agregats
% - conditions de bord pour la temperature
% - csat(T, Chimie), Tliq(al2o3, Chimie), K(Chimie),
%   deltaHdiss(Chimie), Chimie
% - Reaction d'electrolyse comme condition de bord,
% - Densite de courant dependant de sigma(al2o3, chimie)
% - Reaction d'electrolyse au bord dependant de \sigma(al2o3, chimie)
%
