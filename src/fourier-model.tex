Soit $\Lambda$ un ouvert borné de $\mathbb R^2$ de bord Lipschitzien
$\partial\Lambda$. Soit $\thickness > 0$ un nombre réel donné
correspondant à la dimension verticale d'une cuve. On définit le
domaine de $\mathbb R^3$ correspondant à une simplification
géométrique de la cuve d'électrolyse
\begin{equation}\label{eq:domain}
  \Omega = \Lambda \times (0,\thickness).
\end{equation}
On suppose que le domaine $\Omega$ est occupé par l'électrolyte, un
fluide newtonien incompressible de viscosité tensorielle $\mu$. On suppose de plus
que le tenseur de viscosité est tel que:
\begin{itemize}
  \item $\mu$ est symétrique, \ie, $\mu_{ij} = \mu_{ji}$, $1\leq
    i,j\leq 3$,
  \item $\exists \chi_0 > 0$ tel que $\mu_{ij} > \chi_0$, $1\leq
    i,j\leq 3$,
    \item $\mu$ est indépendent de $x_3$.
\end{itemize}
Si $u$ est la
vitesse de l'écoulement, le tenseur des contraintes visqueuses dans le
fluide \cite{Landau1987} est donné par
\begin{equation}
  \stresstensor_{i,j}(u) = 2\electrolyteviscosity_{i,j}
  \straintensor_{i,j}(u),\quad i,j = 1,2,3.
\end{equation}
Ici on a noté $\straintensor$ le tenseur du taux de déformation du
fluide qui s'écrit en fonction de la vitesse d'écoulement $u$:
\begin{equation}
  \straintensor_{i,j}(u) = \frac{1}{2}\parent{\frac{\partial u_i}{\partial x_j} + \frac{\partial u_j}{\partial x_i}},\quad i,j = 1,2,3.
\end{equation}
Étant donné un champ de forces $f:\Omega\to \mathbb R^3$, on suppose
que la vitesse d'écoulement $u:\Omega \to \mathbb R^3$ et la pression
$p:\Omega \to \mathbb R$ du fluide satisfont le système de Stokes stationnaire
\begin{align}
  &- \div(\stresstensor(u)) + \nabla p = f,\label{eq:stokes-u}\\
  &\div \parent{u} = 0\label{eq:stokes-p}
\end{align}
dans $\Omega$. Dans la suite nous noterons $u = (u_1, u_2, u_3)$ et $f =
(f_1, f_2, f_3)$. De plus, on demande à ce que l'écoulement $u$ satisfasse
les conditions aux limites suivantes. Sur les faces latérales, la
vitesse satisfait la condition d'adhérence
\begin{equation}
  u = 0\quad \text{ sur } \quad\partial \Lambda\times(0,\thickness).\label{eq:stokes-bc-1}
\end{equation}
Sur les faces horizontales supérieures et inférieures, la vitesse
d'écoulement satisfait une condition de glissement total. On a
\begin{align}
  &u_3(x_1, x_2, x_3) = 0,\label{eq:stokes-bc-2}\\
  &\parent{\stresstensor(u)\cdot \nu}\cdot t_i = 0,\quad i = 1,2\label{eq:stokes-bc-3}
\end{align}
sur $\Lambda \times\cparent{0,\thickness}$, avec $t_1$, $t_2$ les deux
vecteurs tangents unités.
Les conditions (\ref{eq:stokes-bc-2}), (\ref{eq:stokes-bc-3})
correspondent à demander à ce que le fluide ne pénètre pas les faces
horizontales de $\partial\Omega$ et à ce que les contraintes
tangentielles sur ces faces soient nulles. En utilisant
(\ref{eq:stokes-bc-2}), la condition (\ref{eq:stokes-bc-3}) se réécrit
\begin{equation}
\frac{\partial u_1}{\partial x_3} + \frac{\partial u_3}{\partial x_1}
  = 0 \quad \text{et} \quad \frac{\partial u_2}{\partial x_3} + \frac{\partial u_3}{\partial x_2}
  = 0\quad \text{ sur }\quad\partial \Lambda \times\cparent{0,\thickness}.
\end{equation}

\paragraph{Décomposition en séries de Fourier}
Pour résoudre le système d'équations (\ref{eq:stokes-u}),
(\ref{eq:stokes-p}) avec les conditions aux limites (\ref{eq:stokes-bc-1}) à
(\ref{eq:stokes-bc-3}), on exprime les inconnues sous forme de séries
de Fourier dans la direction $x_3$. Soit le domaine $\Omega^+ = \Lambda
\times (-\thickness, \thickness)$, on définit les fonctions suivantes
$\forall (x_1, x_2, x_3)\in \Omega^+$:
\begin{align}
  & U_i(x_1, x_2, x_3) = \left\{
    \begin{array}{lll}
      u_i(x_1, x_2, x_3) &\text{ si } x_3 \geq 0,\\
      u_i(x_1, x_2, -x_3) &\text{ si } x_3 < 0,     & \quad i = 1,2,
    \end{array}
  \right.\label{eq:decomp-1}\\
  %
  & U_3(x_1, x_2, x_3) = \left\{
    \begin{array}{ll}
       u_3(x_1, x_2, x_3) &\text{ si } x_3 \geq 0,\\
      -u_3(x_1, x_2, -x_3) &\text{ si } x_3 < 0,
    \end{array}
  \right.\label{eq:decomp-2}\\
  %
  & P(x_1, x_2, x_3) = \left\{
    \begin{array}{ll}
      p(x_1, x_2, x_3) &\text{ si } x_3 \geq 0,\\
      p(x_1, x_2, -x_3) &\text{ si } x_3 < 0,
    \end{array}
  \right.\label{eq:decomp-3}\\
  %
  & F_i(x_1, x_2, x_3) = \left\{
    \begin{array}{lll}
      f_i(x_1, x_2, x_3) &\text{ si } x_3 \geq 0,\\
      f_i(x_1, x_2, -x_3) &\text{ si } x_3 < 0,     & \quad i = 1,2,
    \end{array}
  \right.\label{eq:decomp-4}\\
  %
  & F_3(x_1, x_2, x_3) = \left\{
    \begin{array}{ll}
      f_3(x_1, x_2, x_3) &\text{ si } x_3 \geq 0\\
     -f_3(x_1, x_2, -x_3) &\text{ si } x_3 < 0.
    \end{array}
  \right.\label{eq:decomp-5}
\end{align}
En vertu de (\ref{eq:stokes-bc-2}) et (\ref{eq:stokes-bc-3}), les
dérivées selon $x_3$ des fonctions $U_1$, $U_2$ et $U_3$ ainsi
définies ne présentent pas de masse de Dirac en $x_3 = 0$. La figure
\ref{fig:solution-periodique} représente schématiquement le
comportement et la régularité de ces fonctions de part et d'autre de
l'origine de l'axe $x_3$.

Avec les définitions ci-dessus on vérifie que $U = (U_1, U_2, U_3)$ et
$P$ satisfont les équations
\begin{align}
  &-\div\parent{\mathcal T(U)} + \nabla P = F,\label{eq:stokes-periodic-1}\\
  &\div\parent{U} = 0\label{eq:stokes-periodic-2}
\end{align}
dans $\Omega^+$. De plus on a
\begin{equation}
U_3 = 0 \quad\text{ sur } \Lambda\times \cparent{-\thickness} \text{ et } \Lambda\times \cparent{\thickness}
\end{equation}
et
\begin{equation}
  \frac{\partial U_i}{\partial x_3} = 0\quad \text{ sur } \Lambda\times
  \cparent{-\thickness} \text{ et } \Lambda\times \cparent{\thickness}
\end{equation}
pour $i = 1,2$, c'est-à-dire que
\begin{equation}
\parent{\stresstensor(U)\cdot \nu}\cdot t_i = 0\quad i = 1,2, \text{ sur } \Lambda\times
\cparent{-\thickness, \thickness}.
\end{equation}

On prolonge toutes les fonctions dans la variable $x_3$ sur tout
$\mathbb R$ en des fonctions périodiques de période $2\thickness$ en
$x_3$. On note encore par $U_1$, $U_2$, $U_3$, $P$, $F_1$, $F_2$,
$F_3$ ces prolongements, c'est-à-dire que ces fonctions sont
considérées comme fonctions de $(x_1, x_2, x_3)\in \Lambda\times
\mathbb R$ et $2\thickness$-périodiques selon $x_3$. Les fonctions
$U_1$, $U_2$, $U_3$ sont $\mathcal C^2$ par morceaux, $P$ est
$\mathcal C^1$ par morceaux et $F_1$, $F_2$, $F_3$ sont $C^0$ par
morceaux.

\begin{figure}[h!]
  \begin{center}
    \input{../media/fourier/solution-periodique/solution-periodique.pdf_tex}
    \caption{Représentation schématique de la réplication des fonction
    $u_i$, $i = 1,2,3$, $p$, et $f_i$, $i = 1,2,3$ sur l'intervalle
      $(0, -\thickness)$. En $x_3 = 0$, les fonctions $U_i$,
      $i=1,2,3$ sont $\mathcal C^1$, $P$, $F_1$, $F_2$ sont $\mathcal
      C^0$, et $F_3$ est discontinue.}
    \label{fig:solution-periodique}
  \end{center}
\end{figure}

On pose pour alléger l'écriture $\beta^k = \frac{\pi
k}{\thickness}$. Les décompositions en séries de Fourier selon la
variable $x_3$ des fonctions $U_i$, $F_i$, $i = 1,2,3$ et $P$ s'écrivent
\begin{align}
  &U_i(x_1, x_2, x_3) = u_i^0(x_1, x_2) %
                       + \sum_{k > 0} u_i^k(x_1, x_2)%
                       \cos\parent{\beta^kx_3}, &\quad i = 1, 2,\label{eq:fourier-coeff-def-1}\\
  &U_3(x_1, x_2, x_3) = \sum_{k > 0} u_3^k(x_1, x_2)%
                        \sin\parent{\beta^kx_3},\label{eq:fourier-coeff-def-2}\\
  &P(x_1, x_2, x_3) = p^0(x_1, x_2) %
                      + \sum_{k>0}p^k(x_1, x_2)%
                      \cos\parent{\beta^kx_3},\label{eq:fourier-coeff-def-3}\\
  &F_i(x_1, x_2, x_3) = f_i^0(x_1, x_2) %
                        + \sum_{k > 0}f_i^k(x_1, x_2)%
                        \cos\parent{\beta^kx_3}, &\quad i = 1,2,\label{eq:fourier-coeff-def-4}\\
  &F_3(x_1, x_2, x_3) = \sum_{k > 0}f_3^k(x_1,x_2)%
                        \sin\parent{\beta^kx_3}.\label{eq:fourier-coeff-def-5}
\end{align}
On note $u^0 = (u^0_1, u^0_2)$, $f^0 = (f^0_1, f^0_2)$, $u^k =
(u^k_1,u^k_2,u^k_3)$ et $f^k = (f^k_1, f^k_2, f^k_3)$, $k \geq 1$. On
obtient les équations que les coefficients de Fourier $u^k$, $p^k$,
$k>0$ doivent satisfaire en substituant les définitions
(\ref{eq:fourier-coeff-def-1}) à (\ref{eq:fourier-coeff-def-5}) dans
le système d'équation de Stokes
(\ref{eq:stokes-periodic-1}),(\ref{eq:stokes-periodic-2}). Les bases
de Fourier $\cparent{\cos(\beta^k x_3)}_{k > 0}$ et
$\cparent{\sin(\beta^kx_3)}_{k>0}$ sur $\mathbb R$ étant
respectivement orthogonales, on peut identifier les équations pour
chaque mode indépendamment des autres. On obtient pour $k = 0$:
\begin{align}
  &-\sum_{j = 1}^2\frac{\partial}{\partial
    x_j}\parent{\mu_{ji}\parent{\frac{\partial u_i^0}{\partial x_j} +
      \frac{\partial u_j^0}{\partial x_i}}} + \frac{\partial
    p^0}{\partial x_i} = f_i^0, \quad i = 1,2,\label{eq:fourier-fund-1}\\
  &\sum_{j = 1}^2 \frac{\partial u_j^0}{\partial x_j} = 0\label{eq:fourier-fund-2}
\end{align}
sur $\Lambda$ et $u^0 = 0$ sur $\partial \Lambda$.

On obtient pour $k
\geq 1$:
\begin{align}
  &-\sum_{j = 1}^2\frac{\partial}{\partial
    x_j}\parent{\mu_{ji}\parent{\frac{\partial u_i^k}{\partial x_j} +
      \frac{\partial u_j^k}{\partial x_i}}} -
  \mu_{3i}\parent{\beta^k\frac{\partial u_3^k}{\partial x_i} -
    \parent{\beta^k}^2 u_i^k} + \frac{\partial p^k}{\partial x_i} = f_i^k,\quad
  i = 1,2,\label{eq:fourier-harm-1}\\
  &-\sum_{j = 1}^2\frac{\partial}{\partial
    x_j}\parent{\mu_{j3}\parent{\frac{\partial u_3^k}{\partial x_j} -
      \beta^k u_j^k}} + 2\mu_{33}\parent{\beta^k}^2 u_3^k - \beta^k p^k =
  f_3^k,\label{eq:fourier-harm-2}\\
  &\sum_{j = 1}^2\frac{\partial u_j^k}{\partial x_j} + \beta^k u_3^k = 0\label{eq:fourier-harm-3}
\end{align}
sur $\Lambda$ et $u^k = 0$ sur $\partial \Lambda$.

Étant données les forces $F_i$, $i = 1,2,3$, les coefficients $f^k$,
$k \geq 0$ qui apparaissent dans les membres de droite des équations
(\ref{eq:fourier-fund-1}), (\ref{eq:fourier-harm-1}) et
(\ref{eq:fourier-harm-2}) s'obtiennent par une projection $\mathrm
L^2$ sur la base de Fourier correspondante. Pour $k = 0$ on a
\begin{align}
  f_i^0(x_1, x_2) &=
  \frac{1}{2\thickness}\int_{-\thickness}^{\thickness}F_i(x_1, x_2, x_3)\,\mathrm
  dx_3, \quad i = 1,2,\nonumber\\
  &= \frac{1}{\thickness}\int_{0}^{\thickness}f_i(x_1, x_2, x_3)\,\mathrm
  dx_3, \quad i = 1,2,\label{eq:f-1}
\end{align}
et pour $k > 0$ on a
\begin{align}
f_i^k(x_1, x_2) &=
\frac{1}{\thickness}\int_{-\thickness}^{\thickness}F_i(x_1, x_2,
x_3)\cos\parent{\beta^k x_3}\,\mathrm dx_3,\quad i =
1,2,\\
&=
\frac{2}{\thickness}\int_{0}^{\thickness}f_i(x_1, x_2,
x_3)\cos\parent{\beta^k x_3}\,\mathrm dx_3,\quad i =
1,2,\label{eq:f-2}\\
f_3^k(x_1, x_2) &= \frac{1}{\thickness}\int_{-\thickness}^{\thickness}
F_3(x_1, x_2,x_3) \sin\parent{\beta^k x_3}\,\mathrm
dx_3,\\
&= \frac{2}{\thickness}\int_{0}^{\thickness}
f_3(x_1, x_2,x_3) \sin\parent{\beta^k x_3}\,\mathrm
dx_3.\label{eq:f-3}
\end{align}


\paragraph{Formulation faible}\label{sec:stokes-fourier-weak}
Nous énonçons maintenant les formulations variationnelles du problème
formé par les équations (\ref{eq:fourier-fund-1}) et
(\ref{eq:fourier-fund-2}) qui correspondent au mode fondamental de
l'écoulement $u$, et des problèmes formés par les équations
(\ref{eq:fourier-harm-1}) à (\ref{eq:fourier-harm-3}) qui
correspondent à chacune des harmoniques de l'écoulement $u$. Pour ce
faire, nous aurons besoin des espaces fonctionnels suivants. L'espace
$L^2(\Lambda)$ est l'ensemble
\begin{align}
  L^2(\Lambda) = \cparent{f:\Lambda\to\mathbb R \mathrel{\Big|} \int_\Lambda
    \abs{f}^2\mathrm dx < \infty}.
\end{align}
Il est muni du produit scalaire
\begin{align}
  (f, g)_{L^2(\Lambda)} = \int_\Lambda f(x)g(x)\mathrm dx
\end{align}
pour tout $f, g\in L^2(\Lambda)$, et de la norme
\begin{align}
  \norm{f}_{L^2(\Lambda)} = \sqrt{(f, f)_{L^2(\Lambda)}}\quad\forall f\in L^2(\Lambda).
\end{align}
On notera encore $H^1(\Lambda)$ l'espace de Sobolev
$W^{1,2}(\Lambda)$, c'est-à-dire que
\begin{align}
  H^1(\Lambda) = \cparent{f\in L^2(\Lambda)\mathrel{\bigg|} \int_\Lambda
    \abs{\nabla f}^2 \mathrm dx < \infty}
\end{align}
et
\begin{align}
  H^1_0(\Lambda) = \cparent{f \in H^1(\Lambda) \mathrel{\bigg|}
    f|_{\partial \Lambda} = 0}.
\end{align}
Nous aurons finalement besoin des espaces de fonctions à moyenne nulle
\begin{align}
  &L^2_0(\Lambda) = \cparent{f\in L^2(\Lambda)\mathrel{\bigg|}
    \int_\Lambda f \,\mathrm dx = 0},\\
  \text{et }\quad &H^1_{0,0}(\Lambda) = \cparent{f\in H^1_0(\Lambda)\mathrel{\bigg|}
    \int_\Lambda f \,\mathrm dx = 0}.
\end{align}
Remarquons que $L^2_0(\Lambda)$ est un sous-espace fermé de
$L^2(\Lambda)$ et $H^1_{0,0}(\Lambda)$ est un sous-espace fermé de
$H^1_0(\Lambda)$ ce qui sera important pour répondre aux questions
d'existence et d'unicité par la suite.

Commençons tout d'abord par traiter le mode fondamental. Pour
simplifier l'écriture, on note $\overline{\mu}$ et
$\reducedstraintensor$ la viscosité et le tenseur du taux de
déformation réduits aux composantes 1 et 2:
\begin{equation}
 \overline{\mu} = \begin{bmatrix}
  \mu_{1,1} & \mu_{1,2} \\
  \mu_{2,1} & \mu_{2,2}
 \end{bmatrix},\quad \text{et} \quad \reducedstraintensor_{ij}(u) =
 \straintensor_{ij}(u),\ i,j = 1,2.
\end{equation}
Le problème faible correspondant aux équations
(\ref{eq:fourier-fund-1}) et (\ref{eq:fourier-fund-2}) consiste à
chercher les fonctions $u_1^0,u_2^0 \in H^1_0(\Lambda)$, $p^0 \in
L^2_0(\Lambda)$, telles que
\begin{align}
  &\int_\Lambda \electrolyteviscosity\otimes \reducedstraintensor(u^0) : \reducedstraintensor(v) \intd{x_1}\intd{x_2} -
  \int_\Lambda p^0\div v \intd{x_1}\intd{x_2} = \int_\Lambda f^0\cdot v
  \intd{x_1}\intd{x_2},\label{eq:fourier-weak-fund-1}\\
  &\int_\Lambda q\div u^0 \intd{x_1}\intd{x_2} = 0,\label{eq:fourier-weak-fund-2}
\end{align}
pour toutes fonctions $v \in \parent{H^1_0(\Lambda)}^2$, $q \in
L_0^2(\Lambda)$. Ici $\otimes$ est le produit tensoriel, \ie,
$\parent{\mu\otimes\reducedstraintensor}_{i,j} =
\mu_{ij}\reducedstraintensor_{ij}$ et $\mu\otimes\reducedstraintensor(u):\reducedstraintensor(v) =
\sum_{i,j = 1}^2\mu_{ij}\reducedstraintensor_{ij}(u)\reducedstraintensor_{ij}(v)$.
En admettant que $\partial \Lambda$ est Lipschitzien et en supposant
que $f^0\in \mathrm L^2(\Lambda)$, il est connu que la formulation
faible (\ref{eq:fourier-weak-fund-1}),(\ref{eq:fourier-weak-fund-2})
admet une unique solution \cite{Temam1977}.

Traitons à présent les problèmes pour les coefficients $u^k$ et
$p^k$. Soit $k > 0$ et soit $\tilde \straintensor^k$ le tenseur $3\times
3$ défini par
\begin{equation}
  \tilde\straintensor^k(u) = \begin{bmatrix}
     2\frac{\partial u_1}{\partial x_1}
    & \frac{\partial u_1}{\partial x_2} + \frac{\partial u_2}{\partial x_1}
    & \frac{\partial u_3}{\partial x_1} - \beta_k u_1 \\
    %
       \frac{\partial u_2}{\partial x_1} + \frac{\partial u_1}{\partial x_2}
    & 2\frac{\partial u_2}{\partial x_2}
    &  \frac{\partial u_3}{\partial x_2} - \beta_k u_2 \\
    %
       \frac{\partial u_3}{\partial x_1} - \beta_k u_1
    &  \frac{\partial u_3}{\partial x_2} - \beta_k u_2
    & 2\beta_k u_3
  \end{bmatrix}
\end{equation}
On obtient la formulation faible pour chaque harmonique en multipliant
les équations (\ref{eq:fourier-harm-1}) à (\ref{eq:fourier-harm-3})
par des fonctions test $v_1$, $v_2$, $v_3$ et $q$ et en intégrant par
parties les termes qui comportent des dérivées secondes. Le problème
faible pour chaque harmonique $k$ consiste à chercher les fonctions
$u^k = (u_1^k,u_2^k,u_3^k)\in \parent{H^1_0(\Lambda)}^3$ et $p^k\in
L^2(\Lambda)$ telles que
\begin{align}
  &\int_\Lambda 2\mu\otimes\tilde\straintensor^k(u^k):\tilde\straintensor^k(v)
  - \int_\Lambda p^k\parent{\frac{\partial v_1}{\partial x_1} +
    \frac{\partial v_2}{\partial x_2} + \beta^k v_3}
  = \int_\Lambda f^k\cdot v,\label{eq:fourier-weak-harm-1}\\
  &\int_\Lambda \parent{\frac{\partial u_1^k}{\partial x_1} + \frac{\partial u_2^k}{\partial x_2} + \beta^k u_3^k}q = 0\label{eq:fourier-weak-harm-2}
\end{align}
pour toutes fonctions $v = (v_1,v_2,v_3)\in \parent{H^1_0(\Lambda)}^3$
et $q\in L^2(\Lambda)$. Ici on adopte les mêmes notations que
précédemment pour le produit tensorielle avec $i,j =
1,2,3$. Remarquons que l'on omet $\intd{x_1}\intd{x_2}$ dans les
intégrales pour simplifier l'écriture. Nous reformulons maintenant ce
problème sous une forme plus adéquate pour l'analyse qui suit.  On
montre en utilisant le théorème de la divergence et en posant $\tilde
p^k = p^k - C^k$ avec $C^k = \abs{\Lambda}^{-1} \int_\Lambda
p^k\intd{x_1}\intd{x_2}$ que le problème
(\ref{eq:fourier-weak-harm-1}),(\ref{eq:fourier-weak-harm-2}) est
équivalent au problème suivant: trouver $u^k\in H^1_0(\Lambda)^3$,
$\tilde p^k \in L^2_0(\Lambda)$ et $C^k\in\mathbb R$ tels que pour
tout $v\in H^1_0(\Lambda)^3$ et $q \in L^2_0(\Lambda)$:
\begin{align}
  &\int_\Lambda 2\mu\otimes\tilde\straintensor^k(u^k):\tilde\straintensor^k(v)
  - \int_\Lambda \tilde p^k\parent{\frac{\partial v_1}{\partial x_1} +
    \frac{\partial v_2}{\partial x_2}} - \int_\Lambda(\tilde p^k +
  C^k)\beta^k v_3
  = \int_\Lambda f^k\cdot v,\label{eq:fourier-weak-harm-v2-1}\\
  &\int_\Lambda \parent{\frac{\partial u_1^k}{\partial x_1} +
    \frac{\partial u_2^k}{\partial x_2} + \beta^k u_3^k}q =
  0,\label{eq:fourier-weak-harm-v2-2}\\
  &\int_\Lambda u_3^k = 0.\label{eq:fourier-weak-harm-v2-3}
\end{align}
Soit maintenant $\psi \in H^1_0(\Lambda)$ tel que $\int_\Lambda
\psi\,\mathrm dx \neq 0$. Alors on peut écrire $H^1_0(\Lambda) =
H^1_{0,0}(\Lambda) \oplus W$, où $W =\mathrm{span}(\psi)$. En
cherchant $u_3^k \in H^1_{0,0}(\Lambda)$ (voir éq. (\ref{eq:fourier-weak-harm-v2-3})), on obtient
le problème que l'on montrera bien posé suivant: trouver $u^k \in
H_0^1(\Lambda)^2 \times H_{0,0}^1(\Lambda)$, $\tilde p^k \in
L^2_0(\Lambda)$ tels que pour tout $v \in H_0^1(\Lambda)^2 \times
H_{0,0}^1(\Lambda)$, $q \in L^2_0(\Lambda)$ on ait
\begin{align}
  &\int_\Lambda 2\mu\otimes\tilde\straintensor^k(u^k):\tilde\straintensor^k(v)
  - \int_\Lambda \tilde p^k\parent{\frac{\partial v_1}{\partial x_1} +
    \frac{\partial v_2}{\partial x_2} + \beta^k v_3}
  = \int_\Lambda f^k\cdot v,\label{eq:fourier-weak-harm-v3-1}\\
  &\int_\Lambda \parent{\frac{\partial u_1^k}{\partial x_1} + \frac{\partial u_2^k}{\partial x_2} + \beta^k u_3^k}q = 0\label{eq:fourier-weak-harm-v3-2}
\end{align}
La constante $C^k$ est unique et peut être calculée a posteriori en
connaissant les fonctions $u^k$ et $\tilde p^k$. Nous reportons la
discussion de ce point à la fin de cette section.

Afin de rendre plus explicite la structure sous-jacente de cette
formulation, on peut l'exprimer en des termes plus abstraits. Notons
$V = H^1_0(\Lambda)^2\times H^1_{0,0}(\Lambda)$. Soit $a^k:V\times V
\to \mathbb R$ les formes bilinéaires continues définies pour tout $k
> 0$ par
\begin{equation}
  a^k(u,v) = \int_\Lambda
  2\mu\otimes\tilde\straintensor^k(u):\tilde\straintensor^k(v)
\end{equation}
et les formes bilinéaires continues $b^k:V\times L^2_0(\Lambda)\to\mathbb
R$ définies pour tout $k > 0$ par
\begin{equation}
  b^k(u, q) = \int_\Lambda \parent{\frac{\partial u_1}{\partial x_1} +
    \frac{\partial u_2}{\partial x_2} + \beta^k u_3}q.
\end{equation}
Ainsi pour $k$ fixé, le problème (\ref{eq:fourier-weak-harm-v3-1}),
(\ref{eq:fourier-weak-harm-v3-2}) est équivalent au problème de
chercher $(u^k,\tilde p^k)\in V \times L^2_0(\Lambda)$ tel que
\begin{align}
  &a^k(u^k,v) - b^k(v,p^k) = \int_\Lambda f^k\cdot v,\label{eq:abstr-weak-harm-1}\\
  & b^k(u^k, q) = 0\label{eq:abstr-weak-harm-2}
\end{align}
pour tout $(v, q)\in V \times L^2_0(\Lambda)$.

\paragraph{Existence d'une solution faible pour les harmoniques}
On donne maintenant une preuve de l'existence et de l'unicité du
problème faible (\ref{eq:abstr-weak-harm-1}),
(\ref{eq:abstr-weak-harm-2}) pour chaque coefficient $(u^k,p^k)$. Dans ce
but, nous introduisons au préalable les deux lemmes suivants.

\begin{lemme}\label{lem:1}
Soit un entier $k > 0$ fixé et une fonction $u \in H^1_0(\Lambda)^3$
telle que $\frac{\partial u_1}{\partial x_1} + \frac{\partial
  u_2}{\partial x_2} + \beta^k u_3 = 0$. On a la relation
\begin{equation}
\int_\Lambda \abs{\tilde\straintensor^k(u)}^2 = \int_\Lambda
\parent{\abs{\reducedstraintensor(u)}^2 +
  \frac{1}{2}\parent{\parent{\frac{\partial u_3}{\partial x_1}}^2 +
    \parent{\frac{\partial u_3}{\partial x_2}}^2 +
    \parent{\beta^k u_1}^2  + \parent{\beta^k u_2}^2}}
\end{equation}
\end{lemme}

\begin{proof}
  Le calcul de $\abs{\tilde\straintensor^k(u)}^2$ une fois
  intégré sur $\Lambda$ donne:
  \begin{align}
    \int_\Lambda \abs{\tilde\straintensor^k(u)}^2 =
    &\int_\Lambda \parent{\abs{\reducedstraintensor(u)}^2
    + \frac{1}{2}\parent{
      \parent{\frac{\partial u_3}{\partial x_1}}^2
      + \parent{\frac{\partial u_3}{\partial x_2}}^2
      + \parent{\beta^k u_1}^2
      + \parent{\beta^k u_2}^2}}\nonumber\\
    &+ \int_\Lambda \beta^k\parent{
      - u_1 \frac{\partial u_3}{\partial x_1}
      - u_2 \frac{\partial u_3}{\partial x_2}
      + \beta^k\parent{u_3}^2
    }.\label{eq:lemme-inter-result}
  \end{align}
  Pour obtenir le résultat souhaité, il reste à voir que le
  dernier terme de (\ref{eq:lemme-inter-result}) est nul.
  En utilisant le théorème de la divergence et les conditions limites
  de $u_3$ sur $\partial \Lambda$ ($u_3 = 0$ sur $\partial \Lambda$) on obtient
  \begin{align}
    \int_\Lambda \beta^k\parent{
      - u_1 \frac{\partial u_3}{\partial x_1}
      - u_2 \frac{\partial u_3}{\partial x_2}
      + \beta^k\parent{u_3}^2} =
    \int_\Lambda\beta^k\parent{
        \frac{\partial u_1}{\partial x_1}u_3
      + \frac{\partial u_2}{\partial x_2}u_3 +
      \beta^k{u_3}^2}\label{eq:lemme-nul-term}
  \end{align}
  En utilisant l'hypothèse que $\frac{\partial u_1}{\partial x_1} +
  \frac{\partial u_2}{\partial x_2} + \beta^k u_3 = 0$ on obtient le
  résultat annoncé.
\end{proof}

\begin{lemme}\label{lem:2}
  Soit un entier $k > 0$ fixé. Il existe une constante positive $\chi >
  0$ telle que
  \begin{equation}
\chi \norm{\nabla u}_{L^2(\Lambda)} \leq
\norm{\tilde\straintensor^k(u)}_{L^2(\Lambda)},
  \end{equation}
  pour tout $u \in H^1_0(\Lambda)^3$ qui satisfait $\frac{\partial
    u_1}{\partial x_1} + \frac{\partial u_2}{\partial x_2} +
  \beta^k u_3 = 0$. Ici on a noté
  \begin{equation}
    \norm{\nabla u}_{L^2(\Lambda)}^2 = \sum_{i,j = 1}^3 \norm{\frac{\partial u_i}{\partial x_j}}^2_{L^2(\Lambda)}
  \end{equation}
  et
  \begin{equation}
    \norm{\tilde\straintensor(u)}^2_{L^2(\Lambda)} = \int_\Lambda \abs{\tilde\straintensor(u)}^2.
  \end{equation}
\end{lemme}

\begin{proof}
Il est connu que l'inégalité de Korn en dimension 2 est vraie, \ie, il
existe une constante $\chi > 0$ qui satisfait
\begin{equation}
\chi\sum_{i,j = 1}^2 \norm{\frac{\partial u_i}{\partial
    x_j}}^2_{L^2(\Lambda)} \leq \int_\Lambda \abs{\reducedstraintensor(u)}^2.
\end{equation}
Le lemme \ref{lem:1} permet de conclure.
\end{proof}

\begin{proposition}\label{prop:2}
  Si le tenseur de viscosité $\mu$ satisfait
\begin{equation}
  \mu_{i,j}(x_1,x_2) \geq \chi_0\quad \forall (x_1, x_2)\in \Lambda,\ 1
  \leq i,j \leq 3,\label{eq:hypothesis}
\end{equation}
où $\chi_0 > 0$ est une constante positive indépendante de $(x_1,
x_2)\in \Lambda$, alors le problème (\ref{eq:abstr-weak-harm-1}),
(\ref{eq:abstr-weak-harm-2}) admet une unique solution.
\end{proposition}

\begin{proof}
  Pour montrer la proposition \ref{prop:2}, il suffit de vérifier
  que la forme $a^k(.,.)$ est coercive sur $V_0$ où $V_0 = \cparent{v
    \in V\mid b^k(v, q) = 0\ \forall q \in
    L^2_0(\Lambda)}$, et que la condition classique inf-sup sur la forme
  bilinéaire $b^k$ est satisfaite.

  Montrons tout d'abord que $\frac{\partial u_1}{\partial x_1} +
  \frac{\partial u_2}{\partial x_2} + \parent{\beta^k}^2 u_3 = 0$,
  c'est-à-dire que
  \begin{equation}
    \int_\Omega\parent{\frac{\partial u_1}{\partial x_1} +
  \frac{\partial u_2}{\partial x_2} + \parent{\beta^k}^2 u_3}q =
    0,\quad \forall q\in L^2(\Lambda).
  \end{equation}
  Remarquons que
  \begin{equation}
    \int_\Omega\parent{\frac{\partial u_1}{\partial x_1} +
      \frac{\partial u_2}{\partial x_2} + \parent{\beta^k}^2 u_3}q =
    b^k(u, q),
  \end{equation}
  et on sait déjà que $b^k(u, q) = 0$ $\forall q \in
  L_0^2(\Lambda)$. De plus, on a que $b^k(u, q) = 0$ lorsque $q = 1$,
  puisque
  \begin{equation}
    \int_\Lambda u_3  = 0 \quad \text{et}\quad \int_\Lambda
    \frac{\partial u_1}{\partial x_1} + \frac{\partial u_2}{\partial
      x_2} = 0.
  \end{equation}

  Ainsi le lemme \ref{lem:1} s'applique, et les lemmes \ref{lem:1} et
  \ref{lem:2} avec l'hypothèse (\ref{eq:hypothesis}) montre bien que
  $a^k$ est coercive sur $V_0$. D'autre part en utilisant l'inégalité
  concernant la condition inf-sup dans $\mathbb R^2$ et si $q\in
  L^2_0(\Lambda)$, on a que
  \begin{align*}
    \sup_{\norm{v}_{H^1_0(\Lambda)^3} = 1} b^k(v,q) = &\sup_{\norm{v}_{H^1_0(\Lambda)^3} = 1} \int_{\Lambda}\parent{\frac{\partial v_1}{\partial x_1} + \frac{\partial v_2}{\partial x_2} + \beta^k v_3}q\\
    \geq & \sup_{\norm{(v_1, v_2, 0)}_{H^1_0(\Lambda)^3} =
      1}\int_{\Lambda}\parent{\frac{\partial v_1}{\partial x_1} +
      \frac{\partial v_2}{\partial x_2}}q\\
    = & \sup_{\norm{(v_1, v_2)}_{H^1_0(\Lambda)^2} = 1}\int_{\Lambda}\parent{\frac{\partial v_1}{\partial x_1} + \frac{\partial v_2}{\partial x_2}}q\\
    \geq& \gamma \norm{q}_{L^2_0(\Lambda)},
  \end{align*}
  où $\gamma > 0$. Ainsi on a bien
  \begin{equation*}
    \inf_{q\in L^2_0(\Lambda)}\sup_{v\in V} \frac{b(v,
      q)}{\norm{q}_{L^2_0(\Lambda)}\norm{v}_{V}} \geq \gamma > 0,
  \end{equation*}
  et la proposition est prouvée.
\end{proof}

Afin d'obtenir le coefficient de Fourier de la pression $p^k$, il
reste à calculer la constante $C^k$ et à montrer son unicité. En
prenant la fonction test $v = (0,0,\psi)$ dans
(\ref{eq:fourier-weak-harm-v2-1}) on obtient
\begin{align}
\int_\Lambda 2\mu\otimes
\tilde\straintensor^k(u^k):\tilde\straintensor^k(v) -
\beta^k\int_\Lambda (\tilde p^k + C^k) = \int_\Lambda f_3 \psi, \label{eq:fourier-pressure-constant}
\end{align}
ce qui permet de calculer $C^k$.

On montre enfin que le calcul de la constante $C^k$ ne dépend pas du
choix de la fonction de base $\psi$ de $W$. En effet, en considérant
une autre fonction $\phi \in H^1_0(\Lambda)$ telle que $\int_\Lambda
\phi \neq 0$ alors $\phi = \bar\phi + \tilde \phi$ où $\tilde \phi \in
H^1_{0,0}(\Lambda)$ et $\bar \phi \in W$ puisque $H^1_0(\Lambda) =
H^1_{0,0} \oplus W$. Cette décomposition implique qu'il existe une
unique constante $\gamma \in \mathbb R$ telle que $\bar \phi = \gamma
\psi$. Si, au lieu de $v = (0,0,\psi)$ on prend $v = (0,0,\phi) =
(0,0,\tilde \phi) + (0,0,\gamma \psi)$ dans
(\ref{eq:fourier-weak-harm-v2-1}), alors $(0,0,\tilde \phi) \in
H^1_0(\Lambda)^2\times H^1_{0,0}(\Lambda)$. Cette fonction test est
comprise dans (\ref{eq:fourier-weak-harm-v3-1}). D'autre part
$(0,0,\gamma \psi)$ donne la même constante $C^k$ que la relation
(\ref{eq:fourier-pressure-constant}).
