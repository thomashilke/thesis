Soit $\Lambda$ un ouvert borné de $\mathbb R^2$ de bord
$\partial\Lambda$. Soit $\thickness > 0$ un nombre réel donné
correspondant à la dimension verticale d'une cuve. On définit le
domaine de $\mathbb R^3$ correspondant à une simplification
géométrique de la cuve d'électrolyse
\begin{equation}\label{eq:domain}
  \Omega = \Lambda \times (0,\thickness).
\end{equation}
On suppose que le domaine $\Omega$ est occupé par l'électrolyte, un
fluide newtonien incompressible de viscosité $\mu$. Si $u$ est la
vitesse de l'écoulement, le tenseur des contraintes visqueuses dans le
fluide \cite{Landau1987} est donné par
\begin{equation}
  \stresstensor_{i,j}(u) = 2\electrolyteviscosity_{i,j}
  \straintensor_{i,j}(u),\quad i,j = 1,2,3.
\end{equation}
Ici on a noté $\straintensor$ le tenseur du taux de déformation du
fluide qui s'écrit en fonction de la vitesse d'écoulement $u$:
\begin{equation}
  \straintensor_{i,j}(u) = \frac{1}{2}\parent{\frac{\partial u_i}{\partial x_j} + \frac{\partial u_j}{\partial x_i}},\quad i,j = 1,2,3.
\end{equation}
Étant donné un champ de forces $f:\Omega\to \mathbb R^3$, on suppose
que la vitesse d'écoulement $u:\Omega \to \mathbb R^3$ et la pression
$p:\Omega \to \mathbb R$ du fluide satisfont le système de Stokes
\begin{align}
  &- \div(\stresstensor(u)) + \nabla p = f,\label{eq:stokes-u}\\
  &\div \parent{u} = 0\label{eq:stokes-p}
\end{align}
dans $\Omega$. Dans la suite nous noterons $u = (u_1, u_2, u_3)$ et $f =
(f_1, f_2, f_3)$. De plus, on demande à ce que l'écoulement $u$ satisfasse
les conditions aux limites suivantes. Sur les faces latérales, la
vitesse satisfait la condition d'adhérence
\begin{equation}
  u = 0\quad \text{ sur } \quad\partial \Lambda\times(0,\thickness).\label{eq:stokes-bc-1}
\end{equation}
Sur les faces horizontales supérieures et inférieures, la vitesse
d'écoulement satisfait une condition de glissement total. On a
\begin{align}
  &u_3(x_1, x_2, x_3) = 0,\label{eq:stokes-bc-2}\\
  &\stresstensor(u)\cdot \nu = 0\label{eq:stokes-bc-3}
\end{align}
sur $\Lambda \times\cparent{0,\thickness}$.
Les conditions (\ref{eq:stokes-bc-2}), (\ref{eq:stokes-bc-3})
correspondent à demander à ce que le fluide ne pénètre pas les faces
horizontales de $\partial\Omega$ et à ce que le flux de quantité de
mouvement à travers ces faces soit nul. En utilisant
(\ref{eq:stokes-bc-2}), la condition (\ref{eq:stokes-bc-3}) se réécrit
\begin{equation}
\frac{\partial u_1}{\partial x_3}
  = 0 \quad \text{et} \quad \frac{\partial u_2}{\partial x_3}
  = 0\quad \text{ sur }\quad\partial \Lambda \times\cparent{0,\thickness}.
\end{equation}

\paragraph{Décomposition en séries de Fourier}
Pour résoudre le système d'équations (\ref{eq:stokes-u}),
(\ref{eq:stokes-p}) avec les conditions aux limites (\ref{eq:stokes-bc-1}) à
(\ref{eq:stokes-bc-3}), on exprime les inconnues sous forme de séries
de Fourier dans la direction $x_3$. Soit le domaine $\Omega^+ = \Lambda
\times (-\thickness, \thickness)$, on définit les fonctions suivantes
$\forall (x_1, x_2, x_3)\in \Omega^+$:
\begin{align}
  & U_i(x_1, x_2, x_3) = \left\{
    \begin{array}{lll}
      u_i(x_1, x_2, x_3) &\text{ si } x_3 \geq 0,\\
      u_i(x_1, x_2, -x_3) &\text{ si } x_3 < 0,     & \quad i = 1,2,
    \end{array}
  \right.\label{eq:decomp-1}\\
  %
  & U_3(x_1, x_2, x_3) = \left\{
    \begin{array}{ll}
       u_3(x_1, x_2, x_3) &\text{ si } x_3 \geq 0,\\
      -u_3(x_1, x_2, -x_3) &\text{ si } x_3 < 0,
    \end{array}
  \right.\label{eq:decomp-2}\\
  %
  & P(x_1, x_2, x_3) = \left\{
    \begin{array}{ll}
      p(x_1, x_2, x_3) &\text{ si } x_3 \geq 0,\\
      p(x_1, x_2, -x_3) &\text{ si } x_3 < 0,
    \end{array}
  \right.\label{eq:decomp-3}\\
  %
  & F_i(x_1, x_2, x_3) = \left\{
    \begin{array}{lll}
      f_i(x_1, x_2, x_3) &\text{ si } x_3 \geq 0,\\
      f_i(x_1, x_2, -x_3) &\text{ si } x_3 < 0,     & \quad i = 1,2,
    \end{array}
  \right.\label{eq:decomp-4}\\
  %
  & F_3(x_1, x_2, x_3) = \left\{
    \begin{array}{ll}
      f_3(x_1, x_2, x_3) &\text{ si } x_3 \geq 0\\
     -f_3(x_1, x_2, -x_3) &\text{ si } x_3 < 0.
    \end{array}
  \right.\label{eq:decomp-5}
\end{align}
En vertu de (\ref{eq:stokes-bc-2}) et (\ref{eq:stokes-bc-3}), les
dérivées selon $x_3$ des fonctions $U_1$, $U_2$ et $U_3$ ainsi définies ne
présentent pas de masse de Dirac en $x_3 = 0$.

Avec les définitions ci-dessus on vérifie que $U = (U_1, U_2, U_3)$ et
$P$ satisfont les équations
\begin{align}
  &-\div\parent{\mathcal T(U)} + \nabla P = F,\label{eq:stokes-periodic-1}\\
  &\div\parent{U} = 0\label{eq:stokes-periodic-2}
\end{align}
dans $\Omega^+$. De plus on a
\begin{equation}
U_3 = 0 \quad\text{ sur } \Lambda\times \cparent{-\thickness} \text{ et } \Lambda\times \cparent{\thickness}
\end{equation}
et
\begin{equation}
  \frac{\partial U_i}{\partial x_3} = 0\quad \text{ sur } \Lambda\times
  \cparent{-\thickness} \text{ et } \Lambda\times \cparent{\thickness}
\end{equation}
pour $i = 1,2$, c'est-à-dire que
\begin{equation}
\stresstensor(U)\cdot \nu = 0\quad \text{ sur } \Lambda\times
\cparent{-\thickness, \thickness}.
\end{equation}

On prolonge toutes les fonctions dans la variable $x_3$ sur tout
$\mathbb R$ en des fonctions périodiques de période $2\thickness$. On
note encore par $U_1$, $U_2$, $U_3$, $P$, $F_1$, $F_2$, $F_3$ ces
prolongements, c'est-à-dire que ces fonctions sont considérées comme
fonctions de $(x_1, x_2, x_3)\in \Lambda\times \mathbb R$ et
$2\thickness$-périodiques selon $x_3$. Les fonctions $U_1$, $U_2$,
$U_3$ sont $\mathcal C^2$ par morceaux, $P$ est $\mathcal C^1$ par
morceaux et $F_1$, $F_2$, $F_3$ sont $C^0$ par morceaux.

On pose pour alléger l'écriture $\beta^k = \frac{\pi
k}{\thickness}$. Les décompositions en séries de Fourier selon la
variable $x_3$ des fonctions $U_i$, $F_i$, $i = 1,2,3$ et $P$ s'écrivent
\begin{align}
  &U_i(x_1, x_2, x_3) = u_i^0(x_1, x_2) %
                       + \sum_{k > 0} u_i^k(x_1, x_2)%
                       \cos\parent{\beta^kx_3}, &\quad i = 1, 2,\label{eq:fourier-coeff-def-1}\\
  &U_3(x_1, x_2, x_3) = \sum_{k > 0} u_3^k(x_1, x_2)%
                        \sin\parent{\beta^kx_3},\label{eq:fourier-coeff-def-2}\\
  &P(x_1, x_2, x_3) = p^0(x_1, x_2) %
                      + \sum_{k>0}p^k(x_1, x_2)%
                      \cos\parent{\beta^kx_3},\label{eq:fourier-coeff-def-3}\\
  &F_i(x_1, x_2, x_3) = f_i^0(x_1, x_2) %
                        + \sum_{k > 0}f_i^k(x_1, x_2)%
                        \cos\parent{\beta^kx_3}, &\quad i = 1,2,\label{eq:fourier-coeff-def-4}\\
  &F_3(x_1, x_2, x_3) = \sum_{k > 0}f_3^k(x_1,x_2)%
                        \sin\parent{\beta^kx_3}.\label{eq:fourier-coeff-def-5}
\end{align}
On note $u^0 = (u^0_1, u^0_2)$, $f^0 = (f^0_1, f^0_2)$, $u^k =
(u^k_1,u^k_2,u^k_3)$ et $f^k = (f^k_1, f^k_2, f^k_3)$, $k \geq 1$. On
obtient les équations que les coefficients de Fourier $u^k$, $p^k$,
$k>0$ doivent satisfaire en substituant les définitions
(\ref{eq:fourier-coeff-def-1}) à (\ref{eq:fourier-coeff-def-5}) dans
le système d'équation de Stokes
(\ref{eq:stokes-periodic-1}),(\ref{eq:stokes-periodic-2}). Les bases
de Fourier $\cparent{\cos(\beta^k x_3)}_{k > 0}$ et
$\cparent{\sin(\beta^kx_3)}_{k>0}$ sur $\mathbb R$ étant
respectivement orthogonales, on peut identifier les équations pour
chaque mode indépendamment des autres. On obtient pour $k = 0$:
\begin{align}
  &-\sum_{j = 1}^2\frac{\partial}{\partial
    x_j}\parent{\mu_{ji}\parent{\frac{\partial u_i^0}{\partial x_j} +
      \frac{\partial u_j^0}{\partial x_i}}} + \frac{\partial
    p^0}{\partial x_i} = f_i^0, \quad i = 1,2,\label{eq:fourier-fund-1}\\
  &\sum_{j = 1}^2 \frac{\partial u_j^0}{\partial x_j} = 0\label{eq:fourier-fund-2}
\end{align}
sur $\Lambda$ et $u^0 = 0$ sur $\partial \Lambda$. On obtient pour $k
\geq 1$:
\begin{align}
  &-\sum_{j = 1}^2\frac{\partial}{\partial
    x_j}\parent{\mu_{ji}\parent{\frac{\partial u_i^k}{\partial x_j} +
      \frac{\partial u_j^k}{\partial x_i}}} -
  \mu_{3i}\parent{\beta^k\frac{\partial u_3^k}{\partial x_i} -
    \parent{\beta^k}^2 u_i^k} + \frac{\partial p^k}{\partial x_i} = f_i^k,\quad
  i = 1,2,\label{eq:fourier-harm-1}\\
  &-\sum_{j = 1}^2\frac{\partial}{\partial
    x_j}\parent{\mu_{j3}\parent{\frac{\partial u_3^k}{\partial x_j} -
      \beta^k u_j^k}} + 2\mu_{33}\parent{\beta^k}^2 u_3^k - \beta^k p^k =
  f_3^k,\label{eq:fourier-harm-2}\\
  &\sum_{j = 1}^2\frac{\partial u_j^k}{\partial x_j} + \beta^k u_3^k = 0\label{eq:fourier-harm-3}
\end{align}
sur $\Lambda$ et $u^k = 0$ sur $\partial \Lambda$.

Étant données les forces $F_i$, $i = 1,2,3$, les coefficients $f^k$,
$k \geq 0$ qui apparaissent dans les membres de droite des équations
(\ref{eq:fourier-fund-1}), (\ref{eq:fourier-harm-1}) et
(\ref{eq:fourier-harm-2}) s'obtiennent par une projection $\mathrm
L^2$ sur la base de Fourier correspondante. Pour $k = 0$ on a
\begin{align}
  f_i^0(x_1, x_2) =
  \frac{1}{2\thickness}\int_{-\thickness}^{\thickness}F_i(x_1, x_2, x_3)\,\mathrm
  dx_3, \quad i = 1,2,\label{eq:f-1}
\end{align}
et pour $k > 0$ on a
\begin{align}
&f_i^k(x_1, x_2) =
\frac{1}{\thickness}\int_{-\thickness}^{\thickness}F_i(x_1, x_2,
x_3)\cos\parent{\beta^k x_3}\,\mathrm dx_3,\quad i = 1,2,\label{eq:f-2}\\
&f_3^k(x_1, x_2) = \frac{1}{\thickness}\int_{-\thickness}^{\thickness}
F_3(x_1, x_2,x_3) \sin\parent{\beta^k x_3}\,\mathrm dx_3.\label{eq:f-3}
\end{align}

%%\begin{align}
%%  &- \mu\frac{\partial^2 u_1^k}{\partial x_1^2} - \mu\frac{\partial^2 u_1^k}{\partial x_2^2}%
%%  + \mu\parent{\beta^k}^2 u_1^k%
%%  + \frac{\partial p^k}{\partial x_1} %
%%  = f_1^k, \label{eq:fourier-harm-1}\\
%%  %
%%  &- \mu\frac{\partial^2 u_2^k}{\partial x_1^2} - \mu\frac{\partial^2 u_2^k}{\partial x_2^2} %
%%  + \mu\parent{\beta^k}^2 u_2^k %
%%  + \frac{\partial p_k}{\partial x_2} %
%%  = f_2^k, \label{eq:fourier-harm-2} \\
%%  %
%%  &- \mu\frac{\partial^2 u_3^k}{\partial x_1^2} - \mu\frac{\partial^2 u_3^k}{\partial x_2^2} %
%%  + \mu\parent{\beta^k}^2 u_3^k %
%%  - \beta^kp^k %
%%  = f_3^k, \label{eq:fourier-harm-3}\\
%%  %
%%  &  \frac{\partial u_1^k}{\partial x_1} %
%%  + \frac{\partial u_2^k}{\partial x_2} %
%%  + \beta^k u_3^k %
%%  = 0. \label{eq:fourier-incompressibility}
%%\end{align}
%%Les problèmes ci-dessus sont à résoudre dans $\Lambda \subset
%%\mathbb R^2$. Naturellement ils nécessitent les conditions limites
%%d'adhérence
%%\begin{equation*}
%%u_1^0 = u_2^0 = 0 \quad\text{ sur }\partial \Lambda
%%\end{equation*}
%%et
%%\begin{equation*}
%% u_1^k = u_2^k = u_3^k = 0 \quad\text{ sur }\partial \Lambda
%%\end{equation*}
%%pour tout $k \geq 1$, et le calcul au préalable des coefficients
%%de Fourier du champ de force:
%%\begin{align}
%%  f_i^0(x_1, x_2) %
%%  &= \frac{1}{2\thickness}\int_0^{2\thickness}f_i(x_1, x_2, x_3)%
%%  \,\mathrm dx_3,& i = 1,2,\label{eq:force-coefficient-1}\\
%%  %
%%  f_i^k(x_1, x_2)
%%\end{align}
%%et pour $k > 0$,
%%\begin{align}
%%  &= \frac{1}{\thickness}\int_0^{2\thickness}f_i(x_1, x_2, x_3)%
%%  \cos\parent{\beta^kx_3}\,\mathrm dx_3,& i = 1,2,\label{eq:force-coefficient-2}\\
%%  %
%%  f_3^k(x_1, x_2)%
%%  &= \frac{1}{\thickness}\int_0^{2\thickness}f_3(x_1, x_2, x_3)%
%%  \sin\parent{\beta^kx_3}\,\mathrm dx_3.\label{eq:force-coefficient-3}
%%\end{align}

\paragraph{Formulation faible}\label{sec:stokes-fourier-weak}
Nous énonçons maintenant les formulations variationnelles du problème
formé par les équations (\ref{eq:fourier-fund-1}) et
(\ref{eq:fourier-fund-2}) qui correspondent au mode fondamental de
l'écoulement $u$, et des problèmes formés par les équations
(\ref{eq:fourier-harm-1}) à (\ref{eq:fourier-harm-3}) qui
correspondent à chacune des harmoniques de l'écoulement $u$. Pour ce
faire, nous aurons besoins des espaces fonctionnels suivants. L'espace
$L^2(\Lambda)$ est l'ensemble
\begin{align}
  L^2(\Lambda) = \cparent{f:\Lambda\to\mathbb R \mathrel{\Big|} \int_\Lambda
    \abs{f}^2\mathrm dx < \infty}.
\end{align}
Il est muni du produit scalaire
\begin{align}
  (f, g)_{L^2(\Lambda)} = \int_\Lambda f(x)g(x)\mathrm dx
\end{align}
pour tout $f, g\in L^2(\Lambda)$, et de la norme
\begin{align}
  \norm{f}_{L^2(\Lambda)} = \sqrt{(f, f)}\quad\forall f\in L^2(\Lambda).
\end{align}
On définit encore
\begin{align}
  L^2_0(\Lambda) = \cparent{f\in L^2(\Lambda)\mathrel{\bigg|} f(x) = 0 \forall
    x\in \partial\Lambda},\\
  H^1(\Lambda) = \cparent{f\in L^2(\Lambda)\mathrel{\bigg|} \int_\Lambda
    \abs{\nabla f}^2 \mathrm dx < \infty}
\end{align}
et
\begin{align}
  H^1_0(\Lambda) = H^1(\Lambda) \cap L^2_0.
\end{align}

Commençons tout d'abord par traiter le mode fondamental. Pour
simplifier l'écriture, on note $\overline{\mu}$ et
$\reducedstraintensor$ la viscosité et le tenseur du taux de
déformation réduits aux composantes 1 et 2:
\begin{equation}
 \overline{\mu} = \begin{bmatrix}
  \mu_{1,1} & \mu_{1,2} \\
  \mu_{2,1} & \mu_{2,2}
 \end{bmatrix},\quad \text{et} \quad \reducedstraintensor_{ij}(u) =
 \straintensor_{ij}(u),\ i,j = 1,2.
\end{equation}
Le problème faible correspondant aux équations
(\ref{eq:fourier-fund-1}) et (\ref{eq:fourier-fund-2}) consiste à
chercher les fonctions $u_1^0,u_2^0 \in H^1(\Lambda)$, $p^0 \in
L^2_0(\Lambda)$, telles que
\begin{align}
  &\int_\Lambda \electrolyteviscosity\otimes \reducedstraintensor(u^0) : \reducedstraintensor(v) \intd{x_1}\intd{x_2} -
  \int_\Lambda p^0\div v \intd{x_1}\intd{x_2} = \int_\Lambda f^0\cdot v
  \intd{x_1}\intd{x_2},\label{eq:fourier-weak-fund-1}\\
  &\int_\Lambda q\div u \intd{x_1}\intd{x_2} = 0\label{eq:fourier-weak-fund-2}
\end{align}
pour toutes fonctions $v \in \parent{H^1(\Lambda)}^2$, $q \in
L_0^2(\Lambda)$. En admettant une régularité adéquate pour le bord
$\partial \Lambda$ et en supposant que $f^0\in \mathrm L^2(\Lambda)$,
il est connu que la formulation faible
(\ref{eq:fourier-weak-fund-1}),(\ref{eq:fourier-weak-fund-2}) admet une unique
solution \cite{Temam1977}.

Traitons à présent les problèmes pour les coefficients $u^k$ et
$p^k$. Soit $k > 0$ et soit $\tilde \straintensor$ le tenseur $3\times
3$ définit par
\begin{equation}
  \tilde\straintensor^k(u) = \begin{bmatrix}
     2\frac{\partial u_1}{\partial x_1}
    & \frac{\partial u_1}{\partial x_2} + \frac{\partial u_2}{\partial x_1}
    & \frac{\partial u_3}{\partial x_1} - \beta_k u_1 \\
    %
       \frac{\partial u_2}{\partial x_1} + \frac{\partial u_1^k}{\partial x_2}
    & 2\frac{\partial u_2}{\partial x_2}
    &  \frac{\partial u_3}{\partial x_2} - \beta_k u_2 \\
    %
       \frac{\partial u_3}{\partial x_1} - \beta_k u_1
    &  \frac{\partial u_3}{\partial x_2} - \beta_k u_2
    & 2\beta_k u_3
  \end{bmatrix}
\end{equation}
On obtient la formulation faible pour chaque harmonique en multipliant
les équations (\ref{eq:fourier-harm-1}) à (\ref{eq:fourier-harm-3})
par des fonctions test $v_1$, $v_2$, $v_3$ et $q$ et en intégrant par
parties les termes qui comportent des dérivées secondes. Le problème
faible pour chaque harmonique $k$ consiste à chercher les fonctions
$u^k = (u_1^k,u_2^k,u_3^k)\in \parent{H^1_0(\Lambda)}^3$ et $p^k\in
L^2(\Lambda)$ telles que
\begin{align}
  &\int_\Lambda 2\mu\otimes\tilde\straintensor^k(u^k):\tilde\straintensor^k(v)
  - \int_\Lambda p^k\parent{\frac{\partial v_1}{\partial x_1} +
    \frac{\partial v_2}{\partial x_2} + \beta^k v_3}
  = \int_\Lambda f^k\cdot v,\label{eq:fourier-weak-harm-1}\\
  &\int_\Lambda \parent{\frac{\partial u_1^k}{\partial x_1} + \frac{\partial u_2^k}{\partial x_2} + \beta^k u_3^k}q = 0\label{eq:fourier-weak-harm-2}
\end{align}
pour toutes fonctions $v = (v_1,v_2,v_3)\in \parent{H^1_0(\Lambda)}^3$ et $q\in
L^2(\Lambda)$. Afin de rendre plus explicite la structure sous-jacente
de cette formulation, on peut l'exprimer en des termes plus
abstraits. Soit $a^k:H^1_0(\Lambda)^3\times H^1_0(\Lambda)^3 \to \mathbb
R$ les formes bilinéaires continues définies pour tout $k > 0$ par
\begin{equation}
  a^k(u,v) = \int_\Lambda
  2\mu\otimes\tilde\straintensor^k(u):\tilde\straintensor^k(v)
\end{equation}
et les formes bilinéaires continues $b^k:H^1_0(\Lambda)^3\times L^2(\Lambda)\to\mathbb
R$ définies pour tout $k > 0$ par
\begin{equation}
  b^k(u, q) = \int_\Lambda \parent{\frac{\partial u_1}{\partial x_1} +
    \frac{\partial u_2}{\partial x_2} + \beta^k u_3}q.
\end{equation}
Ainsi pour $k$ fixé, le problème (\ref{eq:fourier-weak-harm-1}),
(\ref{eq:fourier-weak-harm-2}) est équivalent au problème de
chercher $(u^k,p^k)\in H^1_0(\Lambda)^3 \times L^2(\Lambda)$ tel que
\begin{align}
  &a^k(u^k,v) - b^k(v,p^k) = \int_\Lambda f^k\cdot v,\label{eq:abstr-weak-harm-1}\\
  & b^k(u^k, q) = 0\label{eq:abstr-weak-harm-2}
\end{align}
pour tout $(v, q)\in H^1_0(\Lambda)^3 \times L^2(\Lambda)$.

\paragraph{Existence d'une solution faible pour les harmoniques}
On donne maintenant une preuve de l'existence et de l'unicité du
problème faible (\ref{eq:abstr-weak-harm-1}),
(\ref{eq:abstr-weak-harm-2}) pour chaque coefficient $(u^k,p^k)$. Dans ce
but, nous introduisons au préalable les deux lemmes suivants.

\begin{lemme}\label{lem:1}
Soit un entier $k > 0$ fixé et une fonction $u \in H^1_0(\Lambda)^3$
telle que $\frac{\partial u_1}{\partial x_1} + \frac{\partial
  u_2}{\partial x_2} + \beta^k u_3 = 0$. On a la relation
\begin{equation}
\int_\Lambda \abs{\tilde\straintensor^k(u)}^2 = \int_\Lambda
\parent{\abs{\reducedstraintensor(u)}^2 +
  \frac{1}{2}\parent{\parent{\frac{\partial u_3}{\partial x_1}}^2 +
    \parent{\frac{\partial u_3}{\partial x_2}}^2 +
    \parent{\beta^k u_1}^2  + \parent{\beta^k u_2}^2}}
\end{equation}
\end{lemme}

\begin{proof}
  Le calcul de $\abs{\tilde\straintensor^k(u)}^2$ une fois
  intégré sur $\Lambda$ donne:
  \begin{align}
    \int_\Lambda \abs{\tilde\straintensor^k(u)}^2 =
    &\int_\Lambda \parent{\abs{\reducedstraintensor(u)}^2
    + \frac{1}{2}\parent{
      \parent{\frac{\partial u_3}{\partial x_1}}^2
      + \parent{\frac{\partial u_3}{\partial x_2}}^2
      + \parent{\beta^k u_1}^2
      + \parent{\beta^k u_2}^2}}\nonumber\\
    &+ \int_\Lambda \beta^k\parent{
      - u_1 \frac{\partial u_3}{\partial x_1}
      - u_2 \frac{\partial u_3}{\partial x_2}
      + \beta^k\parent{u_3}^2
    }.\label{eq:lemme-inter-result}
  \end{align}
  Pour obtenir le résultat souhaité, il reste à voir que le
  dernier terme de (\ref{eq:lemme-inter-result}) est nul.
  En utilisant le théorème de la divergence on obtient
  \begin{align}
    \int_\Lambda \beta^k\parent{
      - u_1 \frac{\partial u_3}{\partial x_1}
      - u_2 \frac{\partial u_3}{\partial x_2}
      + \beta^k\parent{u_3}^2} =
    &\int_\Lambda\beta^k\parent{
        \frac{\partial u_1}{\partial x_1}u_3
      + \frac{\partial u_2}{\partial x_2}u_3 +
      \beta^k{u_3}^2}\\
    &+ \int_{\partial \Lambda} u_3\parent{u_1,
      u_2}^t\cdot\mathrm dl\label{eq:lemme-nul-term}
  \end{align}
  En utilisant l'hypothèse que $\frac{\partial u_1}{\partial x_1} +
  \frac{\partial u_2}{\partial x_2} + \beta^k u_3 = 0$ et les
  conditions aux limites de Dirichlet homogènes $u = 0$ sur
  $\partial \Lambda$ on conclut que l'expression
  (\ref{eq:lemme-nul-term}) est identiquement nul et on obtient le
  résultat annoncé.
\end{proof}

\begin{lemme}\label{lem:2}
  Soit un entier $k > 0$ fixé. Il existe une constante positive $\chi >
  0$ telle que
  \begin{equation}
\chi \norm{\nabla u}_{L^2(\Lambda)} \leq
\norm{\tilde\straintensor^k(u)}_{L^2(\Lambda)},
  \end{equation}
  pour tout $u \in H^1_0(\Lambda)^3$ qui satisfait $\frac{\partial
    u_1}{\partial x_1} + \frac{\partial u_2}{\partial x_2} +
  \beta^k u_3 = 0$. Ici on a noté
  \begin{equation}
    \norm{\nabla u}_{L^2(\Lambda)}^2 = \sum_{i,j = 1}^3 \norm{\frac{\partial u_i}{\partial x_j}}^2
  \end{equation}
  et
  \begin{equation}
    \norm{\tilde\straintensor(u)}^2_{L^2(\Lambda)} = \int_\Lambda \abs{\tilde\straintensor(u)}^2.
  \end{equation}
\end{lemme}

\begin{proof}
Il est connu que l'inégalité de Korn en dimension 2 est vraie, \ie, il
existe une constante $\chi > 0$ qui satisfait
\begin{equation}
\chi\sum_{i,j = 1}^2 \norm{\frac{\partial u_i}{\partial
    x_j}}^2_{L^2(\Lambda)} \leq \int_\Lambda \abs{\reducedstraintensor(u)}^2.
\end{equation}
Le lemme \ref{lem:1} permet de conclure.
\end{proof}

\begin{proposition}\label{prop:2}
  Si le tenseur de viscosité $\mu$ satisfait
\begin{equation}
  \mu_{i,j}(x_1,x_2) \geq \mu_0\quad \forall (x_1, x_2)\in \Lambda,\ 1
  \leq i,j \leq 3,\label{eq:hypothesis}
\end{equation}
où $\mu_0 > 0$ est une constante positive indépendante de $(x_1,
x_2)\in \Lambda$, alors le problème (\ref{eq:abstr-weak-harm-1}),
(\ref{eq:abstr-weak-harm-2}) admet une unique solution.
\end{proposition}

\begin{proof}
  Pour montrer la proposition \ref{prop:2}, il suffit de vérifier
  que la forme $a^k(.,.)$ est coercive sur $V_0$ où $V_0 = \cparent{v
    \in H_0^1(\Lambda)^3\mid b^k(v, q) = 0\ \forall q \in
    L^2(\Lambda)}$, et que la condition classique inf-sup sur la forme
  bilinéaire $b$ est satisfaite.

  Le lemme \ref{lem:2} avec l'hypothèse (\ref{eq:hypothesis}) montre
  bien que $a^k$ est coercive sur $V_0$. D'autre part en utilisant
  l'inégalité concernant la condition inf-sup dans $\mathbb R^2$ et si
  $q\in L^2(\Lambda)$, on a que
  \begin{align*}
    \sup_{\norm{v}_{H^1_0} = 1} b^k(v,q)
    = \sup_{\norm{v}_{H^1_0} = 1} \int_{\Lambda}\parent{\frac{\partial v_1}{\partial x_1} + \frac{\partial v_2}{\partial x_2} + \beta^k v_3}
    \geq & \sup_{\norm{v}_{H^1_0} = 1}\int_{\Lambda}\parent{\frac{\partial v_1}{\partial x_1} + \frac{\partial v_2}{\partial x_2}}\\
    &\geq \gamma \norm{q}_{L^2},
  \end{align*}
  où $\gamma > 0$, et où on a noté $\norm{v}_{H_0^1}^2 = \sum_{i =
    1}^3 \norm{v_i}_{H_0^2}^2$. Ainsi on a bien
  \begin{equation}
    \inf_{q\in L^2}\sup_{v\in (H_0^1)^3} \frac{b(v,
      q)}{\norm{q}_{L^2}\norm{v}_{H^1_0}} \geq \gamma > 0,
  \end{equation}
  et la proposition est prouvée.
\end{proof}

%%Deuxièmement, pour tout $k = 1, 2, \dots$, nous cherchons les
%%fonctions $u_{1,k},u_{2,k},u_{3,k} \in H^1(\Lambda)$, $p_k \in
%%L_0^2(\Lambda)$ et les nombres réels $C_k\in\mathbb R$ tels que
%%\begin{align}
%%  &\electrolyteviscosity \int_\Lambda \parent{\nabla u_{1,k}\cdot \nabla v_1 %
%%                           + \parent{\beta^k}^2 u_{1,k}v_1}\,\mathrm dx_1\mathrm dx_2 %
%%  - \int_\Lambda p_k\frac{\partial v_1}{\partial x_1}\,\mathrm dx_1\mathrm dx_2 %
%%  = \int_\Lambda f_{1,k}v_1\mathrm dx_1\mathrm dx_2, \label{eq:stokes-fourier-weak-1}\\
%%  %
%%  &\electrolyteviscosity \int_\Lambda \parent{\nabla u_{2,k}\cdot \nabla v_2 %
%%                           + \parent{\beta^k}^2 u_{2,k}v_2}\,\mathrm dx_1\mathrm dx_2 %
%%  - \int_\Lambda p_k\frac{\partial v_2}{\partial x_2}\,\mathrm dx_1\mathrm dx_2 %
%%  = \int_\Lambda f_{2,k}v_1\mathrm dx_1\mathrm dx_2, \label{eq:stokes-fourier-weak-2}\\
%%  %
%%  &\electrolyteviscosity \int_\Lambda \parent{\nabla u_{3,k}\cdot \nabla v_3 %
%%                           + \parent{\beta^k}^2 u_{3,k}v_3}\,\mathrm dx_1\mathrm dx_2 %
%%  - \int_\Lambda \parent{p_k + C_k}v_3\,\mathrm dx_1\mathrm dx_2 %
%%  = \int_\Lambda f_{3,k}v_1\mathrm dx_1\mathrm dx_2, \label{eq:stokes-fourier-weak-3}\\
%%  %
%%  &\int_\Lambda \parent{\frac{\partial u_{1,k}}{\partial x_1} %
%%                       + \frac{\partial u_{2,k}}{\partial x_2} + \beta^k u_{3,k}}q\,\mathrm dx_1\mathrm dx_2 %
%%  %
%%  = 0,\label{eq:stokes-fourier-weak-4}\\
%%  %
%%  &\int_\Lambda u_{3,k}\,\mathrm dx_1\mathrm dx_2 %
%%  = 0, \label{eq:stokes-fourier-weak-5}
%%\end{align}
%%pour toute fonctions $v_1$, $v_2$, $v_3$ $\in H^1(\Lambda)$, $q \in
%%L_0^2(\Lambda)$. Il est bien connu que la formulation variationnelle
%%du problème de Stokes
%%(\ref{eq:stokes-weak-1}),(\ref{eq:stokes-weak-2}) dans $\Lambda$ admet
%%une solution unique \cite{Temam1977}. La proposition qui suit établi l'existence et
%%l'unicité de la solution du problème variationnel
%%(\ref{eq:stokes-fourier-weak-1})-(\ref{eq:stokes-fourier-weak-5}) pour
%%chacun des coefficients de la série de Fourier.

%%\begin{remarque}
%%  Si la viscosité du fluide est constante et isotrope,
%%  c'est-à-dire que le tenseur de viscosité $\mu_{ij}(x_1, x_2) =
%%  \mu_0$, le problème de Stokes (\ref{eq:stokes-periodic-1}),
%%  (\ref{eq:stokes-periodic-2}) se réécrit sous la forme
%%  \begin{align}
%%    & -\mu_0 \Delta U + \nabla P = F,\\
%%    & \div(U) = 0.
%%  \end{align}
%%\end{remarque}

%%\paragraph{Formulation avec viscosité tensorielle variable}
%%Considérons maintenant le problème de Stokes avec une viscosité
%%tensorielle et variable. On suppose que le tenseur de viscosité est
%%symétrique et est indépendant de la coordonnée verticale $x_3$,
%%c'est-à-dire que $\mu_{i,j} = \mu_{i,j}(x_1, x_2)$, $i,j = 1,2,3$. On
%%cherche la vitesse $u$ et la pression $p$ qui vérifient:
%%\begin{align}
%%  &-\div\parent{2\mu\otimes \epsilon(u)} + \nabla p = f, %
%%  \quad \text{dans } \Omega,\\
%%  %
%%  &\div(u) = 0, %
%%  \quad \text{dans } \Omega,
%%\end{align}
%%avec les conditions aux limites
%%\begin{align}
%%  &u_1 = u_2 = 0, &\quad \text{ sur } \partial \Lambda \times
%%  (0,\thickness),\\
%%  &u_3 = 0, &\quad \text{ sur }  \Lambda \times \cparent{0,\thickness},\\
%%  &\mu_{13}\parent{\frac{\partial u_1}{\partial x_3} + \frac{\partial u_3}{\partial
%%      x_1}} = 0, &\quad \text{ sur }  \Lambda \times \cparent{0,\thickness},\\
%%  &\mu_{23}\parent{\frac{\partial u_2}{\partial x_3} + \frac{\partial u_3}{\partial
%%  x_2}} = 0, &\quad \text{ sur }  \Lambda \times \cparent{0,\thickness}.
%%\end{align}
%%Ici on a noté
%%\begin{equation}
%%  [\mu \otimes \straintensor(u)]_{i,j} %
%%  = \frac{1}{2}\mu_{i,j}\parent{\frac{\partial u_i}{\partial x_j} %
%%                              + \frac{\partial u_j}{\partial x_i}},%
%%  \quad 1\leq i,j\leq 3.
%%\end{equation}
%%En reprenant les notations introduites dans la section
%%(\ref{sec:fourier}), on obtient pour $\beta > 0$, et en omettant
%%l'indice du mode de Fourier $k$:
%%\begin{align}
%%  - 2\frac{\partial}{\partial x_1}%
%%  \parent{\mu_{1,1}\frac{\partial u_1}{\partial x_1}} %
%%  -  \frac{\partial}{\partial x_2}%
%%  \parent{\mu_{1,2}\parent{\frac{\partial u_1}{\partial x_2} %
%%                           + \frac{\partial u_2}{\partial x_1}}} %
%%  - \mu_{1,3}\parent{- \beta^2 u_1 %
%%                     + \beta\frac{\partial u_3}{\partial x_1}} %
%%  + \frac{\partial p}{\partial x_1} %
%%  &= f_{1}, \label{eq:stokes-fourier-visc-1}\\
%%  %
%%  - \frac{\partial }{\partial x_1}%
%%  \parent{\mu_{2,1}\parent{\frac{\partial u_2}{\partial x_1} %
%%                           + \frac{\partial u_1}{\partial x_2}}}
%%  - 2\frac{\partial}{\partial x_2}%
%%  \parent{\mu_{2,2}\frac{\partial u_2}{\partial x_2}}
%%  - \mu_{2,3}\parent{- \beta^2 u_2 %
%%                     + \beta\frac{\partial u_3}{\partial x_2}} %
%%  + \frac{\partial p}{\partial x_2} %
%%  &= f_{2}, \label{eq:stokes-fourier-visc-2}\\
%%  %
%%  - \frac{\partial}{\partial x_1}%
%%  \parent{\mu_{3,1}\parent{ \frac{\partial u_3}{\partial x_1} %
%%                           - \beta u_1}} %
%%  - \frac{\partial }{\partial x_2}%
%%  \parent{\mu_{3,2}\parent{ \frac{\partial u_3}{\partial x_2} %
%%                           - \beta u_2}} %
%%  + 2\mu_{3,3}\beta^2 u_3 %
%%  - \beta p %
%%  &= f_{3}. \label{eq:stokes-fourier-visc-3}
%%\end{align}
%%
%%En définissant maintenant le tenseur $3\times 3$ des déformations:
%%\begin{equation}\label{eq:stokes-fourier-strain-tensor-3d}
%%  E(u) = \frac{1}{2} \begin{bmatrix}
%%    2\displaystyle\frac{\partial u_1}{\partial x_1} %
%%    &  \displaystyle\frac{\partial u_1}{\partial x_1} + \displaystyle\frac{\partial u_2}{\partial x_1} %
%%    & - \beta u_1 + \displaystyle\frac{\partial u_3}{\partial x_1}\\
%%    %
%%    \displaystyle\frac{\partial u_1}{\partial x_2}+ \displaystyle\frac{\partial u_2}{\partial x_1} %
%%    & 2 \displaystyle\frac{\partial u_2}{\partial x_2} %
%%    & -\beta u_2 + \displaystyle\frac{\partial u_3}{\partial x_2} \\
%%    %
%%    - \beta u_1 + \displaystyle\frac{\partial u_3}{\partial x_1} %
%%    & - \beta u_2 + \displaystyle\frac{\partial u_3}{\partial x_2} & 2\beta u_3
%%  \end{bmatrix}
%%\end{equation}
%%et celui des viscosités:
%%\begin{equation}\label{eq:stokes-fourier-visc-tensor-3d}
%%\mu = \begin{bmatrix}
%%  \mu_{1,1} & \mu_{1,2} & \mu_{1,3} \\
%%  \mu_{2,1} & \mu_{2,2} & \mu_{2,3} \\
%%  \mu_{3,1} & \mu_{3,2} & \mu_{3,3}
%%\end{bmatrix}
%%\end{equation}
%%et en multipliant (\ref{eq:stokes-fourier-visc-1}),
%%(\ref{eq:stokes-fourier-visc-2}), (\ref{eq:stokes-fourier-visc-3}) par
%%$v_1$, $v_2$, $v_3$, puis en intégrant par partie, on obtient:
%%\begin{align}
%%  \begin{aligned}[b]
%%    \int_\Lambda 2\mu\otimes E(u):E(v)\,\mathrm dx_1\mathrm dx_2 %
%%    &- \int_\Lambda p\parent{\frac{\partial v_1}{\partial x_1}
%%                            + \frac{\partial v_2}{\partial x_2}}\,\mathrm dx_1\mathrm dx_2 \\
%%    &- \beta \int_\Lambda \parent{p + C}v_3\,\mathrm dx_1\mathrm dx_2
%%  \end{aligned}
%%  &= \int_\Lambda f\cdot v\,\mathrm dx_1\mathrm dx_2,\label{eq:stokes-fourier-visc-weak-1}\\
%%  %
%%  \int_\Lambda \parent{\frac{\partial u_1}{\partial x_1} %
%%    + \frac{\partial u_2}{\partial x_2}}q\,\mathrm dx_1\mathrm dx_2 %
%%  + \beta \int_\Lambda u_3 q \,\mathrm dx_1\mathrm dx_2 %
%%  &= 0,\label{eq:stokes-fourier-visc-weak-2}\\
%%  %
%%  \int_\Lambda u_3 \,\mathrm dx_1\mathrm dx_2 %
%%  &= 0.\label{eq:stokes-fourier-visc-weak-3}
%%\end{align}
%%
%%\begin{remarque}
%%Le système d'équations
%%(\ref{eq:stokes-fourier-visc-weak-1})-(\ref{eq:stokes-fourier-visc-weak-3})
%%permet de traiter le cas ``historique'':
%%\begin{equation}
%%  \mu = \begin{bmatrix}
%%    10 & 10 & 0.5\\
%%    10 & 10 & 0.5\\
%%    0.5 & 0.5 & 1
%%  \end{bmatrix}.
%%\end{equation}
%%\end{remarque}
%%
%%\begin{remarque}
%%  Le mode fondamental $k = 0$ consiste à chercher  $u_0\in\parent{H_0^1(\Lambda)}^2$, $p_0\in L_0^2(\Lambda)$, tels que
%%pour tout $v\in H_0^1(\Lambda)^2$ et $q\in L_0^2(\Lambda)$:
%%\begin{align}
%%  \int_\Lambda 2\tilde\mu\otimes \tilde \straintensor(u_0):\tilde \straintensor(v)\,\mathrm dx_1\mathrm dx_2 %
%%  - \int_\Lambda p\parent{\frac{\partial v_1}{\partial x_1} %
%%                          + \frac{\partial v_2}{\partial x_2}}\,\mathrm dx_1\mathrm dx_2 %
%%  &= \int_\Lambda f\cdot v\,\mathrm dx_1\mathrm dx_2,\\
%%  %
%%  \int_\Lambda \parent{\frac{\partial u_1}{\partial x_1} %
%%    + \frac{\partial u_2}{\partial x_2}}q \,\mathrm dx_1\mathrm dx_2 %
%%  &= 0,
%%\end{align}
%%avec:
%%\begin{equation}\label{eq:stokes-fourier-strain-tensor-2d}
%%  \tilde \straintensor(u) = \frac{1}{2}\begin{bmatrix}
%%    2 \displaystyle\frac{\partial u_1}{\partial x_1} %
%%    & \displaystyle\frac{\partial u_1}{\partial x_2} + \displaystyle\frac{\partial u_2}{\partial x_1}\\
%%    %
%%    \displaystyle\frac{\partial u_1}{\partial x_2} + \displaystyle\frac{\partial u_2}{\partial x_1} %
%%    & 2 \frac{\partial u_2}{\partial x_2}
%%  \end{bmatrix}
%%\end{equation}
%%et:
%%\begin{equation}\label{eq:stokes-fourier-visc-tensor-2d}
%%\tilde \mu = \begin{bmatrix}
%%  \mu_{1,1} & \mu_{1,2}\\
%%  \mu_{2,1} & \mu_{2,2}
%%\end{bmatrix}.
%%\end{equation}
%%On trouvera ci-dessous une démonstration plus simple que celle faite
%%pour démontrer la proposition \ref{prop:existance-unicite-1} de la
%%section \ref{sec:stokes-fourier-weak}.
%%\end{remarque}

%\paragraph{Existence et unicité de la solution}
Tout d'abord on gardera les notations introduites dans
(\ref{eq:stokes-fourier-strain-tensor-3d}) et
(\ref{eq:stokes-fourier-strain-tensor-2d}) pour $E(u)$ et $\tilde E(u)$
respectivement. D'autre part on note toujours $\abs{E(u)}^2 =
\sum_{i,j = 0}^3E_{i,j}^2(u)$ et $\abs{\tilde E(u)}^2 = \sum_{i,j =
  0}^2\tilde E_{i,j}^2(u)$. Pour simplifier, on supposera que le
domaine $\Lambda$ est un rectangle de dimensions $L_1,L_2$,
c'est-à-dire $\Lambda = \cparent{(x_1, x_2) \setsuchthat
  0<x_1<L_1,\ 0<x_2<L_2}$.

\begin{lemme}\label{lem:lemme-1}
  Sous l'hypothèse $\frac{\partial u_1}{\partial x_1} +
  \frac{\partial u_2}{\partial x_2} + \beta u_3 = 0$, on a la
  relation:
  \begin{equation}
    \int_\Lambda \abs{E(u)}^2\,\mathrm dx_1\mathrm dx_2 %
    = \int_\Lambda \parent{\abs{\tilde E(u)}^2 %
      + \frac{1}{2}\parent{  \parent{\frac{\partial u_3}{\partial x_1}}^2 %
                           + \parent{\frac{\partial u_3}{\partial x_2}}^2 %
                           + \beta^2 u_1^2 %
                           + \beta^2 u_2^2}}\,\mathrm dx_1\mathrm dx_2.
  \end{equation}
\end{lemme}

\begin{proof}
  Le calcul de $\abs{E(u)}^2$ donne:
  \begin{equation}\label{eq:proof-details}
    \begin{split}
      \abs{E(u)}^2 = \abs{\tilde E(u)}^2 %
      + \frac{1}{2}\parent{  \parent{\frac{\partial u_3}{\partial x_1}}^2 %
                           + \parent{\frac{\partial u_3}{\partial x_2} %
                                     + \beta^2u_1^2 %
                                     + \beta^2 u_2^2}}\\
      %
      - \beta u_1 \frac{\partial u_3}{\partial x_1} %
      - \beta u_2 \frac{\partial u_3}{\partial x_2} %
      + \beta^2 u_3.
    \end{split}
  \end{equation}
  En intégrant par partie le terme $u_1 \frac{\partial u_3}{\partial x_1}$ on obtient:
  \begin{equation}
    \begin{split}
      \int_\Lambda u_1 \frac{\partial u_3}{\partial x_1}\,\mathrm dx_1\mathrm dx_2 %
      &= \int_0^{L_2}\mathrm dx_2\int_0^{L_1} %
           u_1 \frac{\partial u_3}{\partial x_1}\,\mathrm dx_1\\
      &= - \int_0^{L_2}\mathrm dx_2 \int_0^{L_1} %
           u_3 \frac{\partial u_1}{\partial x_1}\,\mathrm dx_1\\
      &= - \int_\Lambda %
           u_3 \frac{\partial u_1}{\partial x_1}\,\mathrm dx_1\mathrm dx_2.
    \end{split}
  \end{equation}
  De même on aura:
  \begin{equation}
    \int_\Lambda u_2 \frac{\partial u_3}{\partial x_2}\,\mathrm dx_1\mathrm dx_2 %
    = - \int_\Lambda u_3 \frac{\partial u_2}{\partial x_2}\,\mathrm dx_1\mathrm dx_2.
  \end{equation}
  En reprenant (\ref{eq:proof-details}), en intégrant sur $\Lambda$
  et en utilisant l'hypothèse $\frac{\partial u_1}{\partial x_1} +
  \frac{\partial u_2}{\partial x_2} + \beta u_3 = 0$, on obtient le
  résultat annoncé.
\end{proof}

\begin{lemme}\label{lem:lemme-2}
  Il existe une constante positive $\chi > 0$ telle que:
  \begin{equation}
    \chi \norm{\nabla u}_{L^2(\Lambda)} \leq \norm{E(u)}_{L^2(\Lambda)},
  \end{equation}
  pour tout $u\in H_0^1(\Lambda)^3$ qui satisfait $\frac{\partial
    u_1}{\partial x_1} + \frac{\partial u_2}{\partial x_2} + \beta u_3
  = 0$. Ici $\norm{\nabla u}^2_{L^2(\Lambda)} =
  \sum_{i,j=1}^3\norm{\frac{\partial u_i}{\partial
      x_j}}^2_{L^2(\Lambda)}$ et $\norm{E(u)}^2_{L^2(\Lambda)} =
  \int_\Lambda\abs{E(u)}^2\,\mathrm dx_1\mathrm dx_2$.
\end{lemme}

\begin{proof}
  Il est connu que l'inégalité de Korn en dimension 2 est vraie,
  i. e. l'existance d'une constante $\chi > 0$ qui satisfait:
  \begin{equation}
    \chi \sum_{i,j = 1}^2\norm{\frac{\partial u_i}{\partial x_j}}^2_{L^2(\Lambda)} %
    \leq \int_\Lambda \abs{\tilde E(u)}^2\,\mathrm dx_1\mathrm dx_2.
  \end{equation}
  Le lemme (\ref{lem:lemme-1}) permet de conclure.
\end{proof}

\begin{proposition}\label{prop:proposition-2}
  Si le tenseur $\mu$ satisfait:
  \begin{equation}\label{eq:prop-2-hypothesis}
    \mu_{i,j}(x_1, x_2) \geq \mu_0 \quad \forall (x_1, x_2)\in \Lambda,\ 1\leq i,j\leq 3,
  \end{equation}
  où $\mu_0 > 0$ est une constante positive indépendante de $(x_1,
  x_2)\in\Lambda$, alors le problème
  (\ref{eq:stokes-fourier-visc-weak-1})-(\ref{eq:stokes-fourier-visc-weak-3})
  admet une et une seule solution.
\end{proposition}

\begin{proof}
  Définissont l'espace $V = H_0^1(\Lambda)^2\times
  H_{0,0}^1(\Lambda)$ où $H_{0,0}^1(\Lambda) = H_0^1(\Lambda) \cap
  L_0^2(\Lambda)$. Clairement l'espace $H_{0,0}^1(\Lambda)$ est un
  sous-espace fermé de $H_0^1(\Lambda)$ de codimension 1. Soit
  encore $a:V\times V\to \mathbb R$ la forme bilinéaire continue
  définie par:
  \begin{equation}
    a(u,v) = \int_\Lambda 2\mu\otimes E(u):E(v)\,\mathrm dx_1\mathrm dx_2.
  \end{equation}
  Définissons encore la forme bilinéaire continue $b:V\times Y\to\mathbb R$, par:
  \begin{equation}
    b(u, q) = \int_\Lambda \parent{\frac{\partial u_1}{\partial x_1} %
      + \frac{\partial u_2}{\partial x_2} %
      + \beta u_3}q\,\mathrm dx_1\mathrm dx_2,
  \end{equation}
  où ici $Y = L_0^2(\Lambda)$.

  Il est facile de voir que le problème
  (\ref{eq:stokes-fourier-visc-weak-1})-(\ref{eq:stokes-fourier-visc-weak-3})
  est équivalent au problème de chercher $(u,p)\in V\times Y$ tel
  que:
  \begin{equation}
    \begin{split}
      a(u,v) - b(v, p) = \int_\Lambda f\cdot v\,\mathrm dx_1\mathrm dx_2,%
        \quad \forall v\in V,\\
      b(u,q) = 0,%
        \quad \forall q\in Y.
    \end{split}
  \end{equation}
  La constante $C$ peut s'obtenir a posteriori en considérant
  (\ref{eq:stokes-fourier-visc-weak-1})-(\ref{eq:stokes-fourier-visc-weak-3})
  avec les fonctions tests $v = (0,0,s)$ où $s\in H_0^1(\Lambda)$
  est dans l'orthogonal de $H_{0,0}^1(\Lambda)$.

  Pour démontrer la proposition \ref{prop:proposition-2}, il suffit
  de vérifier que la forme $a(.,.)$ est coercive sur $V_0$, où
  $V_0 = \cparent{v\in V\setsuchthat b(v,q) = 0\ \forall q\in Y}$, et
  que la condition classique inf--sup sur la forme bilinéaire $b$
  est satisfaite.

  Le lemme \ref{lem:lemme-2} avec l'hypothèse
  (\ref{eq:prop-2-hypothesis}) montrent bien que $a$ est coercive sur
  $V_0$. D'autre part en utilisant l'inégalité concernant la
  condition inf--sup dans $\mathbb R^2$ on a si $q\in L_0^2(\Lambda)$:
  \begin{equation}
    \begin{split}
      \sup_{\norm{v}_{H_0^1} = 1}b(v,q) %
      &\geq \sup_{\norm{(v_1, v_2, 0)}_{H_0^1}}b(v,q)\\
      &= \sup_{\norm{(v_1, v_2, 0)}_{H_0^1}}\int_\Lambda \parent{\frac{\partial v_1}{\partial x_1} %
        + \frac{\partial v_2}{\partial x_2}}q\,\mathrm dx_1\mathrm dx_2\\
      &\geq \delta \norm{q}_{L_0^2},
    \end{split}
  \end{equation}
  où $\delta > 0$. Ainsi on a prouvé la proposition \ref{prop:proposition-2}.
\end{proof}

