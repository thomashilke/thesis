Soit $\Lambda$ un ouvert borné de $\mathbb R^2$ de bord
$\partial\Lambda$. Soit $\thickness > 0$ un nombre réel donné
correspondant à la dimension verticale d'une cuve. On définit le
domaine de $\mathbb R^3$ correspondant à une simplification
géométrique de la cuve d'électrolyse
\begin{equation}
  \Omega = \Lambda \times (0,\thickness).
\end{equation}
On suppose que le domaine $\Omega$ est occupé par un fluide
incompressible de viscosité $\mu$. Étant donné un champ de forces
$f:\Omega\to \mathbb R^3$, on suppose que la vitesse $u:\Omega \to
\mathbb R^3$ et la pression $p:\Omega \to \mathbb R$ du fluide
satisfont le système de Stokes:
\begin{align}
  &- \div(\mu\nabla u) + \nabla p = f,\label{eq:stokes-u}\\
  &\div u = 0.\label{eq:stokes-p}
\end{align}
Dans la suite nous notons $u = (u_1, u_2, u_3)$ et $f = (f_1, f_2,
f_3)$. La vitesse du fluide satisfait des conditions aux limites
d'adhérence sur les faces latérales et des conditions de glissement
total sur les faces horizontales, c'est-à-dire que
\begin{align}
  &u(x_1, x_2, x_3) = 0 &\text{ si } (x_1, x_2) \in \partial \Lambda%
  \text{ et } x_3\in(0, \thickness),\label{eq:stokes-bc-1}\\
%
  &u_3(x_1, x_2, x_3) = 0 &\text{ si } (x_1, x_2) \in \Lambda \text{ et } x_3 = 0 \text{ ou } x_3 = \thickness,\label{eq:stokes-bc-2}\\
%
  &\frac{\partial u_1}{\partial x_3}(x_1, x_2, x_3) = \frac{\partial u_2}{\partial x_3}(x_1, x_2, x_3) = 0 &\text{ si }
  (x_1, x_2) \in \Lambda \text{ et } x_3 = 0 \text { ou } x_3 = \thickness.\label{eq:stokes-bc-3}
\end{align}

\paragraph{Décomposition en séries de Fourier}
Pour résoudre le système d'équation (\ref{eq:stokes-u}),
(\ref{eq:stokes-p}) avec les conditions (\ref{eq:stokes-bc-1}) à
(\ref{eq:stokes-bc-3}), on exprime les inconnues sous forme de séries
de Fourier dans la direction $x_3$.Soit le domaine $\Omega^+ = \Lambda
\times (-\thickness, \thickness)$, on définit les fonctions suivantes
$\forall (x_1, x_2, x_3)\in \Omega^+$:
\begin{align}
  & U_i(x_1, x_2, x_3) = \left\{
    \begin{array}{lll}
      u_i(x_1, x_2, x_3) &\text{ si } x_3 \geq 0,\\
      u_i(x_1, x_2, -x_3) &\text{ si } x_3 < 0,     & \quad i = 1,2,
    \end{array}
  \right.\\
  %
  & U_3(x_1, x_2, x_3) = \left\{
    \begin{array}{ll}
       u_3(x_1, x_2, x_3) &\text{ si } x_3 \geq 0,\\
      -u_3(x_1, x_2, -x_3) &\text{ si } x_3 < 0,
    \end{array}
  \right.\\
  %
  & P(x_1, x_2, x_3) = \left\{
    \begin{array}{ll}
      p(x_1, x_2, x_3) &\text{ si } x_3 \geq 0,\\
      p(x_1, x_2, -x_3) &\text{ si } x_3 < 0,
    \end{array}
  \right.\\
  %
  & F_i(x_1, x_2, x_3) = \left\{
    \begin{array}{lll}
      f_i(x_1, x_2, x_3) &\text{ si } x_3 \geq 0,\\
      f_i(x_1, x_2, -x_3) &\text{ si } x_3 < 0,     & \quad i = 1,2,
    \end{array}
  \right.\\
  %
  & F_3(x_1, x_2, x_3) = \left\{
    \begin{array}{ll}
      f_3(x_1, x_2, x_3) &\text{ si } x_3 \geq 0\\
     -f_3(x_1, x_2, -x_3) &\text{ si } x_3 < 0.
    \end{array}
  \right.
\end{align}

Ainsi on a $U_3(x_1, x_2, 0) = 0\ \forall(x_1, x_2)\in\Lambda$ et les
conditions de glissement impliquent en $x_3 = 0$ que
\begin{equation}
  \frac{\partial
U_1}{\partial x_3}(x_1,x_2, 0) = \frac{\partial U_2}{\partial
  x_3}(x_1, x_2, 0) = 0\quad \forall (x_1, x_2) \in \Lambda
\end{equation}
et en $x_3 = \thickness$ que
\begin{align}
  &\frac{\partial U_1}{\partial x_3}(x_1, x_2, \thickness) = \frac{\partial U_1}{\partial x_3}(x_1, x_2, -\thickness) = 0,\\
  &\frac{\partial U_2}{\partial x_3}(x_1, x_2, \thickness) = \frac{\partial U_2}{\partial x_3}(x_1, x_2, -\thickness) = 0.
\end{align}
Les dérivées secondes en $x_3$ des fonction $U_1$, $U_2$, $U_3$ ne
présenteront pas de masse de Dirac en $x_3 = 0$.

Avec les définitions ci-dessus on peut ainsi vérifier que $U$ et $P$
satisfont les équations
\begin{align}
  &-\mu\Delta U + \Delta P = F& \quad \text{sur }\Omega^+,\\
  &\div U = 0,& \quad \text{sur }\Omega^+.
\end{align}

On prolonge toutes les fonctions dans la variable $x_3$ sur tout
$\mathbb R$ en des fonctions périodiques de période $2\thickness$. On
note encore par $U_1$, $U_2$, $U_3$, $P$, $F_1$, $F_2$, $F_3$ ces
prolongements, c'est-à-dire que ces fonctions sont considérées comme
fonctions de $(x_1, x_2, x_3)\in \Lambda\times \mathbb R$ et
$2\thickness$-périodiques selon $x_3$. Les fonctions $U_1$, $U_2$,
$U_3$ sont $\mathcal C^2$ par morceaux, $P$ est $\mathcal C^1$ par
morceaux et $F_1$, $F_2$, $F_3$ sont $C^0$ par morceaux.

Leurs décomposition en série de Fourier selon la variable $x_3$ s'écrivent:
\begin{align}
  &U_i(x_1, x_2, x_3) = u_{i,0}(x_1, x_2) %
                       + \sum_{k > 0} u_{i,k}(x_1, x_2)%
                       \cos\parent{\frac{\pi k}{\thickness}x_3}, \quad i = 1, 2,\label{eq:fourier-coeff-def-1}\\
  &U_3(x_1, x_2, x_3) = \sum_{k > 0} u_{3,k}(x_1, x_2)%
                        \sin\parent{\frac{\pi k}{\thickness}x_3},\label{eq:fourier-coeff-def-2}\\
  &P(x_1, x_2, x_3) = p_0(x_1, x_2) %
                      + \sum_{k>0}p_k(x_1, x_2)%
                      \cos\parent{\frac{\pi k}{\thickness}x_3},\label{eq:fourier-coeff-def-3}\\
  &F_i(x_1, x_2, x_3) = f_{i, 0}(x_1, x_2) %
                        + \sum_{k > 0}f_{i,k}(x_1, x_2)%
                        \cos\parent{\frac{\pi k}{\thickness}x_3}, \quad i = 1,2,\label{eq:fourier-coeff-def-4}\\
  &F_3(x_1, x_2, x_3) = \sum_{k > 0}f_{3,k}(x_1,x_2)%
                        \sin\parent{\frac{\pi k}{\thickness}x_3}.\label{eq:fourier-coeff-def-5}
\end{align}
En identifiant les différents modes, on obtient pour $k = 0$:
\begin{align}
  -\mu \Delta_{x_1, x_2}u_{1, 0} + \frac{\partial p_0}{\partial x_1} %
  &= f_{1,0},\label{eq:fourier-fund-1}\\
  %
  -\mu \Delta_{x_1, x_2}u_{2, 0} + \frac{\partial p_0}{\partial x_2} %
  &= f_{2,0},\label{eq:fourier-fund-2}\\
  %
  \frac{\partial u_{1, 0}}{\partial x_1} +   \frac{\partial u_{2, 0}}{\partial x_2} %
  &= 0,\label{eq:fourier-fund-3}
\end{align}
et pour $k = 1, 2, 3,\dots$ :
\begin{align}
  - \mu\Delta_{x_1, x_2}u_{1, k}%
  + \mu\parent{\frac{\pi k}{\thickness}}^2 u_{1,k}%
  + \frac{\partial p_k}{\partial x_1} %
  &= f_{1,k}, \label{eq:fourier-harm-1}\\
  %
  - \mu\Delta_{x_1, x_2}u_{2, k} %
  + \mu\parent{\frac{\pi k}{\thickness}}^2 u_{2,k} %
  + \frac{\partial p_k}{\partial x_2} %
  &= f_{2,k}, \label{eq:fourier-harm-2} \\
  %
  - \mu\Delta_{x_1, x_2}u_{3, k} %
  + \mu\parent{\frac{\pi k}{\thickness}}^2 u_{3,k} %
  - \frac{\pi k}{\thickness}p_k %
  &= f_{3,k}, \label{eq:fourier-harm-3}\\
  %
    \frac{\partial u_{1,k}}{\partial x_1} %
  + \frac{\partial u_{2,k}}{\partial x_2} %
  + \frac{\pi k}{\thickness} u_{3, k} %
  &= 0. \label{eq:fourier-incompressibility}
\end{align}
Les problèmes ci-dessus sont à résoudre dans $\Lambda \subset
\mathbb R^2$. Naturellement ils nécessitent des
conditions limites d'adhérence sur $\partial \Lambda$:
\begin{align}
  & u_{1,0} = u_{2,0} = 0 & \text{ sur } \partial \Lambda,\\
  & u_{1,k} = u_{2,k} = u_{3,k} = 0 &\text{ sur } \partial \Lambda,\ k
  = 1, 2, 3,\dots,
\end{align}
et le calcul au préalable des coefficients de Fourier du champ de force:
\begin{align}
  f_{i,0}(x_1, x_2) %
  &= \frac{1}{2\thickness}\int_0^{2\thickness}f_i(x_1, x_2, x_3)%
  \,\mathrm dx_3,\quad i = 1,2,\label{eq:force-coefficient-1}\\
  %
  f_{i,k}(x_1, x_2) %
  &= \frac{1}{\thickness}\int_0^{2\thickness}f_i(x_1, x_2, x_3)%
  \cos\parent{\frac{\pi k}{\thickness}}\,\mathrm dx_3,\quad i = 1,2,\ k > 0,\label{eq:force-coefficient-2}\\
  %
  f_{3,k}(x_1, x_2)%
  &= \frac{1}{\thickness}\int_0^{2\thickness}f_3(x_1, x_2, x_3)%
  \sin\parent{\frac{\pi k}{\thickness}}\,\mathrm dx_3,\quad k > 0.\label{eq:force-coefficient-3}
\end{align}

\paragraph{Formulation faible}\label{sec:stokes-fourier-weak}
La formulation variationnelle associée au problèmes formés par les
équations (\ref{eq:fourier-fund-1}) à
(\ref{eq:fourier-incompressibility}) s'énonce de la façon
suivante. Premièrement, nous cherchons les fonctions $u_{1,0},u_{2,0} \in
H^1(\Lambda)$ $p_0 \in L^2_0(\Lambda)$, telles
que
\begin{align}
&\int_\Lambda \electrolyteviscosity \nabla u : \nabla v \intd{x_1}\intd{x_2} -
\int_\Lambda p\div v \intd{x_1}\intd{x_2} = \int_\Lambda f v
\intd{x_1}\intd{x_2},\label{eq:stokes-weak-1}\\
&\int_\Lambda q\div u \intd{x_1}\intd{x_2} = 0\label{eq:stokes-weak-2}
\end{align}
pour toute fonctions $v \in \parent{H^1(\Lambda)}^2$, $q \in
L_0^2(\Lambda)$. Notons pour alléger l'écriture $\beta_k = \frac{\pi
  k}{\thickness}$. Deuxièmement, pour tout $k = 1, 2, \dots$, nous cherchons les
fonctions $u_{1,k},u_{2,k},u_{3,k} \in H^1(\Lambda)$, $p_k \in
L_0^2(\Lambda)$ et les nombres réels $C_k\in\mathbb R$ tels que
\begin{align}
  &\electrolyteviscosity \int_\Lambda \parent{\nabla u_{1,k}\cdot \nabla v_1 %
                           + \beta_k^2 u_{1,k}v_1}\,\mathrm dx_1\mathrm dx_2 %
  - \int_\Lambda p_k\frac{\partial v_1}{\partial x_1}\,\mathrm dx_1\mathrm dx_2 %
  = \int_\Lambda f_{1,k}v_1\mathrm dx_1\mathrm dx_2, \label{eq:stokes-fourier-weak-1}\\
  %
  &\electrolyteviscosity \int_\Lambda \parent{\nabla u_{2,k}\cdot \nabla v_2 %
                           + \beta_k^2 u_{2,k}v_2}\,\mathrm dx_1\mathrm dx_2 %
  - \int_\Lambda p_k\frac{\partial v_2}{\partial x_2}\,\mathrm dx_1\mathrm dx_2 %
  = \int_\Lambda f_{2,k}v_1\mathrm dx_1\mathrm dx_2, \label{eq:stokes-fourier-weak-2}\\
  %
  &\electrolyteviscosity \int_\Lambda \parent{\nabla u_{3,k}\cdot \nabla v_3 %
                           + \beta_k^2 u_{3,k}v_3}\,\mathrm dx_1\mathrm dx_2 %
  - \int_\Lambda \parent{p_k + C_k}v_3\,\mathrm dx_1\mathrm dx_2 %
  = \int_\Lambda f_{3,k}v_1\mathrm dx_1\mathrm dx_2, \label{eq:stokes-fourier-weak-3}\\
  %
  &\int_\Lambda \parent{\frac{\partial u_{1,k}}{\partial x_1} %
                       + \frac{\partial u_{2,k}}{\partial x_2} + \beta_k u_{3,k}}q\,\mathrm dx_1\mathrm dx_2 %
  %
  = 0,\label{eq:stokes-fourier-weak-4}\\
  %
  &\int_\Lambda u_{3,k}\,\mathrm dx_1\mathrm dx_2 %
  = 0, \label{eq:stokes-fourier-weak-5}
\end{align}
pour toute fonctions $v \in \parent{H^1(\Lambda)}^3$, $q \in
L_0^2(\Lambda)$. Il est bien connu que la formulation variationnelle
du problème de Stokes
(\ref{eq:stokes-weak-1}),(\ref{eq:stokes-weak-2}) dans $\Lambda$ admet
une solution unique \cite{}. La proposition qui suit établi l'existance et
l'unicité de la solution du problème variationnel
(\ref{eq:stokes-fourier-weak-1})-(\ref{eq:stokes-fourier-weak-5}) pour
chacun des coefficients de la série de Fourier.

%\begin{proposition}\label{prop:existance-unicite-1}
  Pour $k > 0$ fixé, le problème de chercher $u_{i,k}\in H_0^1(\Lambda),\ i =
  1,2,3,\ p_k^0 \in L_0^2(\Lambda)$ et $C_k\in\mathbb R$, qui satisfont
  les équations
  (\ref{eq:stokes-fourier-weak-1})-(\ref{eq:stokes-fourier-weak-5})
  pour tout $v_1$, $v_2$, $v_3$ $\in H_0^1(\Lambda)$ et $q\in
  L^2(\Lambda)$ admet une et une seule solution.
\end{proposition}

\begin{proof}
  Soit $g\in L_0^2(\Lambda)$ et résolvont le problème
  suivant. Trouver $u_1$, $u_2\in H_0^1(\Lambda)$ et $p\in
  L_0^2(\Lambda)$ qui satisfont:

  \begin{align}
    &\mu\int_\Lambda \parent{\nabla u_1\cdot \nabla v_1 %
                            + \beta_k^2 u_1 v_1}\,\mathrm dx_1\mathrm dx_2 %
    - \int_\Lambda p\frac{\partial v_1}{\partial x_1} %
    = \int_\Lambda f_{1,k} v\, \mathrm dx_1\mathrm dx_2,\label{eq:t-weak-form-1}\\
    %
    &\mu\int_\Lambda \parent{\nabla u_2\cdot \nabla v_2 %
                            + \beta_k^2 u_2 v_2}\,\mathrm dx_1\mathrm dx_2 %
    - \int_\Lambda p\frac{\partial v_2}{\partial x_2} %
    = \int_\Lambda f_{2,k} v\, \mathrm dx_1\mathrm dx_2,\label{eq:t-weak-form-2}\\
    %
    &\int_\Lambda \parent{\frac{\partial u_1}{\partial x_1} %
                         + \frac{\partial u_2}{\partial x_2}} q \, \mathrm dx_1\mathrm dx_2 %
    + \beta_k \int_\Lambda g q\,\mathrm dx_1\mathrm dx_2 %
    = 0,\label{eq:t-weak-form-3}
  \end{align}
  pour toutes fonctions tests $v_1,v_2\in H_0^1(\Lambda)$ et $q\in
  L_0^2(\Lambda)$. Il est bien connu que la condition:
  \begin{equation}\label{eq:inf-sup}
    \inf_{q\in L_0^2(\Lambda)} %
    \sup_{v\in H_0^1(\Lambda)} %
    \frac{\displaystyle\int_\Lambda q\div(v)\,\mathrm dx_1\mathrm dx_2}%
         {\norm{q}_{L_0^2}\norm{v}_{H_0^1}}> 0
  \end{equation}
  est vraie et qu'elle implique que le problème
  (\ref{eq:t-weak-form-1})-(\ref{eq:t-weak-form-2}) admet une solution
  unique $(u_1, u_2, p)$ qui dépend bien évidemment de $g$. De
  plus on a:
  \begin{equation}\label{eq:t-estim-1}
    \norm{u_1}_{H_0^1} %
    + \norm{u_2}_{H_0^1} %
    + \norm{p}_{L_0^2} %
    \leq K\parent{\norm{f_{1,k}}_{L^2} %
    + \norm{f_{2,k}}_{L^2} %
    + \norm{g}_{L^2}}
  \end{equation}
  où ici $K$ est une constante indépendante de $f_{1,k}$, $f_{2,k}$,
  $f_{3,k}$ et $g$. Soit maintenant $u \in H_0^1(\Lambda)$ qui satisfait:
  \begin{equation}\label{eq:t-u-def}
    \electrolyteviscosity\int_\Lambda \parent{\nabla u\cdot \nabla v %
                         + \beta_k ^2 uv}\,\mathrm dx_1\mathrm dx_2 %
    = \beta_k\int_\Lambda pv\,\mathrm dx_1\mathrm dx_2 %
    + \int_\Lambda f_{3,k}v\,\mathrm dx_1\mathrm dx_2,%
    \quad \forall v\in H_0^1(\Lambda).
  \end{equation}
  Clairement, $u$ existe et est unique. Ici encore $u$ dépend de $g$
  puisque $p$ en dépend. On a
  \begin{equation}\label{eq:t-estim-2}
\norm{u}_{H_0^1} \leq K\parent{\norm{p}_{L^2_0} + \norm{f_{3,k}}_{H^1_0}}
  \end{equation}
  et avec (\ref{eq:t-estim-1})
  \begin{equation}\label{eq:t-estim-3}
    \norm{u}_{H_0^1} %
    \leq K\parent{\norm{f_{1,k}}_{L^2_0} %
      + \norm{f_{2,k}}_{L^2_0} %
      + \norm{f_{3,k}}_{L^2_0} %
      + \norm{g}_{L^2_0}}.
  \end{equation}
  Soit encore $w \in H_0^1(\Lambda)$, indépendant de $g$, qui satisfait
  \begin{equation}\label{eq:t-w-def}
    \mu \int_\Lambda \parent{\nabla w\cdot\nabla v %
      + \beta_k^2 wv}\,\mathrm dx_1\mathrm dx_2 %
    = \beta_k \int_\Lambda v\,\mathrm dx_1\mathrm dx_2 %
    \quad \forall v\in H_0^1(\Lambda).
  \end{equation}
  On définit $C\in\mathrm R$ par:
  \begin{equation}\label{eq:t-c-def}
    C = \frac{\displaystyle\int_\Lambda u\,\mathrm dx_1\mathrm dx_2}%
             {\displaystyle\int_\Lambda w\,\mathrm dx_1\mathrm dx_2}
  \end{equation}
  et $u_3\in H_0^1(\Lambda)$ par:
  \begin{equation}\label{eq:t-u3-def}
    u_3 = u + Cw.
  \end{equation}
  Ainsi on aura bien évidemment
  \begin{equation}
    \int_\Lambda u_3\,\mathrm dx_1\mathrm dx_2 = 0,
  \end{equation}
  et en utilisant (\ref{eq:t-u-def}), (\ref{eq:t-w-def}),
  (\ref{eq:t-u3-def}) on obtient
  \begin{equation}
    \mu\int_\Lambda \parent{\nabla u_3\cdot \nabla v %
                            + \beta_k^2 u_3 v}\,\mathrm dx_1\mathrm dx_2 %
    = \beta_k\int_\Lambda \parent{p + C}v \,\mathrm dx_1\mathrm dx_2 %
    + \int_\Lambda f_{3,k}v\,\mathrm dx_1\mathrm dx_2, %
    \quad v\in H_0^1(\Lambda).
  \end{equation}
  La fonction $w$ est indépendante de $f_{i,k}$, $i = 1,2,3$ et
  $g$. Ainsi (\ref{eq:t-c-def}) implique que $\abs{C}\leq
  K\norm{u}_{L^2}$, et en utilisant (\ref{eq:t-estim-3}) et
  (\ref{eq:t-u3-def}) on a:
  \begin{equation}\label{eq:t-estim-4}
    \norm{u_3}_{H^1} %
    \leq K\parent{\norm{f_{1,k}}_{L^2} %
      + \norm{f_{2,k}}_{L^2} %
      + \norm{f_{3,k}}_{L^2} %
      + \norm{g}_{L^2}}.
  \end{equation}
  Pour terminer on définit l'opérateur:
  \begin{equation}
    T: g\in L_0^2(\Lambda)\to T(g)\doteqdot u_3\in L_0^2(\Lambda).
  \end{equation}
  On vérifie, en utilisant l'inégalité (\ref{eq:t-estim-4}), que
  $T$ est un opérateur compact et borné. En utilisant le
  théorème de Leray-Schauder, l'opérateur $T$ a au moins un
  point fixe, et donc il existe $\hat g\in L_0^2(\Lambda)$ tel que
  $\hat g = T(\hat g)$.

  Si, pour cette fonction $\hat g$, à laquelle lui correspondent
  $u_1,\ u_2,\ p$, $C$ et $u_3 = T(\hat g) = \hat g$, nous posons
  $u_{1,k} = u_1$, $u_{2,k} = u_2$, $u_{3,k} = u_3$, $p_k^0 = p$ et
  $C_k = C$,
  alors nous vérifions que les équations
  (\ref{eq:stokes-fourier-weak-1}) à
  (\ref{eq:stokes-fourier-weak-5}) sont satisfaites. Ainsi le
  problème
  (\ref{eq:stokes-fourier-weak-1})-(\ref{eq:stokes-fourier-weak-5}) a
  au moins une solution $(u_{1,k},u_{2,k},u_{3,k},p_k^0,C_k)$.  Pour
  montrer l'unicité de cette solution, il suffit de sommer les
  équations (\ref{eq:stokes-fourier-weak-1}) à
  (\ref{eq:stokes-fourier-weak-5}) avec $f_{1,k} = f_{2,k} = f_{3,k} =
  0$ et $v_1 = u_{1,k}$, $v_2 = u_{2,k}$, $v_3 = u_{3,k}$, $q =
  p_{k}^0$ pour voir que la solution triviale du problème linéaire
  homogène
  (\ref{eq:stokes-fourier-weak-1})-(\ref{eq:stokes-fourier-weak-5})
  est unique.
\end{proof}


\paragraph{Formulation avec viscosité tensorielle variable}
Considérons maintenant le problème de Stokes avec une viscosité
tensorielle et variable. On suppose que le tenseur de viscosité est
symétrique et est indépendent de la coordonnée verticale $x_3$,
c'est-à-dire que $\mu_{i,j} = \mu_{i,j}(x_1, x_2)$, $i,j = 1,2,3$. On
cherche la vitesse $u$ et la pression $p$ qui vérifient:
\begin{align}
  -\div\parent{2\mu\otimes \epsilon(u)} + \nabla p &= f, %
  \quad \text{dans } \Omega,\\
  %
  \div(u) &= 0, %
  \quad \text{dans } \Omega,
\end{align}
avec les conditions aux limites
\begin{align}
  &u_1 = u_2 = 0, &\quad \text{ sur } \partial \Lambda \times
  (0,\thickness),\\
  &u_3 = 0, &\quad \text{ sur }  \Lambda \times \cparent{0,\thickness},\\
  &\mu_{13}\parent{\frac{\partial u_1}{\partial x_3} + \frac{\partial u_3}{\partial
      x_1}} = 0, &\quad \text{ sur }  \Lambda \times \cparent{0,\thickness},\\
  &\mu_{23}\parent{\frac{\partial u_2}{\partial x_3} + \frac{\partial u_3}{\partial
  x_2}} = 0, &\quad \text{ sur }  \Lambda \times \cparent{0,\thickness}.
\end{align}
Ici on a noté
\begin{equation}
  [\mu \otimes \straintensor(u)]_{i,j} %
  = \frac{1}{2}\mu_{i,j}\parent{\frac{\partial u_i}{\partial x_j} %
                              + \frac{\partial u_j}{\partial x_i}},%
  \quad 1\leq i,j\leq 3.
\end{equation}
En reprenant les notations introduites dans la section
(\ref{sec:fourier}), on obtient pour $\beta > 0$, et en omettant
l'indice du mode de Fourier $k$:
\begin{align}
  - 2\frac{\partial}{\partial x_1}%
  \parent{\mu_{1,1}\frac{\partial u_1}{\partial x_1}} %
  -  \frac{\partial}{\partial x_2}%
  \parent{\mu_{1,2}\parent{\frac{\partial u_1}{\partial x_2} %
                           + \frac{\partial u_2}{\partial x_1}}} %
  - \mu_{1,3}\parent{- \beta^2 u_1 %
                     + \beta\frac{\partial u_3}{\partial x_1}} %
  + \frac{\partial p}{\partial x_1} %
  &= f_{1}, \label{eq:stokes-fourier-visc-1}\\
  %
  - \frac{\partial }{\partial x_1}%
  \parent{\mu_{2,1}\parent{\frac{\partial u_2}{\partial x_1} %
                           + \frac{\partial u_1}{\partial x_2}}}
  - 2\frac{\partial}{\partial x_2}%
  \parent{\mu_{2,2}\frac{\partial u_2}{\partial x_2}}
  - \mu_{2,3}\parent{- \beta^2 u_2 %
                     + \beta\frac{\partial u_3}{\partial x_2}} %
  + \frac{\partial p}{\partial x_2} %
  &= f_{2}, \label{eq:stokes-fourier-visc-2}\\
  %
  - \frac{\partial}{\partial x_1}%
  \parent{\mu_{3,1}\parent{ \frac{\partial u_3}{\partial x_1} %
                           - \beta u_1}} %
  - \frac{\partial }{\partial x_2}%
  \parent{\mu_{3,2}\parent{ \frac{\partial u_3}{\partial x_2} %
                           - \beta u_2}} %
  + 2\mu_{3,3}\beta^2u_3 %
  - \beta p %
  &= f_{3}. \label{eq:stokes-fourier-visc-3}
\end{align}

En définissant mainenant le tenseur $3\times 3$ des déformations:
\begin{equation}\label{eq:stokes-fourier-strain-tensor-3d}
  E(u) = \frac{1}{2} \begin{bmatrix}
    2\displaystyle\frac{\partial u_1}{\partial x_1} %
    &  \displaystyle\frac{\partial u_1}{\partial x_1} + \displaystyle\frac{\partial u_2}{\partial x_1} %
    & - \beta u_1 + \displaystyle\frac{\partial u_3}{\partial x_1}\\
    %
    \displaystyle\frac{\partial u_1}{\partial x_2}+ \displaystyle\frac{\partial u_2}{\partial x_1} %
    & 2 \displaystyle\frac{\partial u_2}{\partial x_2} %
    & -\beta u_2 + \displaystyle\frac{\partial u_3}{\partial x_2} \\
    %
    - \beta u_1 + \displaystyle\frac{\partial u_3}{\partial x_1} %
    & - \beta u_2 + \displaystyle\frac{\partial u_3}{\partial x_2} & 2\beta u_3
  \end{bmatrix}
\end{equation}
et celui des viscosités:
\begin{equation}\label{eq:stokes-fourier-visc-tensor-3d}
\mu = \begin{bmatrix}
  \mu_{1,1} & \mu_{1,2} & \mu_{1,3} \\
  \mu_{2,1} & \mu_{2,2} & \mu_{2,3} \\
  \mu_{3,1} & \mu_{3,2} & \mu_{3,3}
\end{bmatrix}
\end{equation}
et en multipliant (\ref{eq:stokes-fourier-visc-1}),
(\ref{eq:stokes-fourier-visc-2}), (\ref{eq:stokes-fourier-visc-3}) par
$v_1$, $v_2$, $v_3$, puis en intégrant par partie, on obtient:
\begin{align}
  \begin{aligned}[b]
    \int_\Lambda 2\mu\otimes E(u):E(v)\,\mathrm dx_1\mathrm dx_2 %
    &- \int_\Lambda p\parent{\frac{\partial v_1}{\partial x_1}
                            + \frac{\partial v_2}{\partial x_2}}\,\mathrm dx_1\mathrm dx_2 \\
    &- \beta \int_\Lambda \parent{p + C}v_3\,\mathrm dx_1\mathrm dx_2
  \end{aligned}
  &= \int_\Lambda f\cdot v\,\mathrm dx_1\mathrm dx_2,\label{eq:stokes-fourier-visc-weak-1}\\
  %
  \int_\Lambda \parent{\frac{\partial u_1}{\partial x_1} %
    + \frac{\partial u_2}{\partial x_2}}q\,\mathrm dx_1\mathrm dx_2 %
  + \beta \int_\Lambda u_3 q \,\mathrm dx_1\mathrm dx_2 %
  &= 0,\label{eq:stokes-fourier-visc-weak-2}\\
  %
  \int_\Lambda u_3 \,\mathrm dx_1\mathrm dx_2 %
  &= 0.\label{eq:stokes-fourier-visc-weak-3}
\end{align}

\begin{remarque}
Le système d'équations
(\ref{eq:stokes-fourier-visc-weak-1})-(\ref{eq:stokes-fourier-visc-weak-3})
permet de traiter le cas ``historique'':
\begin{equation}
  \mu = \begin{bmatrix}
    10 & 10 & 0.5\\
    10 & 10 & 0.5\\
    0.5 & 0.5 & 1
  \end{bmatrix}.
\end{equation}
\end{remarque}

\begin{remarque}
Le mode fondamental s'obtient en prennant $\beta = 0$, i. e. en
cherchant $u\in H_0^1(\Lambda)^2$, $p\in L_0^2(\Lambda)$, tels que
pour tout $v\in H_0^1(\Lambda)^2$ et $q\in L_0^2(\Lambda)$:
\begin{align}
  \int_\Lambda 2\tilde\mu\otimes \tilde E(u):\tilde E(v)\,\mathrm dx_1\mathrm dx_2 %
  - \int_\Lambda p\parent{\frac{\partial v_1}{\partial x_1} %
                          + \frac{\partial v_2}{\partial x_2}}\,\mathrm dx_1\mathrm dx_2 %
  &= \int_\Lambda f\cdot v\,\mathrm dx_1\mathrm dx_2,\\
  %
  \int_\Lambda \parent{\frac{\partial u_1}{\partial x_1} %
    + \frac{\partial u_2}{\partial x_2}}q \,\mathrm dx_1\mathrm dx_2 %
  &= 0,
\end{align}
avec:
\begin{equation}\label{eq:stokes-fourier-strain-tensor-2d}
  \tilde E(u) = \frac{1}{2}\begin{bmatrix}
    2 \frac{\partial u_1}{\partial x_1} %
    & \frac{\partial u_1}{\partial x_2} + \frac{\partial u_2}{\partial x_1}\\
    %
    \frac{\partial u_1}{\partial x_2} + \frac{\partial u_2}{\partial x_1} %
    & 2 \frac{\partial u_2}{\partial x_2}
  \end{bmatrix}
\end{equation}
et:
\begin{equation}\label{eq:stokes-fourier-visc-tensor-2d}
\tilde \mu = \begin{bmatrix}
  \mu_{1,1} & \mu_{1,2}\\
  \mu_{2,1} & \mu_{2,2}
\end{bmatrix}.
\end{equation}
On trouvera ci-dessous une démonstration plus simple que celle faite
pour démontrer la proposition \ref{prop:existance-unicite-1} de la
section \ref{sec:stokes-fourier-weak}.
\end{remarque}

%\paragraph{Existence et unicité de la solution}
Tout d'abord on gardera les notations introduites dans
(\ref{eq:stokes-fourier-strain-tensor-3d}) et
(\ref{eq:stokes-fourier-strain-tensor-2d}) pour $E(u)$ et $\tilde E(u)$
respectivement. D'autre part on note toujours $\abs{E(u)}^2 =
\sum_{i,j = 0}^3E_{i,j}^2(u)$ et $\abs{\tilde E(u)}^2 = \sum_{i,j =
  0}^2\tilde E_{i,j}^2(u)$. Pour simplifier, on supposera que le
domaine $\Lambda$ est un rectangle de dimensions $L_1,L_2$,
c'est-à-dire $\Lambda = \cparent{(x_1, x_2) \setsuchthat
  0<x_1<L_1,\ 0<x_2<L_2}$.

\begin{lemme}\label{lem:lemme-1}
  Sous l'hypothèse $\frac{\partial u_1}{\partial x_1} +
  \frac{\partial u_2}{\partial x_2} + \beta u_3 = 0$, on a la
  relation:
  \begin{equation}
    \int_\Lambda \abs{E(u)}^2\,\mathrm dx_1\mathrm dx_2 %
    = \int_\Lambda \parent{\abs{\tilde E(u)}^2 %
      + \frac{1}{2}\parent{  \parent{\frac{\partial u_3}{\partial x_1}}^2 %
                           + \parent{\frac{\partial u_3}{\partial x_2}}^2 %
                           + \beta^2 u_1^2 %
                           + \beta^2 u_2^2}}\,\mathrm dx_1\mathrm dx_2.
  \end{equation}
\end{lemme}

\begin{proof}
  Le calcul de $\abs{E(u)}^2$ donne:
  \begin{equation}\label{eq:proof-details}
    \begin{split}
      \abs{E(u)}^2 = \abs{\tilde E(u)}^2 %
      + \frac{1}{2}\parent{  \parent{\frac{\partial u_3}{\partial x_1}}^2 %
                           + \parent{\frac{\partial u_3}{\partial x_2} %
                                     + \beta^2u_1^2 %
                                     + \beta^2 u_2^2}}\\
      %
      - \beta u_1 \frac{\partial u_3}{\partial x_1} %
      - \beta u_2 \frac{\partial u_3}{\partial x_2} %
      + \beta^2 u_3.
    \end{split}
  \end{equation}
  En intégrant par partie le terme $u_1 \frac{\partial u_3}{\partial x_1}$ on obtient:
  \begin{equation}
    \begin{split}
      \int_\Lambda u_1 \frac{\partial u_3}{\partial x_1}\,\mathrm dx_1\mathrm dx_2 %
      &= \int_0^{L_2}\mathrm dx_2\int_0^{L_1} %
           u_1 \frac{\partial u_3}{\partial x_1}\,\mathrm dx_1\\
      &= - \int_0^{L_2}\mathrm dx_2 \int_0^{L_1} %
           u_3 \frac{\partial u_1}{\partial x_1}\,\mathrm dx_1\\
      &= - \int_\Lambda %
           u_3 \frac{\partial u_1}{\partial x_1}\,\mathrm dx_1\mathrm dx_2.
    \end{split}
  \end{equation}
  De même on aura:
  \begin{equation}
    \int_\Lambda u_2 \frac{\partial u_3}{\partial x_2}\,\mathrm dx_1\mathrm dx_2 %
    = - \int_\Lambda u_3 \frac{\partial u_2}{\partial x_2}\,\mathrm dx_1\mathrm dx_2.
  \end{equation}
  En reprenant (\ref{eq:proof-details}), en intégrant sur $\Lambda$
  et en utilisant l'hypothèse $\frac{\partial u_1}{\partial x_1} +
  \frac{\partial u_2}{\partial x_2} + \beta u_3 = 0$, on obtient le
  résultat annoncé.
\end{proof}

\begin{lemme}\label{lem:lemme-2}
  Il existe une constante positive $\chi > 0$ telle que:
  \begin{equation}
    \chi \norm{\nabla u}_{L^2(\Lambda)} \leq \norm{E(u)}_{L^2(\Lambda)},
  \end{equation}
  pour tout $u\in H_0^1(\Lambda)^3$ qui satisfait $\frac{\partial
    u_1}{\partial x_1} + \frac{\partial u_2}{\partial x_2} + \beta u_3
  = 0$. Ici $\norm{\nabla u}^2_{L^2(\Lambda)} =
  \sum_{i,j=1}^3\norm{\frac{\partial u_i}{\partial
      x_j}}^2_{L^2(\Lambda)}$ et $\norm{E(u)}^2_{L^2(\Lambda)} =
  \int_\Lambda\abs{E(u)}^2\,\mathrm dx_1\mathrm dx_2$.
\end{lemme}

\begin{proof}
  Il est connu que l'inégalité de Korn en dimension 2 est vraie,
  i. e. l'existance d'une constante $\chi > 0$ qui satisfait:
  \begin{equation}
    \chi \sum_{i,j = 1}^2\norm{\frac{\partial u_i}{\partial x_j}}^2_{L^2(\Lambda)} %
    \leq \int_\Lambda \abs{\tilde E(u)}^2\,\mathrm dx_1\mathrm dx_2.
  \end{equation}
  Le lemme (\ref{lem:lemme-1}) permet de conclure.
\end{proof}

\begin{proposition}\label{prop:proposition-2}
  Si le tenseur $\mu$ satisfait:
  \begin{equation}\label{eq:prop-2-hypothesis}
    \mu_{i,j}(x_1, x_2) \geq \mu_0 \quad \forall (x_1, x_2)\in \Lambda,\ 1\leq i,j\leq 3,
  \end{equation}
  où $\mu_0 > 0$ est une constante positive indépendante de $(x_1,
  x_2)\in\Lambda$, alors le problème
  (\ref{eq:stokes-fourier-visc-weak-1})-(\ref{eq:stokes-fourier-visc-weak-3})
  admet une et une seule solution.
\end{proposition}

\begin{proof}
  Définissont l'espace $V = H_0^1(\Lambda)^2\times
  H_{0,0}^1(\Lambda)$ où $H_{0,0}^1(\Lambda) = H_0^1(\Lambda) \cap
  L_0^2(\Lambda)$. Clairement l'espace $H_{0,0}^1(\Lambda)$ est un
  sous-espace fermé de $H_0^1(\Lambda)$ de codimension 1. Soit
  encore $a:V\times V\to \mathbb R$ la forme bilinéaire continue
  définie par:
  \begin{equation}
    a(u,v) = \int_\Lambda 2\mu\otimes E(u):E(v)\,\mathrm dx_1\mathrm dx_2.
  \end{equation}
  Définissons encore la forme bilinéaire continue $b:V\times Y\to\mathbb R$, par:
  \begin{equation}
    b(u, q) = \int_\Lambda \parent{\frac{\partial u_1}{\partial x_1} %
      + \frac{\partial u_2}{\partial x_2} %
      + \beta u_3}q\,\mathrm dx_1\mathrm dx_2,
  \end{equation}
  où ici $Y = L_0^2(\Lambda)$.

  Il est facile de voir que le problème
  (\ref{eq:stokes-fourier-visc-weak-1})-(\ref{eq:stokes-fourier-visc-weak-3})
  est équivalent au problème de chercher $(u,p)\in V\times Y$ tel
  que:
  \begin{equation}
    \begin{split}
      a(u,v) - b(v, p) = \int_\Lambda f\cdot v\,\mathrm dx_1\mathrm dx_2,%
        \quad \forall v\in V,\\
      b(u,q) = 0,%
        \quad \forall q\in Y.
    \end{split}
  \end{equation}
  La constante $C$ peut s'obtenir a posteriori en considérant
  (\ref{eq:stokes-fourier-visc-weak-1})-(\ref{eq:stokes-fourier-visc-weak-3})
  avec les fonctions tests $v = (0,0,s)$ où $s\in H_0^1(\Lambda)$
  est dans l'orthogonal de $H_{0,0}^1(\Lambda)$.

  Pour démontrer la proposition \ref{prop:proposition-2}, il suffit
  de vérifier que la forme $a(.,.)$ est coercive sur $V_0$, où
  $V_0 = \cparent{v\in V\setsuchthat b(v,q) = 0\ \forall q\in Y}$, et
  que la condition classique inf--sup sur la forme bilinéaire $b$
  est satisfaite.

  Le lemme \ref{lem:lemme-2} avec l'hypothèse
  (\ref{eq:prop-2-hypothesis}) montrent bien que $a$ est coercive sur
  $V_0$. D'autre part en utilisant l'inégalité concernant la
  condition inf--sup dans $\mathbb R^2$ on a si $q\in L_0^2(\Lambda)$:
  \begin{equation}
    \begin{split}
      \sup_{\norm{v}_{H_0^1} = 1}b(v,q) %
      &\geq \sup_{\norm{(v_1, v_2, 0)}_{H_0^1}}b(v,q)\\
      &= \sup_{\norm{(v_1, v_2, 0)}_{H_0^1}}\int_\Lambda \parent{\frac{\partial v_1}{\partial x_1} %
        + \frac{\partial v_2}{\partial x_2}}q\,\mathrm dx_1\mathrm dx_2\\
      &\geq \delta \norm{q}_{L_0^2},
    \end{split}
  \end{equation}
  où $\delta > 0$. Ainsi on a prouvé la proposition \ref{prop:proposition-2}.
\end{proof}


\subsection{Un modèle de transport et diffusion d'alumine}\label{sec:harmonic-c}
L'objectif ultime étant d'obtenir une approximation de la
concentration d'alumine dans la cuve, on propose un modèle de
transport et diffusion d'alumine dissoute dans le domaine $\Omega$.

On suppose donné un champ de convection stationnaire $u:\Omega\to\mathbb R^3$,
qui correspond à l'écoulement des fluides dans le domaine
$\Omega$. Le champ $u$ est typiquement calculé à l'aide d'un
schéma numérique introduit dans la section
\ref{sec:harmonic-discretization}.

Soit $S$ le terme source de la concentration d'alumine dissoute
correspondant à l'injection de particules, $\dot
q$ le terme source correspondant à la consommation d'alumine
dissoute par la réaction d'électrolyse, et la diffusivité de la
concentration d'alumine dans le bain $D_C > 0$.

On cherche le champ de concentration d'alumine stationnaire
$c:\Omega\to\mathbb R$ solution de l'équation d'advection-diffusion:
\begin{equation}\label{eq:stat-concentration}
  u\cdot \nabla c - D_c \Delta c = S + \dot q\quad \text{dans } \Omega,
\end{equation}
avec des conditions de Neumann homogènes sur le bord $\partial
\Omega$:
\begin{equation}
  \frac{\partial c}{\partial n} = 0,\quad\text{sur } \partial \Omega.
\end{equation}
La solution $c$ de l'équation (\ref{eq:stat-concentration}) étant
définie à une constante près, on considère la contrainte
supplémentaire:
\begin{equation}
  \frac{1}{\abs{\Omega}}\int_\Omega c\,\mathrm dx = \bar c,
\end{equation}
où $\bar c$ est la concentration moyenne dans le domaine $\Omega$,
supposée donnée.

Le terme source $S$ est construit à partir des conditions initiales
$S_k$ des populations de particules d'alumines introduite dans la
section \ref{sec:math-np}. On définit $S$ comme la moyenne
temporelle du débit de masse de particules d'alumine définit par
les fonctions $S_k$:
\begin{equation}
  S(x_1,x_2,x_3) = \lim_{T\to\infty}\frac{1}{T}\int_0^T \sum_{k>0}
  S_k(x_1, x_2, x_3) \delta(t - \tau_k)\,\mathrm dt, \quad (x_1, x_2, x_3)\in\Omega.
\end{equation}

Le terme source $\dot q$ correspond à la consommation d'alumine
dissoute par la réaction d'électrolyse. Le débit total d'alumine
consommée dans la cuve étant donné par la relation
(\ref{eq:mass-consumption}), on pose:
\begin{equation}
  \dot q(x_1, x_2, x_3) = -\frac{IM}{6F\abs{\Omega}}.
\end{equation}

\begin{remarque}
  Par construction des fonctions $\cparent{S_k}_{k>0}$, la moyenne de $S$ sur
  $\Omega$ est exactement égale à la moyenne de $\dot q$ sur
  $\Omega$, i. e.:
  \begin{equation}
    \int_\Omega S + q\,\mathrm dx = 0,
  \end{equation}
  condition nécessaire pour que le problème
  (\ref{eq:stat-concentration}) admette une solution.
\end{remarque}

On utilisera pour approximer la solution du problème
(\ref{eq:stat-concentration}) une méthode d'éléments finis sur
un maillage tétraédrique de $\Omega$ telle que décrite dans
l'annexe \ref{chap:annexe-finite-element}.
