
%%% Math typography
\newcommand{\ie}{i.e.}
\renewcommand{\div}{\mathrm{div}}
\newcommand{\norm}[1]{\left\lVert #1 \right\rVert}
\newcommand{\parent}[1]{\left( #1 \right)}
\newcommand{\abs}[1]{\left\lvert #1 \right\rvert}
\newcommand{\cparent}[1]{\left\{ #1 \right\}}
\newcommand{\setsuchthat}{\ \mid|\ }


%%% Theorems
\theoremstyle{definition}
\newtheorem{proposition}{Proposition}
\newtheorem{lemme}{Lemme}
\newtheorem{remarque}{Remarque}

%%% Quantities
\newcommand{\temperature}{\Theta} % temperature du bain en fonction de
% x et t.
\newcommand{\temperaturer}{\tilde{\Theta}} % temperature du bain en fonction de x et t.
\newcommand{\tinj}{\temperature_\text{Inj}} % temperature d'injection des particules
\newcommand{\tliq}{\temperature_\text{Liq}} % temperature du liquidus du bain
\newcommand{\tinit}{\temperature_\text{Init}} % temperature initiale du bain
% (constant ou fonction de x)
\newcommand{\telectrolyte}{\temperature_\mathrm{e}} % temperature de
                                % l'electrolyte (liquide)

\newcommand{\electrolytedensity}{\rho_\mathrm{e}}
\newcommand{\electrolytehc}{C_{p,e}}
\newcommand{\electrolyteshc}{C_{p,e,s}}
\newcommand{\electrolytelhc}{C_{p,e,l}}
\newcommand{\fusionenthalpy}{\Delta H_\mathrm{sl}}
\newcommand{\electrolytetdiff}{D_\temperature}
\newcommand{\electrolytestdiff}{D_{\temperature,s}}
\newcommand{\electrolyteltdiff}{D_{\temperature,l}}

\newcommand{\concentration}{c}
\newcommand{\enthalpy}{H}
\newcommand{\enthalpydensity}{h}
\newcommand{\enthalpydensityr}{\tilde{h}}

\newcommand{\tlat}{T_\text{Lat}}


\newcommand{\rplus}{\mathbb R_+}
\newcommand{\nstar}{\mathbb N^*}

\newcommand{\intd}[1]{\,\mathrm d#1}

\chapter*{Notations}
\addcontentsline{toc}{chapter}{Notations}

\begin{tabularx}{\textwidth}{@{}ll@{}}
  \toprule
  Symbol & Description \\
  \midrule

  \bottomrule
\end{tabularx}
