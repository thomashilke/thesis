
%%% Math typography
\newcommand{\ie}{i.e.}
\renewcommand{\div}{\mathrm{div}}
\newcommand{\norm}[1]{\left\lVert #1 \right\rVert}
\newcommand{\parent}[1]{\left( #1 \right)}
\newcommand{\jump}[1]{\left[ #1 \right]}
\newcommand{\abs}[1]{\left\lvert #1 \right\rvert}
\newcommand{\cparent}[1]{\left\{ #1 \right\}}
\newcommand{\setsuchthat}{\ \mid|\ }
\newcommand{\rplus}{\mathbb R_+}
\newcommand{\nstar}{\mathbb N^*}
\newcommand{\intd}[1]{\,\mathrm d#1}

%%% Theorems
\theoremstyle{definition}
\newtheorem{proposition}{Proposition}
\newtheorem{lemme}{Lemme}
\newtheorem{remarque}{Remarque}

%%% Quantities
\newcommand{\temperature}{\Theta} % temperature du bain en fonction de
% x et t.
\newcommand{\temperaturer}{\tilde{\temperature}} % temperature du bain en fonction de x et t.
\newcommand{\tinj}{\temperature_\text{Inj}} % temperature d'injection des particules
\newcommand{\tliq}{\temperature_\text{Liq}} % temperature du liquidus du bain
\newcommand{\tinit}{\temperature_\text{Init}} % temperature initiale du bain
\newcommand{\tinitr}{\tilde\temperature_\text{Init}} % temperature initiale du bain en coordonnee spheriques
% (constant ou fonction de x)
\newcommand{\telectrolyte}{\temperature_\mathrm{e}} % temperature de
                                % l'electrolyte (liquide)

\newcommand{\aluminahc}{C_{\mathrm{p,Al}}}
\newcommand{\aluminadensity}{\rho_\mathrm{Al}}

\newcommand{\electrolytedensity}{\rho_\mathrm{e}}
\newcommand{\electrolytehc}{C_{\mathrm{p,e}}}
\newcommand{\electrolyteshc}{C_{\mathrm{p,e,1}}}
\newcommand{\electrolytelhc}{C_{\mathrm{p,e,2}}}
\newcommand{\fusionenthalpy}{\Delta H_\mathrm{sl}}
\newcommand{\electrolytetdiff}{D_\temperature}
\newcommand{\electrolytestdiff}{D_{\temperature,\mathrm{1}}}
\newcommand{\electrolyteltdiff}{D_{\temperature,\mathrm{2}}}

\newcommand{\concentration}{c}
\newcommand{\enthalpy}{H}
\newcommand{\enthalpydensity}{H}
\newcommand{\enthalpydensityr}{\tilde{\enthalpydensity}}

\newcommand{\tlat}{T_\text{Lat}}

\newcommand{\Omegae}{{\Omega_\text{e}}}
\newcommand{\Omegam}{{\Omega_\text{m}}}
\newcommand{\rhoal}{{\rho_\text{Al}}}
\newcommand{\rhoe}{{\rho_\text{e}}}
\newcommand{\rhom}{{\rho_\text{m}}}
\newcommand{\rmax}{R_\mathrm{Max}}


\chapter*{Notations}
\addcontentsline{toc}{chapter}{Notations}

\begin{tabularx}{\textwidth}{@{}ll@{}}
  \toprule
  Symbol & Description \\
  \midrule

  \bottomrule
\end{tabularx}
