
%%% Math typography
\newcommand{\ie}{i.e.}
\renewcommand{\div}{\mathrm{div}}
\newcommand{\norm}[1]{\left\lVert #1 \right\rVert}
\newcommand{\parent}[1]{\left( #1 \right)}
\newcommand{\jump}[1]{\left[ #1 \right]}
\newcommand{\abs}[1]{\left\lvert #1 \right\rvert}
\newcommand{\cparent}[1]{\left\{ #1 \right\}}
\newcommand{\setsuchthat}{\ \mid|\ }
\newcommand{\rplus}{\mathbb R_+}
\newcommand{\nstar}{\mathbb N^*}
\newcommand{\intd}[1]{\,\mathrm d#1}
\newcommand{\dt}{\Delta t}
\newcommand{\dr}{\Delta r}
\newcommand{\heaviside}{H}

%%% Theorems
\theoremstyle{definition}
\newtheorem{proposition}{Proposition}
\newtheorem{lemme}{Lemme}
\newtheorem{remarque}{Remarque}


%%%
\newcommand{\thetime}{t}
\newcommand{\timestep}{\Delta t}


%%% Quantities
\newcommand{\temperature}{\Theta} % temperature du bain en fonction de
% x et t.
\newcommand{\temperaturer}{\tilde{\temperature}} % temperature du bain en fonction de x et t.
\newcommand{\tinj}{\temperature_\text{Inj}} % temperature d'injection des particules
\newcommand{\tliq}{\temperature_\text{Liq}} % temperature du liquidus du bain
\newcommand{\tinit}{\temperature_\text{Init}} % temperature initiale du bain
\newcommand{\tinitr}{\tilde\temperature_\text{Init}} % temperature initiale du bain en coordonnee spheriques
\newcommand{\tsur}{\temperature_\text{Sur}} % temperature initiale du bain
% (constant ou fonction de x)
\newcommand{\telectrolyte}{\temperature_\mathrm{e}} % temperature de
                                % l'electrolyte (liquide)

\newcommand{\aluminahc}{C_{\mathrm{p,Al}}}
\newcommand{\aluminadensity}{\rho_\mathrm{Al}}

\newcommand{\electrolytedensity}{\rho_\mathrm{e}}
\newcommand{\electrolytehc}{C_{\mathrm{p,e}}}
\newcommand{\electrolyteshc}{C_{\mathrm{p,e,1}}}
\newcommand{\electrolytelhc}{C_{\mathrm{p,e,2}}}
\newcommand{\fusionenthalpy}{\Delta H_\mathrm{sl}}
\newcommand{\electrolytetdiff}{D_\temperature}
\newcommand{\electrolytestdiff}{D_{\temperature,\mathrm{1}}}
\newcommand{\electrolyteltdiff}{D_{\temperature,\mathrm{2}}}
\newcommand{\electrolytecdiff}{D_{\concentration}}
\newcommand{\electrolytemoldiff}{D^{\mathrm M}}
\newcommand{\electrolyteturbdiff}{D^{\mathrm T}}

\newcommand{\faradayyield}{\eta_\mathrm{F}}

\newcommand{\concentration}{c}
\newcommand{\enthalpy}{H}
\newcommand{\enthalpydensity}{H}
\newcommand{\enthalpydensityr}{\tilde{\enthalpydensity}}

\newcommand{\tlat}{\temperature_\text{Lat}}

\newcommand{\Omegae}{{\Omega_\text{e}}}
\newcommand{\Omegam}{{\Omega_\text{m}}}
\newcommand{\rhoal}{{\rho_\text{Al}}}
\newcommand{\rhoe}{{\rho_\text{e}}}
\newcommand{\rhom}{{\rho_\text{m}}}
\newcommand{\rmax}{R_\mathrm{Max}}

\newcommand{\dissolutionrate}{\kappa}
\newcommand{\csat}{\concentration_\text{Sat}}
\newcommand{\csatwp}{\concentration_{\text{Sat},\%\mathrm w}}
\newcommand{\cinit}{\concentration_\text{Init}}
\newcommand{\cinitwp}{\concentration_{\text{Init},\%\mathrm w}}
\newcommand{\tcrit}{\temperature_\text{Crit}}

\newcommand{\population}{n_p}
\newcommand{\dragforce}{F_{\mathrm D}}
\newcommand{\gravityforce}{F_{\mathrm g}}
\newcommand{\buoyancyforce}{F_{\mathrm A}}

\newcommand{\reynolds}{R_{\mathrm e}}
\newcommand{\electrolyteviscosity}{\mu}
\newcommand{\electrolyteturbviscosity}{\mu}

\newcommand{\tend}{T}
\newcommand{\faraday}{F}
\newcommand{\cdiffusivity}{D_\concentration}
\newcommand{\dirac}{\delta}
\newcommand{\kronecker}{\delta}
\newcommand{\aluminadissolutionenthalpy}{\Delta H_\text{Diss}}
\newcommand{\conductivity}{\sigma}
\newcommand{\electrolyteturbulentviscosity}{\mu_\mathrm{T}}
\newcommand{\electrolytelaminarviscosity}{\mu_\mathrm{L}}

\newcommand{\stresstensor}{\mathcal T} % Tenseur des contraintes visqueuses
\newcommand{\straintensor}{\mathcal E} % Tenseur du taux de
% deformation (symetrique)
\newcommand{\reducedstraintensor}{\overline{\mathcal E}} % Tenseur du
                                % taux de deformation (symetrique)
                                % reduit au premier block 2x2
\newcommand{\thickness}{\varepsilon}

\chapter*{Notations}
\addcontentsline{toc}{chapter}{Notations}

\begin{tabularx}{\textwidth}{@{}ll@{}}
  \toprule
  Symbol & Description \\
  \midrule
  $t$ & Temps \\
  $\timestep$ & Pas de temps \\
  $t^n$ & $n$-ième pas de temps\\
  $T$ & Temps final \\
  $N$ & Nombre de pas de temps \\
  $r$ & Rayon d'une particule ou d'une collection de particules d'alumine \\
  $r_0$ & Rayon initial d'une particule \\
  $M$ & Nombre de discrétisations du rayon des particules \\
  $ [g]_\Gamma $ & Saut d'une fonction $g$ sur une interface $\Gamma$\\
  $\Omega$ & Domaine ouvert de $\mathbb R^3$ \\
  $\partial\Omega$ & Bord d'un domaine ouvert $\Omega$\\
  $\mathcal B(x, r)$ & Boule de rayon $r$ et de centre $x$\\
  $\Gamma$ & Surface dans $\mathbb R^3$\\
  $\Gamma_\mathrm{N}$ & Partie du bord d'un ouvert sur laquelle sont imposées des conditions aux limite de Neumann\\
  $\Gamma_\mathrm{D}$ & Partie du bord d'un ouvert sur laquelle sont imposées des conditions aux limite de Dirichlet\\
  $\nu$ & Vecteur normal au bord d'un ouvert\\
  $\temperature$ & Température des fluides dans une cuve\\
  $H$ & Enthalpie dans l'alumine ou l'électrolyte environnant \\
  $\beta$ & Relation entre la densité d'enthalpie et la température locale de l'électrolyte\\
  $L_\beta$ & Constante de Lipschitz de la fonction $\beta$ \\
  $\mu$ & Paramètre de relaxation \\
  $f_s$ & Fraction solide \\
  $\tinj$ & Température des particules au moment de l'injection\\
  $\tliq$ & Température du liquidus du bain électrolytique \\
  $\tinit$ & Température initiale des fluides\\
  $\tcrit$ & Température critique de dissolution de l'alumine\\
  $\tsur$ & Température de surchauffe de l'électrolyte\\
  $\telectrolyte$ & Température de l'électrolyte\\
  $D_\temperature$ & Diffusivité thermique\\
  $D_\concentration$ & Diffusivité moléculaire de l'électrolyte\\
  $\rhoe$ & Densité de l'électrolyte\\
  $\rhoal$ & Densité de l'alumine\\
  \bottomrule
\end{tabularx}


\begin{tabularx}{\textwidth}{@{}ll@{}}
  \toprule
  Symbol & Description \\
  \midrule
  $C_\mathrm{p,Al}$ & Chaleur spécifique de l'alumine\\
  $C_\mathrm{p,e}$ & Chaleur spécifique de l'électrolyte\\
  $\fusionenthalpy$ & Enthalpie de transition de phase solide-liquide\\
  $\tlat$ & Temps de latence avant dissolution des particules d'alumine\\
  $\faradayyield$ & Rendement de Faraday\\
  $I$ & Courant électrique total traversant une cuve\\
  $\faraday$ & Constante de Faraday\\
  $\concentration$ & Concentration molaire d'alumine dissoute dans l'électrolyte\\
  $\csat$ & Concentration molaire de saturation de l'alumine dissoute dans l'électrolyte\\
%  $\csatwp$ & \\
  $\cinit$ & Concentration molaire initial d'alumine dissoute dans l'électrolyte\\
  $f(r, \concentration,\temperature)$ & Vitesse de dissolution d'une particule d'alumine dans le bain\\
  $\kappa(\concentration, \temperature)$ & Taux de dissolution d'une particule d'alumine dans le bain\\
  $K$ & Taux limite de dissolution l'une particule d'alumine dans le bain \\
  $\population$ & Densité de particule d'alumine dans l'électrolyte en fonction de leur rayon\\
%  $n_{p,0}$ & \\
  $F_\mathrm{D}$ & Force de trainée exercée par un fluide visqueux sur un objet en mouvement relatif\\
  $F_\mathrm{g}$ & Force de gravité\\
  $F_\mathrm{A}$ & Force d'Archimède\\
  $\mu_\mathrm{T}$ & Viscosité turbulente\\
  $\mu_\mathrm{L}$ & Viscosité dynamique\\
  $g$ & Accélération de la gravité \\
  $v_0$ & Vitesse verticale initiale d'une particule d'alumine \\
  $w(r)$ & Vitesse limite de chute d'une sphère de rayon $r$\\
  $R_\mathrm{e}$ & Nombre de Reynolds \\
  $j$ & Densité de courant électrique \\
  $u$ & Vitesse d'écoulement du bain électrolytique \\
  $K$ & Nombre total d'injections \\
  $\tau^k$ & Temps d'injection \\
  $S^k$ & Densité initiale de particules pour l'injection $k$ \\
  $n_p^k(t, x, r)$ & Densité de particules dans l'électrolyte issue de l'injection $k$\\
  $\bar{k}(t)$ & Indice de la dernière population à se dissoudre au temps $t$ \\
  $p^k$ & Indice du dernier pas de temps qui précède l'injection $k$\\
  $q^k$ & Indice du dernier pas de temps qui précède le début de la dissolution de l'injection $k$\\
  $\delta(t)$ & Masse de Dirac \\
  $\delta_{i,j}$ & Symbol de Kronecker \\
%  $H$ & Fonction de Heaviside \\
  $\sigma$ & conductivité électrique du bain\\
  $q_1$ & Alumine dissoute consommée par l'électrolyse\\
  $q_2$ & Source d'alumine provenant de la dissolution des particules\\
  $p_1$ & Chaleur absorbée pour réchauffer les particules après leur injection\\
  $p_2$ & Chaleur absorbée par la réaction de dissolution de l'alumine\\
  $p_3$ & Source de chaleur liée à l'effet Joule\\
  \bottomrule
\end{tabularx}

\begin{tabularx}{\textwidth}{@{}ll@{}}
  \toprule
  Symbol & Description \\
  \midrule
  $D_c^M$ & Constante de diffusivité moléculaire de l'alumine dissoute dans l'électrolyte\\
  $D_c^T$ & Constante de diffusivité turbulente de l'alumine dissoute dans l'électrolyte\\
  $D_\temperature^M$ & Constante de diffusivité thermique de l'électrolyte\\
  $D_\temperature^T$ & Constante de diffusivité thermique de l'électrolyte issue des turbulences\\
  $\Lambda$ & Domaine ouvert de $\mathbb R^2$\\
  $\mathcal T_{i,j}$ & Tenseur des contraintes visqueuses \\
  $\mathcal E_{i,j}$ & Tenseur des déformations du fluide \\
  $\bar{\mathcal E}$ & Tenseur des déformations restreint aux composantes 1 et 2 \\
  $f$ & Force exercée sur le fluide \\
  $p$ & Pression hydrostatique \\
  $\mathcal M_h$ & Maillage conforme en simplexes de diamètre maximal $h$ d'un domaine ouvert de $\mathbb R^2$ ou $\mathbb R^3$\\
  $\delta$ & Coefficient de stabilisation \\
  $h$ & Taille de maille \\
  $\thickness$ & Epaisseur de la couche de fluide \\
  $H^1(\Omega)$ & L'ensemble $W^{1, 2}(\Omega)$\\
  $L^2(\Omega)$ & L'ensemble $W^{0, 2}(\Omega)$\\
  $L^2_0(\Omega)$ & L'ensemble $\cparent{f \in L^2(\Omega) \mid \displaystyle\int_\Omega f\,\mathrm dx = 0}$\\
  $H^1_0(\Omega)$ & L'ensemble $\cparent{f \in H^1(\Omega) \mid \displaystyle\int_\Omega f\,\mathrm dx = 0}$\\
  $V_h$ & Approximation de l'espace $H^1(\Omega)$ par des fonctions linéaires par morceaux\\
  $B_h$ & Espace $V_h$ enrichi par une fonction bulle\\
  \bottomrule
\end{tabularx}
