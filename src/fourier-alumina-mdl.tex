L'objectif ultime étant d'obtenir une approximation de la
concentration d'alumine dans la cuve, on propose un modèle de
transport et diffusion stationnaire d'alumine dissoute dans le domaine
$\Omega$.  On suppose donné un champ de convection stationnaire
$u:\Omega\to\mathbb R^3$, qui correspond à l'écoulement des fluides
dans le domaine $\Omega$. Le champ $u$ sera typiquement calculé à
l'aide du schéma numérique introduit dans la section
\ref{sec:fourier-discretisation}.  Soit $S$ le terme source de la
concentration d'alumine dissoute correspondant à l'injection de
particules, $\dot q$ le terme source correspondant à la consommation
d'alumine dissoute par la réaction d'électrolyse, et la diffusivité
moléculaire de la concentration d'alumine dans le bain $D_C^M > 0$.
On cherche le champ de concentration d'alumine stationnaire
$c:\Omega\to\mathbb R$ solution de l'équation d'advection-diffusion
\begin{equation}\label{eq:stat-concentration}
  u\cdot \nabla c - D_c^M \Delta c = S + \dot q\quad \text{dans } \Omega,
\end{equation}
avec des conditions de Neumann homogènes sur le bord $\partial
\Omega$:
\begin{equation}
  \frac{\partial c}{\partial n} = 0,\quad\text{sur } \partial \Omega.
\end{equation}
La solution $c$ de l'équation (\ref{eq:stat-concentration}) étant
définie à une constante près, on considère la contrainte
supplémentaire:
\begin{equation}\label{eq:stat-alumina-avg-c}
  \frac{1}{\abs{\Omega}}\int_\Omega c\,\mathrm dx = \bar c,
\end{equation}
où $\bar c$ est la concentration moyenne dans le domaine $\Omega$,
supposée donnée.
Le terme source $S$ est construit à partir des conditions initiales
$S_k$ des populations de particules d'alumines introduite dans la
section \ref{sec:populations-model}. On définit $S$ comme la moyenne
temporelle du débit de masse de particules d'alumine définit par
les fonctions $S_k$:
\begin{equation}
  S(x_1,x_2,x_3) = \lim_{T\to\infty}\frac{1}{T}\int_0^T \sum_{k>0}
  S_k(x_1, x_2, x_3) \delta(t - \tau_k)\,\mathrm dt, \quad (x_1, x_2, x_3)\in\Omega.
\end{equation}

Le terme source $\dot q$ correspond à la consommation d'alumine
dissoute par la réaction d'électrolyse. Le débit total d'alumine
consommée dans la cuve étant donné par la relation
(\ref{eq:mass-consumption}), on pose:
\begin{equation}\label{eq:stat-alumina-consommation}
  \dot q(x_1, x_2, x_3) = -\frac{I}{6\faraday\abs{\Omega}},
\end{equation}
où $I$ est le courant électrique total qui traverse la cuve, et
$\faraday$ la constante de Faraday.

\begin{remarque}
  Par construction des fonctions $\cparent{S_k}_{k>0}$, la moyenne de $S$ sur
  $\Omega$ est exactement égale à la moyenne de $\dot q$ sur
  $\Omega$, i. e.:
  \begin{equation}
    \int_\Omega \parent{S + q}\,\mathrm dx = 0.
  \end{equation}
  C'est une condition nécessaire pour que le problème
  (\ref{eq:stat-concentration}) admette une solution.
\end{remarque}

On utilisera pour approcher la solution du problème
(\ref{eq:stat-concentration}) une méthode d'éléments finis continus
linéaires par morceaux, stabilisée par une méthode de type SUPG, sur un
maillage tetraédrique de $\Omega$.
