\makeatletter
\@openrightfalse
\makeatother


\chapter*{Abstract}
\addcontentsline{toc}{chapter}{Abstract}

Metallic aluminium plays a key role in our modern economy. Primary
metallic aluminium is produced by the transformation of aluminium
oxide with the Hall-Héroult industrial process. This process, which
requires enormous energy quantities, consists in performing the
electrolysis of an aluminium oxide solute in large pots and with
hundreds of thousands of amperes of electrical current.

The topic of this thesis is the study of some selected aspects of the
modelisation of the electrolysis process from the point of view of
numerical simulation. This thesis is split in two parts.

The first part is focused on the numerical modelisation of the alumina
powder dissolution and transport in the electrolytic bath as a
function the bath temperature. We provide a mathematical model for the
transport and dissolution of the alumina powder, followed by its time
and space discretisation by means of a finite element
method. Eventually, we study the behavior of this numerical model in
the case of an industrial electrolysis pot.

The second part devoted to the development of a numerical scheme for
the approximation of the fluid flows in an electrolysis pot. The
scheme on a Fourier basis decomposition of the unknowns. The amplitude
of each Fourier componant satisfy a partial differential equation
which is explicitely derived. The solution of this equation is
approximated by means of a finite element method. Eventually, the
approximate fluid flow obtained with this novel method is compared
with the solution provided by the reference model in an industrial
electrolysis pot.

\paragraph{Keywords}
Numerical simulations,
Finite element methods,
Partial differential equations,
Advection-diffusion equation,
Electrolysis,
Alumina,
Dissolution,
Temperature.


\makeatletter
\@openrighttrue
\makeatother
