Dans cette section, nous nous intéressons à la modélisation de la
dissolution d'une unique particule d'alumine dans un bain
électrolytique. On suppose que cette particule, après avoir été
injectée à température $\tinj$, s'est réchauffée et se trouve à la
même température $\temperature$ que le bain environnant.

La problématique de la dissolution de l'alumine dans le bain d'une
cuve d'électrolyse d'aluminium est un phénomène complexe. Ce sujet a
déjà fait l'objet de nombreux travaux de recherche, voir par exemple
\cite{Dassylva2015}, \cite{Kvande1986}, \cite{Gerlach1975},
\cite{Solheim1995}.

La taille initiale des particules qui forme une dose est comprise pour
la plupart entre $\num{20}\si{\micro\meter}$ et
$\num{100}\si{\micro\meter}$ \cite{Fini2017}. Si les particules sont
trop fines celles-ci deviennent suffisamment volatiles pour être
soufflées loin du trou d'injection par les gaz s'en échappant. Si
elles atteignent malgré tout l'électrolyte, celui-ci a du mal à les
mouiller et à permettre leur dissolution. A l'inverse, des particules
trop grandes sont trop longues à dissoudre et risquent de sédimenter
au fond de la cuve.

Les particules d'alumine sont le résultat de l'agglomération de
micro-cristaux et peuvent présenter une porosité importante, jusqu'à
\num{75}\%. Cette porosité joue un rôle sur la capacité du bain à
mouiller la surface de la particule et à permettre sa dissolution.

L'oxyde d'aluminium a une forte affinité avec l'eau
\cite{Patterson2001}. Lorsque les particules entrent en contact avec
l'électrolyte en fusion, le gaz qui résulte de la désorption des
molécules d'eau crée une forte agitation qui favorise la dispersion et
la dissolution des particules. Par exemple, Haverkamp et
al. \cite{Haverkamp1994} ont montré que le temps de dissolution de
particules d'alumine hydratées peut être jusqu'à \num{40}\% plus
faible par rapport à des particules sèches. La formation d'agrégats
et de radeaux est également réduite.

La dissolution de l'oxyde d'aluminium dans l'électrolyte est une réaction
endothermique. La vitesse de cette réaction dépend de la chimie du
bain, \ie, des concentrations respectives des différentes espèces qui
le constituent et de la surchauffe du bain à proximité de la
particule. Au niveau microscopique, deux mécanismes entrent en
compétition au voisinage de la surface de la particule d'alumine. Il y a
d'une part la réaction de dissolution elle-même qui consiste à
détacher une molécule d'\ce{Al2O3} et à former différents complexes
avec les ions \ce{AlF6^{3-}} \cite{Haupin1995}, \cite{Kvande1986}, et
d'autre part le mécanisme de diffusion moléculaire qui transporte
l'alumine dissoute loin de la particule. La diffusion moléculaire est
essentielle pour maintenir la concentration à la surface de la
particule inférieure à la concentration de saturation $\csat$ et
permettre la réaction de dissolution. Clairement, si la concentration
est saturée dans l'ensemble du bain, la dissolution ne peut pas avoir
lieu. La réaction de dissolution étant elle-même endothermique, elle
est contrôlée par la disponibilité d'énergie sous forme thermique à
proximité de la particule, \ie, de la surchauffe du bain.

Dans cette section, on propose un modèle qui décrive la vitesse de
dissolution d'une unique particule d'alumine dans le bain d'une cuve
d'électrolyse d'aluminium, basé sur le travail de thèse de T. Hofer
\cite{Hofer2011}.

On fait l'hypothèse que les conditions d'exploitation de la cuve
d'électrolyse varient peu au cours du temps, c'est-à-dire que la
concentration des différentes espèces chimiques sont constantes au
cours du temps, à l'exception de la concentration d'alumine dissoute. On
suppose que la concentration d'alumine reste dans un intervalle qui
permette de négliger les variations de la température du liquidus
$\tliq$ \cite{Skybakmoen1997}. De même, on suppose des variations de températures
suffisamment faible pour se permettre de négliger son influence sur
la concentration de saturation de l'alumine $\csat$.

Enfin, on négligera la possible hydratation de l'alumine qui constitue
la particule et son influence sur sa dissolution puisqu'en pratique
celle-ci est généralement maintenue inférieure à \num{1}\% et que sont
effet est pricipalement d'aider à la dispersion des particules
\cite{Fini2017}.

On suppose que cette particule est une sphère parfaite, non poreuse,
et qu'elle est complètement mouillée par le bain, \ie, que la totalité
de sa surface soit en contact avec de l'électrolyte. Dans le travail
\cite{Hofer2011}, la vitesse de dissolution d'une particule est une
fonction $f$ qui décrit la vitesse de variation de son rayon $r$, et
qui dépend d'une part de $r$ et d'autre part de la concentration
locale d'alumine dissoute $c$. Dans le présent travail, nous
considèrerons en plus l'influence de la température locale de
l'électrolyte $\temperature$ sur sa vitesse de dissolution,
c'est-à-dire que
\begin{equation}\label{eq:radius-edo}
  \frac{\mathrm dr}{\mathrm dt} = f(r, \concentration, \temperature).
\end{equation}
On propose d'écrire la fonction $f$ comme
\begin{equation}\label{eq:dissolution-velocity}
  f(r, c, \temperature) = -\frac{\dissolutionrate(c, \temperature)}{r},
\end{equation}
où $\dissolutionrate$ est le taux de dissolution qui modélise les
aspects liés à la réaction de dissolution uniquement. Le facteur $1/r$
est lié à la géométrie sphérique de la particule et à l'hypothèse que
la diffusion de la concentration contrôle la dissolution, comme l'ont
montré Wang et Flanagan \cite{Wang1999}.

On suppose que la concentration de saturation $\csat$ est donnée, de
même que la température du liquidus $\tliq$. Clairement, le taux de
dissolution $\dissolutionrate$ doit être maximal lorsque $c = 0$, et
s'annuler lorsque l'électrolyte est saturé en alumine dissoute, \ie, lorsque $c
= \csat$. De même, lorsque le bain se trouve à la température du
liquidus $\tliq$, il n'y a pas d'énergie thermique à disposition pour
la réaction de dissolution, et donc $\dissolutionrate$ doit s'annuler
si $\temperature = \tliq$. A l'inverse, $\dissolutionrate$ doit être
maximal lorsque $\temperature\gg\tliq$.

Motivé par ces dernières considérations, on propose la forme suivante
pour l'expression du taux de dissolution:
\begin{equation}\label{eq:dissolution-rate}
  \dissolutionrate(\concentration, \temperature) =
  \left\{
  \begin{array}{ll}
    K\displaystyle\frac{\csat - \concentration}{\csat}\parent{1 -
      \exp\parent{-\displaystyle\frac{\temperature - \tliq}{\tcrit -
          \tliq}}} & \text{si }\temperature \geq \tliq\text{ et }
    0\leq c
    \leq \csat,\\
    0                                     & \text{sinon,}
  \end{array}
  \right.
\end{equation}
où $K > 0$ et $\tcrit > \tliq$ sont deux paramètres du modèle. Le taux
de dissolution limite $K$ est tel que
\begin{equation}
  \lim_{\temperature \to\infty}\dissolutionrate(0, \temperature) = K.
\end{equation}
Le paramètre $\tcrit$ caractérise l'intervalle dans lequel la
température à une influence sur la vitesse de dissolution. La figure
\ref{fig:diss-rate} illustre la relation entre le taux de dissolution
et la température dans le cas où la concentration locale
$\concentration = 0$ et le taux de dissolution limite $K = 1$.

\begin{figure}[h]
  \begin{center}
    \input{../media/particles/diss-rate/diss-rate.tex}
    \caption{Relation entre la température $\temperature$ et le taux de
      dissolution $\dissolutionrate$ lorsque $c = 0$ et que le taux
      de dissolution limite $K = 1$. En vert sont les traits de
      construction, et donnent une interprétation du paramètre
      $\tcrit$.}
    \label{fig:diss-rate}
  \end{center}
\end{figure}

La vitesse de dissolution $f$ décrite ci-dessus forme la base qui
permet la description de l'évolution d'une population constituée de
particules de tailles différentes dans un bain électrolytique, et qui
fait l'objet de la section suivante.
