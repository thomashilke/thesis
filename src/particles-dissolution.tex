Dans cette section, nous nous intéressons à la modélisation de
la dissolution d'une particules d'alumine individuelle dans un bain
électrolytique. On suppose que cette particule, après avoir été
injectée à température $\tinj$, s'est maintenant réchauffée et se
trouve à la même température $\temperature$ que le bain environnant.

On donne ci-dessous un aperçu de la compléxité du phénomène de
dissolution des particules d'alumine dans le bain
électrolytique. C'est un sujet qui a déjà fait l'objet de nombreux
travaux de recherche, voir par exemple \cite{Dassylva2015},
\cite{Kvande1986}, \cite{Gerlach1975}, \cite{Solheim1995}. La
taille initiale des particules qui forme une dose est comprise pour la
plupart entre $\num{20}\si{\micro\meter}$ et
$\num{100}\si{\micro\meter}$. Ces particules sont seulement
approximativement sphériques. Elles sont le résultat de
l'agglomération de micro-cristaux et peuvent présenter une porosité
importante, jusqu'à \num{75}\%. Cette porosité joue un rôle sur la
capacité du bain à mouiller la surface de la particule, et à permettre
sa dissolution. De même, si les particules sont suffisamment fines
celle-ci deviennent volatiles et sont facilement soufflées loin du
trou d'injection par les gaz s'en échappant. Si ces particules fines
atteignent malgré tout l'électrolyte, celui-ci a du mal à les mouiller
et a permettre leur dissolution. A l'inverse, des particules trop
grandes sont trop longues à dissoudre et risquent de sédimenter au
fond de la cuve.

L'alumine a une forte affinité avec l'eau
\cite{Patterson2001}. Lorsque les particules entrent en contact avec
le l'électrolyte en fusion, le gaz qui résulte de la désorption des
molécule d'eau crée une forte agitation qui favorise la dispersion et
la dissolution des particules. Par exemple, Haverkamp et
al. \cite{Haverkamp1994} ont montré que le temps de dissolution de
particules d'alumine hydratées peut être jusqu'à \num{40}\% plus
faible par rapport à des particules sèches.

La dissolution de l'oxyde d'aluminium l'électrolyte est une réaction
endothermique. La vitesse de cette réaction dépend de la chimie du
bain, \ie, des concentrations respectives des différentes espèces qui
le constitue, et de la surchauffe du bain à proximité et de la
particule. La température du liquidus $\tliq$, dont dépend la
surchauffe, est elle-même fonction de la concentration d'alumine
dissoute et des autres espèces chimiques. La concentration de
saturation de l'alumine $\csat$ dépend aussi de la température de
l'électrolyte.

Finalement, l'écoulement des fluide à proximité de la surface de la
particule peut influencer sa dissolution. Au niveau microscopique,
deux mécanismes entrent en compétition au niveau de la surface de la
particule d'alumine. Il y a d'une part la réaction de dissolution
elle-même qui consiste à détacher une molécule d'\ce{Al2O3} et à
former différents complexes avec les ions \ce{AlF6^{3-}}
\cite{Haupin1995}, \cite{Kvande1986}, et d'autre part le mécanisme de
diffusion moléculaire qui transporte l'alumine dissoute loin de la
particule. La diffusion moléculaire est essentielle pour maintenir la
concentration à la surface de la particule inférieure à la
concentration de saturation $\csat$ et permettre la réaction de
dissolution. Clairement, si la concentration est saturée dans
l'ensemble du bain, la dissolution ne peut pas avoir lieu. La réaction
de dissolution étant elle-même endothermique, elle est contrôlée par
la disponibilité d'énergie sous forme thermique à proximité de la
particule, \ie, de la surchauffe du bain.

Dans cette section, on propose un modèle qui décrive la vitesse de
dissolution d'une particule dans le bain d'une cuve d'électrolyse
d'aluminium, basé sur le travail de thèse de T. Hofer
\cite{Hofer2011}.

On fait l'hypothèse que les particules d'alumine sont des sphères
parfaites et non poreuses. On négligera la possible hydratation de
l'alumine qui les constituent et son effet sur leur dissolution. Dans
le travail de T. Hofer \cite{Hofer2011}, la vitesse de dissolution
d'une particule est une fonction $f$ qui décrit la variation de son
rayon et qui qui dépend d'une part de son rayon actuel $r$, et d'autre
part de la concentration locale $c$. Dans ce travail, nous
considèrerons en plus l'influence de la température locale de
l'électrolyte $\temperature$ sur sa vitesse de dissolution,
c'est-à-dire que
\begin{equation}\label{eq:radius-edo}
  \frac{\mathrm dr}{\mathrm dt} = f(r, \concentration, \temperature).
\end{equation}

On fait l'hypothèse que la dissolution des particules est
contrôlée par la diffusion de la concentration dans l'électrolyte et
que la concentration de saturation $\csat$ est indépendante de la
température $\temperature$ de l'électrolyte.

En suivant le travail de T. Hofer \cite{Hofer2011}, on propose
d'écrire la fonction $f$ comme
\begin{equation}\label{eq:dissolution-velocity}
  f(r, c, \temperature) = -\frac{\dissolutionrate(c, \temperature)}{r},
\end{equation}
où $\dissolutionrate$ est le taux de dissolution. Le facteur $1/r$ est
lié à l'hypothèse que la diffusion de la concentration contrôle
dissolution, comme l'ont montré Wang et Flanagan \cite{Wang1999}.

Clairement, le taux de dissolution $\dissolutionrate$ doit être
maximal lorsque $c = 0$, et s'annuler lorsque l'électrolyte est saturé
en alumine, \ie, lorsque $c = \csat$. De même, lorsque le bain se
trouve à la température du liquidus $\tliq$, il n'y a pas d'énergie
thermique à disposition pour la réaction de dissolution, et donc
$\dissolutionrate$ doit s'annuler si $\temperature = \tliq$. A
l'inverse, $\dissolutionrate$ doit être maximal lorsque
$\temperature\gg\tliq$.

Motivé par ces dernières considérations, on propose la forme suivante
pour l'expression du taux de dissolution:
\begin{equation}\label{eq:dissolution-rate}
  \dissolutionrate(\concentration, \temperature) =
  \left\{
  \begin{array}{ll}
    K\displaystyle\frac{\csat - \concentration}{\csat}\parent{1 -
      \exp\parent{-\displaystyle\frac{\temperature - \tliq}{\tcrit -
          \tliq}}} & \text{si }\temperature \geq \tcrit\text{ et }
    0\leq c
    \leq \csat,\\
    0                                     & \text{sinon,}
  \end{array}
  \right.
\end{equation}
où $K > 0$ et $\tcrit > \tliq$ sont deux paramètres du modèle. Le taux
de dissolution limite $K$ est tel que
\begin{equation}
  \lim_{\temperature \to\infty}\dissolutionrate(0, \temperature) = K.
\end{equation}
La figure \ref{fig:diss-rate} illustre la relation entre le taux de
dissolution et la température dans le cas où la concentration locale
$\concentration = 0$ et le taux de dissolution limite $K = 1$.

\begin{figure}[h]
  \begin{center}
    \input{../media/particles/diss-rate/diss-rate.tex}
    \caption{Relation entre la température $\temperature$ et le taux de
      dissolution $\dissolutionrate$ lorsque $c = 0$ et que le taux
      de dissolution limite $K = 1$. En vert sont les traits de
      construction, et donnent une interprétation du paramètre
      $\tcrit$.}
    \label{fig:diss-rate}
  \end{center}
\end{figure}

La vitesse de dissolution $f$ décrite ci-dessus constitue la base qui
permet la description de l'évolution d'une population de particules
dans un bain électrolytique qui fait l'objet de la section suivante.
