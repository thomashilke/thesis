
% réaction chimique
Le procédé consiste à dissoudre l'oxyde d'aluminium dans un bain
composé de cryolite, puis à réduire les ions d'aluminium en effectuant
l'électrolyse de la solution:
\begin{align}
  \cee{Al2O3 &-> 2Al^{3+} + 2O^{2-}},\\
  \cee{Al^{3+} + 3e^{-} &-> Al}.
\end{align}

% contexte industriel
Dans un cadre industriel, la réaction a lieu dans une récipient dont
une coupe verticale schématique est représentée par la figure
\fig{pot}.

% structure de la cuve
Le bain électrolytique est maintenu à une température de
\num{950}\si{\celsius} environ dans un contenant formé par une
épaisseurs de briques réfractaires au niveau des parois verticales et
une cathode en carbone au fond. Le tout est soutenu par une structure
appelée cuve. Les briques réfractaires jouent le rôle d'isolant
thermique d'une part, et protègent le reste de la structure des
attaques corrosives de l'électrolyte d'autre part.

% métal liquide


% circuit électrique
L'électrolyte est alimenté en courant électrique à travers une série
d'anodes plongées dans le bain et maintenues dans la partie supérieure
de celui-ci par un pont qui sert à la fois de support mécanique et de
conducteur électrique. Le courant électrique traverse l'électrolyte,
le métal liquide et atteint la cathode en carbone placée au fond de la
cuve. Finalement, le courant est conduit en dehors de la cuve à
travers des conducteurs appelés {\em bus barres} placés sous la
cathode. L'intensité du courant électrique qui traverse une cuve
d'élecrolyse industrielle peut atteindre \num{500}\si{\kilo\ampere},
et jusqu'à \num{1}\si{\mega\ampere} dans le cuves les plus modernes.


% structure de la cuve
D'un point de vue industriel, la réaction a lieu dans une cuve, dont
la structure schématique est représentée par la figure \fig{pot}. Le
bain électrolytique, maintenu à l'état liquide à
\num{950}\si{\celsius} environ \needcite, est contenu dans une cuve,
dont les parois sont recouvertes d'une épaisseur de briques
réfractaires jouant le rôle d'isolant thermique, et d'une couche de
brique de carbone protégeant le contenant des attaques corrosives de
la cryolite. Au fond de la cuve sont disposées une série de barres
électriquement conductrices dont le but est de collecter le courant
électrique. Une cathode, typiquement en carbone fait le contact entre
les barres collectrices et une couche d'aluminium métallique
liquide. La cuve est maintenue dans une structure appelée caisson,
dont le rôles sont multiples. En plus d'offrir un support mécanique à
l'ensemble, le matériau du caisson est choisi de sorte à écranter
partiellement les champs d'induction électromagnétique intenses à
proximité. Ceci permet de diminuer l'intensité des forces de Lorentz
qui s'exercent sur le fluide, et tend à stabiliser l'écoulement des
fluides. Finalement, le caisson permet de contrôler plus finement les
pertes thermiques du système dans des zones clés.

Le bain électrolytique est constitué de cryolite (\ce{Na3.AlF6}),
solide à température ambiante, et d'autres sels en proportions
inférieures. Les proportions exactes qui font l'objet de secrets
industriels permettent de ramener le point de fusion de ce mélange à
environ \num{950}\si{\celsius}. Dans ces conditions, l'aluminium
métallique étant immiscible dans le bain, l'ensemble forme un système
bi-fluide séparé par une frontière libre. La légère différence de
densité entre le métal et le bain assure que, grâce à la gravité,
l'interface soit essentiellement horizontale.

Le bain électrolytique présente une densité volumique légèrement plus
faible d'environ \num{10}\% que celle de l'aluminium métallique, ce
qui lui permet de surnager sur celui-ci. Dans le bain sont maintenues
deux rangées d'anodes faites de carbone également. Le courant
électrique provient des anodes et traverse le bain pour atteindre
l'interface avec le métal liquide pour être finalement collecté par la
cathode. Cette épaisseur de bain dans lequel a lieu la réaction
d'électrolyse est appelée ACD\footnote{Acronyme anglais de \em{Anode
Cathode Distance}}. Dans une cuve industrielle moderne, l'ACD est de
l'ordre de \num{20}\si{\milli\meter} à \num{20}\si{\milli\meter}. Le
bain ayant une faible conductivité, on cherche à diminuer au maximum
cette distance, afin de minimiser les pertes énergétiques par effet
Joule. Le procédé d'électrolyse étant continu, il faut remplacer
régulièrement l'alumine dissoute consommée par l'électrolyse. Une
série d'injecteurs alimentent le bain en oxyde d'aluminium solide sous
forme de fine poudre. La surface du bain est protégée par une croûte
de bain solidifié et de poudre d'alumine. Pour cette raison, les
injecteurs doivent préalablement percer la croûte avant de pouvoir
déposer une dose de poudre d'alumine dans le bain. L'alimentation ne
peut se faire que de manière discontinue. L'ensemble de la cuve est
recouvert par un couvercle qui limite les dissipations thermiques et
permet la collecte des émanations gazeuses potentiellement toxiques
résultant de la réaction d'électrolyse, et protège les opérateurs
travaillant à proximité.

Par la réaction d'électrolyse, l'alumine dissoute est transformée en
aluminium métallique au niveau de l'interface avec le métal liquide,
qui joue le rôle de cathode. Du côté de l'anode, l'électrolyse produit
de l'oxygène, qui se recombine immédiatement avec le carbone de
l'anode. Le \ce{CO2} qui en résulte forme des bulles
millimétriques qui remontent à la surface pour s'échapper. C'est cette
production de \ce{CO2} qui provoque l'érosion des anodes.


L'ensemble des anodes est fixé sur un pont offrant un support
mécanique, mais également conducteur électrique. Chaque anode est
séparée de ses voisines par un espace de quelques centimètres appelé
canal. La position verticale de chaque anode peut être ajustée pour
compenser l'érosion de celle-ci par la réaction
d'électrolyse. Régulièrement, chaque anode arrivant en fin de vie doit
être remplacée par une anode neuve. Ceci intervient environ une fois
par jour et par cuve.

Comme indiqué ci-dessus, de l'alumine doit être ajoutée dans le bain
régulièrement afin de compenser celle qui a été consommée par la
réaction d'électrolyse. En raison de la disposition des anodes et de
la formation de croûte de bain gelée à la surface de celui-ci, la
poudre d'alumine est injectée par dose de \num{1}\si{\kilo\gram}
environ au niveau des canaux et à intervalle régulier.

L'alumine se trouve dans un état cristallin, à une température
d'environ \num{350}\si{\celsius}. Lorsque les particules d'alumine
entrent en contact avec le bain, une couche de bain solidifié se
forme immédiatement à la surface de celles-ci, et sont convectée par
l'écoulement dans le fluide et chute dans le bain sous l'action de
la gravité.

Une fois la couche de bain gelé qui enveloppe les particules
refondues, celles-ci se dissolvent peu à peu. Dans certaines
conditions, les particules s'agglomèrent et forment des agrégats
qui descendent rapidement dans le bain et atteignent l'interface. Si
la taille des agrégats est suffisante, il arrive qu'ils pénètrent
le métal et terminent leur course au fond de la cuve, risquant de
recouvrir peu à peu la cathode et de l'isoler électriquement.

La température du bain joue un rôle crucial au niveau du comportement
de l'alumine nouvellement injectée. Pour des raisons d'économie
d'énergie, le bain est maintenu aussi proche que possible de la
température de fusion de la cryolite, avec des surchauffes typiques de
l'ordre de \num{10}\si{\celsius} ou moins. Plus cette surchauffe est
faible, plus l'impact sur la dissolution de l'alumine est
important. Le temps nécessaire pour refondre la couche de bain gelé à
la surface des particules augmente lorsque la surchauffe diminue. Dans
la limite où la surchauffe est nulle, ce temps devient infini. De
plus, la réaction de dissolution de l'alumine cristalline dans le bain
est un processus endothermique. Il faut alors fournir un equantité
d'énergie thermique non négligeable pour dissoudre les
particules. Cette énergie est disponible uniquement si la surchauffe
du bain environnant est suffisante.

