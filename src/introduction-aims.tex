% maîtrise de l'état de la cuve
Comme on a pu l'apprécier ci-dessus, l'opération d'une cuve
d'électrolyse d'aluminium industrielle est une tâche complexe qui
nécessite de maîtriser, entre autre, la production d'énergie thermique
par effet Joule et sa dissipation à l'extérieur de la cuve, les
écoulements dans les fluides et la répartition de la concentration
d'alumine dans le bain.

% l'optimisation n'est pas un nouveau problème
De fait, l'optimisation de l'exploitation d'une cuve d'électrolyse de
Hall-Héroult, tant en terme de consommation énergétique qu'en terme de
qualité de métal produit et d'impact sur l'environnement est une tâche
qui fait l'objet de recherche et développements s'étalant maintenant
sur plus de 100 ans.

% 1er sujet de cette thèse: dissolution (problematique)
Dans cette thèse nous nous intéresserons à la modélisation numérique
de certains phénomènes physiques qui interviennent dans le
fonctionnement d'une cuve d'électrolyse. La distribution de la
concentration d'alumine dissoute dans le bain est un aspect important
de l'opération d'une cuve d'électrolyse. Si la distribution de la
concentration d'alumine n'est pas suffisamment uniforme, certaines
zones seront suralimentées, tandis que d'autres seront
sous-alimentées. Dans un cas comme dans l'autre, la réaction
d'électrolyse ne peut pas avoir lieu correctement dans ces zones, ce
qui tend à déstabiliser l'ensemble du système, péjore le rendement du
système et engendre des émanations gazeuses toxiques. Nous aborderons
en particulier la modélisation de l'interaction entre la température
du bain électrolytique au cours du temps et la dissolution et
diffusion de la poudre d'alumine.

Nous proposerons une généralisation du modèle proposé par
T. Hofer \cite{Hofer2011} pour prendre en compte les nouveaux effets
évoqués, dans le but d'améliorer son pouvoir de prédiction.

% 2ème sujet de cette thèse: calculs rapides
L'état d'une cuve d'électrolyse n'est pas figé: il évolue sans cesse
au fil des changements d'anodes, des variations de cadence des
injecteurs, ou encore des perturbations des cuves voisines de la
série. Le modèle numérique existant, introduit par G. Steiner dans son
travail de thèse \cite{Steiner2009} et implémenté dans le logiciel
Alucell, permet d'obtenir une approximation numérique de l'état des
écoulements des fluides et de la concentration d'alumine dans une cuve
d'électrolyse avec un temps de calcul de l'ordre de \num{24} à
\num{48}\si{\hour}. Il est donc impossible d'utiliser ce modèle pour
obtenir des informations sur l'état d'une cuve à chaque instant de sa
vie, en fonction de paramètres extérieurs. Dans cette thèse nous
proposerons un modèle d'écoulement de fluide et de concentration
d'alumine dissoute stationnaire simplifié, qui permet d'obtenir des
informations sur l'état d'une cuve dans un laps de temps bien plus
court, de l'ordre de la minute.

Les opérateurs des cuves industrielles devraient alors pouvoir tirer
parti des prédictions de ce modèle pour contrôler l'état des cuves
et agir plus rapidement et de manière mieux informée dans les
conditions d'exploitation exceptionnelles.
