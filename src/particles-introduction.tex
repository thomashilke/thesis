% Dans cette section on doit introduire les 3 phénomènes physiques qui
% jour un rôle (on l'espère) sur la dissolution de l'alumine que l'on
% va étudier dans ce travail. Premièrement, on doit introduire le
% phénomène de formation de gel autour des particules d'alumine
% lorsque qu'on injecte celle-ci dans le bain
% électrolytique. Deuxièmement, on doit introduire l'interaction entre
% la température et la vitesse de dissolution des particules, liee au
% fait que la réaction de dissolution est endothermique. Finalement,
% on doit parler de la petite étude qui traite de la chute des
% particules dans le bain sous l'action de la gravite.
