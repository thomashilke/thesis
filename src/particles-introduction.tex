% Plan de l'introduction:
% * Introduction du chapitre: aspects de l'operation d'une cuve lies
%   aux particules d'alumine dans un bain
%   conclusion: les technologies de cuve moderne repose de plus en
%   plus sur la dissolution efficace et rapide de l'alumine dans le bain.
% * histoire d'une particule d'alumine de l'injection a la dissolution
%   * proprietes des grains, structures cristallines, taille, etc
%   * technologie d'injection, piqueurs, injecteurs, doses, etc
%   * contact avec le bain, aggregation, dispersion, transport
% * aspect importants de la dissolution de l'alumine que l'on retient
%   et etudie dans ce chapitre
% * Organisation du chapitre

Au cours de l'opération d'une cuve d'électrolyse, de l'oxyde
d'aluminium doit être injecté dans le bain électrolytique afin
de compenser l'alumine dissoute qui est consommée par la
réaction d'électrolyse.

En raison de la faible surchauffe\footnote{La surchauffe du bain
  électrolytique est définie comme la différence entre la température
  du bain et la température du liquidus $\tliq$.} du bain, la surface
de celui-ci est recouverte, en tout temps, par une croûte
solide. Cette croûte est principalement constituée par de
l'électrolyte solidifié. Sa présence est
désirable, elle joue le rôle d'isolant thermique, protège la structure
supérieure de la cuve des éclaboussures et facilite la canalisation
des émanations gazeuses.

Cependant, la présence de la croûte limite l'accès à la surface du
bain, et en particulier, l'injection d'alumine nécessite la mise en
place d'un mécanisme qui permette de la percer. Ces dispositifs
appelés piqueurs, percent mécaniquement des ouvertures circulaires dans
la croûte en plusieurs endroits et à intervalles réguliers, et
maintiennent des accès libres à la surface du bain.

Ces accès permettent à des injecteurs de déposer, à la surface du
bain, des doses de poudre d'oxyde d'aluminium à intervalles
réguliers. Cette poudre est constituée de particules grossièrement
sphériques, sous forme cristalline et dont la température $\tinj$ se
situe entre \num{100}\si{\celsius} et \num{150}\si{\celsius}. Le
diamètre des particules est pour la plupart compris entre
\num{20}\si{\micro\meter} et \num{100}\si{\micro\meter}.
Dans une situation idéale, après leur injection dans le bain, les
particules se dispersent dans celui-ci et se dissolvent peu-à-peu tout
en étant transportées par l'écoulement des fluides. Cette alumine
dissoute vient contribuer à la concentration d'alumine dissoute. Les
conditions d'exploitation des cuves d'électrolyse modernes sont de
plus en plus dépendantes d'une dissolution rapide et uniforme des
particules dans l'électrolyte. Malheureusement, dans un système
industriel réel, de nombreux phénomènes viennent entraver le bon
déroulement de ce processus.

Par exemple, il arrive fréquement que les points d'injections
aménagés par les piqueurs se bouchent \cite{Dion2017}. Les raisons
précises pour lequelles un accès se bouche sont encore mal
comprises. Une telle situation, une fois identifiée, nécessite de
prendre des mesures particulières pour permettre à l'alumine
d'atteindre l'électrolyte. Ensuite, lorsqu'une dose d'alumine est
déposée à la surface du bain, les particules ont tendance à
s'agglomérer et former de petit radeau qui flottent et se maintiennent
à la surface du bain (\cite{Dassylva2015}, \cite{Kaszas2017}). Par
rapport à l'alumine qui se disperse immédiatement dans le volume
du bain, les particules qui constituent ces radeaux sont beaucoup plus
difficiles à dissoudre. De plus, lorsque ces radeaux coulent ils
peuvent, selon leur masse et leur taille, pénétrer dans la couche
de metal et se retrouver au fond de la cuve au niveau de la
cathode. Il forment alors des boues qui isolent électriquement la
cathode, et provoque une usure mécanique prématurée.

Dans ce travail, nous nous intéresserons à trois aspects distincts
liés au transport des particules d'alumine dans le bain électrolytique
et à leur dissolution. Dans chacun des cas, nous supposerons que les
particules sont suffisamment dispersées dans le fluide pour
s'autoriser à considérer des particules individuelles. En particulier,
on négligera toutes interaction, directes ou indirectes par
l'intermédiaire du bain, entre les particules présentes dans ledit bain.

Premièrement, nous nous pencherons sur les phénomènes thermiques à
proximité des particules immédiatement après leur injection, et en
particulier à la formation et refonte d'une couche de bain solidifiée
à la surface des particules. Tant qu'elle est présente, la couche de
gel empêche la dissolution de la particule. Nous proposerons un modèle
mathématique qui décrive l'évolution de la température dans une
particule et dans l'électrolyte environnant, et la position du front
de transition de phase. Nous utiliserons un modèle numérique pour
estimer le temps nécessaire à la refonte de la couche de gel qui se
forme typiquement à la surface des particules après leur injection.

Ensuite, nous nous intéresserons au rôle de la température du bain sur
la capacité de dissolution des particules d'alumine. Nous proposerons
un modèle qui décrive la dissolution, c'est-à-dire l'évolution du
rayon d'une particule en fonction de sa surface de contact avec le
bain, de la concentration et de la température locale du bain. On en
dérivera un modèle qui décrive l'évolution d'une population de
particules caractérisées par leur rayon.

Finalement, nous considérerons l'effet de force de gravité sur la
trajectoire des particules. Lorsque les particules d'alumine sont
déposées à la surface du bain, celle-ci sont entraînées par le fluide
en mouvement par le biais de forces de trainée. De plus, la force de
gravité les entraîne vers le fond de la cuve. Dans son travail,
T. Hofer \cite{Hofer2011} a négligé l'effet de la gravité sur le
transport des particules dans le bain électrolytique. Nous
déterminerons dans ce travail les conditions dans lesquelles cette
hypothèse se vérifie.





%%Tout d'abord, commençons par noter que la taille des particules est
%%contrainte pour différentes raisons.
%%Le diamètre des particules est pour la plupart compris entre
%%\num{20}\si{\micro\meter} et \num{100}\si{\micro\meter}.
%%Des particules trop fines présentent des
%%difficulté de manipulation au niveau mécanique, et la faible
%%capacité du bain à les mouiller rend leur dissolution difficile. De
%%plus, leur présence en trop grand nombre dans le bain peu avoir un
%%impact négatif sur la conductivité électrique de celui-ci. A
%%l'inverse, des particule trop grande mettent trop de temps à se
%%dissoudre \cite{Fini2017}.
%%
%%Ceci est du aux gaz adsorbés à la surface des particules et au
%%taux d'humidité qu'elle contiennent.
%%
%%Ce chapitre est dédié au traitement d'une série de phénomènes
%%physiques qui entre en jeu lorsque les particules d'une dose d'alumine
%%entre en contact avec le bain électrolytique et commencent à se
%%dissoudre dans celui-ci.
%%
%%Ce chapitre est dédié à la discussion de différents phénomènes qui
%%déterminent la capacité des particules d'alumine à se dissoudre dans
%%ledit bain électrolytique, et à contribuer à la concentration d'oxyde
%%d'aluminium dissoute. On discutera d'une part des phénomènes
%%liés à la température du bain électrolytique au voisinage des
%%particules, et d'autre part à l'effet de la gravité sur la
%%trajectoire des particules dans le bain électrolytique.
%%
%%La température du bain électrolytique influe sur la capacité des
%%particules d'alumine à se dissoudre pour plusieurs raisons. Faisons
%%remarquer tout d'abord qu'on cherche à maintenir la température du
%%bain électrolytique au-dessus, mais aussi proche que possible de la
%%température de liquidus $\tliq$. En effet, plus la température de
%%l'électrolyte est basse, plus les pertes d'énergie par diffusion
%%thermiques sont minimisées. Cependant, lorsque la température du
%%bain se rapproche de $\tliq$, la cuve d'électrolyse
%%devient de plus en plus sensible à des écarts de température. On
%%cherche alors un équilibre délicat en cherchant à diminuer la
%%dissipation thermique tout en maintenant la cuve d'électrolyse dans
%%un état stable.
%%
%%En raison de cette faible surchauffe du bain, la surface de celui-ci
%%est systématiquement recouverte par une croûte solide.
%%
%%Lors de l'injection d'une dose d'alumine, la différence de température
%%entre les particules et le bain provoque la solidification de celui-ci
%%à la surface des particules. Tant que cette couche de bain solidifiées
%%persiste, les particules sont transportées par l'écoulement sans
%%qu'elles se dissolvent. Progressivement, l'équilibre thermique
%%entre la dose de particules et l'électrolyte se rétablit et le bain
%%solidifié autour des particules se résorbe. C'est seulement à partir
%%de cet instant que celle-ci peuvent se dissoudre et contribuer à
%%l'oxyde d'aluminium dissout dans le bain électrolytique. Ce laps de
%%temps pendant lequel la particule ne peut pas se dissoudre est le
%%temps de latence.
%%
%%Le temps de latence d'une particule dépend essentiellement de la
%%surchauffe du bain au voisinage de celle-ci, de sa température
%%initiale ainsi que de son rayon. Plus le volume d'une
%%particule est grand, plus la couche de bain gelé est importante, et
%%plus long est le temps nécessaire à refondre celle-ci. Il en va de
%%même pour sa température initiale: plus celle-ci est
%%froide, plus la couche de bain gelé est importante, et plus le temps
%%de refonte est long. Finalement, une température de surchauffe
%%importante diminue le temps de latence, puisque les flux de chaleur
%%qui établissent l'équilibre thermique au voisinage de la particule
%%sont plus importants.
%%
%%% parler ici de la dependance de la dissolution par rapport a la
%%% temperature.
%%
%%Il y a bien entendu d'autres effets qui jouent un rôle dans la
%%dissolution des particules d'alumine, mais que l'on négligera dans
%%le présent travail. En particulier, on notera que les particules,
%%bien qu'approximativement sphériques, sont en réalité des
%%structures cristallines complexes partiellement poseuses
%%\cite{Ostbo2002}. La dissolution individuelle de chaque grain est un
%%processus complexe qui fait intervenir des transformations entre différentes phases
%%cristallines et une désintégration partielle avant que ceux-ci soit
%%entièrement dissout. La composition chimique du bain électrolytique
%%joue également un rôle sur la réaction de dissolution de l'alumine,
%%par l'intermédiaire de l'enthalpie de dissolution. Au moment de
%%l'injection d'une dose de poudre d'alumine, une partie des particules
%%peuvent se retrouver densément compactées. La présence du bain
%%électrolytique et les haute température peuvent provoquer un
%%phénomène de frittage entre les particules. Les particules ainsi
%%solidarisées forment des agrégats de tailles macroscopiques
%%\cite{Ostbo2002}.
%%Ces agrégats sont nuisible au bon fonctionnement des cuves
%%d'électrolyse industrielles.
%%
%%- non-sphericite des particules,
%%- agglomeration,
%%- changement de structure cristalline,
%%- structure de l'écoulement du bain au voisinage des particules, turbulences,
