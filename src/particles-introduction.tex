Au cours de l'opération d'une cuve d'électrolyse, de l'oxyde
d'aluminium doit être injecté dans le bain électrolytique afin
de compenser l'alumine dissoute qui est consommée par la
réaction d'électrolyse.

En raison de la faible surchauffe\footnote{La surchauffe du bain
  électrolytique est définie comme la différence entre la température
  du bain et la température du liquidus $\tliq$.} du bain, la surface
de celui-ci est recouverte, en tout temps, par une croûte
solide. Cette croûte est principalement constituée par de
l'électrolyte solidifié. Sa présence est désirable, elle joue le rôle
d'isolant thermique, protège la structure supérieure de la cuve des
éclaboussures et facilite la canalisation des émanations gazeuses.

Cependant, la présence de la croûte limite l'accès à la surface du
bain, et en particulier, l'injection d'alumine nécessite la mise en
place d'un mécanisme qui permette de la percer. Ces dispositifs
appelés piqueurs, percent mécaniquement des ouvertures circulaires dans
la croûte en plusieurs endroits et à intervalles réguliers, et
maintiennent des accès libres à la surface du bain.

Ces accès permettent à des injecteurs de déposer, à la surface du
bain, des doses de poudre d'oxyde d'aluminium à intervalles
réguliers. Cette poudre est constituée de particules grossièrement
sphériques, sous forme cristalline et dont la température $\tinj$ se
situe entre \num{100} \si{\celsius} et \num{150} \si{\celsius}. Le
diamètre des particules est pour la plupart compris entre
\num{20} \si{\micro\meter} et \num{100} \si{\micro\meter}.
Dans une situation idéale, après leur injection dans le bain, les
particules se dispersent dans celui-ci et se dissolvent peu-à-peu tout
en étant transportées par l'écoulement des fluides. Cette alumine
dissoute vient contribuer à la concentration d'alumine dissoute. Les
conditions d'exploitation des cuves d'électrolyse modernes sont de
plus en plus dépendantes d'une dissolution rapide et uniforme des
particules dans l'électrolyte. Malheureusement, dans un système
industriel réel, de nombreux phénomènes viennent entraver le bon
déroulement de ce processus.

Par exemple, il arrive fréquement que les points d'injections
aménagés par les piqueurs se bouchent \cite{Dion2017}. Les raisons
précises pour lequelles un accès se bouche sont encore mal
comprises. Une telle situation, une fois identifiée, nécessite de
prendre des mesures particulières pour permettre à l'alumine
d'atteindre l'électrolyte. Ensuite, lorsqu'une dose d'alumine est
déposée à la surface du bain, les particules ont tendance à
s'agglomérer et former de petit radeau qui flottent et se maintiennent
à la surface du bain (\cite{Dassylva2015}, \cite{Kaszas2017}). Par
rapport à l'alumine qui se disperse immédiatement dans le volume
du bain, les particules qui constituent ces radeaux sont beaucoup plus
difficiles à dissoudre. De plus, lorsque ces radeaux coulent ils
peuvent, selon leur masse et leur taille, pénétrer dans la couche
de metal et se retrouver au fond de la cuve au niveau de la
cathode. Il forment alors des boues qui isolent électriquement la
cathode, et provoque une usure mécanique prématurée.

Dans ce travail, nous nous intéresserons à trois aspects distincts
liés au transport des particules d'alumine dans le bain électrolytique
et à leur dissolution. Dans chacun des cas, nous supposerons que les
particules sont suffisamment dispersées dans le fluide pour
s'autoriser à considérer des particules individuelles. En particulier,
on négligera toutes interaction, directes ou indirectes par
l'intermédiaire du bain, entre les particules présentes dans ledit bain.

Premièrement, nous nous pencherons sur les phénomènes thermiques à
proximité des particules immédiatement après leur injection, et en
particulier à la formation et refonte d'une couche de bain solidifiée
à la surface des particules. Tant qu'elle est présente, la couche de
gel empêche la dissolution de la particule. Nous proposerons un modèle
mathématique qui décrive l'évolution de la température dans une
particule et dans l'électrolyte environnant, et la position du front
de transition de phase. Nous utiliserons un modèle numérique pour
estimer le temps nécessaire à la refonte de la couche de gel qui se
forme typiquement à la surface des particules après leur injection.

Ensuite, nous nous intéresserons au rôle de la température du bain sur
la capacité de dissolution des particules d'alumine. Nous proposerons
un modèle qui décrive la dissolution, c'est-à-dire l'évolution du
rayon d'une particule en fonction de sa surface de contact avec le
bain, de la concentration et de la température locale du bain. On en
dérivera un modèle qui décrive l'évolution d'une population de
particules caractérisées par leur rayon.

Finalement, nous considérerons l'effet de force de gravité sur la
trajectoire des particules. Lorsque les particules d'alumine sont
déposées à la surface du bain, celle-ci sont entraînées par le fluide
en mouvement par le biais de forces de trainée. De plus, la force de
gravité les entraîne vers le fond de la cuve. Dans son travail,
T. Hofer \cite{Hofer2011} a négligé l'effet de la gravité sur le
transport des particules dans le bain électrolytique. Nous
déterminerons dans ce travail les conditions dans lesquelles cette
hypothèse se vérifie.
