%\makeatletter
%\@openrightfalse
%\makeatother


\chapter*{Résumé}
\addcontentsline{toc}{chapter}{Résumé}

L'aluminium métallique joue un rôle crucial dans l'économie
moderne. L'aluminium métallique primaire est obtenu par transformation
de l'oxyde d'aluminium par le procédé industriel de Hall-Héroult. Ce
procédé, qui consomme d'énormes quantités d'énergie, consiste à
réaliser l'électrolyse d'une solution d'oxyde d'aluminium dans une
cuve de grande dimension à l'aide de courant de plusieurs centaines de
milliers d'ampères.

Le sujet de cette thèse est l'étude de certains aspects de la
modélisation de l'électrolyse de l'oxyde d'aluminium du point de vue
de la simulation numérique. Cette thèse est divisée en deux parties.

La première partie est consacrée à la modélisation numérique du
phénomène de dissolution et transport de la poudre d'alumine dans le
bain électrolytique d'une cuve en fonction de la température du
bain. Nous proposons un modèle mathématique du transport et de la
dissolution de la poudre d'alumine, puis sa discrétisation en temps et
en espace par une méthode d'éléments finis. Finalement, nous étudions
le comportement de ce modèle dans le contexte d'une cuve d'électrolyse
industrielle.

La deuxième partie est consacrée au développement d'une méthode
numérique pour le calcul de l'écoulement des fluides d'une cuve
d'électrolyse. Le calcul de la vitesse de l'électrolyte et du métal
liquide est basée sur une décomposition en modes de
Fourier. L'amplitude de chaque mode satisfait une équation aux
dérivées partielles que l'on explicite, dont la solution est approchée
par une méthode d'éléments finis. Enfin, l'écoulement des fluides
obtenu avec cette méthode est comparé aux solutions obtenue avec le
modèle de référence dans le contexte d'une cuve d'électrolyse
industrielle.

\paragraph{Mots clés} Simulations numériques, Méthode des
éléments finis, Équations aux dérivées partielles, Équation
d'advection-diffusion, Électrolyse, Alumine, Dissolution, Température.


%\makeatletter
%\@openrighttrue
%\makeatother
