Selon la distribution de tailles des particules d'alumine, une dose de
\num{1} \si{\kilo\gram} contient typiquement entre $10^9$ et $10^{12}$
particules. Il n'est donc pas envisageable de suivre individuellement
l'évolution de chacune d'elles dans le bain. On adopte une approche
statistique \cite{Hofer2011} et on décrit l'ensemble des particules
présentes dans le bain par l'intermédiaire d'une distribution
continue. On note $n_p(t, r)\intd{r}$ le nombre de particules dont le
rayon est compris dans l'intervalle $[r, r + \intd{r}]$ à l'instant
$t$. On suppose que cette population $\population$ se dissout selon
l'équation
\begin{align}
&\frac{\partial \population}{\partial t} + \frac{\partial
}{\partial r}\parent{f(r,c,\temperature)n_p} = 0\quad \text{si } r > 0,\ t > 0,\label{eq:population-pde}\\
&n_p(0, r) = n_{p,0}(r) \quad \text{si } r > 0,\label{eq:population-ic}
\end{align}
où la vitesse de dissolution $f$ est donnée par l'expression
(\ref{eq:dissolution-velocity}) et $n_{p,0}$ est une distribution de
particules initiale donnée.

En tenant compte de (\ref{eq:dissolution-velocity}), l'équation
(\ref{eq:population-pde}) pour la population de particule
$\population$ se réécrit
\begin{equation}\label{eq:population-pde-2}
  \frac{\partial n_p}{\partial t} -
  \frac{\dissolutionrate}{r}\frac{\partial n_p}{\partial r} +
  \frac{\dissolutionrate}{r^2}n_p = 0.
\end{equation}
\newcommand{\rcharacteristic}{R_{(\bar t, \bar r)}}

\begin{figure}
  \begin{center}
    \input{../media/particles/characteristics/characteristics.pdf_tex}
    \caption{Ligne caractéristique d'une particule de rayon initial
      $r_0$ qui se dissout avec une vitesse $\kappa$ selon
      l'expression (\ref{eq:particle-exact}).}
    \label{fig:characteristics}
  \end{center}
\end{figure}

Dans le plan $(t, r)$ on fixe un point $(\bar t, \bar r)$ avec $\bar t
> 0$, $\bar r > 0$, et on veut déterminer $\population(\bar t, \bar
r)$ à partir de $n_{p,0}(r)$ par la méthode des caractéristiques rétrograde.
L'équation de la ligne caractéristique $\rcharacteristic$ qui passe par $(\bar t,
\bar r)$ est donnée par
\begin{equation*}
\frac{\mathrm d}{\mathrm dt}R_{(\bar t, \bar r)}(t) =
-\frac{\dissolutionrate}{R_{(\bar t, \bar r)}(t)}.
\end{equation*}
Cette situation est illustrée par la figure \ref{fig:characteristics}.
Ainsi $R_{(\bar t, \bar r)}(t) = \sqrt{\bar r^2 - 2\dissolutionrate (t
  - \bar t)}$, en vertu de (\ref{eq:particle-exact}). Notons $N_p(t) =
n_p(t, R_{(\bar t, \bar r)}(t))$ la valeur de $n_p$ sur cette
caractéristique. En dérivant par rapport à $t$ et en utilisant
(\ref{eq:population-pde-2}) on a
\begin{align*}
  \frac{\mathrm d}{\mathrm dt}N_p(t) %
  &= \frac{\partial n_p}{\partial t}(t, \rcharacteristic(t)) %
  + \frac{\partial}{\partial r}n_p(t,\rcharacteristic(t)) %
    \frac{\mathrm d}{\mathrm dt}\rcharacteristic(t)\\
  &= \frac{\partial n_p}{\partial t}(t, \rcharacteristic(t))
     - \frac{\dissolutionrate}{\rcharacteristic(t)}\frac{\partial}{\partial r} n_p(t, \rcharacteristic(t))\\
  &= -\frac{\dissolutionrate}{R_{(\bar t, \bar r)}^2(t)}N_p(t). %
\end{align*}
Ainsi
\begin{equation*}
  \frac{1}{N_p(t)}\frac{\mathrm d}{\mathrm dt}N_p(t) =
  -\frac{\dissolutionrate}{R^2_{(\bar t, \bar r)}(t)}
\end{equation*}
qui implique que
\begin{equation*}
  \frac{\mathrm d}{\mathrm dt}\ln(N_p(t)) =
  -\frac{\dissolutionrate}{\parent{\bar r^2 - 2\dissolutionrate (t - \bar t)}}.
\end{equation*}
En intégrant on obtient
\begin{equation*}
  \ln(N_p(t)) = \frac{1}{2}\ln\parent{\bar r^2 -
    2\dissolutionrate\parent{t - \bar t}} + D
\end{equation*}
et donc
$N_p(t) = \parent{\bar r^2 - 2\dissolutionrate\parent{t - \bar t}}^{1/2} \exp(D)$. Si $t = 0$, on a:
\begin{equation*}
  N_p(0) = \exp(D)\sqrt{\bar r^2 +
    2\dissolutionrate \bar t} = n_{p,0}(\rcharacteristic(0)) = n_{p,0}(\sqrt{\bar r^2 +
    2\dissolutionrate \bar t}).
\end{equation*}
Ainsi
$\exp(D) = \frac{n_{p,0}\parent{\sqrt{\bar r^2 + 2\dissolutionrate \bar t\,}}}{\sqrt{\bar r^2 + 2\dissolutionrate \bar t\,}}$
et
\begin{equation*}
  N_p(t) = \frac{n_{p,0}(\sqrt{\bar r^2 + 2\dissolutionrate \bar
      t})}{\sqrt{\bar r^2 + 2\dissolutionrate \bar t\,}}\parent{\bar r^2 -
    2\dissolutionrate \parent{t - \bar t}}^{1/2}.
\end{equation*}
Puisque $N_p(\bar t) = n_p(\bar t, \bar r)$, on a
\begin{equation}\label{eq:population-exact-solution}
  n_p(\bar t, \bar r) = \bar r\frac{n_{p,0}(\sqrt{\bar r^2 +
      2\dissolutionrate \bar t})}{\sqrt{\bar r^2+2\dissolutionrate \bar t}}.
\end{equation}

\begin{remarque}\label{rem:population-exact-step}
  La solution (\ref{eq:population-exact-solution}) est exacte pour
  autant que le taux de dissolution $\dissolutionrate$ soit constant
  sur l'intervalle de temps $[0, T]$. Cependant, si $\dissolutionrate$
  est constant sur le sous-intervalle $[\bar t - \Delta t, \bar t]$ et
  que la distribution de particules $\population$ est connue au temps
  $\bar t - \Delta t$, alors on exprime facilement $\population$ au
  temps $\bar t$ en fonction de $\population$ au temps $\bar t - \dt$
  en fonction de $\population$ au temps $\bar t - \mathrm dt$.

  En effet, supposons connu $n_p(\bar t - \Delta t, r)$, $\forall
  r\in (0, \infty)$. Nous avons ainsi
  \begin{equation*}
    R_{(\bar t, \bar r)}(\bar t - \Delta t) = \sqrt{\bar r^2 +
      2\dissolutionrate \Delta t\,}.
  \end{equation*}
  Puisque $N_p(t) = C\parent{\bar r^2 - 2\dissolutionrate\parent{t -
      \bar t}}^{1/2}$, alors
  \begin{equation*}
N_p(\bar t - \Delta t) = C\parent{\bar r^2 + 2\dissolutionrate \Delta
  t}^{1/2} = n_p\parent{t - \Delta t, \sqrt{\bar r^2 + 2\dissolutionrate
    \Delta t\,}}.
  \end{equation*}
  Ainsi
  \begin{equation*}
    C = \frac{n_p(t - \Delta t, \sqrt{\bar r^2 + 2\dissolutionrate
        \Delta t})}{\parent{\bar r^2 + 2\dissolutionrate\Delta t}^{1/2}}
  \end{equation*}
  et donc
  \begin{equation*}
    N(t) = \frac{n_p(t - \Delta t, \sqrt{\bar r^2 + 2\dissolutionrate
        \Delta t})}{\parent{\bar r^2 + 2\dissolutionrate\Delta t}^{1/2}}\parent{\bar
      r^2 - 2\dissolutionrate\parent{t - \bar t}}^{1/2}.
  \end{equation*}
  Puisque $N(\bar t) = n_p(\bar t, \bar r)$, on obtient
  \begin{equation*}
    n_p(\bar t, \bar r) = \bar r\frac{n_p(\bar t - \Delta t, \sqrt{\bar
        r^2 + 2\dissolutionrate \Delta t})}{\sqrt{\bar r^2 +
        2\dissolutionrate \Delta t}}.
  \end{equation*}
\end{remarque}

\paragraph{Approximation numérique} Pour discrétiser l'équation
(\ref{eq:population-pde}) on procède comme dans Hofer \cite{Hofer2011}
et on utilise une méthode des caractéristiques \cite{Pironneau1989}.

Soit $T > 0$ le temps final et soit $N\in \mathbb N^*$ le nombre de
pas de temps. Soient $\Delta t = T/N$ et $t_n = n\tau$, $n = 0,1,
\dots, N$ une subdivision uniforme de $[0, T]$. De plus on suppose
donnée une suite de nombres $\dissolutionrate^n \geq 0$, $n = 0, 1,
\dots, N - 1$. Le nombre $\dissolutionrate^n$ est la valeur du taux de
dissolution sur l'intervalle de temps $[t^n, t^{n+1}]$.

Soit $\rmax > 0$ et $M \in \mathbb N^*$ le nombre de subdivisions de
l'intervalle $[0, \rmax]$. Soient $\Delta r = \rmax/M$ et $r_j =
j\Delta r$ une subdivision de $[0, \rmax]$.

Pour que le problème soit bien posé sur le domaine $(0,
\rmax)$ et puisque la vitesse de dissolution $f$ est négative, on
considère en plus la condition limite sur le bord entrant $r = \rmax$:
\begin{equation*}
  n_p(t, \rmax) = 0,\quad\forall t \in [0, T].
\end{equation*}

Notons $n_{p,j}^n$ l'approximation de $\population(t_n, r_j)$ que l'on
suppose connue pour $j = 1,2, \dots, M$. Les valeurs $n_{p,j}^{n+1}$
sont calculées de la manière suivante. On utilise $\dissolutionrate^n$
comme taux de dissolution sur l'intervalle de temps $[t^n,
  t^{n+1}]$. Soit $R_j^n = \sqrt{r_j^2 + 2\dissolutionrate^n \Delta
  t}$ la position à l'instant $t_n$ de la caractéristique passant par le point
$(t_{n+1}, r_j)$. Soit $s_j^n \in \mathbb N$
l'indice tel que $r_{s_j^n} < R_j^n \leq r_{s_j^n + 1}$. Pour obtenir
une approximation de $\population(t_n, R_j^n)$ on interpole
linéairement les valeurs de $n_{p,s_j^n}^n$ et
$n_{p,s_j^n+1}^n$. Ainsi en
utilisant le résultat de la remarque \ref{rem:population-exact-step}
on pose pour $j = 1,\dots, M-1$
\begin{equation}\label{eq:population-characteristic-scheme}
  n_{p,j}^{n+1} = \left\{
  \begin{array}{ll}
    \parent{\frac{r_{s_j^n + 1} - R_j^n}{\Delta r}
      n_{p,s_j^n}^n + \frac{R_j^n - r_{s_j^n}}{\Delta r} n_{p,s_j^n +
        1}^n} \displaystyle\frac{r_j}{\sqrt{r_j^2 + 2\dissolutionrate^n \Delta
        t}},&\text{ si } s_j < M,\\
    0, & \text{ sinon.}
  \end{array}
  \right.
\end{equation}


\paragraph{Validation numérique} On propose maintenant de valider
l'implémentation du schéma numérique
(\ref{eq:population-characteristic-scheme}) en étudiant la
convergence de l'erreur entre une solution exacte et son approximation
numérique.

On fixe $\rmax = \num{100}$ \si{\micro\meter}, $T = 5$ \si{\second} et
le taux de dissolution
$\dissolutionrate^n = \num{0.5e-9}$ pour tout $n$ tel que $0\leq n\leq
N-1$. On choisit la distribution de particules initiale
\begin{equation}\label{eq:population-test-ic}
  n_{p,0}(r) = \exp\parent{1 - \frac{1}{1 - \frac{16}{\rmax^2}\parent{r - \frac{\rmax^2}{2}}}}.
\end{equation}
Alors la solution exact $n_p(t, r)$ est donnée par l'expression
(\ref{eq:population-exact-solution}). On note $n_{p,\Delta r}^N$ la
reconstruction continue linéaire par morceau des valeurs $n_{p,j}^N$
sur la subdivision $\cparent{r_j}_{j=0}^M$, c'est-à-dire que
\begin{equation}
  n_{p,\Delta r}^N(r) = n_{p,j}^N + \frac{r - r_j}{\Delta
    r}\parent{n_{p,j+1}^N - n_{p,j}^N}\quad \text{si } r\in[r_j, r_{j+1}].
\end{equation}
On considère l'erreur $L^2(0,\rmax)$ entre la solution exacte
$\population(T, \cdot)$ et $n_{p,\Delta r}$. La figure
\ref{fig:population-timelapse} illustre la solution et l'approximation
numérique à différents instants. La figure
\ref{fig:population-convergence} présente l'erreur $\norm{n_p(T, .) -
  n_{p,\Delta r}^N}_{L^2(0,\rmax)}$ pour $\Delta r = 1/32$, $1/64$,
$1/128$, $1/256$, $1/512$, $1/1024$, $1/2048$ et $1/4096$. On a
choisit $\Delta t = \alpha\Delta r$ avec $\alpha =
\num{0.934}$. L'erreur est comparée à une droite de pente 1. On
constate que l'erreur de l'approximation numérique est un $O(\Delta
r)$.

\begin{figure}[h!]
  \begin{center}
    %\input{../media/particles/dissolution/timelapse-0.tex}
    \input{../media/particles/dissolution/timelapse-1.tex}
    %\input{../media/particles/dissolution/timelapse-2.tex}
    \input{../media/particles/dissolution/timelapse-3.tex}
    %\input{../media/particles/dissolution/timelapse-4.tex}
    \input{../media/particles/dissolution/timelapse-5.tex}
    %\input{../media/particles/dissolution/timelapse-6.tex}
    \input{../media/particles/dissolution/timelapse-7.tex}
    %\input{../media/particles/dissolution/timelapse-8.tex}
    \input{../media/particles/dissolution/timelapse-9.tex}
    %\input{../media/particles/dissolution/timelapse-10.tex}
    \input{../media/particles/dissolution/timelapse-11.tex}
  \end{center}
  \begin{center}
    \caption{Solution exacte $n_p$ et approximation numérique avec
      $\Delta r = 1/64$, $\Delta t = \num{1.46e-2}$ et
      $\dissolutionrate^n = \num{0.5e-9}$ à différents temps. La
      condition initiale est donnée par
      (\ref{eq:population-test-ic}.). En vert: solution exacte. En
      violet: approximation numérique.}
    \label{fig:population-timelapse}
  \end{center}
\end{figure}

\begin{figure}[h!]
  \begin{center}
    \input{../media/particles/dissolution/dissolution-convergence.tex}
    \caption{Convergence de l'erreur $L^2(0, \rmax)$ entre $n_p(T, .)$
      et la reconstruction linéaire par morceau des approximations
      $n_{p,j}^N$ sur la subdivision $\cparent{r_j}_{j = 0}^M$. Le pas
      de temps est choisi tel que $\Delta t \propto \Delta r$.}
    \label{fig:population-convergence}
  \end{center}
\end{figure}

La section suivante tient lieu de conclusion de ce chapitre, et discute
de la sédimentation des particules d'alumine dans le bain sous
l'action de la force de gravité.
