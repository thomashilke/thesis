Selon la distribution de tailles des particules d'alumine, une dose de
\num{1}\si{\kilo\gram} contient typiquement entre \num{1.e+9} et
\num{1.e+12} particules. Il n'est donc pas envisageable de suivre
individuellement l'évolution de chacune d'elles dans le bain. En
suivant le travail de T. Hofer \cite{Hofer2011} on adopte une approche
statistique et on décrit l'ensemble des particules dans le bain par
l'intermédiaire d'une distribution continue. On note $n_p(t, r)\intd{r}$ le
nombre de particules dont le rayon est compris dans l'intervalle $[r,
  r + \intd{r}]$ à l'instant $t$. En suivant le travail de T. Hofer \cite{Hofer2011}, on suppose que
cette population $\population$ se dissout selon l'équation
\begin{align}
&\frac{\partial \population}{\partial t} - \frac{\partial
}{\partial r}\parent{f(r,c,\temperature)n_p} = 0\quad \text{si } r > 0,\ t > 0,\label{eq:population-pde}\\
&n_p(0, r) = n_{p,0}(r) \quad \text{si } r > 0,\label{eq:population-ic}
\end{align}
où la vitesse de dissolution $f$ est donnée par l'expression
(\ref{eq:dissolution-velocity}) et $n_{p,0}$ est une distribution de
particules initiale donnée.

Supposons maintenant que $\concentration$ et $\temperature$ sont des
constantes données. Alors
$\dissolutionrate(\concentration,\temperature)$ est un réel fixé, et
on omettra d'indiquer ses arguments dans la suite, lorsqu'il n'y a pas
de risque d'ambiguïté. Dans ces conditions, on peut trouver une
solution du système d'équations l'équation
(\ref{eq:population-pde})-(\ref{eq:population-ic}) sous forme
analytique par la méthode des caractéristiques.

Considérons tout d'abord une particule individuelle de rayon initiale
$r_0$. Son rayon varie au cours du temps selon l'équation
(\ref{eq:radius-edo}) que l'on réécrit en explicitant le membre de
droite:
\begin{align*}
  &\frac{\mathrm d r}{\mathrm d t} = -\frac{\dissolutionrate}{r} \quad \text{si } t
  > 0,\\
  & r(0) = r_0.
\end{align*}
Ainsi, en notant $\dot r = \frac{\mathrm d}{\mathrm dt}$ on obtient:
\begin{align}
  &r\dot r = -\dissolutionrate,\nonumber\\
  &\frac{1}{2}\frac{\mathrm d}{\mathrm dt}r^2 = -\dissolutionrate,\nonumber\\
  &r^2 = -2\dissolutionrate t + r_0^2,\nonumber\\
  &r(t) = \sqrt{r_0^2-2\kappa t}.\label{eq:particle-exact}
\end{align}
Si $\bar t = \frac{r_0^2}{2\dissolutionrate}$ on obtient $r(\bar
t) = 0$. La particule de rayon $r_0$ sera dissoute après un temps $\bar
t = \frac{r_0^2}{2\dissolutionrate}$.

Revenons maintenant à l'équation pour la population de particule
$\population$. On a:
\begin{equation}\label{eq:population-pde-2}
  \frac{\partial n_p}{\partial t} -
  \frac{\dissolutionrate}{r}\frac{\partial n_p}{\partial r} +
  \frac{\dissolutionrate}{r^2}n_p = 0.
\end{equation}
\newcommand{\rcharacteristic}{R_{(\bar t, \bar r)}}
Si, dans le plan $(t, r)$ on fixe $t = \bar t > 0$, $r = \bar r > 0$,
l'équation de la courbe caractéristique $\rcharacteristic$ qui passe par $(\bar t,
\bar r)$ est donnée par
\begin{equation*}
\frac{\mathrm d}{\mathrm dt}R_{(\bar t, \bar r)}(t) =
-\frac{\dissolutionrate}{R_{(\bar t, \bar r)}(t)}.
\end{equation*}
Ainsi $R_{(\bar t, \bar r)}(t) = \sqrt{\bar r^2 - 2\dissolutionrate (t
  - \bar t)}$, en vertu de (\ref{eq:particle-exact}). Notons $N_p(t) =
n_p(t, R_{(\bar t, \bar r)}(t))$ la valeur de $n_p$ sur cette
caractéristique. En dérivant par rapport à $t$ et en utilisant
(\ref{eq:population-pde-2}) on a
\begin{align*}
  \frac{\mathrm d}{\mathrm dt}N_p(t) %
  &= \frac{\partial n_p}{\partial t}(t, \rcharacteristic(t)) %
  + \frac{\partial}{\partial r}n_p(t,\rcharacteristic(t)) %
    \frac{\mathrm d}{\mathrm dt}\rcharacteristic(t)\\
  &= \frac{\partial n_p}{\partial t}(r, \rcharacteristic) - \frac{\dissolutionrate}{\rcharacteristic}\frac{\partial}{\partial r} n_p(t, \rcharacteristic(t))\\
  &= -\frac{\dissolutionrate}{R_{(\bar t, \bar r)}^2(t)}N_p(t). %
\end{align*}
Ainsi
\begin{equation*}
  \frac{1}{N_p(t)}\frac{\mathrm d}{\mathrm dt}N_p(t) =
  -\frac{\dissolutionrate}{R^2_{(\bar t, \bar r)}(t)}
\end{equation*}
qui implique que
\begin{equation*}
  \frac{\mathrm d}{\mathrm dt}\ln(N_p(t)) =
  -\frac{\dissolutionrate}{\parent{\bar r^2 - 2\dissolutionrate (t - \bar t)}}.
\end{equation*}
En intégrant on obtient
\begin{equation*}
  \ln(N_p(t)) = \frac{1}{2}\ln\parent{\bar r^2 -
    2\dissolutionrate\parent{t - \bar t}} + D
\end{equation*}
et donc
$N_p(t) = \parent{\bar r^2 - 2\dissolutionrate\parent{t - \bar t}}^{1/2} \exp(D)$. Si $t = 0$, on a:
\begin{equation*}
  N_p(0) = \exp(D)\sqrt{\bar r^2 +
    2\dissolutionrate \bar t} = n_{p,0}(R(0)) = n_{p,0}(\sqrt{\bar r^2 +
    2\dissolutionrate \bar t})
\end{equation*}
Ainsi
$\exp(D) = \frac{n_{p,0}\parent{\sqrt{\bar r^2 + 2\dissolutionrate \bar t\,}}}{\sqrt{\bar r^2 + 2\dissolutionrate \bar t\,}}$
et
\begin{equation*}
  N_p(t) = \frac{n_{p,0}(\sqrt{\bar r^2 + 2\dissolutionrate \bar
      t})}{\sqrt{\bar r^2 + 2\dissolutionrate \bar t\,}}\parent{\bar r^2 -
    2\dissolutionrate \parent{t - \bar t}}^{1/2}.
\end{equation*}
Clairement
$N_p(\bar t) = n_p(\bar t, \bar r)$
et donc
\begin{equation}\label{eq:population-exact-solution}
  n_p(\bar t, \bar r) = \bar r\frac{n_{p,0}(\sqrt{\bar r^2 +
      2\dissolutionrate \bar t})}{\sqrt{\bar r^2+2\dissolutionrate \bar t}}.
\end{equation}

\begin{remarque}\label{rem:population-exact-step}
  Supposons connu $n_p(\bar t - \Delta t, r)$. Nous avons ainsi
  \begin{equation*}
    R_{(\bar t, \bar r)}(\bar t - \Delta t) = \sqrt{\bar r^2 +
      2\dissolutionrate \Delta t\,}.
  \end{equation*}
  Puisque $N_p(t) = C\parent{\bar r^2 - 2\dissolutionrate\parent{t -
      \bar t}}^{1/2}$, alors
  \begin{equation*}
N_p(\bar t - \Delta t) = C\parent{\bar r^2 + 2\dissolutionrate \Delta
  t} = n_p\parent{t - \Delta t, \sqrt{\bar r^2 + 2\dissolutionrate
    \Delta t\,}}.
  \end{equation*}
  Ainsi
  \begin{equation*}
    C = \frac{n_p(t - \Delta t, \sqrt{\bar r^2 + 2\dissolutionrate
        \Delta t})}{\bar r^2 + 2\dissolutionrate\Delta t}
  \end{equation*}
  et donc
  \begin{equation*}
    N(t) = \frac{n_p(t - \Delta t, \sqrt{\bar r^2 + 2\dissolutionrate
        \Delta t})}{\bar r^2 + 2\dissolutionrate\Delta t}\parent{\bar
      r^2 - 2\dissolutionrate\parent{t - \bar t}}^{1/2}.
  \end{equation*}
  Puisque $N(\bar t) = n_p(\bar t, \bar r)$, on obtient
  \begin{equation*}
    n_p(\bar t, \bar r) = \bar r\frac{n_p(t - \Delta t, \sqrt{\bar
        r^2 + 2\dissolutionrate \Delta t})}{\sqrt{\bar r^2 +
        2\dissolutionrate \Delta t}}.
  \end{equation*}
\end{remarque}

Dans cette partie nous avons présenté le modèle décrit l'évolution d'une
distribution de particules qui se dissout dans un bain électrolytique
selon la fonction (\ref{eq:dissolution-velocity}). On a montré que
l'évolution de la population $n_p$ s'exprime sous forme exacte,
donnée par (\ref{eq:population-exact-solution}), pour
autant que la concentration $c$ et la température $\temperature$
sont maintenue constante.

Nous utiliserons le résultat de la remarque
\ref{rem:population-exact-step} lors de la discusssion de la
discrétisation du modèle de transport et dissolution complet dans le
chapitre \ref{chap:populations}.

La section suivante tient lieu de conclusion ce chapitre, et discute
de la sédimentation des particules d'alumine dans le bain sous
l'action de la force de gravité.

%%\paragraph{Approximation numérique}
%%Soit $T>0$ le temps final, $N \in \mathbb N^*$ le nombre de pas de
%%temps, $\tau = \frac{T}{N}$, $\Delta r = \frac{\rmax}{M}$,
%%$t_i = i \tau$, $r_j = j\Delta r$ constantes
