Dans ce dernier chapitre, nous avons proposé une méthode numérique
pour calculer l'écoulement de fluides dans un domaine
$\Omega = \Lambda\times(0,\thickness)$ de $\mathbb R^3$, basée sur une
décomposition en harmoniques de Fourier des inconnues et des
méthodes d'éléments finis pour chaque coefficients des séries de
Fourier. Cette formulation nécessite d'imposer des conditions
d'adhérence sur les bords verticaux du domaine et des conditions de
glissement total sur les bords horizontaux.

Une caractésistique essentielle de cette méthode est de calculer une
approximation de l'écoulement tridimensionnel en résolvant une
série de problèmes bidimensionnels découplés les uns des autres. Cette
approche devient de plus en plus intéressante lorsque l'épaisseur du
domaine $\thickness$ s'approche de 0. En effet, on observe d'une part
que le temps CPU est essentiellement indépendent de $\thickness$, mais
en plus l'erreur d'approximation diminue lorsque $\thickness$
diminue. C'est un clair avantage par rapport à une méthode
d'éléments finis classiques basée sur un maillage tetraédrique. En
raison de la péjoration du conditionnement des matrice éléments
finis lorsque $\thickness$ tend vers 0, \ie, lorsque le rapport
d'aspect du maillage devient de plus en plus grand, la convergence des
méthodes iteratives devient de plus en plus lente. De plus, l'erreur
d'approximation tend accroître lorsque $\thickness$ diminue.

Cette méthode est donc bien adaptée au calcule d'écoulement de
fluides en couches minces, pour autant que les conditions aux limites
soient adaptées.
