Dans ce dernier chapitre, nous avons proposé une méthode numérique
pour calculer l'écoulement de fluides dans un domaine
$\Omega = \Lambda\times(0,\thickness)$ de $\mathbb R^3$, basée sur une
décomposition en harmoniques de Fourier des inconnues et des
méthodes d'éléments finis pour chaque coefficients des séries de
Fourier. Cette formulation nécessite d'imposer des conditions
d'adhérence sur les bords verticaux du domaine et des conditions de
glissement total sur les bords horizontaux.

Une caractésistique essentielle de cette méthode est de calculer une
approximation de l'écoulement tridimensionnel en résolvant une
série de problèmes bidimensionnels découplés les uns des autres. Cette
approche devient de plus en plus intéressante lorsque l'épaisseur du
domaine $\thickness$ s'approche de 0. En effet, on observe d'une part
que le temps CPU est essentiellement indépendent de $\thickness$, mais
en plus l'erreur d'approximation diminue lorsque $\thickness$
diminue. C'est un clair avantage par rapport à une méthode
d'éléments finis classiques basée sur un maillage tetraédrique. En
raison de la péjoration du conditionnement des matrice éléments
finis lorsque $\thickness$ tend vers 0, \ie, lorsque le rapport
d'aspect du maillage devient de plus en plus grand, la convergence des
méthodes iteratives devient de plus en plus lente. De plus, l'erreur
d'approximation tend a croître lorsque $\thickness$ diminue.
Cette méthode est donc bien adaptée au calcule d'écoulement de
fluides en couches minces, pour autant que les conditions aux limites
soient adaptées.

La formulation du modèle de Stokes-Fourier impose des contraintes
particulières au niveau de la géométrie du domaine, des conditions
aux limites sur certaines parties du bord du domaine. Dans le cas
d'une cuve d'électrolyse d'aluminium, on est contraint de négliger la
présence des l'ensemble des canaux et de l'inclinaison des parois
latérales de la cuve. Au niveau des conditions aux limites, on est
forcés d'imposer des conditions de glissement parfait sur les partie
du bord en contact avec les anodes ou la cathode, et des conditions
d'adhérence parfaite sur les parois verticales. Dans une cuve
d'électrolyse, alors que le choix des conditions aux limites sur les
parois est cohérent avec le modèle S3D, les conditions aux limites sur
la cathodes ne correspondent pas. Le choix des conditions aux limites
sur les anodes est sujet à discussion en raison de la présence de
film gazeux à leur surface et à la présence de bulles de \ce{CO2}
qui provoquent une forte agitation.

La structure des champs de vitesse $u_h^\mathrm{SF}$ est
similaire au modèle de référence $u_h^\mathrm{S3D}$ lorsque
toutes les anodes sont activées. Cependant, lorsque des anodes sont
désactivées l'écoulement de rérérence $u_h^\mathrm{S3D}$ développe des
petites structures et des tourbillons additionnels à proximité de
l'anodes désactivée que la solution $u_h^\mathrm{S3D}$ peine à
reproduire.

Le calcul de distribution stationnaire $c_h^\mathrm{SF}$ dépend
directement de la vitesse d'écoulement $u_h^\mathrm{SF}$. La
distribution $c_h^\mathrm{SF}$ certaines caractéristique
significatives de la distribution de référence $c_h^\mathrm{S3D}$,
tels que les excès de concentration à proximité des points
d'injection. Cependant, la distribution d'alumine dans la cuve
dépend de manière sensible à la position des tourbillons principaux
de l'écoulement $u_h^\mathrm{SF}$. Il suffit que ces tourbillons
soient légérement décalés par rapport à l'emplacement des injecteurs
pour que la distribution d'alumine dissoute change drastiquement.
