Dans cette section on propose de valider l'implémentation des schémas
numériques décrits dans la section
(\ref{sec:fourier-discretisation}). Dans un premier temps, on
introduit une solution exacte du problème
(\ref{eq:stokes-u}),(\ref{eq:stokes-p}) qui satisfait les conditions
aux limites (\ref{eq:stokes-bc-1})-(\ref{eq:stokes-bc-3}). Dans un
deuxième temps, on analysera la convergence de l'erreur entre la
solution exacte et l'approximation $u_{h,K}$ telle que définie par
(\ref{eq:u-h-12}), (\ref{eq:u-h-3}).

\subsection{Une solution exacte polynomiale non triviale}
On pose $\Lambda = (-1, 1)^2$, on rappelle que $\Omega =
\Lambda\times(0,\thickness)$. On choisit dans cette partie
\ref{sec:fourier-validation} une viscosité scalaire unité,
c'est-à-dire que $\mu_{i,j} = 1\ \forall i,j$. On vérifie alors que les
fonctions $u_1, u_2, u_3:\Omega\to\mathbb R$ et $p:\Omega\to\mathbb R$
définies par:
\begin{align}
  &u_1(x_1, x_2, x_3) = u_2(x_1, x_2, x_3) = \parent{x_1^2 - 1} %
  \parent{x_2^2 - 1}
  \parent{x_3^3 - \frac{3}{2}\thickness x_3^2 + \frac{1}{4}\thickness ^3},\label{eq:stokes-exact-sol-1}\\
  &u_3(x_1, x_2, x_3) = %
  \begin{aligned}[t]
    &\parent{-4x_1\parent{x_1^2 - 1} %
                                   \parent{x_2^2 - 1}^2 %
                              -4x_2\parent{x_2^2 - 1} %
                                   \parent{x_1^2 - 1}^2} \\
    &\cdot\parent{\frac{1}{4}x_3^4 - \frac{1}{2}\thickness x_3^3 + \frac{1}{4}\thickness ^3x_3},
  \end{aligned}\label{eq:stokes-exact-sol-2}\\
  &p(x_1, x_2, x_3) = x_1^3 x_2^4 x_3^5\label{eq:stokes-exact-sol-3}
\end{align}
satisfont (\ref{eq:stokes-u}), (\ref{eq:stokes-p}) avec les
conditions aux limites (\ref{eq:stokes-bc-1})-(\ref{eq:stokes-bc-3}),
pour autant que $f$ dans (\ref{eq:stokes-u}) soit définie par
\begin{multline}
  f_1(x_1, x_2, x_3) =
  -\left(
       8x_1^2
       \parent{x_2^2 - 1}^2
       \parent{\frac{1}{4}\thickness ^4 - \frac{3}{2}\thickness x_3^2 + x_3^3}
     - \parent{x_1^2 - 1}
       \parent{x_2^2 - 1}^2
       \parent{3\thickness  - 6x_3}
   \right.\\
  + 8x_2^2
  \parent{x_1^2-1}^2
  \parent{\frac{1}{4} \thickness ^3 - \frac{3}{2}\thickness x_3^3 + x_3^3}
  + 4\parent{x_1^2 - 1}
     \parent{x_2^2 - 1}^2
     \parent{\frac{1}{4}\thickness ^3 -\frac{3}{2} \thickness x_3^2 + x_3^3}\\
    + \left.4\parent{x_1^2 - 1}^2\parent{x_2^2 - 1}
      \parent{\frac{1}{4}\thickness ^3 - \frac{3}{2} \thickness x_3^2 + x_3^3}\right)\\
    + 3x_1^2 x_2^3x_3^5,
\end{multline}
\begin{multline}
  f_2(x_1, x_2, x_3) =
  -\left(8x_1^2 \parent{x_2^2 - 1}^2
    \parent{\frac{1}{4}\thickness ^4
      - \frac{3}{2}\thickness x_3^2 + x_3^3}
    - \parent{x_1^2 - 1}
    \parent{x_2^2 - 1}^2
    \parent{3\thickness  - 6x_3}\right.\\
    + 8x_2^2
    \parent{x_1^2-1}^2
    \parent{\frac{1}{4}\thickness ^3 - \frac{3}{2}\thickness x_3^3 + x_3^3}\\
    + 4\parent{x_1^2 - 1}
    \parent{x_2^2 - 1}^2
    \parent{\frac{1}{4}\thickness ^3 -\frac{3}{2} \thickness x_3^2 + x_3^3}\\
    \left. + 4\parent{x_1^2 - 1}^2\parent{x_2^2 - 1}
    \parent{\frac{1}{4}\thickness ^3 -\frac{3}{2} \thickness x_3^2 + x_3^3}\right)\\
    + 4x_1^3 x_2^3x_3^5,
\end{multline}
et
\begin{multline}
  f_3(x_1, x_2, x_3) =
  \parent{\frac{1}{4}\thickness ^3x_3 - \frac{1}{2} \thickness  x_3^3
    + \frac{1}{4}x_3^4} \left(24 x_2
  \parent{x_1^2 - 1}^2\right.\\
  + 32x_1x_2^2\parent{x_1^2 - 1}
  + \left.16x_1\parent{x_1^2 - 1}\parent{x_2^2 - 1}\right)
  + \parent{\frac{1}{4}\thickness ^3x_3 - \frac{1}{2} \thickness
    x_3^3 + \frac{1}{4}x_3^4}
  \left(24 x_1 \parent{x_2^2 - 1}^2\right.\\
  + 32x_1^2x_2\parent{x_2^2 - 1}
  + \left.16x_2\parent{x_1^2 - 1}\parent{x_2^2 - 1}\right)\\
  + \parent{3x_3^2 - 3\thickness x_3}
  \parent{4x_1\parent{x_1^2 - 1}
    \parent{x_2^2 - 1}^2
    + 4x_2\parent{x_1^2 - 1}^2\parent{x_2^2 - 1}}\\
  + 5x_1^3x_2^4x_3^4.
\end{multline}

Les figures \ref{fig:stokes-exact-sol},
\ref{fig:stokes-exact-sol-streamlines} représentent la solution
(\ref{eq:stokes-exact-sol-1})-(\ref{eq:stokes-exact-sol-3}) pour le
choix $\thickness = 2$, i. e. dans le domaine $\Omega =
[-1,1]^2\times[0,\thickness]$.

\begin{figure}[t]
  \begin{center}
    \begin{subfigure}[b]{0.49\textwidth}
      \includegraphics[width=\textwidth]{../media/fourier/fig-exact/print/slice-6.png}
      \caption{$x_3 = 2$}
    \end{subfigure}
    \begin{subfigure}[b]{0.49\textwidth}
      \includegraphics[width=\textwidth]{../media/fourier/fig-exact/print/slice-5.png}
      \caption{$x_3 = 1.6$}
    \end{subfigure}
    \begin{subfigure}[b]{0.49\textwidth}
      \includegraphics[width=\textwidth]{../media/fourier/fig-exact/print/slice-4.png}
      \caption{$x_3 = 1.2$}
    \end{subfigure}
    \begin{subfigure}[b]{0.49\textwidth}
      \includegraphics[width=\textwidth]{../media/fourier/fig-exact/print/slice-3.png}
      \caption{$x_3 = 0.8$}
    \end{subfigure}
    \begin{subfigure}[b]{0.49\textwidth}
      \includegraphics[width=\textwidth]{../media/fourier/fig-exact/print/slice-2.png}
      \caption{$x_3 = 0.4$}
    \end{subfigure}
    \begin{subfigure}[b]{0.49\textwidth}
      \includegraphics[width=\textwidth]{../media/fourier/fig-exact/print/slice-1.png}
      \caption{$x_3 = 0$}
    \end{subfigure}
    \caption{Représentation de la solution exacte
      (\ref{eq:stokes-exact-sol-1})-(\ref{eq:stokes-exact-sol-3}) du
      problème de Stokes (\ref{eq:stokes}). La solution est
      représentée sur 6 plans horizontaux espacés
      régulièrement entre $x_3 = 0$ et $x_3 = 2$. L'échelle de
      couleur correspond à l'amplitude de $u$, tandis que les
      flèches indique la direction de $u$.}
    \label{fig:stokes-exact-sol}
  \end{center}
\end{figure}

\begin{figure}[t]
  \begin{center}
    \includegraphics[width=\rasterimagewidth]{../media/fourier/fig-exact/print/streamlines.png}
    \caption{Lignes de courant de la vitesse $u$ définie par
      (\ref{eq:stokes-exact-sol-1})-(\ref{eq:stokes-exact-sol-2}) dans
      le domaine cubique $\Omega$, c'est-à-dire que $\thickness = 2$.}
    \label{fig:stokes-exact-sol-streamlines}
  \end{center}
\end{figure}


\subsection{Convergence de l'erreur}
On s'intéresse d'une part à la convergence de l'erreur d'approximation
de $u_{h,K}$ donnée par (\ref{eq:u-h-12}),(\ref{eq:u-h-3}) vers la
solution exacte
(\ref{eq:stokes-exact-sol-1})-(\ref{eq:stokes-exact-sol-2}), définie
par:
\begin{equation}
  e_{h,K}^{\mathrm{SF}} =
  \frac{\norm{u - u_{h,K}}_{L^2(\Omega)}}{\norm{u}_{L^2(\Omega)}},
\end{equation}
où le champ de vecteur $u$ est donné par
(\ref{eq:stokes-exact-sol-1})-(\ref{eq:stokes-exact-sol-2}).

On cherchera d'autre part à comparer $u_{h,K}$ (notée
$u_{h,K}^{\mathrm{SF}}$ dans la suite) avec la solution de
(\ref{eq:stokes-u}), (\ref{eq:stokes-p}) obtenue à l'aide d'une
méthode d'éléments finis $\mathbb P_1$bulle-$\mathbb P_1$ sur un
maillage tétraédrique classique (notée
$u_h^\mathrm{S3D}$ dans la suite). On note aussi l'erreur
\begin{equation}
  e_h^{\mathrm{S3D}} = \frac{\norm{u - u_h^{\mathrm{S3D}}}_{L^2(\Omega)}}{\norm{u}_{L^2(\Omega)}}.
\end{equation}

\subsubsection{Convergence de la somme partielle de Fourier}
On s'intéresse à la convergence de l'erreur $e_h^\mathrm{SF}$ en
fonction du nombre d'harmoniques $K$ dans la somme partielle de
$u_{h,K}$. On prend pour le maillage $\mathcal M_h$ du domaine $\Lambda$
une triangulation structurée de taille de maille $h = 2/64$. La
table \ref{tab:h-k-convergence} donne les valeurs de l'erreur
$e_{h,K}^\mathrm{SF}$ pour différents nombres d'harmoniques $K$ et
différentes épaisseurs $\thickness$ du domaine $\Omega$.

\begin{table}[t]
  \caption{Erreur $e_{h,K}^\mathrm{SF}$ entre la solution exacte et la
    série de Fourier tronquée à l'ordre $K$.}
  \label{tab:h-k-convergence}
  \begin{center}
    \begin{tabular}{@{}rrrrrr@{}}
      \toprule
                   %& $K = 1$ & $K = 2$ & $K = 3$ & $K = 4$ & $K = 5$ & $K = 6$ & $K = 7$ & $K = 8$\\
      $\epsilon$   & $K = 1$ & $K = 3$ & $K = 5$ & $K = 7$ & $K = 9$\\
      \cmidrule{1-6}
      $2$      & \num{0.8918E-02} & \num{0.1216E-02} & \num{0.5207E-03} & \num{0.4378E-03} & \num{0.4259E-03} \\
      $0.2$    & \num{0.1238E-01} & \num{0.1654E-02} & \num{0.4547E-03} & \num{0.1882E-03} & \num{0.1117E-03} \\
      $0.02$   & \num{0.1245E-01} & \num{0.1662E-02} & \num{0.4508E-03} & \num{0.1727E-03} & \num{0.8142E-04} \\
      %
      %$h = 2$      & \num{0.89186348E-02} & \num{0.89189885E-02} & \num{0.12167724E-02} & \num{0.12172997E-02} & \num{0.52070528E-03} & \num{0.52114649E-03} & \num{0.43781666E-03} & \num{0.43807398E-03} & \num{0.42595367E-03} \\
      %$h = 0.2$    & \num{0.12389603E-01} & \num{0.12389604E-01} & \num{0.16543681E-02} & \num{0.16543708E-02} & \num{0.45478885E-03} & \num{0.45479262E-03} & \num{0.18823592E-03} & \num{0.18823987E-03} & \num{0.11171968E-03} \\
      %$h = 0.02$   & \num{0.12455460E-01} & \num{0.12455460E-01} & \num{0.16621111E-02} & \num{0.16621111E-02} & \num{0.45086937E-03} & \num{0.45086937E-03} & \num{0.17279275E-03} & \num{0.17279275E-03} & \num{0.81428389E-04} \\
      \bottomrule
    \end{tabular}
  \end{center}
\end{table}


\subsubsection{Convergence en fonction du maillage}
On s'intéresse à la convergence de l'erreur $e_{h,K}^\mathrm{SF}$ en
fonction de la taille de maille $h$ de la subdivision $\mathcal M_h$ du domaine
$\Lambda$. Le tableau \ref{tab:n-h-convergence} donne les valeurs de
l'erreur pour différentes tailles de maille $h$ et différentes
épaisseurs $\thickness$ du domaine $\Omega$.

\begin{table}[t]
  \caption{Erreur $e_{h,K}^\mathrm{SF}$ en fonction de $h$ et de
    la taille de maille $h$ et de l'épaisseur $\thickness$ lorsque $K
    = 20$.}
  \label{tab:n-h-convergence}
  \begin{center}
    \begin{tabular}{@{}rrrrrr@{}}
      \toprule
      $\epsilon$ & $h = 0.25$ & $h = 0.125$ & $h = 0.0625$ & $h = 0.03125$ & $h = 0.015625$ \\
      \cmidrule{1-6}
      $2$      & \num{0.9765E-01} & \num{0.2611E-01} & \num{0.6698E-02} & \num{0.1688E-02} & \num{0.4232E-03} \\
      $0.2$    & \num{0.4184E-02} & \num{0.2342E-02} & \num{0.9966E-03} & \num{0.2962E-03} & \num{0.7794E-04} \\
      $0.02$   & \num{0.3977E-04} & \num{0.1166E-04} & \num{0.9187E-05} & \num{0.1121E-04} & \num{0.1210E-04} \\
      %$h = 2$      & \num{0.97650317E-01} & \num{0.26117175E-01} & \num{0.66981324E-02} & \num{0.16888002E-02} & \num{0.42326195E-03} \\
      %$h = 0.2$    & \num{0.41848104E-02} & \num{0.23426567E-02} & \num{0.99662780E-03} & \num{0.29620176E-03} & \num{0.77949986E-04} \\
      %$h = 0.02$   & \num{0.39774924E-04} & \num{0.11660001E-04} & \num{0.91876790E-05} & \num{0.11211307E-04} & \num{0.12101410E-04} \\
      \bottomrule
    \end{tabular}
  \end{center}
\end{table}


\subsubsection{Comparaison des erreur $e_h^{\mathrm{SF}}$ et
  $e_h^{\mathrm{S3D}}$} On s'intéresse maintenant à comparer la
précision des calculs $u_{h,K}^\mathrm{SF}$ et
$u_h^{\mathrm{S3D}}$. Ici on a choisi $K = 20$ et $h = 2/64$. La table
\ref{tab:e-sf-e-s3d-convergence} donne la valeur des erreurs pour
différents épaisseurs $\thickness$ du domaine $\Omega$. La figure
\ref{fig:e-sf-e-s3d-convergence} représente les même données sous
forme graphique.

\begin{table}[t]
  \caption{Erreurs $e_{h,K}^\mathrm{SF}$ et $e_h^\mathrm{S3D}$ en fonction
    de l'épaisseur $\thickness$ lorsque $K = 20$ et $h = 2/64$.}
  \label{tab:e-sf-e-s3d-convergence}
  \begin{center}
    \begin{tabular}{@{}rrrrrrrrrrr@{}}
      \toprule
      & $\thickness = 0.020$
      %& $\thickness = 0.033$
      & $\thickness = 0.055$
      %& $\thickness = 0.092$
      & $\thickness = 0.154$
      %& $\thickness = 0.258$
      & $\thickness = 0.430$
      %& $\thickness = 0.718$
      & $\thickness = 1.198$ \\
      %& $\thickness = 2$ \\
      \cmidrule{2-6}
      $e_{h,K}^\mathrm{SF}$  & \num{0.1121E-04} & \num{0.8356E-04} & \num{0.2478E-03} & \num{0.4944E-03} & \num{0.1225E-02}\\
      $e_h^\mathrm{S3D}$ & \num{0.1426E-02} & \num{0.1351E-02} & \num{0.1175E-02} & \num{0.9317E-03} & \num{0.6221E-03}\\
      %$e_h^\mathrm{SF}$  & \num{0.11211307E-04} & \num{0.30500446E-04} & \num{0.83562271E-04} & \num{0.16100704E-03} & \num{0.24780044E-03} & \num{0.34930482E-03} & \num{0.49448033E-03} & \num{0.76153942E-03} & \num{0.12254762E-02} & \num{0.16888002E-02} \\
      %$e_h^\mathrm{S3D}$ & \num{0.14265133E-02} & \num{0.14010261E-02} & \num{0.13516086E-02} & \num{0.12761443E-02} & \num{0.11759665E-02} & \num{0.10615871E-02} & \num{0.93173182E-03} & \num{0.78897224E-03} & \num{0.62210445E-03} & \num{0.84532577E-03} \\
      \bottomrule
    \end{tabular}
  \end{center}
\end{table}

\begin{figure}
  \begin{center}
    \input{../media/fourier/convergence/sf-convergence.tex}
    \caption{Erreurs $e_{h,K}^\mathrm{SF}$ et $e_h^\mathrm{S3D}$ en fonction
    de l'épaisseur $\thickness$ lorsque $K = 20$ et $h = 2/64$.}
    \label{fig:e-sf-e-s3d-convergence}
  \end{center}
\end{figure}

Finalement, on compare les temps CPU pour chaque schéma, ainsi
que le nombre d'itérations de l'algorithme GMRES pour résoudre le
système linéaire dans le cas du schéma S3D.

\begin{table}[t]
  \caption{Comparaison des performances entre les schémas permettant
    de calculer $u_{h,K}^{\mathrm{SF}}$ et $u_h^\mathrm{S3D}$ en
    fonction de l'épaisseur $\thickness$. Les calculs
    sont effectués sur un machine équipée d'un processeur Intel$^{\tiny{\textregistered}}$ Xeon$^{\tiny{\textregistered}}$
    E5-2620 v2 @ 2.1\si{\giga\hertz} et 32Go de RAM.}
  \label{tab:e-sf-e-s3d-cpu-cost}
  \begin{center}
    \begin{tabular}{@{}lrrrrr@{}}
      \toprule
      & $\thickness = 0.020$
      %& $\thickness = 0.033$
      & $\thickness = 0.055$
      %& $\thickness = 0.092$
      & $\thickness = 0.154$
      %& $\thickness = 0.258$
      & $\thickness = 0.430$
      %& $\thickness = 0.718$
      & $\thickness = 1.198$ \\
      %& $\thickness = 2$ \\
      \cmidrule{2-6}
      F3D  \\
      \hphantom{a} $T_\text{CPU}$ [\si{\second}] & 4587 & 2694 & 1343 & 753 & 753 \\
      \hphantom{a} itérations GMRES           & 901 & 609  & 327  & 184 & 133 \\
      SF \\
      \hphantom{a} $T_\text{CPU}$ [\si{\second}] & 43 & 45   & 43   & 44  & 45  \\
      \bottomrule
    \end{tabular}
  \end{center}
\end{table}
