Comme annoncé dans le chapitre \ref{chap:introduction}, une cuve
d'électrolyse typique dans une halle de production n'est jamais dans
un état stationnaire \cite{Steiner2009,Flotron2013}. La cause
majeure pour laquelle ces cuves se trouvent dans un état de
déséquilibre constant est liée aux changements d'anodes en fin de vie,
qui interviennent toutes les 24 heures environ.

Cet état de déséquilibre entraîne d'importantes variations des
écoulements des fluides dans la cuve, et ces variations ont un impact
non négligeable sur la répartition de l'alumine dissoute dans le bain
électrolytique. Pour palier à ces variations de répartition de
concentration d'alumine, les opérateurs peuvent modifier les cadences
d'injection de chaque injecteur. Malheureusement, il ne disposent pour
l'heure d'aucun outil qui leur permette de déterminer comment adapter
ces cadences en fonction de l'état de la cuve.

Le modèle de dissolution et transport d'alumine proposé dans
\cite{Hofer2011} et dont nous avons étudié certaines extensions dans
la première partie de ce travail, peut apporter un début de réponse: la
conductivité des différentes anodes dépend de leur âge, ce qui modifie
le champ de vitesse dans le bain électrolytique et dans l'aluminium
liquide par l'intermédiaire de la densité de courant
électrique et des forces de Lorentz, ainsi que la distribution d'alumine
dissoute. Malheureusement, ce calcul ne peut pas être fait en temps
réel, et donc la cadence des injecteurs qui optimise la répartition de
l'alumine dissoute dans le bain ne peut pas non plus être obtenu en
temps réel.

Dans ce chapitre nous proposons une méthode pour calculer le champ de
vitesse dans une cuve d'électrolyse bien plus rapidement qu'avec les
modèles de \cite{Steiner2009} et \cite{Hofer2011}, au prix de
certaines hypothèses supplémentaires que l'on discutera.

La géométrie d'une cuve d'aluminium est particulière: l'essentiel de
l'écoulement des fluides prend place dans un domaine approximativement
parallélépipédique qui présente une forte anisotropie. Les dimensions
horizontales ($\approx \num{14}\times\num{4}$ \si{\meter}) sont bien plus
grandes que la dimension verticales ($\approx \num{0.2}$ \si{\meter}). Par
conséquent, l'écoulement est essentiellement contraint dans le plan
horizontal, si l'on néglige ce qui se passe dans les canaux.

Le modèle de référence implémenté avec le logiciel \alucell
\cite{Steiner2009,Flotron2013,Hofer2011,Rochat2016} calcule une
approximation des écoulements dans les fluides en utilisant une
méthode d'éléments finis Lagrange continu linéaire par morceau sur un
maillage en tétraèdres du domaine fluide.

Or, la géométrie anisotrope du domaine fluide d'une cuve d'électrolyse
pose de sérieuses difficultés au niveau numérique. En effet,
l'utilisation de mailles isotropes n'est pas possible puisque la
taille de maille dans la direction verticale est de l'ordre de
\num{0.01} \si{\meter}.  La seule alternative est de considérer des
mailles anisotropes de l'ordre de \num{0.01} \si{\meter} dans la
direction verticale et \num{0.1} \si{\meter} dans les directions
horizontales. Cependant, l'utilisation de mailles anisotropes
augmente le conditionnement des matrices et donc le nombre
d'itérations nécessaires à la résolution du système linéaire et
dégrade la précision du calcul.

Le but de ce chapitre est de proposer une méthode pour calculer des
écoulements de Stokes dans des domaines parallélépipédiques présentant
une forte anisotropie. Dans la section \ref{sec:fourier-model} nous
proposons une décomposition de Fourier de la vitesse selon la
direction verticale, puis nous proposons dans la section
\ref{sec:fourier-discretisation} une méthode numérique de type
éléments finis pour approcher chaque harmonique de la décomposition de
Fourier. Dans la section \ref{sec:fourier-validation} nous validons
l'implémentation du schéma ainsi obtenu et étudions la convergence de
l'erreur par rapport à une solution non triviale. Finalement, dans la
section \ref{sec:fourier-application} nous appliquons cette méthode au
cas industriel d'une cuve d'électrolyse et évaluons les propriétés de
cette méthode.
