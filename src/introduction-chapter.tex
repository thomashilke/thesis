L'aluminium est un métal léger qui figure parmi les éléments
métalliques les plus abondant sur Terre, avec une proportion d'environ
\num{8}\% de la masse totale de la croûte terrestre. Malgré tout il
n'apparaît jamais sous forme métallique en raison de sa forte affinité
avec l'oxygène, et se trouve sous forme d'oxyde d'aluminium \ce{Al2O3}
(appelé également alumine). L'alumine est un constituant de différents
minerais, mais seul la bauxite est utilisée pour la production de
l'aluminium.

La bauxite est constituée principalement par de l'oxyde d'aluminium
hydraté, de silicate, d'oxyde de fer et d'impuretés en plus faibles
proportions. Elle provient essentiellement d'Australie, de Chine et du
Brésil. Par exemple, ces trois pays combinés ont extrait plus de
\num{179} millions de tonnes de bauxite au cours de l'année
\num{2014}. Le procédé de Bayer permet ensuite d'isoler et purifier
l'alumine de la bauxite. Après concassage et broyage du minerais, la
soude caustique chaude dissout l'alumine:
\begin{align*}
  \cee{NaOH &-> Na+ + OH-},\\
  \cee{Al2O3.(H2O) + 2OH- &-> 2AlO2- + 2H2O},\\
  \cee{Al2O3.(H2O)3 + 2OH- &-> 2AlO2- + 4H2O}.
\end{align*}

Les autres composés restent insolubles, et sont éliminés par
filtration. En refroidissant le filtrat, de l'hydrate d'alumine est
récupéré par précipitation. Finalement, cet hydrate d'alumine est
calciné pour éliminer les molécules d'eau:
\begin{equation*}
  \cee{Al2O3.(H2O)3 -> Al2O3}.
\end{equation*}

La deuxième partie du processus consiste à inverser la réaction
d'oxydation pour finalement obtenir de l'aluminium sous forme
métallique. C'est cet aspect du procédé, qui porte les noms de ses
deux inventeurs Paul Héroult et Mark Hall, qui nous intéresse dans
ce travail et que nous décrivons plus en détail dans le paragraphe
qui suit.
