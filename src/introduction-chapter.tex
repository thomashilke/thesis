L'aluminium est un métal léger qui figure parmi les éléments
métalliques les plus abondant sur Terre, avec une proportion d'environ
\num{8}\% de la masse totale de la croûte terrestre. Malgré tout il
n'apparaît jamais sous forme métallique en raison de sa forte affinité
avec l'oxygène, et se trouve sous forme d'oxyde d'aluminium \ce{Al2O3}
(appelé également alumine). L'alumine est un constituant de différents
minerais, mais seul la bauxite est utilisée pour la production de
l'aluminium \cite{Hydro2018}.

La bauxite est constituée principalement par de l'oxyde d'aluminium
hydraté, de silicate, d'oxyde de fer et d'impuretés en plus faibles
proportions. Elle provient essentiellement d'Australie, de Chine et du
Brésil. Par exemple, ces trois pays combinés ont extrait plus de
\num{182.9} millions de tonnes de bauxite au cours de l'année \num{2015}
\cite{USGS2015}. Le procédé de Bayer permet ensuite d'isoler et
purifier l'alumine de la bauxite. Après concassage et broyage du
minerais, la soude caustique chaude dissout l'alumine:
\begin{align*}
  \cee{NaOH &-> Na+ + OH-},\\
  \cee{Al2O3.(H2O) + 2OH- &-> 2AlO2- + 2H2O},\\
  \cee{Al2O3.(H2O)3 + 2OH- &-> 2AlO2- + 4H2O}.
\end{align*}

Les autres composés restent insolubles, et sont éliminés par
filtration. En refroidissant le filtrat, de l'hydrate d'alumine est
récupéré par précipitation. Finalement, cet hydrate d'alumine est
calciné pour éliminer les molécules d'eau:
\begin{equation*}
  \cee{Al2O3.(H2O)3 -> Al2O3}.
\end{equation*}
La deuxième partie du processus consiste à inverser la réaction
d'oxydation pour finalement obtenir de l'aluminium sous forme
métallique.

Cette deuxième partie, qui porte les noms de ses deux inventeurs Paul
Héroult et Mark Hall, s'effectue industriellement dans des cuves
d'électrolyse, à haute température et sous l'influence de forts champs
d'inductions. Ce travail traite de la modélisation numérique de certains
aspects du procédé de Hall-Héroult, que nous décrivons plus en détail
dans la section qui suit.


\section{Procédé de Hall-Héroult}
\label{sec:introduction-hall-heroult}

% réaction chimique
Le procédé consiste à dissoudre l'oxyde d'aluminium dans un bain
composé de cryolite, puis à réduire les ions d'aluminium en effectuant
l'électrolyse de la solution:
\begin{align}
  \cee{Al2O3 &-> 2Al^{3+} + 2O^{2-}},\\
  \cee{Al^{3+} + 3e^{-} &-> Al}.
\end{align}

% contexte industriel
Dans un cadre industriel, la réaction a lieu dans une récipient dont
une coupe verticale schématique est représentée par la figure
\fig{pot}.

% structure de la cuve
Le bain électrolytique est maintenu à une température de
\num{950}\si{\celsius} environ dans un contenant formé par une
épaisseurs de briques réfractaires au niveau des parois verticales et
une cathode en carbone au fond. Le tout est soutenu par une structure
appelée cuve. Les briques réfractaires jouent le rôle d'isolant
thermique d'une part, et protègent le reste de la structure des
attaques corrosives de l'électrolyte d'autre part.

% métal liquide


% circuit électrique
L'électrolyte est alimenté en courant électrique à travers une série
d'anodes plongées dans le bain et maintenues dans la partie supérieure
de celui-ci par un pont qui sert à la fois de support mécanique et de
conducteur électrique. Le courant électrique traverse l'électrolyte,
le métal liquide et atteint la cathode en carbone placée au fond de la
cuve. Finalement, le courant est conduit en dehors de la cuve à
travers des conducteurs appelés {\em bus barres} placés sous la
cathode. L'intensité du courant électrique qui traverse une cuve
d'élecrolyse industrielle peut atteindre \num{500}\si{\kilo\ampere},
et jusqu'à \num{1}\si{\mega\ampere} dans le cuves les plus modernes.


% structure de la cuve
D'un point de vue industriel, la réaction a lieu dans une cuve, dont
la structure schématique est représentée par la figure \fig{pot}. Le
bain électrolytique, maintenu à l'état liquide à
\num{950}\si{\celsius} environ \needcite, est contenu dans une cuve,
dont les parois sont recouvertes d'une épaisseur de briques
réfractaires jouant le rôle d'isolant thermique, et d'une couche de
brique de carbone protégeant le contenant des attaques corrosives de
la cryolite. Au fond de la cuve sont disposées une série de barres
électriquement conductrices dont le but est de collecter le courant
électrique. Une cathode, typiquement en carbone fait le contact entre
les barres collectrices et une couche d'aluminium métallique
liquide. La cuve est maintenue dans une structure appelée caisson,
dont le rôles sont multiples. En plus d'offrir un support mécanique à
l'ensemble, le matériau du caisson est choisi de sorte à écranter
partiellement les champs d'induction électromagnétique intenses à
proximité. Ceci permet de diminuer l'intensité des forces de Lorentz
qui s'exercent sur le fluide, et tend à stabiliser l'écoulement des
fluides. Finalement, le caisson permet de contrôler plus finement les
pertes thermiques du système dans des zones clés.

Le bain électrolytique est constitué de cryolite (\ce{Na3.AlF6}),
solide à température ambiante, et d'autres sels en proportions
inférieures. Les proportions exactes qui font l'objet de secrets
industriels permettent de ramener le point de fusion de ce mélange à
environ \num{950}\si{\celsius}. Dans ces conditions, l'aluminium
métallique étant immiscible dans le bain, l'ensemble forme un système
bi-fluide séparé par une frontière libre. La légère différence de
densité entre le métal et le bain assure que, grâce à la gravité,
l'interface soit essentiellement horizontale.

Le bain électrolytique présente une densité volumique légèrement plus
faible d'environ \num{10}\% que celle de l'aluminium métallique, ce
qui lui permet de surnager sur celui-ci. Dans le bain sont maintenues
deux rangées d'anodes faites de carbone également. Le courant
électrique provient des anodes et traverse le bain pour atteindre
l'interface avec le métal liquide pour être finalement collecté par la
cathode. Cette épaisseur de bain dans lequel a lieu la réaction
d'électrolyse est appelée ACD\footnote{Acronyme anglais de \em{Anode
Cathode Distance}}. Dans une cuve industrielle moderne, l'ACD est de
l'ordre de \num{20}\si{\milli\meter} à \num{20}\si{\milli\meter}. Le
bain ayant une faible conductivité, on cherche à diminuer au maximum
cette distance, afin de minimiser les pertes énergétiques par effet
Joule. Le procédé d'électrolyse étant continu, il faut remplacer
régulièrement l'alumine dissoute consommée par l'électrolyse. Une
série d'injecteurs alimentent le bain en oxyde d'aluminium solide sous
forme de fine poudre. La surface du bain est protégée par une croûte
de bain solidifié et de poudre d'alumine. Pour cette raison, les
injecteurs doivent préalablement percer la croûte avant de pouvoir
déposer une dose de poudre d'alumine dans le bain. L'alimentation ne
peut se faire que de manière discontinue. L'ensemble de la cuve est
recouvert par un couvercle qui limite les dissipations thermiques et
permet la collecte des émanations gazeuses potentiellement toxiques
résultant de la réaction d'électrolyse, et protège les opérateurs
travaillant à proximité.

Par la réaction d'électrolyse, l'alumine dissoute est transformée en
aluminium métallique au niveau de l'interface avec le métal liquide,
qui joue le rôle de cathode. Du côté de l'anode, l'électrolyse produit
de l'oxygène, qui se recombine immédiatement avec le carbone de
l'anode. Le \ce{CO2} qui en résulte forme des bulles
millimétriques qui remontent à la surface pour s'échapper. C'est cette
production de \ce{CO2} qui provoque l'érosion des anodes.


L'ensemble des anodes est fixé sur un pont offrant un support
mécanique, mais également conducteur électrique. Chaque anode est
séparée de ses voisines par un espace de quelques centimètres appelé
canal. La position verticale de chaque anode peut être ajustée pour
compenser l'érosion de celle-ci par la réaction
d'électrolyse. Régulièrement, chaque anode arrivant en fin de vie doit
être remplacée par une anode neuve. Ceci intervient environ une fois
par jour et par cuve.

Comme indiqué ci-dessus, de l'alumine doit être ajoutée dans le bain
régulièrement afin de compenser celle qui a été consommée par la
réaction d'électrolyse. En raison de la disposition des anodes et de
la formation de croûte de bain gelée à la surface de celui-ci, la
poudre d'alumine est injectée par dose de \num{1}\si{\kilo\gram}
environ au niveau des canaux et à intervalle régulier.

L'alumine se trouve dans un état cristallin, à une température
d'environ \num{350}\si{\celsius}. Lorsque les particules d'alumine
entrent en contact avec le bain, une couche de bain solidifié se
forme immédiatement à la surface de celles-ci, et sont convectée par
l'écoulement dans le fluide et chute dans le bain sous l'action de
la gravité.

Une fois la couche de bain gelé qui enveloppe les particules
refondues, celles-ci se dissolvent peu à peu. Dans certaines
conditions, les particules s'agglomèrent et forment des agrégats
qui descendent rapidement dans le bain et atteignent l'interface. Si
la taille des agrégats est suffisante, il arrive qu'ils pénètrent
le métal et terminent leur course au fond de la cuve, risquant de
recouvrir peu à peu la cathode et de l'isoler électriquement.

La température du bain joue un rôle crucial au niveau du comportement
de l'alumine nouvellement injectée. Pour des raisons d'économie
d'énergie, le bain est maintenu aussi proche que possible de la
température de fusion de la cryolite, avec des surchauffes typiques de
l'ordre de \num{10}\si{\celsius} ou moins. Plus cette surchauffe est
faible, plus l'impact sur la dissolution de l'alumine est
important. Le temps nécessaire pour refondre la couche de bain gelé à
la surface des particules augmente lorsque la surchauffe diminue. Dans
la limite où la surchauffe est nulle, ce temps devient infini. De
plus, la réaction de dissolution de l'alumine cristalline dans le bain
est un processus endothermique. Il faut alors fournir un equantité
d'énergie thermique non négligeable pour dissoudre les
particules. Cette énergie est disponible uniquement si la surchauffe
du bain environnant est suffisante.




\section{But du travail}
\label{sec:introduction-aims}
% maîtrise de l'état de la cuve
Comme on a pu l'apprécier ci-dessus, l'opération d'une cuve
d'électrolyse d'aluminium industrielle est une tâche complexe qui
nécessite de maîtriser, entre autre, la production d'énergie thermique
par effet Joule et sa dissipation à l'extérieur de la cuve, les
écoulements dans les fluides et la répartition de la concentration
d'alumine dans le bain.

% l'optimisation n'est pas un nouveau problème
De fait, l'optimisation de l'exploitation d'une cuve d'électrolyse de
Hall-Héroult, tant en terme de consommation énergétique qu'en terme de
qualité de métal produit et d'impact sur l'environnement est une tâche
qui fait l'objet de recherche et développements s'étalant maintenant
sur plus de 100 ans.

% 1er sujet de cette thèse: dissolution (problematique)
Dans cette thèse nous nous intéresserons à la modélisation numérique
de certains phénomènes physiques qui interviennent dans le
fonctionnement d'une cuve d'électrolyse. La distribution de la
concentration d'alumine dissoute dans le bain est un aspect important
de l'opération d'une cuve d'électrolyse. Si la distribution de la
concentration d'alumine n'est pas suffisamment uniforme, certaines
zones seront suralimentées, tandis que d'autres seront
sous-alimentées. Dans un cas comme dans l'autre, la réaction
d'électrolyse ne peut pas avoir lieu correctement dans ces zones, ce
qui tend à déstabiliser l'ensemble du système, péjore le rendement du
système et engendre des émanations gazeuses toxiques. Nous aborderons
en particulier la modélisation de l'interaction entre la température
du bain électrolytique au cours du temps et la dissolution et
diffusion de la poudre d'alumine.

Nous proposerons une généralisation du modèle proposé par
T. Hofer \cite{Hofer2011} pour prendre en compte les nouveaux effets
évoqués, dans le but d'améliorer son pouvoir de prédiction.

% 2ème sujet de cette thèse: calculs rapides
L'état d'une cuve d'électrolyse n'est pas figé: il évolue sans cesse
au fil des changements d'anodes, des variations de cadence des
injecteurs, ou encore des perturbations des cuves voisines de la
série. Le modèle numérique existant, introduit par G. Steiner dans son
travail de thèse \cite{Steiner2009} et implémenté dans le logiciel
Alucell, permet d'obtenir une approximation numérique de l'état des
écoulements des fluides et de la concentration d'alumine dans une cuve
d'électrolyse avec un temps de calcul de l'ordre de \num{24} à
\num{48}\si{\hour}. Il est donc impossible d'utiliser ce modèle pour
obtenir des informations sur l'état d'une cuve à chaque instant de sa
vie, en fonction de paramètres extérieurs. Dans cette thèse nous
proposerons un modèle d'écoulement de fluide et de concentration
d'alumine dissoute stationnaire simplifié, qui permet d'obtenir des
informations sur l'état d'une cuve dans un laps de temps bien plus
court, de l'ordre de la minute.

Les opérateurs des cuves industrielles devrait alors pouvoir tirer
parti des prédictions de ce modèle pour contrôler l'état des cuves
et agir plus rapidement et de manière mieux informée dans les
conditions d'exploitation exceptionnelles.

%%%%
%%
%%% introduction du travail de Hofer
%%Dans son travail, T. Hofer \cite{Hofer2011} propose une première
%%approche de modélisation numérique des phénomènes de transport
%%et dissolution d'alumine à l'état solide sous forme de poudre, et
%%de transport et consommation de l'alumine dissoute à l'échelle
%%d'une cuve d'électrolyse industrielle.
%%
%%
%%Le présent travail se base sur celui de T. Hofer \cite{Hofer2011}, et
%%se décline en deux parties. D'une part nous investiguerons des
%%extensions au modèle existant, en proposant des modèles mathématiques
%%motivés et justifiés par la physique des phénomènes en présence. En
%%particulier, nous nous intéresserons aux interactions entre la
%%température du bain électrolytique et la distribution de la
%%concentration d'alumine dissoute dans le bain. Nous nous pencherons
%%également sur le phénomène de la chute des particules dans le
%%bain. Nous en dériverons des modèles numériques, qui seront
%%implémentés et validés dans le logiciel Alucell. Ces modèles seront
%%finalement appliqués à des situations concrètes de cuves d'électrolyse
%%industrielles, afin d'en évaluer les bénéfices.
%%
%%
%%Dans un deuxième temps, nous proposerons une approche alternative pour
%%obtenir une approximation de la concentration d'alumine à un coût bien
%%moindre, au prix d'un certain nombre d'approximations. Le modèle
%%numérique de transport et dissolution d'alumine introduit en première
%%partie est particulièrement coûteux en temps de calcul. Selon le
%%maillage et le choix de divers paramètres, il faut environ
%%\num{30}\si{\hour} de calcul pour obtenir d'une part un écoulement
%%dans le bain électrolytique en résolvant un système d'équations
%%stationnaires, et d'autre part la distribution stationnaire de
%%concentration d'alumine en résolvant un problème transitoire de
%%convection et diffusion.
%%
%%
%%Comme déjà évoqué, une cuve d'électrolyse industrielle ne se trouve
%%jamais dans un état stationnaire particulier, mais passe continûment à
%%travers une série d'états au fil du cycle des changements d'anodes,
%%des variations des conditions d'opérations et éventuellement de l'état
%%des cuves voisines. Sous cet angle, le modèle de T. Hofer
%%\cite{Hofer2011} et le nôtre ne donne d'informations qu'à propos d'un
%%état moyen d'une cuve. Or, on constate typiquement, lors des
%%changements d'anodes,que l'écoulement peut s'écarter de façon
%%significative d'un écoulement moyen, et ceci pendant des intervalles
%%de temps importants. La conséquence directe est que la distribution de
%%concentration d'alumine s'écarte de l'état supposé optimal.
%%
%%La conductivité du bain électrolytique dépend fortement de la
%%concentration d'alumine dissoute. Une concentration d'alumine optimale
%%pour une cuve en opération minimise la conductivité du bain. Si la
%%conductivité s'écarte de la concentration optimale, la conductivité
%%augmente rapidement. Si le bain est sousalimenté, la concentration
%%d'alumine dissoute est trop faible pour que la réaction
%%d'électrolyse puisse avoir lieu, et c'est l'électrolyte qui réagit en
%%produisant typiquement des gaz fluorés. Si le bain est suralimenté, le
%%rendement de Faraday de l'électrolyse diminue rapidement. De plus,
%%dans un cas comme dans l'autre, les anodes subissent une usure
%%anormale et développent des défauts caractéristiques qui nécessitent
%%un remplacement prématuré. Il est clair que maintenir une
%%concentration uniforme d'alumine dissoute dans le bain électrolytique
%%est essentiel au bon fonctionnement d'une cuve.



\section{Organisation du document}
\label{sec:introduction-organisation}
Ce travail se divise en deux parties. Dans la partie
\ref{part:alumina}, on considère plusieurs effets physiques liés à la
gravité et aux interactions thermiques entre les particules d'alumine
et le bain environnant, que l'on exprime sous forme de modèles
mathématiques. On propose ensuite une extension du modèle de transport
et dissolution d'alumine, introduit dans le travail de thèse de
T. Hofer \cite{Hofer2011}, pour tenir compte de l'effet de la gravité
sur la trajectoire des particules d'alumine et de l'effet de la
température du bain sur leur dissolution.  Dans la partie
\ref{part:fluid} on s'intéresse à diminuer le coût de calcul de
l'écoulement dans le bain électrolytique et le métal liquide. On
propose une méthode numérique basée sur une décomposition en
harmoniques de Fourier du problème de Stokes dans la dimension
verticale. On propose de plus une méthode pour approximer la
distribution de concentration d'alumine dans le bain électrolytique,
sans recourir au calcul d'une solution transitoire.

