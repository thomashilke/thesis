L'aluminium est un métal léger qui figure parmi les éléments
métalliques les plus abondant sur Terre, avec une proportion d'environ
\num{8}\% de la masse totale de la croûte terrestre. Malgré tout il
n'apparaît jamais sous forme métallique en raison de sa forte affinité
avec l'oxygène, et se trouve sous forme d'oxyde d'aluminium \ce{Al2O3}
(appelé également alumine). L'alumine est un constituant de différents
minerais, mais seul la bauxite est utilisée pour la production de
l'aluminium.

La bauxite est constituée principalement par de l'oxyde d'aluminium
hydraté, de silicate, d'oxyde de fer et d'impuretés en plus faibles
proportions. Elle provient essentiellement d'Australie, de Chine et du
Brésil. Par exemple, ces trois pays combinés ont extrait plus de
\num{179} millions de tonnes de bauxite au cours de l'année
\num{2014}. Le procédé de Bayer permet ensuite d'isoler et purifier
l'alumine de la bauxite. Après concassage et broyage du minerais, la
soude caustique chaude dissout l'alumine:
\begin{align*}
  \cee{NaOH &-> Na+ + OH-},\\
  \cee{Al2O3.(H2O) + 2OH- &-> 2AlO2- + 2H2O},\\
  \cee{Al2O3.(H2O)3 + 2OH- &-> 2AlO2- + 4H2O}.
\end{align*}

Les autres composés restent insolubles, et sont éliminés par
filtration. En refroidissant le filtrat, de l'hydrate d'alumine est
récupéré par précipitation. Finalement, cet hydrate d'alumine est
calciné pour éliminer les molécules d'eau:
\begin{equation*}
  \cee{Al2O3.(H2O)3 -> Al2O3}.
\end{equation*}

La deuxième partie du processus consiste à inverser la réaction
d'oxydation pour finalement obtenir de l'aluminium sous forme
métallique. C'est cet aspect du procédé, qui porte les noms de ses
deux inventeurs Paul Héroult et Mark Hall, qui nous intéresse dans
ce travail et que nous décrivons plus en détail dans le paragraphe
qui suit.

\section{Procédé de Hall-Héroult}
\label{sec:introduction-hall-heroult}
On donne dans cette section un bref aperçu du procédé de
Hall-Héroult. Le lecteur intéressé peut trouver de plus amples
informations dans le travail de \cite{grjotheim1977aluminium}.

\begin{figure}[t]
\begin{center}
  \input{../media/introduction/electrolysis-pot/electrolysis-pot.pdf_tex}
  \begin{multicols}{2}
    \small
    \begin{enumerate}[label=(\alph*)]
    \item Injecteur de dose d'alumine
    \item Broches de connections des anodes
    \item Bain électrolytique
    \item Aluminium métallique liquide
    \item Pâte de brasquage
    \item Isolation réfractaire
    \item Caisson
    \item Couvercle
    \item Anodes
    \item Talus, croûte de bain solidifiée
    \item Interface bain-métal
    \item Isolation réfractaire
    \item Cathode
    \item Barres bus collectrices
    \end{enumerate}
  \end{multicols}
  \caption{Représentation de la structure des éléments internes
    d'une cuve d'électrolyse d'aluminium. Vue de coupe.}
  \label{fig:electrolysis-pot}
\end{center}
\end{figure}

% réaction chimique
Le procédé consiste à dissoudre l'oxyde d'aluminium dans un bain
composé de cryolite, puis à réduire les ions d'aluminium en effectuant
l'électrolyse de la solution:
\begin{align}
  \cee{Al2O3 &-> 2Al^{3+} + 2O^{2-}},\label{reac:dissociation}\\
  \cee{Al^{3+} + 3e^{-} &-> Al}.\label{reac:electrolysis}
\end{align}

% contexte industriel
Dans un cadre industriel, la réaction a lieu dans un bain
électrolytique contenu dans un récipient dont une coupe verticale
schématique est représentée par la figure \fig{electrolysis-pot}. Le
bain électrolytique, dont le rôle est de permettre la dissociation des
ions \ce{Al^{3+}} et \ce{O^{2-}} nécessaire à la réaction
d'électrolyse, est essentiellement constitué par de la cryolite
(\ce{Na3.AlF6}). Un nombre d'additifs sont ajoutés dans le bain afin
de ramener la température de fusion de celui-ci à environ
\num{950}\si\celsius.

% structure de la cuve
Le bain électrolytique est maintenu sous forme liquide un contenant
constitué par une épaisseurs de briques réfractaires au niveau des
parois verticales et une cathode en carbone au fond. Le tout est
soutenu par une structure appelée cuve. Les briques réfractaires
jouent le rôle d'isolant thermique d'une part, et protègent le reste
de la structure des attaques corrosives de l'électrolyte d'autre part.

% métal liquide
La réaction d'électrolyse (\ref{reac:electrolysis}) produit de
l'aluminium métallique au niveau de la cathode. La température de
fusion de l'aluminium étant de \num{660.3}\si{\celsius}, le métal
produit est maintenu sous forme liquide. L'aluminium métallique est
immiscible dans l'électrolyte. En raison de sa densité supérieure
d'environ \num{10}\% à celle de l'électrolyte, le métal forme une
nappe au fond de la cuve et recouvre la cathode. C'est le métal
liquide qui joue effectivement le rôle de cathode. Le bain
électrolytique et le métal forment un système à deux phases séparées
par une interface libre.

% plan anodique
Dans une cuve industrielle, l'anode est subdivisée en une série de
blocs anodiques d'environ \num{1}\si{\cubic\meter} chacun. Les blocs
anodiques sont typiquement arrangés selon deux rangées
parallèles. L'espace qui sépare les deux rangées est appelé canal
central. Cette subdivisions en blocs anodiques permet d'adapter
individuellement la position verticale de chaque bloc, ainsi que de
les remplacer individuellement sans perturber l'opération de la cuve.

% circuit électrique
L'électrolyte est alimenté en courant électrique à travers les blocs
anodiques plongés dans le bain et maintenus dans la partie supérieure
de celui-ci par un pont qui sert à la fois de support mécanique et de
conducteur électrique. Le courant électrique traverse l'électrolyte,
le métal liquide et atteint la cathode en carbone placée au fond de la
cuve. Finalement, le courant est conduit en dehors de la cuve à
travers des conducteurs horizontaux appelés {\em barres bus} placés
sous la cathode. L'intensité du courant électrique qui traverse une
cuve d'électrolyse industrielle peut atteindre
\num{500}\si{\kilo\ampere}, et jusqu'à \num{1}\si{\mega\ampere} dans
le cuves les plus modernes.

% champs magnétiques et forces de lorentz
Les courants électriques importants qui traversent la cuve
d'électrolyse ainsi que les nombreuses cuves voisines dans une usine
de production génèrent des champs d'induction magnétique
importants. Le passage du courant électrique dans l'électrolyte et le
métal en présence du champ d'induction magnétique engendre des forces
de Lorentz qui mettent les fluides en mouvement.

% La réaction d'électrolyse, erosion des anodes
Par la réaction d'électrolyse, l'alumine dissoute est transformée en
aluminium métallique au niveau de l'interface avec le métal liquide,
qui joue le rôle de cathode. Du côté de l'anode, l'électrolyse produit
de l'oxygène, qui se recombine immédiatement avec le carbone de
l'anode. Le \ce{CO2} qui en résulte forme des bulles millimétriques
qui remontent à la surface pour s'échapper. Cette production de
\ce{CO2} liée à l'oxydation du carbone provoque l'érosion des anodes
au cours du temps. Régulièrement, la position verticale de chaque
anode est ajustée pour compenser sont érosion. Finalement, chaque
anode arrivant en fin de vie doit être remplacée par une anode
neuve. Ceci intervient environ une fois par jour et par cuve.

% écoulement dans les fluides, mouvements de l'interface
Le mouvement des fluides dans les cuves d'électrolyse est un aspect
important pour plusieurs raisons, et qui a déjà fait l'objet de
nombreuses études. On peut consulter par exemple les travaux de
S. Pain \cite{Pain2006}, Y. Safa \cite{Safa2009}, G. Steiner
\cite{Steiner2009} et S. Flotron \cite{Flotron2013}. Le bain
électrolytique est l'élément du circuit électrique qui est
responsable, en raison de sa faible conductivité électrique, de
l'essentiel de la production d'énergie thermiques par effet
Joule. Pour minimiser ces pertes d'énergie thermiques, on cherche à
minimiser la distance parcourue par le courant électrique entre les
anodes et l'interface entre le bain et le métal. Cependant, les
mouvements dans les fluides perturbe la forme de l'interface. En
particulier, on cherche à tout prix à éviter que le métal entre en
contact avec les anodes, et crée un court circuit. Une telle
situation, qui risque de déstabiliser le système et perturber les
cuves voisines, est à éviter à tout prix. Dans une cuve industrielle
moderne, l'écart entre le plan anodique et l'interface, abrégée
ACD\footnote{ACD est l'acronyme anglais de {\em Anode Cathode
    Distance}}, est de l'ordre de \num{20}\si{\milli\meter} à
\num{40}\si{\milli\meter}.

% écoulement dans les fluides, transport des particules et de la
% concentration
A intervalles réguliers, de l'oxyde d'aluminium sous forme de poudre
solide est injecté dans le bain afin de compenser la concentration
consommée par la réaction d'électrolyse (\ref{reac:electrolysis}).  La
surface du bain est protégée par une croûte de bain solidifié et de
poudre d'alumine. Pour pouvoir injecter de l'alumine dans le bain, un
piqueur perce cette croûte mécaniquement. Pour cette raison,
l'injection d'alumine ne peut avoir lieu que de façon discontinue.  La
poudre est ensuite libérée par l'injecteur par doses d'environ
\num{1}\si{\kilo\gram} au niveau de la surface du bain électrolytique,
dans le canal central entre les deux rangées d'anodes. Le transport de
cette poudre dans l'ensemble du bain et sa dissolution dépendent
fortement du mouvement du fluide. On cherche un équilibre fin entre
des écoulements qui déforment peu l'interface bain-métal, mais
suffisamment fort pour assurer le transport et la dissolution de la
poudre d'alumine dans toutes les parties du bain électrolytique.

% maîtrise de la température
Le bain électrolytique, maintenu au-dessus de sa température de
fusion, est un milieu extrêmement corrosif, y compris pour les
structures qui servent à le contenir. Pour allonger la durée de vie
des cuves, on cherche à minimiser les surfaces de contact avec le bain
liquide. En particulier, on essaye de maintenir une couche de bain
solide le long des parois, appelée talus, pour les protéger des
attaques corrosives du bain liquide. On y arrive en cherchant un
équilibre thermique qui dépend du flux d'énergie au niveau des parois
et de la convection de la chaleur produite dans le bain par les
mouvements des fluides. Pour ces différentes raisons, un bonne
compréhension et maîtrise des écoulements dans une cuve d'électrolyse
est essentielle à la bonne opération de celle-ci.

% Formation de bain gelé autour des particules froides
Au moment de l'injection, l'alumine se trouve dans un état cristallin,
à une température d'environ \num{350}\si{\celsius}. Lorsque les
particules d'alumine entrent en contact avec le bain, une couche de
bain solidifié se forme immédiatement à la surface de celles-ci,
sont convectée par l'écoulement dans le fluide et chute dans le bain
sous l'action de la gravité.

% dissolution, agrégats
Une fois la couche de bain gelé qui enveloppe les particules
refondues, celles-ci se dissolvent peu à peu. Dans certaines
conditions, les particules s'agglomèrent et forment des agrégats
qui descendent rapidement dans le bain et atteignent l'interface. Si
la taille des agrégats est suffisante, il arrive qu'ils pénètrent
le métal et terminent leur course au fond de la cuve, risquant de
recouvrir peu à peu la cathode et de l'isoler électriquement \cite{Geay2016}.

% rôle de la température du bain
La température du bain joue un rôle crucial au niveau du comportement
de l'alumine nouvellement injectée. Pour des raisons d'économie
d'énergie, le bain est maintenu aussi proche que possible de la
température de fusion de la cryolite, avec des surchauffes typiques de
l'ordre de \num{10}\si{\celsius} ou moins. Plus cette surchauffe est
faible, plus l'impact sur la dissolution de l'alumine est
important. Le temps nécessaire pour refondre la couche de bain gelé à
la surface des particules augmente lorsque la surchauffe diminue. Dans
la limite où la surchauffe est nulle, ce temps devient infini. De
plus, la réaction de dissolution de l'alumine cristalline dans le bain
est un processus endothermique. Il faut alors fournir une quantité
d'énergie thermique non négligeable pour dissoudre les
particules. Cette énergie est disponible uniquement si la surchauffe
du bain environnant est suffisante.

% caisson ferromagnetique
L'ensemble de la cuve est placée dans un caisson, dont le matériau est
choisi pour ses propriétés ferromagnétiques. La présence du caisson
permet d'écranter une partie du champ d'induction magnétique et
réduire l'intensité de celui-ci dans les fluides. Les forces de
Lorentz peuvent ainsi être contrôlées, et diminuer l'intensité des
vitesses dans les fluides de la cuve \cite{Descloux1},
\cite{Descloux2}.  Le caisson permet également de contrôler plus
finement la dissipation thermique sur les bords de la cuve et la
formation des talus.

% couvercle
L'ensemble de la partie supérieure de la cuve est recouverte par un
couvercle qui limite les dissipations thermiques, permet la collecte
des émanations gazeuses potentiellement toxiques résultant de la
réaction d'électrolyse, et protège les opérateurs travaillant à
proximité.


