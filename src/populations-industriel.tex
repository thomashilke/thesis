Dans cette section nous appliquons le modèle de transport et
dissolution d'alumine en fonction de la température proposé dans la
section \ref{sec:populations-model} dans le cadre d'une cuve
d'électrolyse industrielle pour déterminer la répartition de l'alumine
dissoute dans le bain de celle-ci. Nous utilisons la cuve AP32, qui
exploite la technologie de cuve d'électrolyse AP
Technology\texttrademark\ développée par RioTinto. Les premières cuves
basées sur la technologie AP ont été mises en production au début des
années 1990, et plus de \num{4000} d'entre elles fonctionnent encore
actuellement en production \cite{RiotintoAP30}. Nous commençons par
présenter le design et le mode d'opération de la cuve AP32. Nous
détaillerons ensuite le choix des différentes données qui
interviennent dans modèle numérique proposé dans la section
\ref{sec:populations-discretisation} et dans le cadre de la cuve
AP32. Finalement, nous présenterons une sélection de résultats
numériques obtenus.


% detail des aspects importants de l'operation d'une cuve
% d'electrolyse:
%  - geometrie de la cuve (taille, nombre d'anodes, epaisseur de bain
%    et de metal,
%  - alimentation electrique
%  - etat stationnaire de l'ecoulement, interface. remarque sur les
%    talus.
%  - control de la concentration d'alumine
%  - schema d'injection periodique, position des injecteurs
%  - calcul de la conductivite sigma du bain
%  - conditions initiales
%  - informations sur le post-processing et la visualisation.

\paragraph{Géométrie de la cuve AP32} La structure de la cuve AP32
occupe au sol une longueur d'environ \num{17} \si{\meter} et une
largeur d'environ \num{7} \si{\meter}. L'ensemble de la structure
s'élève sur environ \num{5} \si{\meter}. La figure
\ref{fig:ap32-geometry} montre la disposition des différents éléments
à l'intérieur de la cuve. Les fluides s'étendent horizontalement sur
environ \num{14} \si{\meter} par \num{3.5} \si{\meter}. L'épaisseur de
la couche d'aluminium liquide (en jaune sur la figure
\ref{fig:ap32-geometry-elements}) en contact avec la cathode est
d'environ \num{17} \si{\centi\meter}, tandis que l'épaisseur maximale
du bain électrolytique, au niveau des canaux entre les blocs
anodiques, est d'environ \num{20} \si{\centi\meter}. La figure
\ref{fig:ap32-geometry-electrolyte} illustre le volume occupé par le
bain dans lequel on s'intéresse à la concentration d'alumine, en
orange. Les indentation rectangulaire à la surface de celui-ci
correspondent au volume occupé par les anodes partiellement
immergées. L'ACD est typiquement de l'ordre de \num{3}
\si{\centi\meter}. Ces différentes épaisseurs varient d'un point à
l'autre de la cuve à cause des écoulements dans les fluides, de la
déformation de l'interface bain-métal et des irrégularités à la
surface des anodes. De plus, le volume de métal liquide varie
constamment, d'une part à cause du produit de la réaction
d'électrolyse, et d'autre part à cause des opérations de siphonnage du
métal, qui interviennent environ une fois par jour.

\begin{figure}[t]
  \begin{center}
    \begin{subfigure}[b]{0.49\textwidth}
      \includegraphics[width=\textwidth]{../media/populations/ap32-mesh-components/print/metal-bath-anodes-cathode-bus-bars.png}
      \caption{Éléments à proximité des fluides}
      \label{fig:ap32-geometry-elements}
    \end{subfigure}
%
    \begin{subfigure}[b]{0.49\textwidth}
      \includegraphics[width=\textwidth]{../media/populations/ap32-mesh-components/print/bath.png}
      \caption{Bain électrolytique}
      \label{fig:ap32-geometry-electrolyte}
    \end{subfigure}
%
    \caption{Géométrie des éléments importants à proximité
      du bain électrolytique dans une cuve AP32
      (fig. \ref{fig:ap32-geometry-elements}), et détail du volume
      occupé par le bain électrolytique dans cette même cuve
      (fig. \ref{fig:ap32-geometry-electrolyte}). On distingue les
      les anodes en haut et la cathode
      en bas (\textbf{noir}), le bain électrolytique
      (\textbf{orange}), le métal liquide (\textbf{jaune}) et les
      bus bar (\textbf{gris clair}).}
    \label{fig:ap32-geometry}
  \end{center}
\end{figure}

Le plan anodique est composée de deux rangées de 10 anodes chacune,
représentées en noir sur la figure \ref{fig:ap32-geometry-elements}. La
surface du plan anodique est d'environ \num{40.3} \si{\square\meter}
et seulement \num{25}\% de la surface du bain est libre, le reste
étant recouvert par les anodes. La cuve est conçu pour que
l'électrolyte soit traversé par un courant électrique total $I = $
\num{320000} \si{\ampere}, ce qui correspond à une densité de courant
d'environ \num{0.8} \si{\ampere\per\square\centi\meter} à la surface
des anodes. En supposant un rendement de réaction de \num{100}\%, ce
courant électrique permet de réduire par électrolyse \num{29.8}
\si{\gram\per\second} ou \num{2577.1} \si{\kilo\gram} par jour
d'aluminium métallique, \ie, un peu plus qu'\num{1} \si{\cubic\meter}
de métal par jour. Cet accroissement de volume de métal correspond à
une variation de l'épaisseur du métal liquide d'environ \num{2}
\si{\centi\meter}.

Du coté des anodes, la réaction d'électrolyse produit environ
\num{0.8} \si{\mol\per\second} d'oxygène \ce{O2}. Cet oxygène réagit
immédiatement avec le carbone de l'anode pour former du \ce{CO2}. Dans
l'ensemble de la cuve, l'électrolyse produit au total environ \num{80}
\si{\liter} par seconde de gaz, qui remonte vers la surface du bain
par le canal central et les canaux latéraux. La réaction de l'oxygène
avec le carbone des anodes provoque l'érosion de celles-ci à une
vitesse d'environ \num{1100} \si{\kilo\gram} par jour. Étant donné le
nombre total d'anodes et leur taille respectives, chaque anode d'une
cuve AP32 a une durée de vie d'environ 30 jours, après quoi elle doit
être remplacée par une anode neuve.

Pour compenser l'alumine dissoute qui est consommée par la réaction
d'électrolyse, il faut injecter en moyenne au cours du temps $56.3$
\si{\gram\per\second} de poudre d'alumine. Comme déjà mentionné
dans la section \ref{sec:introduction-hall-heroult}, la poudre
d'alumine est déposée à la surface du bain dans le canal central par
une série d'injecteurs dont la position est fixe. Un piqueur vient
percer mécaniquement un trou dans la croûte et créer un accès à la
surface libre du bain avant chaque injection. Ce trou se rebouche
rapidement, et pour cette raison l'injection d'alumine ne peut pas
avoir lieu continûment.

Pour maintenir un rendement énergétique maximum, éviter l'émission de
gaz fluorés et éviter l'occurence des effets d'anodes, il est crucial
que la concentration d'oxyde d'aluminium dissout dans le bain soit
maintenu dans un intervalle très précis. Malheureusement, pour de
nombreuses raisons il est impossible de maintenir un bilan précis de
la quantité d'alumine dans le bain en fonction de ce qui est injecté
et de ce qui est consommé. En effet, l'environnement rend difficile la
pesée précise des quantités déposées, une partie des particules
volatiles ne parviennent jamais dans le bain, des agrégats se forme,
dont une partie s'accumule au fond de la cuve sur la cathode, et des
réactions chimiques parasites viennent, entre autres, grever ce bilan.

Pour contourner cette difficulté, les opérateurs exploitent le fait
que la resistivité du bain électrolytique dépend de la concentration
d'alumine dissoute, et atteint un minimum à la concentration optimale
$\concentration \approx$ \num{3}\% masse. En mesurant la chute de
potentiel électrique à travers le bain électrolytique, on maintient la
concentration d'alumine dissoute au voisinage de la concentration
optimale en alternant une phase de sur-alimentation en alumine et une
phase de sous-alimentation. Durant la phase de sur-alimentation, la
concentration d'alumine va passer au-delà de la concentration optimale
par conséquent accroître la resistivité du bain. Passé un certain
seuil, on débute une phase de sous-alimentation, durant laquelle la
résistivité commence par chuter, puis croît à nouveau. Passé un
certain seuil, on amorce une phase de sur-alimentation, et ainsi de
suite.

Dans chacune des phase de sur-alimentation ou sous-alimentation, les
injecteurs déposent les doses d'alumine selon une cadence préétablie
et périodique. La période et une taille des doses peut être spécifiée
indépendemment pour chaque injecteur.

\begin{figure}[t]
  \begin{center}
    \begin{tikzpicture}
      \begin{axis}[
          hide axis,
          colorbar,
          scale only axis,
          height=0.41\rasterimagewidth,,
          width=\rasterimagewidth,
          colorbar horizontal,
          point meta min=0.00,
          point meta max=0.05,
          colorbar style={
            title=Vitesse $u$ [\si{\meter\per\second}],
            width=7.4cm,
            height=0.3cm,
            xtick={0.00, 0.01, 0.02, 0.03, 0.04, 0.05},
            xticklabel style={
              /pgf/number format/fixed,
              /pgf/number format/fixed zerofill,
              /pgf/number format/precision=2
            },
            scaled x ticks = false,
            at={(0.5\rasterimagewidth,0.4cm)},
            anchor=north
          }
        ]
        \addplot [] coordinates {(0,0)};
        \node (myfirstpic) at (0,0) {\includegraphics[width=\rasterimagewidth]{{../media/populations/ap32-fluid-flow/print/acd-all-anodes-velocity-0.00-0.05}.png}};
      \end{axis}
    \end{tikzpicture}
    \caption{Champ de vitesse $u$ dans le bain électrolytique d'une
      cuve AP32 restreint sur une surface placée à mi-hauteur de
      l'ACD, vue depuis dessus. Cette situation correspond à un état
      d'opération standard.}
    \label{fig:ap32-flow-acd}
  \end{center}
\end{figure}

\begin{figure}[t]
  \begin{center}
    \includegraphics[width=\rasterimagewidth]{../media/populations/ap32-fluid-flow/print/chanel-velocity-streamlines.png}
    \caption{Lignes de courant correspondant au champ de vitesse
      représenté sur la figure \ref{fig:ap32-flow-acd}. Les lignes de
      courant prennent leur origine le long du canal central.}
    \label{fig:ap32-flow-streamlines}
  \end{center}
\end{figure}

\paragraph{Calcul de l'écoulement dans le bain} Une approximation de vitesse
d'écoulement du bain $u$ et de la densité de courant $j$ dans la cuve
AP32 est obtenue par l'intermédiaire du modèle multiphysique
stationnaire proposé par S. Steiner \cite{Steiner2009}, J. Rochat
\cite{Rochat2016} déjà introduit dans la section
\ref{sec:populations-introduction}. La figure \ref{fig:ap32-flow-acd}
représente la vitesse d'écoulement ainsi calculée par le logiciel
Alucell dans le bain électrolytique de la cuve AP32, dans
l'ACD. Lorsque la densité de courant électrique est répartie
uniformément sur toutes les anodes, l'écoulement dans les fluides
forme deux tourbillons principaux qui tournent en sens opposés. Deux
petits tourbillons se forment dans les coins avals. Les vitesse
maximales de l'écoulement (\num{5} \si{\centi\meter\per\second}
environ) sont atteinte dans le canal central au niveau des extrémités
de la cuve, ainsi que le long de la paroi amont, de part et d'autre de
la cuve. Dans le reste du bain et en particulier sous les anodes la
vitesse d'écoulement dépasse rarement \num{2}
\si{\centi\meter\per\second}. La figure
\ref{fig:ap32-flow-streamlines} illustre les lignes de courant de
l'écoulement dans le bain. On remarque les lignes de courant
s'engagent volontiers dans les canaux latéraux et dans le bain en
pourtour des rangées d'anodes.

\paragraph{Conditions sur l'injection et l'effet Joule}
Le schéma numérique proposé dans la section
\ref{sec:populations-discretisation} est conçu de manière à conserver
exactement d'une part la masse d'alumine dans les populations de
particules $\population$ et concentration d'alumine dissoute
$\concentration$, et d'autre part la quantité d'énergie thermique liée
à la température du bain $\temperature$. Pour des raisons déjà
évoquées, ces bilans ne sont pas exactement respectés dans une cuve
industrielle réelle, et il faut en général injecter un peu plus d'alumine
que ce qui est consommé par la réaction d'électrolyse. Quand à
l'énergie thermique, nous avons supposé que le bain est isolé
thermiquement, alors que dans une cuve réelle une quantité non
négligeable d'énergie s'échappe par le métal, les parois latérales de
la cuve, les anodes, la surface du bain et par le \ce{CO2} qui
s'échappe dans l'atmosphère.

Pour éviter que la masse totale d'alumine dans la cuve croisse sans
limite au cours du temps, il faut s'assurer que dans un état pseudo
stationnaire, la masse d'alumine reste proche de la masse d'alumine
initialement présente dans le bain. En d'autres termes, si $M_n$ est
la masse totale d'alumine dans le bain à l'instant $t_n$ telle que
définie dans la section (\ref{sec:populations-discretisation}), alors
on demande à ce que
\begin{equation*}
\lim_{n\to\infty} \frac{M_{n+1} - M_{0}}{t_n} = 0,
\end{equation*}
c'est-à-dire que
\begin{equation}\label{eq:injection-mass-condition}
  \frac{I[\cee{Al2O3}]}{6F}
  =\lim_{n\to \infty}\frac{1}{t_{n+1}}\ \sum_{\mathclap{\substack{k\\ p^k + 1\leq n +
        1}}}\ \int_\Omega\int_0^\infty\aluminadensity S^k \frac{4}{3}\pi r^3\intd{r}\intd{x}.
\end{equation}
Il faut donc choisir la masse des doses d'alumine injectées $S^k$, $k
= 1, 2,\dots$ et les temps d'injection $\tau^k$, $k = 1,2,\dots$ de
sorte à ce que le débit de masse de poudre d'alumine moyen au cours du
temps soit égal à $\frac{I[\cee{Al2O3}]}{6F}$, c'est-à-dire
\num{56.382e-3}\si{\kilo\gram\per\second}.

\begin{figure}
  \begin{center}
    \input{../media/populations/anode-configuration/anode-configuration.pdf_tex}
    \caption{Vue schématique de la partie supérieure du bain
      électrolytique. Les blocs rectangulaires représentent
      l'emplacement des anodes, tandis que les cercles marquent
      l'emplacement des injecteurs disposés le long du canal central.}
    \label{fig:anode-configuration}
  \end{center}
\end{figure}

La cuve AP32 possède 4 injecteurs placés au-dessus du canal central,
numérotés de 1 à 4, dans le sens de la coordonnée $x$ croissante (voir
la figure \ref{fig:anode-configuration}). Les paramètres qui
définissent chaque injecteurs sont regroupé dans la table
\ref{tab:injectors}.

\begin{table}
  \begin{center}
    \caption{Paramètres caractérisant les 4 injecteurs de la cuve AP32.}
    \label{tab:injectors}
    \begin{tabularx}{\textwidth}{@{}rrrrZ@{}}
      \toprule
      Injecteur & Position & Première injection & Intervalle d'injection & Masse de dose\\
      \midrule
      \#1         & \num{-4.4}\si\meter & \num{16}\si\second & \num{16}\si\second  & \num{0.225}\si{\kilo\gram} \\
      \#2         & \num{-1.6}\si\meter & \num{32}\si\second & \num{32}\si\second  & \num{0.451}\si{\kilo\gram} \\
      \#3         & \num{ 1.6}\si\meter & \num{48}\si\second & \num{48}\si\second  & \num{0.676}\si{\kilo\gram} \\
      \#4         & \num{ 4.4}\si\meter & \num{64}\si\second & \num{24}\si\second  & \num{0.338}\si{\kilo\gram} \\
      \bottomrule
    \end{tabularx}
  \end{center}
\end{table}
La masse des doses dans le tableau \ref{tab:injectors} est choisie de
telle sorte à ce que l'ensemble des 4 injecteurs injecte 25\% de la
masse d'alumine en moyenne. La forme des densité de particules
initiales $S^k$ est décrite dans \cite{Hofer2011}. Plus précisément,
la forme spatiale de chaque $S^k$ est une fonction lisse à support
compact centrée autour du point d'injection de l'injecteur
correspondant. La distribution en rayon initiale est approximée par
par une loi log-normale basée sur des mesures expérimentales. Puisque
le rapport des périodes d'injection des 4 injecteurs sont des nombres
rationnels, on peut définir une périodes globale liée à l'ensemble des
4 injecteurs. Étant données les périodes d'injections reportées dans le tableau
\ref{tab:injectors}, le cycle d'injection global est périodique après une
transition initiale de \num{64}\si\second, avec une période
de \num{192}\si\second. La figure \ref{fig:injections} représente les
injections qui ont lieu durant les \num{192} premières secondes de la
simulation.

\begin{figure}
  \begin{center}
    \input{../media/populations/cadence/cadence.tex}
    \caption{Temps d'injections des différents injecteurs de la cuve
      AP32. Chaque cercle représente une injection. Chaque ligne
      correspond à l'un des 4 injecteurs.}
    \label{fig:injections}
  \end{center}
\end{figure}

On détermine maintenant une condition sur la conductivité
électrique $\conductivity$, issue d'un argument similaire sur
l'énergie thermique du bain. Pour que l'énergie thermique du bain
reste proche de l'énergie thermique initiale, on demande à ce que
\begin{equation}
  \lim_{n\to\infty}\frac{\electrolytedensity\electrolytehc\displaystyle\int_\Omega\parent{\temperature_{n+1}
    - \temperature_0}\intd{x}}{t_{n+1}} = 0,
\end{equation}
ce qui correspond à la condition
\begin{eqnarray}
  \conductivity \int_\Omega j\cdot j\intd{x} &=&
  \aluminahc\parent{\tinit - \tinj}\lim_{n\to\infty}\frac{1}{t_{n+1}}\sum_{\mathclap{\substack{k\\ p^k +
  1 \leq n + 1}}}\int_\Omega\int_0^\infty\aluminadensity
  \frac{4}{3}\pi r^3 S^k\intd{r}\intd{x}\nonumber \\
  &&+ {\aluminadissolutionenthalpy}\lim_{n\to\infty}\frac{1}{t_{n+1}}\sum_{m = 0}^{n}\sum_{\mathclap{\substack{k\\ q^k < m + 1}}} \int_\Omega\int_0^\infty \aluminadensity \frac{4}{3}\pi r^3\parent{n_{p,m+1}^k-n_{p,m}^k}\intd{r}\intd{x}\label{eq:energy-condition}
\end{eqnarray}
en considérant (\ref{eq:energy-mass-balance}). Le premier terme du
membre de droite de (\ref{eq:energy-condition}) est égal à
\begin{equation*}
\aluminahc\parent{\tinit-\tinj}  \frac{I[\cee{Al2O3}]}{6F}
\end{equation*}
en vertu de (\ref{eq:injection-mass-condition}). Pour que le deuxième
terme du membre de droite de (\ref{eq:energy-condition}) converge, il
est suffisant que chaque population $k\geq 1$ se dissolve en un temps
fini donné quelque soit $k$. Plus précisement, les populations $n_p^k$
se dissolvent en un temps fini s'il existe un entier positif $q'$ tel
que $n_{p,n}^k = 0$ pour tout $n > q^k + q'$.

Le cas échéant on peut télescoper la somme sur $k$ et en remarquant que
\begin{equation*}
  \int_\Omega n_{p,q^k}^k\intd{x} = \int_\Omega S^k\intd{x}
\end{equation*}
pour tout $k \leq 1$ et pour tout $r > 0$, le deuxième terme du
membre de droite de (\ref{eq:energy-condition}) se réduit à
\begin{equation*}
  {\aluminadissolutionenthalpy}\lim_{n\to\infty}\frac{1}{t_{n+1}}\sum_{\mathclap{\substack{k\\ p^k +
  1 \leq n + 1}}}\int_\Omega\int_0^\infty\aluminadensity
  \frac{4}{3}\pi r^3 S^k\intd{r}\intd{x}.
\end{equation*}
Ainsi finalement,
\begin{equation*}
  \frac{1}{\sigma}\int_\Omega j\cdot j\intd{x}
  =\aluminahc\parent{\tinit-\tinj}\frac{I[\cee{Al2O3}]}{6F} + {\aluminadissolutionenthalpy}\frac{I[\cee{Al2O3}]}{6F}
\end{equation*}
et on fixe $\conductivity$ de sorte à satisfaire cette dernière
relation, c'est-à-dire que
\begin{equation}\label{eq:conductivity-condition}
  \conductivity = \frac{\displaystyle\int_\Omega j\cdot j\intd{x}}{\parent{\aluminahc\parent{\tinit-\tinj} + {\aluminadissolutionenthalpy} }
    \displaystyle\frac{I[\cee{Al2O3}]}{6F}}.
\end{equation}

\paragraph{Diffusivité du bain électrolytique} La vitesse d'écoulement
stationnaire du bain $u$ obtenue par la méthode introduite dans
\cite{Steiner2009}, \cite{Rochat2016} est la vitesse moyenne d'un
écoulement turbulent. Les structures turbulentes du fluide sont
décritent par un modèle de longueur de mélange de Smagorinsky
\cite{Rochat2016}. Dans le modèle de Smagorinsky, les structures
turbulentes de l'écoulement se traduise par une viscosité de
l'écoulement moyen $u$ proportionnelle au tenseur des déformation de
$u$. Dans le présent travail, nous caractérisons la diffusivité
$\electrolytecdiff$ de la concentration $\concentration$ et la
diffusivité $\electrolytetdiff$ de la température $\temperature$ par
deux réels $\electrolytemoldiff$, $\electrolyteturbdiff$ pour tout
$x\in\Omega$ de la manière suivante:
\begin{align}
  \electrolytetdiff(x) =  \electrolytecdiff(x) = \electrolytemoldiff +
  \electrolyteturbdiff\parent{2\sum_{i,j}\mathcal E_{ij}(x)\mathcal E_{ij}(x)}^{1/2},
\end{align}
où $\mathcal E_{ij}$, $i, j = 1,2,3$ est le tenseur des déformation
de l'écoulement défini par
\begin{equation}
  \mathcal E_{ij} = \frac{1}{2}\parent{\frac{\partial u_i}{\partial
      x_j} + \frac{\partial u_j}{\partial x_i}}.
\end{equation}
Les valeurs du paramètre $\electrolytemoldiff$, associé à la diffusion
moléculaire et de $\electrolyteturbdiff$, associé à la diffusion
induite par les turbulences de l'écoulement, sont rapporté dans le
tableau \ref{tab:dissolution-physical-parameters}.


\begin{table}
  \begin{center}
    \caption{Paramètres physiques et paramètres liés à la cuve AP32
      qui interviennent dans le transport et la dissolution de poudre
      d'alumine.}
    \label{tab:dissolution-physical-parameters}
    \begin{tabularx}{\textwidth}{@{}lllX@{}}
      \toprule
      Quantité                         & Valeur        & Unités                                      & Description \\
      \midrule
      $\electrolytedensity$            & \num{2130}    & \si{\kg\per\cubic\meter}                    & Masse volumique du bain électrolytique                          \\
      $\aluminadensity$                & \num{3960}    & \si{\kg\per\cubic\meter}                    & Masse volumique de l'oxyde d'aluminium                          \\
      $\electrolytehc$                 & \num{2945}    & \si{\joule\per\kilo\gram\per\kelvin}        & Chaleur spécifique du bain électrolytique liquide               \\
      $\aluminahc$                     & \num{1200}    & \si{\joule\per\kilo\gram\per\kelvin}        & Chaleur spécifique de l'oxyde d'aluminium                       \\
      $g$                              & \num{9.81}    & \si{\meter\per\square\second}               & Accélération de la gravité terrestre                            \\
      $I$                              & \num{320000}  & \si{\ampere}                                & Courant électrique total                                        \\
      $\faraday$                       & \num{96485.33}& \si{\coulomb\per\mol}                       & Constante de Faraday                                            \\
      $\faradayyield$                  & \num{94.5}    & \%                                          & Rendement de Faraday de l'électrolyse                           \\\relax
      [\ce{Al2O3}]                     & \num{0.102}   & \si{\kilo\gram\per\mol}                     & Masse molaire de l'oxyde d'aluminium                            \\
      $\tinit$                         & \num{1223}    & \si{\kelvin}                                & Température initiale du bain électrolytique                     \\
      $\tinj$                          & \num{423}     & \si{\kelvin}                                & Température d'injection des particules d'alumine                \\
      $\tliq$                          & \num{1219}    & \si{\kelvin}                                & Température du liquidus du bain électrolytique                  \\
      $\tcrit$                         & \num{1219.86} & \si{\kelvin}                                & Température critique de dissolution dans le bain électrolytique \\
      $\aluminadissolutionenthalpy$    & \num{5.3e5}   & \si{\joule\per\kilo\gram}                   & Enthalpie de dissolution de l'oxyde d'aluminium                 \\
      $\conductivity$                  & \num{0}       & \si{\siemens\per\meter}                     & Conductivité électrique du bain électrolytique                  \\
%     $\electrolytelaminarviscosity$   & \num{2e-3}    & \si{\kilo\gram\per\meter\per\second}        & Viscosité laminaire du bain électrolytique                      \\
%     $\electrolyteturbulentviscosity$ & \num{2e-3}    & \si{\kilo\gram\per\meter\per\second}        & Viscosité turbulente du bain électrolytique                     \\
      $\electrolytemoldiff$            & \num{5e-4}    & \si{\squared\meter\per\second}              & Diffusivité moléculaire dans le bain                            \\
      $\electrolyteturbdiff$           & \num{5e-4}    & \si{\squared\meter\per\second}              & Diffusivité turbulente dans le bain                             \\
      $\csat$                          & \num{1689.7}  & \si{\mol\per\cubic\meter}                   & Concentration de saturation de l'alumine dissoute               \\
      $\csatwp$                        & \num{7.8}     & w\%                                         & Concentration de saturation de l'alumine dissoute               \\
      $\cinit$                         & \num{635.3}   & \si{\mol\per\cubic\meter}                   & Concentration initiale de l'alumine dissoute                    \\
      $\cinitwp$                       & \num{3.0}     & w\%                                         & Concentration initiale de l'alumine dissoute                    \\
      \bottomrule
    \end{tabularx}
  \end{center}
\end{table}

\paragraph{Conditions initiales de la concentration et de la
  température}
Les conditions initiales des population de particules $n_p^k$ sont
données implicitement dans la description du modèle par la
relation (\ref{eq:splitting-np1-init}). Plus précisement, aucune
particule n'est présente dans le bain électrolytique à $t = 0$, et
donc $n_p(0, x,r) = 0$ pour $r>0$ et $x\in\Omega$.

On observe expérimentalement sur des cuves d'électrolyse industrielles
que celle-ci atteigne un état stationnaire périodique lié à la période
du cycle d'injection après un temps caractéristique de transition. On
observe typiquement l'établissement de cet état périodique par
d'intermédiaire de mesures indirectes de la résistivité électrique de
la cuve. Par analogie, dans le cadre du modèle de transport et
dissolution d'alumine nous nous attendons à ce que la densité de
particules $n_p$, la concentration $\concentration$ et la température
$\temperature$ atteigne asymptotiquement lorsque $t\to\infty$ un état
périodique stationnaire de période $P$ identique à la période du
cycle d'injection global, soit \num{192}\si{\second} dans notre cas.

Il est clair que plus les conditions initiales pour $c$ et
$\temperature$ sont éloignées de l'état périodique stationnaire
asymptotique, plus le temps nécessaire pour atteindre un tel état est
long. Par conséquent, nous choisissons des conditions initiales
uniformes en espace pour la concentration et la température, et qui
soient égales aux conditions optimales d'opération de la cuve AP32:
\begin{align*}
  & \concentration(0, x) = \cinit,\quad\forall x\in\Omega,\\
  & \temperature(0, x) = \tinit,\quad\forall x\in\Omega.
\end{align*}
Le paramètre $\tinit$ est reporté dans la table
\ref{tab:dissolution-physical-parameters}. La concentration initiale
$\cinit$ en \si{\mol\per\cubic\meter} s'exprime en fonction de la
concentation initiale $\concentration_{\text{init},\%\mathrm w}$ en \% masse à
l'aide de la formule:
\begin{equation*}
  \cinit = \frac{\cinitwp \cdot
    100^{-1}}{\electrolytedensity[\cee{Al2O3}]\parent{1 - \cinitwp\cdot
      100^{-1}\parent{1 - \displaystyle\frac{\electrolytedensity}{\aluminadensity}}}}.
\end{equation*}
Nous exploiterons également cette dernière formule pour convertir les
valeurs du champ de concentration en \% masse lors de la visualisation
des résultats. Les deux valeurs de la concentration initiale selon les
unités physique sont rapportées dans la table
\ref{tab:dissolution-physical-parameters}.
% paramètres numérique. (deltat, maillage, deltar, N_r)

\begin{figure}[t]
  \begin{center}
    \includegraphics[width=\rasterimagewidth]{../media/populations/ap32-mesh-components/print/bath-mesh.png}
    \caption{Aperçu du maillage du domaine occupé par le bain
      électrolytique dans la cuve AP32.}
    \label{fig:bath-mesh}
  \end{center}
\end{figure}

\paragraph{Paramètres de discrétisation}
Le maillage du domaine $\Omega$ est obtenu en même temps que les
champs $u$ et $j$ par la méthode déjà évoquée plus haut proposée par
\cite{Steiner2009}, \cite{Rochat2016}. On peut voir sur la figure
\ref{fig:bath-mesh} un aperçu du maillage de $\Omega$ qui correspond
à l'une des extrémités de la cuve. Le maillage est fortement
anisotrope, avec des rapports d'aspect d'environ \num{25} dans
l'ACD. Le diamètre des mailles est compris entre
\num{38}\si{\centi\meter} (aux extrémités de la cuve) et
\num{9}\si{\centi\meter} (dans l'ACD et les canaux). Le maillage
comporte \num{282240} éléments tetraédriques. On choisit le nombre de
discrétisation des rayons des particules $M = 5$ et $\dr = \num{40}$
\si{\micro\meter}. Enfin on fixe $\dt = 1$ \si{\second} et $T =
\num{10000}$ \si{\second}.


\paragraph{Solution de référence}
