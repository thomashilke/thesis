On répète ici, par soucis de clarté, le système d'équations aux
dérivées partielles qui correspondent au modèle de transport et
dissolution d'alumine en fonction de la température. On a $\forall k =
1, 2, \dots, K$
\begin{align}
  & n_p^k(t, x, r) = 0,
  & 0\leq t < \tau^k,\label{eq:mdl-eq-1}\\
%
  & \frac{\partial n_p^k}{\partial t} + \parent{u(x) + w(r)}\cdot \nabla  n_p^k = 0,
  & \tau^k < t \leq \tau^k + \tlat,\label{eq:mdl-eq-2}\\
%
  & \frac{\partial n_p^k}{\partial t} + \parent{u(x) + w(r)}\cdot \nabla  n_p^k + \frac{\partial }{\partial r}\parent{f(r, \concentration, \temperature)n_p^k} = 0,
  & \tau^k + \tliq < t \leq \tend,\label{eq:mdl-eq-3}
\end{align}
et
\begin{align}
  & \frac{\partial \concentration}{\partial t} + u(x)\cdot\nabla \concentration - \cdiffusivity(x) \Delta \concentration = q_1 + q_2,
  & \forall t\in (0, T),\label{eq:mdl-eq-4}\\
%
  & \frac{\partial \temperature}{\partial t} + u(x)\cdot\nabla\temperature - \electrolytetdiff(x)\Delta \temperature = \frac{1}{\electrolytedensity\electrolytehc}\sum_{i = 1}^{3}p_i,
  & \forall t\in(0, T)\label{eq:mdl-eq-5}
\end{align}
dans $\Omega$.

Ces équations forment un système couplé pour les inconnues $n_p^k$,
$\concentration$ et $\temperature$. En effet, la densité de
particules $n_p$ dépend de la concentration $\concentration$ et de
la température $\temperature$ par l'intermédiaire de la vitesse de
dissolution $f$, tandis que la concentration et la température
dépendent de la densité de particule $n_p$ à travers leurs termes
sources respectifs $q_2$, $p_1$ et $p_2$.

En suivant l'approche adoptée dans \cite{Hofer2011}, on propose de
discrétiser les équations (\ref{eq:mdl-eq-1}) à (\ref{eq:mdl-eq-5})
par une méthode de splitting en temps de la façon suivante. Soit
l'entier $N$, le nombre de pas de temps et $\dt = T/N$ un pas de temps
uniforme. Soient $t_n = n\dt$, $n = 0, 1, \dots, N$, une subdivision de
l'intervalle de temps $[0, T]$. On note $n_{p,n}^k$ une approximation
de $n_p^k(t_n, ., .)$, $n_{p,n}$ de $n_p(t_n, ., .)$, $\concentration_n$ une approximation de
$\concentration(t_n, .)$ et $\temperature_n$ une approximation de
$\temperature(t_n, .)$. Si $k$ et l'indice de l'injection d'une
population de particules, on définit $p^k$ et $q^k$ les plus grands
entiers tels que
\begin{equation}
  t_{p^k} < \tau^k \quad\text{et}\quad t_{q^k} < \tau^k + \tlat.
\end{equation}
En d'autres termes, $p_k$ est le dernier pas de temps qui précède
l'injection de la population $k$, et $q_k$ est le dernier pas de temps
qui précède le début de la dissolution de la population $k$. Bien
entendu, si $\tlat = 0$ alors $p_k = q_k$.

Etant donnés $n_{p,n}^k$ $\forall k=1, 2, \dots, K$,
$\concentration_n$ et $\temperature_n$, on pose
\begin{align}
  & n_{p,n+1}^k = 0, &&\text{si } n \leq p^k,\\
  & n_{p,n+1}^k = S^k, &&\text{si } n = p^k + 1,\\
  & \frac{n_{p,n+1}^k - n_{p,n}^k}{\dt} + u\cdot\nabla n_{p,n+1}^k = 0, && \text{si } p^k + 1 < n \leq q^k.\label{eq:splitting-np1-u}
\end{align}
dans $\Omega$ et pour tout $r > 0$. Puis, si $n$ est tel que $q^k + 1 \leq n < N$, on pose
\begin{align}
  &\displaystyle\frac{\bar n_{p,n+1}^k - n_{p,n}^k}{\dt} +
  u\cdot\nabla \bar n_{p,n+1}^k = 0,\label{eq:splitting-np2-u}\\
  &\displaystyle\frac{\bar{\bar n}_{p,n+1}^k - \bar n_{p,n}^k}{\dt} +
  w\cdot\nabla \bar{\bar n}_{p,n+1}^k = 0,\label{eq:splitting-np2-w}\\
    &\displaystyle\frac{n_{p,n+1}^k - \bar{\bar n}_{p,n+1}^k}{\dt} +
    \displaystyle\frac{\partial}{\partial r}\parent{f(r,
      \concentration_n, \temperature_n)n_{p,n+1}^k} =
    0\label{eq:splitting-dissolution}
\end{align}
dans $\Omega$ et pour tout $r > 0$. Et finalement
\begin{align}
&\frac{\concentration_{n+1} - \concentration_{n}}{\dt} + u\cdot\nabla
\concentration_n - \cdiffusivity \Delta \concentration_n = q_1
+ q_{2,n},\label{eq:splitting-c} \\
&\frac{\temperature_{n+1} - \temperature_{n}}{\dt} + u\cdot\nabla
\temperature_n - \electrolytetdiff \Delta \temperature_n =
\frac{1}{\electrolytedensity \electrolytehc}\parent{\sum_{i =
    1}^2p_{i,n} + p_3}\label{eq:splitting-t}
\end{align}
dans $\Omega$. On précise maintenant la forme des termes sources
discrétisés $q_{2,n}$, $p_{i,n}$, $i = 1,2$.

\paragraph{Discrétisation de la source d'alumine $q_{2}$}
Le terme source $q_{2}$ qui apparaît dans l'équation
(\ref{eq:mdl-eq-4}) correspond à la masse d'alumine qui est transférée
entre les particules qui se dissolvent et l'alumine dissoute par unité
de temps. Afin de permettre une conservation de la masse d'alumine
exacte par le schéma numérique entre les champs $n_p$ et $c$, nous
tirons parti du splitting en temps des équations (\ref{eq:mdl-eq-3})
et (\ref{eq:mdl-eq-4}) \cite{Hofer2011}. Plus précisement, grâce au
splitting en temps, les quantités $n_{p,n+1}^k$, $k = 1, 2, \dots, K$
sont indépendentes de $c_{n+1}$. On pose alors
\begin{equation}
  q_{2,n}(x) = -\frac{1}{\dt} \sum%_\substack{1\leq k\leq K\\ q^k < n}
  \int_{0}^\infty
  \frac{\aluminadensity}{[\ce{Al2O3}]} \frac{4}{3}\pi r^3
  \parent{n_{p,n+1}^k(x) - \bar n_{p,n+1}^k(x)} \intd{r}, \quad x\in\Omega.
\end{equation}
Ici, la somme porte sur toutes les populations de particules qui se
dissolvent à l'instant $t_n$, c'est-à-dire les populations $k$
telles que $t_{q^k} < t_n$.


\paragraph{Discrétisation des source de puissance thermique $p_1$ et
$p_2$} Le terme source de puissance thermique $p_1$ est discrétisé
en régularisant la masse de Dirac sur les intervalles $[t_{p^k},
  t_{p^k} + \dt]$, $k = 1, 2, \dots, K$. On pose
\begin{equation}
p_{1,n}(x) = -\parent{\tinit - \tinj}\sum_{k = 1}^K
\frac{1}{\dt}\kronecker_{n,p^k}\int_{\rplus} \aluminadensity\aluminahc
\frac{4}{3}\pi r^3S^k(x, r)\,\intd{r}
\end{equation}
où $\kronecker_{ij}$ est le symbol de Kronecker définit par
\begin{equation}
  \kronecker_{ij} = \left\{
  \begin{array}{ll}
    1&\quad\text{si } i = j,\\
    0&\quad\text{si } i\neq j.
  \end{array}\right.
\end{equation}
Le terme source $p_2$ qui correspond à la puissance thermique
nécessaire à la dissolution des particules est discrétisé en
utilisant $q_{2,n}$:
\begin{equation}
p_{2,n} = -\aluminadissolutionenthalpy q_{2,n}.
\end{equation}

L'équation (\ref{eq:splitting-dissolution}) est discrétisé selon $r$
à l'aide du schéma de caractéristiques présenté dans la section
\ref{sec:particle-population-dissolution}. Les équations
(\ref{eq:splitting-np1-u}), (\ref{eq:splitting-np2-u}),
(\ref{eq:splitting-np2-w}), (\ref{eq:splitting-c}) et
(\ref{eq:splitting-t}) sont des équations d'advection ou
d'advection-diffusion et sont discrétisées en espace selon la
méthode adoptée dans \cite{Hofer2011}. La discrétisation est basée
sur des éléments finis stabilisée par la méthode SUPG
\cite{Quarteroni2008}. La section suivant traite de l'application de
ce modèle numérique à une cuve d'électrolyse d'aluminium industrielle.
