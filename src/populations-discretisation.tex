On répète ici, par soucis de clarté, le système d'équations aux
dérivées partielles qui correspondent au modèle de transport et
dissolution d'alumine en fonction de la température, soit les
équations (\ref{eq:population-pre-injection}) à
(\ref{eq:population-dissolution}), (\ref{eq:concentration}) et
(\ref{eq:temperature}). On a $\forall k = 1, 2, \dots, K$
\begin{align}
  & n_p^k(t, x, r) = 0,
  & 0\leq t < \tau^k,\label{eq:mdl-eq-1}\\
%
  & \frac{\partial n_p^k}{\partial t} + \parent{u(x) + w(r)}\cdot \nabla  n_p^k = 0,
  & \tau^k < t \leq \tau^k + \tlat,\label{eq:mdl-eq-2}\\
%
  & \frac{\partial n_p^k}{\partial t} + \parent{u(x) + w(r)}\cdot \nabla  n_p^k + \frac{\partial }{\partial r}\parent{f(r, \concentration, \temperature)n_p^k} = 0,
  & \tau^k + \tlat < t \leq \tend,\label{eq:mdl-eq-3}
\end{align}
et
\begin{align}
  & \frac{\partial \concentration}{\partial t} + u(x)\cdot\nabla \concentration - \div\parent{\cdiffusivity(u(x)) \nabla \concentration} = q_1 + q_2,
  & \forall t\in (0, T),\label{eq:mdl-eq-4}\\
%
  & \frac{\partial \temperature}{\partial t} + u(x)\cdot\nabla\temperature - \div\parent{\electrolytetdiff(u(x))\nabla \temperature} = \frac{1}{\electrolytedensity\electrolytehc}\sum_{i = 1}^{3}p_i,
  & \forall t\in(0, T)\label{eq:mdl-eq-5}
\end{align}
dans $\Omega$. Les termes source de la concentration $\concentration$ sont
\begin{equation}
  q_1(x) = -\frac{I}{6\faraday} %
  \frac{\abs{j(x)}}{\displaystyle\int_\Omega\abs{j(x)}\intd{x}}
\end{equation}
et
\begin{equation}
  q_2(t, x) = -\sum_{k = 1}^{\bar
    k(t)}\frac{4\pi\aluminadensity}{[\ce{Al2O3}]}\int_{\rplus}n_p^k(t,
  x, r)f(r, \concentration(t, x), \temperature(t, x))r^2\,\intd{r}.
\end{equation}
Les termes source de la température $\temperature$ sont
\begin{equation}
  p_1(t, x) = -\parent{\tinit-\tinj}\sum_{k = 1}^{K}\dirac(t -
  \tau^k)\int_{\rplus}\aluminadensity\aluminahc\frac{4}{3}\pi r^3 S^k(x, r)\,\intd{r},
\end{equation}
\begin{equation}
  p_2(t, x) = - \aluminadissolutionenthalpy q_2(t, x),
\end{equation}
et
\begin{equation}
  p_3(x) = \frac{j\cdot j}{\conductivity}.
\end{equation}

Ces équations munies de conditions limites et initiales forment un
système couplé pour les inconnues $n_p^k$, $\concentration$ et
$\temperature$. En effet, la densité de particules $n_p$ (voir
(\ref{eq:populations-sum})) dépend de la concentration
$\concentration$ et de la température $\temperature$ par
l'intermédiaire de la vitesse de dissolution $f$, tandis que la
concentration et la température dépendent de la densité de particule
$n_p$ à travers leurs termes sources respectifs $q_2$, $p_1$ et $p_2$.

En suivant l'approche adoptée dans \cite{Hofer2011}, on propose de
discrétiser les équations (\ref{eq:mdl-eq-1}) à (\ref{eq:mdl-eq-5})
par une méthode de splitting en temps de la façon suivante. Soit un
nombre de pas de temps $N$ et soit $\dt = T/N$ un pas de temps
uniforme. Soient $t_n = n\dt$, $n = 0, 1, \dots, N$, une subdivision
de l'intervalle de temps $[0, T]$. On note $n_{p,n}^k$ une
approximation de $n_p^k(t_n, ., .)$, $n_{p,n}$ une approximation de
$n_p(t_n, ., .)$, $\concentration_n$ une approximation de
$\concentration(t_n, .)$ et $\temperature_n$ une approximation de
$\temperature(t_n, .)$. Puisque $k$ est l'indice de l'injection d'une
population de particules, on définit $i^k$ et $j^k$ tels que
\begin{equation}
  i^k = \max\cparent{j\in \mathbb N\ \middle|\ j\dt < \tau^k}, \quad
  j^k = \max\cparent{j \in \mathbb N\ \middle|\ j\dt < \tau^k + \tlat}.
\end{equation}
En d'autres termes, $i^k$ est l'indice du dernier pas de temps qui précède
l'injection de la population $k$, et $j^k$ est l'indice du dernier pas de temps
qui précède le début de la dissolution de la population $k$. Bien
entendu, si $\tlat = 0$ alors $i_k = j_k$.

Étant donnés $n_{p,n}^k$ $\forall k=1, 2, \dots, K$,
$\concentration_n$ et $\temperature_n$, on pose
\begin{align}
  & n_{p,n+1}^k = 0, &&\text{si }  n + 1 < i^k + 1,\label{eq:splitting-np1-init}\\
  & n_{p,n+1}^k = S^k, &&\text{si } n + 1 = i^k + 1,\\
  & \frac{\bar n_{p,n+1}^k - n_{p,n}^k}{\dt} + u\cdot\nabla \bar n_{p,n+1}^k = 0,\\
  & \frac{n_{p,n+1}^k - \bar n_{p,n}^k}{\dt} + w\cdot\nabla n_{p,n+1}^k = 0, && \text{si } i^k + 1 < n+1 \leq j^k,\label{eq:splitting-np1-u}
\end{align}
dans $\Omega$ et pour tout $r > 0$. Puis, si $n$ est tel que $j^k < n
+ 1 \leq N$, on pose
\begin{align}
  &\displaystyle\frac{\bar n_{p,n+1}^k - n_{p,n}^k}{\dt} +
  u\cdot\nabla \bar n_{p,n+1}^k = 0,\label{eq:splitting-np2-u}\\
  &\displaystyle\frac{\bar{\bar n}_{p,n+1}^k - \bar n_{p,n+1}^k}{\dt} +
  w(r)\cdot\nabla \bar{\bar n}_{p,n+1}^k = 0,\label{eq:splitting-np2-w}\\
    &\displaystyle\frac{n_{p,n+1}^k - \bar{\bar n}_{p,n+1}^k}{\dt} +
    \displaystyle\frac{\partial}{\partial r}\parent{f(r,
      \concentration_n, \temperature_n)n_{p,n+1}^k} =
    0\label{eq:splitting-dissolution}
\end{align}
dans $\Omega$ et pour tout $r > 0$. Et finalement
\begin{align}
&\frac{\concentration_{n+1} - \concentration_{n}}{\dt} + u\cdot\nabla
\concentration_{n+1} - \div\parent{\cdiffusivity \nabla \concentration_{n+1}} = q_1
+ q_{2,n+1},\label{eq:splitting-c} \\
&\frac{\temperature_{n+1} - \temperature_{n}}{\dt} + u\cdot\nabla
\temperature_{n+1} - \div\parent{\electrolytetdiff \nabla \temperature_{n+1}} =
\frac{1}{\electrolytedensity \electrolytehc}\parent{\sum_{i =
    1}^2p_{i,n+1} + p_3}\label{eq:splitting-t}
\end{align}
dans $\Omega$. On précise maintenant la forme des termes sources
discrétisés $q_{2,n+1}$, $p_{i,n+1}$, $i = 1,2$.

\begin{remarque}
Les équations (\ref{eq:splitting-np2-u})-(\ref{eq:splitting-dissolution}) forment un ``splitting'' en
temps de (\ref{eq:mdl-eq-3}).
\end{remarque}

\paragraph{Discrétisation de la source d'alumine $q_{2}$.}
Le terme source $q_{2}$ qui apparaît dans l'équation
(\ref{eq:mdl-eq-4}) correspond à la masse d'alumine qui est transférée
entre les particules qui se dissolvent et l'alumine dissoute par unité
de temps. Afin de permettre une conservation exacte de la masse d'alumine
par le schéma numérique entre les champs $n_p$ et $c$, nous
tirons parti du splitting en temps des équations (\ref{eq:mdl-eq-3})
et (\ref{eq:mdl-eq-4}) \cite{Hofer2011}. Plus précisement, grâce au
splitting en temps, les quantités $n_{p,n+1}^k$ et $\bar{\bar n}_{p,n+1}^k$, $k = 1, 2, \dots, K$
sont indépendentes de $c_{n+1}$. On pose alors
\begin{equation}\label{eq:q2}
  q_{2,n+1}(x) = -\frac{1}{\dt} \sum_{\substack{1\leq k\leq K\\ j^k < n+1}}
  \int_\rplus
  \frac{\aluminadensity}{[\ce{Al2O3}]} \frac{4}{3}\pi r^3
  \parent{n_{p,n+1}^k(x) - \bar{\bar n}_{p,n+1}^k(x)} \intd{r}, \quad x\in\Omega.
\end{equation}
Ici, la somme porte sur toutes les populations de particules qui se
dissolvent à l'instant $t_n$, c'est-à-dire les populations $k$
telles que $t_{j^k} < t_n$.


\paragraph{Discrétisation des sources de puissance thermique $p_1$ et
$p_2$.} Le terme source de puissance thermique $p_1$ (voir (\ref{eq:power-heating})) est discrétisé
en régularisant la masse de Dirac sur les intervalles $[t_{i^k},
  t_{i^k} + \dt]$, $k = 1, 2, \dots, K$. On pose
\begin{equation}\label{eq:p1-discret}
p_{1,n+1}(x) = -\parent{\tinit - \tinj}\sum_{k = 1}^K
\frac{1}{\dt}\kronecker_{n+1,i^k}\int_{\rplus} \aluminadensity\aluminahc
\frac{4}{3}\pi r^3S^k(x, r)\,\intd{r}
\end{equation}
où $\kronecker_{i,j}$ est le symbol de Kronecker définit par
\begin{equation}
  \kronecker_{i,j} = \left\{
  \begin{array}{ll}
    1&\quad\text{si } i = j,\\
    0&\quad\text{si } i\neq j.
  \end{array}\right.
\end{equation}
Le terme source $p_2$, qui correspond à la puissance thermique
nécessaire à la dissolution des particules (voir (\ref{eq:p2})), est
discrétisé en utilisant $q_{2,n+1}$:
\begin{equation}\label{eq:p2-discret}
p_{2,n+1} = -[\cee{Al2O3}]\aluminadissolutionenthalpy q_{2,n+1}.
\end{equation}

L'équation (\ref{eq:splitting-dissolution}) est discrétisée selon $r$ à
l'aide du schéma des caractéristiques présenté dans la section
\ref{sec:particle-population-dissolution}. Les équations
(\ref{eq:splitting-np1-u}), (\ref{eq:splitting-np2-u}),
(\ref{eq:splitting-np2-w}), (\ref{eq:splitting-c}) et
(\ref{eq:splitting-t}) sont des équations d'advection ou
d'advection-diffusion et sont discrétisées en espace selon la méthode
adoptée dans \cite{Hofer2011}. La discrétisation est basée sur une
méthode d'éléments finis continus linéaires par morceaux et stabilisée
par une méthode de type SUPG \cite{Quarteroni2008}.

\paragraph{Propriétés de conservation du schéma de discrétisation}
Nous établissons maintenant le bilan de masse d'alumine dans les
champs $n_{p,n+1}^k$, $k = 1,\dots, K$ et $c_{n+1}$ à l'instant
$t_{n+1}$ dans le cas particulier où la vitesse de sédimentation des
particules est nulles, \ie, lorsque $w(r)$ = 0. On note la masse
de particules d'alumine dans la population $k$ à l'instant $t_{n+1}$ par
\begin{equation}\label{eq:np-masse-def}
  N_{p,n+1}^k = \int_\Omega \intd{x}\int_\rplus\intd{r} \aluminadensity
  n_{p,n+1}^k\frac{4}{3}\pi r^3.
\end{equation}
On notera encore la masse totale de particules d'alumine à l'instant
$t_{n+1}$ par $N_{p,n+1} =
\sum_{k = 1}^K N_{p,n+1}^k$ et la masse d'alumine dissoute à l'instant
$t_{n+1}$ par
\begin{equation}\label{eq:c-masse-def}
  C_{n+1} = [\cee{Al2O3}]\int_\Omega c_n\intd{x}.
\end{equation}
Pour rappel, on a noté ici $[\ce{Al2O3}]$ la masse molaire de l'alumine.

En vertu des équations (\ref{eq:splitting-np1-init}) à
(\ref{eq:splitting-dissolution}), l'accroissement de la masse d'alumine
de la population $k$ entre les instants $t_{n}$ et $t_{n + 1}$ s'écrit
\begin{equation}\label{eq:np-accr}
  N_{p,n+1}^k - N_{p,n}^k = \left\{
  \begin{array}{ll}
    0, & \text{si }n+1 \leq i^k, \\
    \displaystyle\int_\Omega\intd{x}\displaystyle\int_\rplus\intd{r} \aluminadensity S^k\frac{4}{3}\pi
    r^3, & \text{si }n+1 = i^k + 1, \\
    0, & \text{si }i^k+1 < n+1 \leq j^k, \\
    \displaystyle\int_\Omega\intd{x}\displaystyle\int_\rplus\intd{r} \aluminadensity \frac{4}{3}\pi
    r^3\parent{n_{p,n+1}^k - \bar{\bar n}_{p,n+1}}, &\text{si } j^k < n+1 \leq N.
  \end{array}
  \right.
\end{equation}
On obtient l'accoissement de la masse totale de particules entre les
instants $t_n$ et $t_{n+1}$ en sommant sur $k = 1$ à $K$:
\begin{eqnarray*}
  N_{p,n+1} - N_{p,n}
  &=& \sum_{\mathclap{\substack{k\\ n+1 \leq i^k}}} \parent{N_{p,n+1}^k - N_{p,n}^k} %
  + \sum_{\mathclap{\substack{k\\ n+1 = i^k + 1}}} \parent{N_{p,n+1}^k - N_{p,n}^k}\\
  && + \sum_{\mathclap{\substack{k\\ i^k + 1 < n+1 \leq j^k}}} \parent{N_{p,n+1}^k - N_{p,n}^k} %
  + \sum_{\mathclap{\substack{k\\ j^k < n+1}}} \parent{N_{p,n+1}^k -
    N_{p,n}^k}.
\end{eqnarray*}
Puisque
\begin{equation*}
\sum_{\mathclap{\substack{k\\ n+1 \leq i^k}}} \parent{N_{p,n+1}^k - N_{p,n}^k} = 0 \quad \text{et}\quad \sum_{\mathclap{\substack{k\\ i^k + 1 < n+1 \leq j^k}}} \parent{N_{p,n+1}^k - N_{p,n}^k} = 0
\end{equation*}
en vertu de (\ref{eq:np-accr}), on obtient
\begin{eqnarray}
  N_{p,n+1} - N_{p,n}
  &=&\sum_{\mathclap{\substack{k\\ n+1 = i^k+1}}} \displaystyle\int_\Omega\intd{x}\displaystyle\int_\rplus\intd{r} \aluminadensity S^k\frac{4}{3}\pi
  r^3 \nonumber\\
  && + \sum_{\mathclap{\substack{k \\ j^k < n+1}}} \displaystyle\int_\Omega\intd{x}\displaystyle\int_\rplus \intd{r}\aluminadensity \frac{4}{3}\pi
    r^3\parent{n_{p,n+1}^k - \bar{\bar n}_{p,n+1}^k}, \label{eq:np-masse-1}
\end{eqnarray}

En intégrant les expression (\ref{eq:splitting-np2-u}),
(\ref{eq:splitting-np2-w}) sur $\Omega$, en utilisant le théorème
de la divergence et en se rappelant que $\div u = 0$ dans $\Omega$,
que $u\cdot\nu = 0$ sur $\partial \Omega$ et que $\frac{\partial
  n_{p,n+1}}{\partial \nu} = 0$ sur $\partial \Omega$, on obtient que
\begin{equation}\label{eq:np-masse-2}
  \int_\Omega \bar{\bar n}_{p,n+1}^k\intd{x} = \int_\Omega n_{p,n}^k\intd{x}.
\end{equation}
En remplaçant (\ref{eq:np-masse-2}) dans (\ref{eq:np-masse-1}),
l'accroissement de masse totale de particules s'écrit
\begin{eqnarray}\label{eq:np-masse-3}
  N_{p,n+1} - N_{p,n} &=&\sum_{\mathclap{\substack{k\\ n+1 = i^k+1}}} \displaystyle\int_\Omega\intd{x}\displaystyle\int_\rplus\intd{r} \aluminadensity S^k\frac{4}{3}\pi
  r^3 \nonumber\\
  && + \sum_{\mathclap{\substack{k \\ j^k < n+1}}} \displaystyle\int_\Omega\intd{x}\displaystyle\int_\rplus\intd{r} \aluminadensity \frac{4}{3}\pi
    r^3\parent{n_{p,n+1}^k - n_{p,n}^k}.
\end{eqnarray}

On dérive maintenant une expression pour l'accroissement de la masse
d'alumine dissoute. En intégrant l'équation (\ref{eq:splitting-c}) où
le terme source $q_{2,n+1}$ est donné par l'expression (\ref{eq:q2}),
en utilisant le théorème de la divergence et en utilisant à nouveau le
fait que $\div u = 0$ dans $\Omega$, que $u\cdot \nu = 0$ et
$\frac{\partial c_n}{\partial \nu} = 0$ sur $\partial \Omega$, et que
$w = 0$ par hypothèse on obtient
\begin{eqnarray*}
[\cee{Al2O3}]\int_\Omega c_{n+1}\intd{x} - [\cee{Al2O3}] \int_\Omega
c_n\intd{x} &=& -\frac{\dt I [\cee{Al2O3}]}{6\faraday}\\
&& - \sum_{\mathclap{\substack{k \\ j^k < n+1}}}
\int_\Omega\intd{x}\int_\rplus\intd{r}\aluminadensity \frac{4}{3}\pi
r^3 \parent{n_{p,n+1} - n_{p,n}},\nonumber\\
\end{eqnarray*}
et ainsi par (\ref{eq:c-masse-def}),
\begin{eqnarray}
  C_{n+1} - C_n = -\frac{\dt I [\cee{Al2O3}]}{6\faraday} -
\sum_{\mathclap{\substack{k \\ j^k < n+1}}}
\int_\Omega\intd{x}\int_\rplus\intd{r}\aluminadensity \frac{4}{3}\pi
r^3 \parent{n_{p,n+1} - n_{p,n}}\label{eq:c-masse-1}
\end{eqnarray}
On définit la masse totale d'alumine dans le bain par $M_{n+1} =
N_{p,n+1} + C_n$, et on obtient l'accroissement total de masse
d'alumine dans le bain entre $t_n$ et $t_{n+1}$ en sommant
(\ref{eq:np-masse-3}) et (\ref{eq:c-masse-1}). On obtient
\begin{equation}
  M_{n+1} - M_{n} = - \dt \frac{I[\cee{Al2O3}]}{6\faraday} +
  \sum_{\mathclap{\substack{k\\ n+1 = i^k + 1}}} \int_\Omega\intd{x}\int_\rplus\intd{r}
  \aluminadensity S^k\frac{4}{3}\pi r^3.
\end{equation}
Par récurrence sur $n$ on obtient
\begin{equation}\label{eq:alumina-mass-balance}
  M_{n+1} - M_{0} = - t_{n+1} \frac{I[\cee{Al2O3}]}{6\faraday} +
  \sum_{\mathclap{\substack{k\\  i^k + 1 \leq n+1}}} \int_\Omega\intd{x}\int_\rplus\intd{r}
  \aluminadensity S^k\frac{4}{3}\pi r^3,
\end{equation}
c'est-à-dire que la masse totale d'alumine dans le bain dépend
uniquement du taux de consommation de l'alumine dissoute par
l'électrolyse (premier terme du membre de droite de
(\ref{eq:alumina-mass-balance})) et de la masse des doses
injectées antérieurement au temps $t_{n+1}$ (deuxième terme du membre
de droite de (\ref{eq:alumina-mass-balance})).

Nous concluons cette section en établissant un bilan de l'énergie
thermique dans le bain. En intégrant l'équation (\ref{eq:splitting-t})
sur $\Omega$ et en multipliant par $\dt\electrolytedensity\electrolytehc$
on a
\begin{multline}
\int_\Omega\electrolytedensity\electrolytehc\parent{\temperature_{n+1}
- \temperature_n}\intd{x}
+\dt\electrolytedensity\electrolytehc\parent{\int_\Omega
  u\cdot\nabla\temperature_{n+1}\intd{x} - \int_\Omega
  \div\parent{\electrolytetdiff\nabla\temperature_{n+1}}\intd{x}} \\
= \dt\sum_{i=1}^2\int_\Omega p_{i,n+1}\intd{x} + \dt\int_\Omega p_3\intd{x}.
\end{multline}
En utilisant le théorème de la divergence, et en utilisant le fait que
$\div u = 0$ dans $\Omega$, que $\frac{\partial
  \temperature_{n+1}}{\partial \nu} = 0$ et $u\cdot \nu = 0$ sur
$\partial \Omega$ il reste
\begin{equation}
  \int_\Omega\electrolytedensity\electrolytehc\parent{\temperature_{n+1} - \temperature_n}\intd{x} = \dt\sum_{i=1}^2\int_\Omega p_{i,n+1}\intd{x} + \dt\int_\Omega p_3\intd{x}.
\end{equation}
En remplaçant les expressions (\ref{eq:p1-discret}), (\ref{eq:p2-discret}) et (\ref{eq:p3}) pour le termes sources $p_{1,n+1}$, $p_{2,n+1}$ et $p_3$ on obtient
\begin{eqnarray*}
  \electrolytedensity\electrolytehc\int_\Omega\parent{\temperature_{n+1} - \temperature_n}\intd{x} &=& \frac{\dt}{ \conductivity}\int_\Omega j\cdot j\intd{x}\\
  && - \aluminahc\parent{\tinit - \tinj}\sum_{\mathclap{\substack{k\\ i^k + 1 = n+1}}}\int_\Omega\intd{x}\int_\rplus\intd{r} \aluminadensity \frac{4}{3}\pi r^3 S^k\\
  && + {\aluminadissolutionenthalpy}\sum_{\mathclap{\substack{k\\ j^k < n+1}}} \int_\Omega\intd{x}\int_\rplus\intd{r} \aluminadensity \frac{4}{3}\pi r^3\parent{n_{p,n+1}^k-n_{p,n}^k}.
\end{eqnarray*}
A nouveau par récurrence sur $n$ on obtient
\begin{eqnarray}
  \electrolytedensity\electrolytehc\int_\Omega\parent{\temperature_{n+1} - \temperature_0}\intd{x} &=& \frac{t_{n+1}}{\conductivity}\int_\Omega j\cdot j\intd{x}\nonumber\\
  && - \aluminahc\parent{\tinit - \tinj}\sum_{\mathclap{\substack{k\\ i^k + 1 \leq n+1}}}\int_\Omega\intd{x}\int_\rplus\intd{r} \aluminadensity \frac{4}{3}\pi r^3 S^k\nonumber\\
  && + {\aluminadissolutionenthalpy}\sum_{m =
    0}^{n}\sum_{\mathclap{\substack{k\\ j^k < m + 1}}}
  \int_\Omega\intd{x}\int_\rplus\intd{r} \aluminadensity \frac{4}{3}\pi
  r^3\parent{n_{p,m+1}^k-n_{p,m}^k},\nonumber\\ \label{eq:energy-mass-balance}
\end{eqnarray}
c'est-à-dire que l'énergie thermique du bain
$\electrolytedensity\electrolytehc\int_\Omega\temperature_{n}\intd{x}$
dépend uniquement de l'énergie thermique initiale, de l'intensité de
l'effet Joule, de la masse des doses d'alumine injectées
antérieurement à l'instant $t_n$ et de la masse de particules
d'alumine dissoutes durant l'intervalle de temps $[0, t_n]$. Nous
utiliserons la relation de bilan d'énergie thermique
(\ref{eq:energy-mass-balance}) pour fixer la conductivité électrique
$\sigma$ dans la section \ref{sec:populations-industriel}. La
conductivité sera choisie de sorte à ce que la variation de l'énergie
thermique $\electrolytedensity \electrolytehc \int_\Omega
\parent{\temperature_{n+1} - \temperature_0}\intd{x}$ soit bornée dans
la limite où $n\to\infty$.

\begin{remarque}
  En général, les bilans de masse (\ref{eq:alumina-mass-balance}) et
  (\ref{eq:energy-mass-balance}) sont valables pour autant que la
  totalité des particules se dissolvent à l'intérieur du bain. Cette
  condition est satisfaite si la vitesse de sédimentation $ w = 0$
  comme on l'a supposé dans ce dernier paragraphe. Cette condition est
  aussi satisfaite si $n_p^k = 0$ sur $\partial \Omega$, puisque le
  terme
  \begin{equation}
    \int_{\partial \Omega} n_p^k w\cdot \nu\intd{s} = 0
  \end{equation}
  également, \ie, le flux de particules à travers la frontière de
  $\Omega$ est nul.
\end{remarque}

\begin{remarque}
  Les bilans de masse et d'énergie ci-dessus ont été dérivés à partir du
  modèle de transport et dissolution semi-discrétisé. Pour que ces
  bilans restent valables pour le problème discrétisé en temps et en
  espace, il est essentiel que la discrétisation en espace des équations
  d'advection-diffusion (\ref{eq:splitting-np1-u}),
  (\ref{eq:splitting-np2-u}), (\ref{eq:splitting-np2-w}),
  (\ref{eq:splitting-c}) (\ref{eq:splitting-t}) conservent exactement
  les intégrales de $n_{p,n}$, $\concentration_n$ et
  $\temperature_n$ sur $\Omega$. Dans les approximations numériques
  des équations de convection-diffusion d'une grandeur physique,
  où le transport par une vitesse $u_h$ donnée ne satisfait pas
  exactement  $\div u_h = 0$ et $u_h\cdot \nu = 0$ sur $\partial
  \Omega$, on obtient un défaut de conservation numérique de
  la grandeur physique. Pour y remédier, on utilise une
  technique donnée dans l'article \cite{flotron2013b}  qui a pour but
  de conserver numériquement l'intégrale de la grandeur physique
  répondant à l'équations de convection-diffusiton discrétisée.
\end{remarque}

La section suivant traite de l'application de ce modèle numérique à
une cuve d'électrolyse d'aluminium industrielle.
