On répète ici, par soucis de clarté, le système d'équations aux
dérivées partielles qui correspondent au modèle de transport et
dissolution d'alumine en fonction de la température (\ref{eq:}). On a $\forall k =
1, 2, \dots, K$
\begin{align}
  & n_p^k(t, x, r) = 0,
  & 0\leq t < \tau^k,\label{eq:mdl-eq-1}\\
%
  & \frac{\partial n_p^k}{\partial t} + \parent{u(x) + w(r)}\cdot \nabla  n_p^k = 0,
  & \tau^k < t \leq \tau^k + \tlat,\label{eq:mdl-eq-2}\\
%
  & \frac{\partial n_p^k}{\partial t} + \parent{u(x) + w(r)}\cdot \nabla  n_p^k + \frac{\partial }{\partial r}\parent{f(r, \concentration, \temperature)n_p^k} = 0,
  & \tau^k + \tlat < t \leq \tend,\label{eq:mdl-eq-3}
\end{align}
et
\begin{align}
  & \frac{\partial \concentration}{\partial t} + u(x)\cdot\nabla \concentration - \div\parent{\cdiffusivity(x) \nabla \concentration} = q_1 + q_2,
  & \forall t\in (0, T),\label{eq:mdl-eq-4}\\
%
  & \frac{\partial \temperature}{\partial t} + u(x)\cdot\nabla\temperature - \div\parent{\electrolytetdiff(x)\nabla \temperature} = \frac{1}{\electrolytedensity\electrolytehc}\sum_{i = 1}^{3}p_i,
  & \forall t\in(0, T)\label{eq:mdl-eq-5}
\end{align}
dans $\Omega$.

Ces équations forment un système couplé pour les inconnues $n_p^k$,
$\concentration$ et $\temperature$. En effet, la densité de
particules $n_p$ dépend de la concentration $\concentration$ et de
la température $\temperature$ par l'intermédiaire de la vitesse de
dissolution $f$, tandis que la concentration et la température
dépendent de la densité de particule $n_p$ à travers leurs termes
sources respectifs $q_2$, $p_1$ et $p_2$.

En suivant l'approche adoptée dans \cite{Hofer2011}, on propose de
discrétiser les équations (\ref{eq:mdl-eq-1}) à (\ref{eq:mdl-eq-5})
par une méthode de splitting en temps de la façon suivante. Soit
l'entier $N$, le nombre de pas de temps et $\dt = T/N$ un pas de temps
uniforme. Soient $t_n = n\dt$, $n = 0, 1, \dots, N$, une subdivision
de l'intervalle de temps $[0, T]$. On note $n_{p,n}^k$ une
approximation de $n_p^k(t_n, ., .)$, $n_{p,n}$ une approximation de
$n_p(t_n, ., .)$, $\concentration_n$ une approximation de
$\concentration(t_n, .)$ et $\temperature_n$ une approximation de
$\temperature(t_n, .)$. Si $k$ et l'indice de l'injection d'une
population de particules, on définit $p^k$ et $q^k$ les plus grands
entiers tels que
\begin{equation}
  t_{p^k} < \tau^k \quad\text{et}\quad t_{q^k} < \tau^k + \tlat.
\end{equation}
En d'autres termes, $p_k$ est le dernier pas de temps qui précède
l'injection de la population $k$, et $q_k$ est le dernier pas de temps
qui précède le début de la dissolution de la population $k$. Bien
entendu, si $\tlat = 0$ alors $p_k = q_k$.

Étant donnés $n_{p,n}^k$ $\forall k=1, 2, \dots, K$,
$\concentration_n$ et $\temperature_n$, on pose
\begin{align}
  & n_{p,n+1}^k = 0, &&\text{si } 1\leq n + 1 < p^k + 1,\label{eq:splitting-np1-init}\\
  & n_{p,n+1}^k = S^k, &&\text{si } n + 1 = p^k + 1,\\
  & \frac{n_{p,n+1}^k - n_{p,n}^k}{\dt} + u\cdot\nabla n_{p,n+1}^k = 0, && \text{si } p^k + 1 < n+1 \leq q^k.\label{eq:splitting-np1-u}
\end{align}
dans $\Omega$ et pour tout $r > 0$. Puis, si $n$ est tel que $q^k < n
+ 1 \leq N$, on pose
\begin{align}
  &\displaystyle\frac{\bar n_{p,n+1}^k - n_{p,n}^k}{\dt} +
  u\cdot\nabla \bar n_{p,n+1}^k = 0,\label{eq:splitting-np2-u}\\
  &\displaystyle\frac{\bar{\bar n}_{p,n+1}^k - \bar n_{p,n+1}^k}{\dt} +
  w\cdot\nabla \bar{\bar n}_{p,n+1}^k = 0,\label{eq:splitting-np2-w}\\
    &\displaystyle\frac{n_{p,n+1}^k - \bar{\bar n}_{p,n+1}^k}{\dt} +
    \displaystyle\frac{\partial}{\partial r}\parent{f(r,
      \concentration_n, \temperature_n)n_{p,n+1}^k} =
    0\label{eq:splitting-dissolution}
\end{align}
dans $\Omega$ et pour tout $r > 0$. Et finalement
\begin{align}
&\frac{\concentration_{n+1} - \concentration_{n}}{\dt} + u\cdot\nabla
\concentration_{n+1} - \div\parent{\cdiffusivity \nabla \concentration_{n+1}} = q_1
+ q_{2,n+1},\label{eq:splitting-c} \\
&\frac{\temperature_{n+1} - \temperature_{n}}{\dt} + u\cdot\nabla
\temperature_{n+1} - \div\parent{\electrolytetdiff \nabla \temperature_{n+1}} =
\frac{1}{\electrolytedensity \electrolytehc}\parent{\sum_{i =
    1}^2p_{i,n+1} + p_3}\label{eq:splitting-t}
\end{align}
dans $\Omega$. On précise maintenant la forme des termes sources
discrétisés $q_{2,n+1}$, $p_{i,n+1}$, $i = 1,2$.

\paragraph{Discrétisation de la source d'alumine $q_{2}$}
Le terme source $q_{2}$ qui apparaît dans l'équation
(\ref{eq:mdl-eq-4}) correspond à la masse d'alumine qui est transférée
entre les particules qui se dissolvent et l'alumine dissoute par unité
de temps. Afin de permettre une conservation exacte de la masse d'alumine
par le schéma numérique entre les champs $n_p$ et $c$, nous
tirons parti du splitting en temps des équations (\ref{eq:mdl-eq-3})
et (\ref{eq:mdl-eq-4}) \cite{Hofer2011}. Plus précisement, grâce au
splitting en temps, les quantités $n_{p,n+1}^k$ et $\bar{\bar n}_{p,n+1}^k$, $k = 1, 2, \dots, K$
sont indépendentes de $c_{n+1}$. On pose alors
\begin{equation}\label{eq:q2}
  q_{2,n+1}(x) = -\frac{1}{\dt} \sum_{\substack{1\leq k\leq K\\ q^k < n+1}}
  \int_{0}^\infty
  \frac{\aluminadensity}{[\ce{Al2O3}]} \frac{4}{3}\pi r^3
  \parent{n_{p,n+1}^k(x) - \bar{\bar n}_{p,n+1}^k(x)} \intd{r}, \quad x\in\Omega.
\end{equation}
Ici, la somme porte sur toutes les populations de particules qui se
dissolvent à l'instant $t_n$, c'est-à-dire les populations $k$
telles que $t_{q^k} < t_n$.


\paragraph{Discrétisation des source de puissance thermique $p_1$ et
$p_2$} Le terme source de puissance thermique $p_1$ est discrétisé
en régularisant la masse de Dirac sur les intervalles $[t_{p^k},
  t_{p^k} + \dt]$, $k = 1, 2, \dots, K$. On pose
\begin{equation}\label{eq:p1-discret}
p_{1,n+1}(x) = -\parent{\tinit - \tinj}\sum_{k = 1}^K
\frac{1}{\dt}\kronecker_{n+1,p^k}\int_{\rplus} \aluminadensity\aluminahc
\frac{4}{3}\pi r^3S^k(x, r)\,\intd{r}
\end{equation}
où $\kronecker_{i,j}$ est le symbol de Kronecker définit par
\begin{equation}
  \kronecker_{i,j} = \left\{
  \begin{array}{ll}
    1&\quad\text{si } i = j,\\
    0&\quad\text{si } i\neq j.
  \end{array}\right.
\end{equation}
Le terme source $p_2$ qui correspond à la puissance thermique
nécessaire à la dissolution des particules est discrétisé en
utilisant $q_{2,n+1}$:
\begin{equation}\label{eq:p2-discret}
p_{2,n+1} = -[\cee{Al2O3}]\aluminadissolutionenthalpy q_{2,n+1}.
\end{equation}

L'équation (\ref{eq:splitting-dissolution}) est discrétisé selon $r$
à l'aide du schéma de caractéristiques présenté dans la section
\ref{sec:particle-population-dissolution}. Les équations
(\ref{eq:splitting-np1-u}), (\ref{eq:splitting-np2-u}),
(\ref{eq:splitting-np2-w}), (\ref{eq:splitting-c}) et
(\ref{eq:splitting-t}) sont des équations d'advection ou
d'advection-diffusion et sont discrétisées en espace selon la
méthode adoptée dans \cite{Hofer2011}. La discrétisation est basée
sur des éléments finis stabilisée par la méthode SUPG
\cite{Quarteroni2008}.

\paragraph{Propriétés de conservation du schéma de discrétisation}
Nous établissons maintenant le bilan de masse d'alumine dans les
champs $n_{p,n+1}^k$, $k = 1,\dots, K$ et $c_{n+1}$ à l'instant
$t_{n+1}$ dans le cas particulier ou la vitesse de sédimentation des
particules est nulles, \ie, lorsque $w(r)$ = 0. On note la masse
de particules d'alumine dans la population $k$ à l'instant $t_{n+1}$
\begin{equation}\label{eq:np-masse-def}
  N_{p,n+1}^k = \int_\Omega\int_0^\infty \aluminadensity
  n_{p,n+1}^k\frac{4}{3}\pi r^3\intd{r}\intd{x},
\end{equation}
la masse totale de particules d'alumine à l'instant $t_{n+1}$, $N_{p,n+1} =
\sum_{k = 1}^K N_{p,n+1}^k$, la masse d'alumine dissoute à l'instant $t_{n+1}$
\begin{equation}\label{eq:c-masse-def}
  C_{n+1} = [\cee{Al2O3}]\int_\Omega c_n\intd{x}
\end{equation}
et la masse totale d'alumine dans le bain $M_{n+1} = N_{p,n+1} +
C_n$. En vertu des équations (\ref{eq:splitting-np1-init}) à
(\ref{eq:splitting-dissolution}), l'accoissement de la masse d'alumine
de la population $k$ entre les instants $t_{n}$ et $t_{n + 1}$ s'écrit
\begin{equation}\label{eq:np-accr}
  N_{p,n+1}^k - N_{p,n}^k = \left\{
  \begin{array}{ll}
    0, & \text{si }n+1 \leq p^k, \\
    \displaystyle\int_\Omega\displaystyle\int_0^\infty \aluminadensity S^k\frac{4}{3}\pi
    r^3\intd{r}\intd{x}, & \text{si }n+1 = p^k + 1, \\
    0, & \text{si }p^k+1 < n+1 \leq q^k, \\
    \displaystyle\int_\Omega\displaystyle\int_0^\infty \aluminadensity \frac{4}{3}\pi
    r^3\parent{n_{p,n+1}^k - \bar{\bar n}_{p,n+1}}\intd{r}\intd{x}, &\text{si } q^k < n+1 \leq N.
  \end{array}
  \right.
\end{equation}
On obtient l'accoissement de la masse totale de particules entre les
instants $t_n$ et $t_{n+1}$ en sommant sur $k = 1$ à $K$:
\begin{eqnarray}
  N_{p,n+1} - N_{p,n}
  &=& \sum_{\mathclap{\substack{k\\ n+1 \leq p^k}}} N_{p,n+1}^k - N_{p,n}^k %
  + \sum_{\mathclap{\substack{k\\ n+1 = p^k + 1}}} N_{p,n+1}^k - N_{p,n}^k \nonumber\\
  && + \sum_{\mathclap{\substack{k\\ p^k + 1 < n+1 \leq q^k}}} N_{p,n+1}^k - N_{p,n}^k %
  + \sum_{\mathclap{\substack{k\\ q^k < n+1}}} N_{p,n+1}^k - N_{p,n}^k,\nonumber\\
  &=&\sum_{\mathclap{\substack{k\\ n+1 = p^k+1}}} \displaystyle\int_\Omega\displaystyle\int_0^\infty \aluminadensity S^k\frac{4}{3}\pi
  r^3\intd{r}\intd{x} \nonumber\\
  && + \sum_{\mathclap{\substack{k \\ q^k < n+1}}} \displaystyle\int_\Omega\displaystyle\int_0^\infty \aluminadensity \frac{4}{3}\pi
    r^3\parent{n_{p,n+1}^k - \bar{\bar n}_{p,n+1}^k}\intd{r}\intd{x}, \label{eq:np-masse-1}
\end{eqnarray}
puisque
\begin{equation*}
\sum_{\mathclap{\substack{k\\ n+1 \leq p^k}}} N_{p,n+1}^k - N_{p,n}^k = 0 \quad \text{et}\quad \sum_{\mathclap{\substack{k\\ p^k + 1 < n+1 \leq q^k}}} N_{p,n+1}^k - N_{p,n}^k = 0
\end{equation*}
en vertu de (\ref{eq:np-accr}). En intégrant les expression (\ref{eq:splitting-np2-u}),
(\ref{eq:splitting-np2-w}) sur $\Omega$, en utilisant le théorème
de la divergence et en se rappelant que $\div u = 0$ dans $\Omega$,
que $u\cdot\nu = 0$ sur $\partial \Omega$ et que $\frac{\partial
  n_{p,n+1}}{\partial \nu} = 0$ sur $\partial \Omega$, on obtient que
\begin{equation}\label{eq:np-masse-2}
  \int_\Omega \bar{\bar n}_{p,n+1}^k\intd{x} = \int_\Omega n_{p,n}\intd{x}.
\end{equation}
En remplaçant (\ref{eq:np-masse-2}) dans (\ref{eq:np-masse-1}),
l'accroissement de masse totale de particules s'écrit
\begin{eqnarray}\label{eq:np-masse-3}
  N_{p,n+1} - N_{p,n} &=&\sum_{\mathclap{\substack{k\\ n+1 = p^k+1}}} \displaystyle\int_\Omega\displaystyle\int_0^\infty \aluminadensity S^k\frac{4}{3}\pi
  r^3\intd{r}\intd{x} \nonumber\\
  && + \sum_{\mathclap{\substack{k \\ q^k < n+1}}} \displaystyle\int_\Omega\displaystyle\int_0^\infty \aluminadensity \frac{4}{3}\pi
    r^3\parent{n_{p,n+1}^k - n_{p,n}^k}\intd{r}\intd{x}.
\end{eqnarray}

On dérive maintenant une expression pour l'accroissement de la masse
d'alumine dissoute. En intégrant l'équation (\ref{eq:splitting-c}) où
le terme source $q_{2,n+1}$ est donné par l'expression (\ref{eq:q2}),
en utilisant le théorème de la divergence et en utilisant à nouveau le
fait que $\div u = 0$ dans $\Omega$, et que $u\cdot \nu = 0$ et
$\frac{\partial c_n}{\partial \nu} = 0$ sur $\partial \Omega$, on
obtient
\begin{eqnarray*}
[\cee{Al2O3}]\int_\Omega c_{n+1}\intd{x} - [\cee{Al2O3}] \int_\Omega
c_n\intd{x} &=& -\frac{\dt I [\cee{Al2O3}]}{6\faraday} - \\
&&\sum_{\mathclap{\substack{k \\ q^k < n+1}}}
\int_\Omega\int_0^\infty\aluminadensity \frac{4}{3}\pi
r^3 \parent{n_{p,n+1} - n_{p,n}}\intd{r}\intd{x},\nonumber\\
\end{eqnarray*}
et ainsi par (\ref{eq:c-masse-def}),
\begin{eqnarray}
  C_{n+1} - C_n = -\frac{\dt I [\cee{Al2O3}]}{6\faraday} -
\sum_{\mathclap{\substack{k \\ q^k < n+1}}}
\int_\Omega\int_0^\infty\aluminadensity \frac{4}{3}\pi
r^3 \parent{n_{p,n+1} - n_{p,n}}\intd{r}\intd{x}\label{eq:c-masse-1}
\end{eqnarray}
On obtient l'accroissement total de masse d'alumine dans le bain entre
$t_n$ et $t_{n+1}$ en sommant (\ref{eq:np-masse-3}) et
(\ref{eq:c-masse-1}). On obtient
\begin{equation}
  M_{n+1} - M_{n} = - \dt \frac{I[\cee{Al2O3}]}{6\faraday} +
  \sum_{\mathclap{\substack{k\\ n+1 = p^k + 1}}} \int_\Omega
  \aluminadensity S^k\frac{4}{3}\pi r^3 \intd{r}\intd{x}.
\end{equation}
Par récurrence sur $n$ on obtient
\begin{equation}\label{eq:alumina-mass-balance}
  M_{n+1} - M_{0} = - t_{n+1} \frac{I[\cee{Al2O3}]}{6\faraday} +
  \sum_{\mathclap{\substack{k\\  p^k + 1 \leq n+1}}} \int_\Omega
  \aluminadensity S^k\frac{4}{3}\pi r^3 \intd{r}\intd{x},
\end{equation}
c'est-à-dire que la masse totale d'alumine dans le bain dépend
uniquement du taux de consommation de l'alumine dissoute par
l'électrolyse (premier terme du membre de droite de
(\ref{eq:alumina-mass-balance})) et de la masse des doses
injectées antérieurement au temps $t_{n+1}$ (deuxième terme du membre
de droite de (\ref{eq:alumina-mass-balance})).

Nous concluons cette section en établissant un bilan de l'énergie
thermique dans le bain. En intégrant l'équation (\ref{eq:splitting-t})
sur $\Omega$ et en multipliant par $\dt\electrolytedensity\electrolytehc$
on a
\begin{multline}
\int_\Omega\electrolytedensity\electrolytehc\parent{\temperature_{n+1}
- \temperature_n}\intd{x}
+\dt\electrolytedensity\electrolytehc\parent{\int_\Omega
  u\cdot\nabla\temperature_{n+1}\intd{x} - \int_\Omega
  \div\parent{\electrolytetdiff\nabla\temperature_{n+1}}} \\
= \dt\sum_{i=1}^2\int_\Omega p_{x,n+1}\intd{x} + \dt\int_\Omega p_3\intd{x}.
\end{multline}
En utilisant le théorème de la divergence, et en utilisant le fait que
$\div u = 0$ dans $\Omega$, que $\frac{\partial
  \temperature_{n+1}}{\partial \nu} = 0$ et $u\cdot \nu = 0$ sur
$\partial \Omega$ il reste
\begin{equation}
  \int_\Omega\electrolytedensity\electrolytehc\parent{\temperature_{n+1} - \temperature_n}\intd{x} = \dt\sum_{i=1}^2\int_\Omega p_{i,n+1}\intd{x} + \dt\int_\Omega p_3\intd{x}.
\end{equation}
En remplaçant les expressions (\ref{eq:p1-discret}), (\ref{eq:p2-discret}) et (\ref{eq:p3}) pour le termes sources $p_{1,n+1}$, $p_{2,n+1}$ et $p_3$ on obtient
\begin{eqnarray*}
  \electrolytedensity\electrolytehc\int_\Omega\parent{\temperature_{n+1} - \temperature_n}\intd{x} &=& \frac{\dt}{ \conductivity}\int_\Omega j\cdot j\intd{x}\\
  && - \aluminahc\parent{\tinit - \tinj}\sum_{\mathclap{\substack{k\\ p^k + 1 = n+1}}}\int_\Omega\int_0^\infty \aluminadensity \frac{4}{3}\pi r^3 S^k\intd{r}\intd{x}\\
  && + {\aluminadissolutionenthalpy}\sum_{\mathclap{\substack{k\\ q^k < n+1}}} \int_\Omega\int_0^\infty \aluminadensity \frac{4}{3}\pi r^3\parent{n_{p,n+1}^k-n_{p,n}^k}\intd{r}\intd{x}.
\end{eqnarray*}
A nouveau par récurrence sur $n$ on obtient
\begin{eqnarray}
  \electrolytedensity\electrolytehc\int_\Omega\parent{\temperature_{n+1} - \temperature_0}\intd{x} &=& \frac{t_{n+1}}{\conductivity}\int_\Omega j\cdot j\intd{x}\nonumber\\
  && - \aluminahc\parent{\tinit - \tinj}\sum_{\mathclap{\substack{k\\ p^k + 1 \leq n+1}}}\int_\Omega\int_0^\infty \aluminadensity \frac{4}{3}\pi r^3 S^k\intd{r}\intd{x}\nonumber\\
  && + {\aluminadissolutionenthalpy}\sum_{m =
    0}^{n}\sum_{\mathclap{\substack{k\\ q^k < m + 1}}}
  \int_\Omega\int_0^\infty \aluminadensity \frac{4}{3}\pi
  r^3\parent{n_{p,m+1}^k-n_{p,m}^k}\intd{r}\intd{x},\nonumber\\ \label{eq:energy-mass-balance}
\end{eqnarray}
c'est-à-dire que l'énergie thermique du bain
$\electrolytedensity\electrolytehc\int_\Omega\temperature_{n}\intd{x}$
dépend uniquement de l'énergie thermique initiale, de l'intensité de
l'effet Joule, de la masse des doses d'alumine injectées
antérieurement à l'instant $t_n$ et de la masse de particule d'alumine
dissoute durant l'intervalle de temps $[0, t_n]$.

\begin{remarque}
  Les bilans de masse (\ref{eq:alumina-mass-balance}) et
  (\ref{eq:energy-mass-balance}) ne sont valable que si la vitesse de
  sédimentation des particules $w$ est strictement nulle, \ie, $w(r) =
  0$ $\forall r > 0$. Dans le cas contraire, il peut exister un flux
  de particules non nulle à travers le bord $\partial \Omega$ que l'on
  ne peut pas contrôler.
\end{remarque}

\begin{remarque}
  Les bilans de masse et d'énergie ci-dessus ont été dérivés à partir du
  modèle de transport et dissolution semi-discrétisé. Pour que ces
  bilans restent valables pour le problème discrétisé en temps et en
  espace, il est essentiel que la discrétisation en espace des équations
  d'advection-diffusion (\ref{eq:splitting-np1-u}),
  (\ref{eq:splitting-np2-u}), (\ref{eq:splitting-np2-w}),
  (\ref{eq:splitting-c}) (\ref{eq:splitting-t}) conservent exactement
  les intégrales de $n_{p,n}$, $\concentration_n$ et
  $\temperature_n$. Cette condition est garantie par le schéma de
  discrétisation proposé dans \cite{Hofer2011} que l'on utilise ici.
\end{remarque}

La section suivant traite de l'application de ce modèle numérique à
une cuve d'électrolyse d'aluminium industrielle.
