% introduction: dans cette section, on introduit le problème de Stefan
%  pour une particule isolée dans un bain électrolytique infini
\subsection*{Introduction}
Dans cette section nous détaillons et étudions numériquement les
phénomènes physiques qui motivent la nécessité de considérer un temps
de latence préalable au début de la dissolution des grains d'alumine
dans le bain électrolytique.


% description du système: particule et bain, symétrie sphérique,
%  caractérisation des matériaux.
\subsection{Un particule de bain solide}
On considère une sphère d'électrolyte, placée à l'origine du système
de coordonnées, de température initiale $\tinj<\tliq$. Cette sphère
est donc gelée, c'est-à-dire sous forme solide. Le reste de l'espace
est occupé par le bain électrolytique à l'état liquide, au repos, et
de température initiale $\tinit > \tliq$.

On s'intéresse à l'évolution de la température au cours du
temps dans le système formé par la sphère de bain gelé et le
bain liquide environnant. En particulier, la température de la
particule étant inférieure à la température du liquidus
$\tliq$, le bain au voisinage de la surface de la particule va
commencer par geler. Puis, la température du système s'équilibrant par
diffusion thermique, cette couche de bain va peu à peu fondre. On
cherche à déterminer l'évolution au cours du temps de l'épaisseur de
cette couche de bain gelée.

Soit $R_p$ le rayon de la particule placée à l'origine du
système de coordonnées, $\electrolytehc$ la chaleur spécifique du bain et
$\electrolytetdiff$ le coefficient de diffusion thermique du bain.

Soit $f_s(\temperature)$ la fraction solide du bain électrolytique. La
fraction solide est définie par:
\begin{equation}
  f_s(\temperature) = \left\{
  \begin{array}{ll}
    0           & \text{ si } \temperature > \tliq,\\
    \frac{1}{2} & \text{ si } \temperature = \tliq,\\
    1           & \text{ si } \temperature < \tliq.
  \end{array}
  \right.
\end{equation}
Le comportement de la fonction $f_s$ au voisinage de $\tliq$ est
représenté sur le graphique de gauche de la figure
\ref{fig:solid-fraction-enthalpy}.

\begin{figure}
  \begin{center}
    \input{../media/particles/solidfraction/solidfraction.tex}
    \input{../media/particles/enthalpy/enthalpy.tex}
    \caption{Gauche: fraction solide du bain $f_s$ en fonction de la
      température. Droite: Enthalpie du bain électrolytique en
      fonction de la température.}
    \label{fig:solid-fraction-enthalpy}
  \end{center}
\end{figure}

Les propriétés thermiques du bain électrolytique sont
caractérisées par la relation entre la température $\temperature$ et
l'enthalpie par unité de masse $\enthalpy$ qui s'exprime de la manière
suivante:
\begin{equation}
  \enthalpy(\temperature) = %
    \int_0^\temperature
      \electrolytedensity\electrolytehc\intd{s} %
    + \fusionenthalpy(1 - f_s(\temperature)),
\end{equation}
où $\electrolytedensity$ est la densité du bain, $\fusionenthalpy$ est
l'enthalpie par unité de masse libérée par la transition entre les
phases solide et liquide, et $f_s$ la fraction solide du bain en
fonction de la température. Dans la suite, on fera l'hypothèse que les
paramètres $\electrolytehc$ et $\electrolytetdiff$ sont des fonctions
de la température $\temperature$, mais constants dans chaque phase,
c'est-à-dire que
\begin{equation}
  \electrolytehc(\temperature) = \left\{
  \begin{array}{ll}
    \electrolyteshc, & \text{ si } \temperature < \tliq,\\
    \electrolytelhc, & \text{ si } \temperature \geq \tliq,
  \end{array}
  \right.
  \quad\text{et}\quad
  \electrolytetdiff(\temperature) = \left\{
  \begin{array}{ll}
    \electrolytestdiff, & \text{ si } \temperature < \tliq,\\
    \electrolyteltdiff, & \text{ si } \temperature \geq \tliq,
  \end{array}
  \right.
\end{equation}
où $\electrolyteshc$, $\electrolytelhc$, $\electrolytestdiff$,
$\electrolyteltdiff$ sont des réels positifs donnés.
Ce choix se justifie par le fait que ces valeurs
varient peu sur l'intervalle de température qui nous intéresse, et
on peut se contenter de considérer une valeur moyenne sur cet intervalle.

Le graphique de droite sur la figure \ref{fig:solid-fraction-enthalpy}
représente le comportement de la fonction $\enthalpy(\temperature)$ au
voisinage de la température de transition $\tliq$.

Bien que la fonction $\enthalpy$ présente une discontinuité en
$\temperature = \tliq$, elle est strictement monotone, et on peut
définir une fonction $\beta(\enthalpy)$ telle que:
\begin{equation}
\beta(\enthalpy(\temperature)) = \temperature
\end{equation}
pour tout $\temperature$ de la manière suivante:
\begin{equation}
  \beta(\enthalpy) = \left\{
  \begin{array}{ll}
    \frac{\enthalpy}{\electrolytehc\electrolytedensity} %
      & \enthalpy < \tliq \electrolytehc\electrolytedensity,\\
    \tliq %
      & \tliq \electrolytehc\electrolytedensity < \enthalpy < \tliq \electrolytehc\electrolytedensity + \fusionenthalpy,\\
    \frac{\enthalpy - \fusionenthalpy}{\electrolytehc\electrolytedensity} %
      & \tliq\electrolytehc\electrolytedensity + \fusionenthalpy < \enthalpy.\\
  \end{array}
  \right.
\end{equation}
La fonction $\beta$ est représentée sur la figure \ref{fig:beta}.
\begin{figure}
  \begin{center}
    \input{../media/particles/beta/beta.tex}
    \caption{Fonction $\beta(\enthalpy)$. On a noté $H_1 =
      \tliq\electrolyteshc\electrolytedensity$ et $H_2 =
      \tliq\electrolyteshc\electrolytedensity + \fusionenthalpy$.}
    \label{fig:beta}
  \end{center}
\end{figure}

% modèle mathématique: définitions des domaines, front de transition
%  de phase, problème de la chaleur dans chaque phase, condition de Stefan, formulation
%  forte, conditions aux limites.
On introduit maintenant le cadre nécessaire à l'écriture d'un modèle
mathématique qui décrive l'évolution thermique de la particule et du
bain. Dans ce travail, nous choisissons de formuler le problème de la
thermique dans le système formé par la particule de bain gelé et le
bain liquide environnant sous forme enthalpique, c'est-à-dire qu'on
cherche à déterminer conjointement l'évolution au cours du temps de la
température $\temperature$ et de la densité d'enthalpie
$\enthalpydensity$ dans tout l'espace.

Soit $\Omega$ un domaine occupé par du bain électrolytique, que l'on
choisit égal à un sous-ensemble de $\mathbb R$ ou $\mathbb R^3$ selon
le contexte. On note $\temperature(t, x)$ et $\enthalpydensity(t, x)$
la température et la densité d'enthalpie de l'électrolyte au point $x
\in \Omega$ et à l'instant $t > 0$.

Soient $\Omega_1(t)\subset \Omega$ et $\Omega_2(t)\subset \Omega$
des ouverts fonctions du temps $t$, définis de la manière suivante:
\begin{equation}
  \Omega_1(t) = \cparent{x\in\Omega \mid %
                         \temperature(t, x) < \tliq},\quad %
  \Omega_2(t) = \cparent{x\in\Omega \mid \temperature(t, x) %
                         \geq \tliq}.
\end{equation}
Bien entendu, $\overline\Omega_1(t)\cup \overline\Omega_2(t) =
\overline\Omega$ et $\Omega_1(t) \cap \Omega_2(t) = \emptyset$
$\forall t > 0$. Par construction, le domaine $\Omega_1(t)$ correspond
à l'électrolyte dans la phase solide, et $\Omega_2(t)$ à l'électrolyte
dans la phase liquide.

On suppose donnée une subdivision $\Gamma_\mathrm{N}$,
$\Gamma_\mathrm{D}\subset \partial \Omega$ telle que
$\Gamma_\mathrm{N}\cap \Gamma_\mathrm{D} = \emptyset$ et
$\Gamma_\mathrm{N}\cup \Gamma_\mathrm{D} = \partial \Omega$. Sur la
partie du bord $\Gamma_\mathrm{N}$ on spécifiera des conditions au
limites de Neumann, tandis que sur la partie du bord
$\Gamma_\mathrm{D}$ on spécifiera des conditions aux limites de
Dirichlet. Finalement, on note $\Gamma(t)\subset \Omega$ la frontière
commune de $\Omega_1(t)$ et $\Omega_2(t)$:
\begin{equation}
  \Gamma(t) = \overline \Omega_1(t)\cap\overline \Omega_2(t)\ \forall
  t > 0.
\end{equation}
L'ensemble $\Gamma(t)$ correspond à la région de transition entre les
phases solide et liquide de l'électrolyte.

La température de l'électrolyte $\temperature$ satisfait une équation
de la chaleur dans chaque phase:
\begin{align}
  &\electrolytedensity\electrolyteshc \frac{\partial
    \temperature}{\partial t} - \electrolytestdiff \Delta\temperature
  = 0, & \forall (t, x) \in [0,\infty)\times \Omega_1(t),\label{eq:heat-solid-phase}\\
    %
    &\electrolytedensity\electrolytelhc \frac{\partial
    \temperature}{\partial t} - \electrolyteltdiff \Delta\temperature
  = 0, & \forall (t, x) \in [0,\infty)\times \Omega_2(t).\label{eq:heat-liquid-phase}
\end{align}
On suppose données les fonctions
$\temperature_\mathrm{D}:\Gamma_\mathrm{D}\to\mathbb R$ et
$\temperature_\mathrm{N}:\Gamma_\mathrm{N}\to\mathbb R$. Les équation
(\ref{eq:heat-solid-phase}), (\ref{eq:heat-liquid-phase}) sont
complétées par les conditions aux limites:
\begin{align}
  &\temperature(t, x) = \tliq, %
  &\forall (t, x) \in [0,\infty)\times\Gamma(t),\label{eq:heat-transition}\\
    %
  &\temperature(t, x) = \temperature_\mathrm{D}(t, x), %
  &\forall (t, x) \in [0,\infty)\times\Gamma_\mathrm{D}(t),\label{eq:heat-dirichlet}\\
    %
  &\frac{\partial \temperature}{\partial n}(t, x) = \temperature_\mathrm{N}(t, x), %
  &\forall (t, x) \in [0,\infty)\times\Gamma_\mathrm{N}(t),\label{eq:heat-neumann}
\end{align}
ainsi qu'une condition initiale appropriée à $t = 0$ pour $\temperature$.

Finalement, on suppose que l'interface entre les phases liquide est
solide satisfait la condition de Stefan. On note $v_\Gamma(t, x)$ la
vitesse à l'instant $t$ d'une particule astreinte à la surface
$\Gamma(t)$ au point $x \in \Gamma(t)$. On note $[g]_s^l$ la valeur du
saut d'une quantité $g$ quelconque définie de part et d'autre de
l'interface $\Gamma$. On note $n: \Gamma(t)\to \mathbb R^3$ le vecteur
unité à la surface $\Gamma(t)$, dirigé vers la phase liquide, \ie,
$\Omega_2$. La condition de Stefan pour l'interface $\Gamma(t)$
s'écrit alors:
\begin{equation}
  \electrolytedensity \fusionenthalpy v_\Gamma = %
  - \left[\electrolytetdiff\nabla\temperature\right]_s^l\cdot n.%
  \label{eq:stefan-condition}
\end{equation}
La condition (\ref{eq:stefan-condition}) correspond à un bilan
d'énergie thermique au niveau de l'interface entre les phases.  Le
membre de gauche de la relation (\ref{eq:stefan-condition}) correspond
à la quantité d'énergie absorbée ou libérée par le déplacement du
front de solidification, tandis que le membre de droite correspond à
la somme des flux d'énergie thermique au niveau du front de
solidification.

Le lecteur intéressé trouvera une discussion détaillée des différentes
formulation des problèmes de Stefan et de leur analyse mathématique
dans l'ouvrage de J. Hill \cite{HillStefanProblems}, ainsi que celui
de L. I. Rubenstein \cite{Rubenstein1971}, par exemple.

\subsection*{Formulation faible du problème de Stefan}
On prétend sans démonstration que le problème de Stefan à deux phases
formé par les équations
(\ref{eq:heat-solid-phase}),(\ref{eq:heat-liquid-phase}) et les
conditions aux limites (\ref{eq:heat-transition}),
(\ref{eq:heat-dirichlet}) et (\ref{eq:heat-neumann}) peut s'écrire
pour la densité d'enthalpie $\enthalpydensity$ sous la forme:
\begin{align}
  & \frac{\partial \enthalpydensity}{\partial t} %
  - \div\parent{ \nabla \beta(\enthalpydensity)} = 0,%
  & \forall (t, x) \in [0,\infty)\times \Omega,\label{eq:heat-one-phase}\\
  &\beta(\enthalpydensity) = \temperature_\mathrm{D}, %
  & \forall (t, x) \in [0,\infty)\times \Gamma_\mathrm{D},\label{eq:heat-one-phase-dirichlet}\\
  &\frac{\partial \beta(\enthalpydensity)}{\partial n} = \temperature_\mathrm{N}, %
  & \forall (t, x) \in [0,\infty)\times \Gamma_\mathrm{N}.\label{eq:heat-one-phase-neumann}
\end{align}
Ici, l'équation (\ref{eq:heat-one-phase}) est à comprendre au sens
faible. Le lecteur intéressé par une preuve de cette équivalence peut
consulter le travail de J. Hill \cite{HillStefanProblems}.\footnote{Ou
plutôt le livre rouge, sur l'étagère en bas au fond à droite de la
bibliothèque de mathématique...}

Étant donné une fonction $\enthalpydensity$ qui satisfait les
équations
(\ref{eq:heat-one-phase})-(\ref{eq:heat-one-phase-neumann}), la
température est obtenue grâce à la fonction $\beta$, \ie,
$\temperature(t, x) = \beta(\enthalpydensity(t, x))$ $\forall (t, x)
\in [0,\infty)\times\Omega$.


% solution exacte: solution de Neumann si $\Omega = \rplus$.
\subsection*{Solution exacte}
Pour certains choix des données, le problème (\ref{eq:})-(\ref{eq:}),
admet une solution exacte. On décrit à présent l'une d'elles,
découverte en premier par J. Neumann. Cette solution à l'avantage de
correspondre à une situation physique que l'on peut facilement
interpréter.

On suppose maintenant que $\electrolytehc(\temperature) = 1$ et
$\electrolytetdiff(\temperature)=1$ $\forall\temperature \in \mathbb
R$, $\electrolytedensity = 1$, $\fusionenthalpy = 1$ et $\tliq =
0$. On s'intéresse à la température dans un matériau qui occupe la
demi-droite positive, c'est-à-dire que $\Omega = \rplus$.

% discrétisation: symétrie sphérique ou plane, discrétisation en temps,
%  discrétisation en espace.
\subsection*{Schéma de discrétisation en espace et en temps}

% validation numérique: convergence de l'erreur.
\subsection*{Validation numérique}

% application: formation et refonte de bain gele autour d'une particule.


Soit $\tinj\in\mathbb R$ la température à laquelle se trouve les
particules d'alumine au moment de leur injection dans la cuve, et
$\telectrolyte$ la température du bain électrolytique. La condition
initiale du système formé par la particule et le bain environnant est
définie par:
\begin{equation}
  \tinit(x) = \left\{
  \begin{array}{ll}
    \tinj & \text{ si } \norm{x} < R_p,\\
    \telectrolyte & \text{ si } \norm{x} \geq R_p,\\
  \end{array}
  \right.\quad \forall x\in\mathbb R^3.\label{eq:initial-temperature}
\end{equation}

La figure \ref{fig:particle-initial-temperature} représente la
température initiale au voisinage de la particule. En dehors de la
particule, la température du bain est suffisante pour que celui-ci
soit dans la phase liquide. Cependant, on suppose que le bain liquide
est au repos en tout temps.

\begin{figure}[h]
  \begin{center}
    \input{../media/particles/temperature/temperature.tex}
    \caption{Température initiale du système formé par la particule de
      bain gelé placée à l'origine du système de coordonnées, et du
      bain électrolytique environnant.}
    \label{fig:particle-initial-temperature}
  \end{center}
\end{figure}

Soit $\enthalpydensity(t, x)$ la densité d'enthalpie au point
$x\in\mathbb R^3$ et au temps $t > 0$ dans le système formé par la
particule de rayon $R_p$ placée à l'origine et le bain électrolytique
occupant le reste de l'espace. La densité d'enthalpie
$\enthalpydensity$ est solution du problème suivant:
\begin{align}
  &\frac{\partial \enthalpydensity}{\partial t}(t,x) -\div %
  \parent{\electrolytetdiff\nabla \temperature(t, x)} = 0,& \quad %
  \forall (t, x) \in [0,\infty)\times \mathbb R^3,\label{eq:stefan-1}\\
  &\temperature(t, x) = \beta(\enthalpydensity(t, x)), & \quad %
  \forall (t, x) \in [0,\infty)\times \mathbb R^3,\label{eq:stefan-2}\\
    &\temperature(0, x) = \tinit(x),& \quad \forall x\in \mathbb R^3.\label{eq:stefan-3}
\end{align}
Puisque la condition initiale (\ref{eq:initial-temperature}) présente
une symétrie sphérique, et que le système formé par les équations
(\ref{eq:stefan-1})-(\ref{eq:stefan-3}) est invariant par symétrie
sphérique, on cherche une solution à symétrie sphérique. On définit
les fonctions $\enthalpydensityr(t, r)$ et $\temperaturer(t, r)$, la
densité d'enthalpie et la température au temps $t$ et à distance $r$
de l'origine.

Les équations (\ref{eq:stefan-1})-(\ref{eq:stefan-3}) se réécrivent en
fonction de la coordonnée $r$ de la manière suivante:
\begin{align}
  &\frac{\partial\enthalpydensityr}{\partial t}(t, r) -
  \frac{1}{r^2}\frac{\partial}{\partial
    r}\parent{\electrolytetdiff r^2\frac{\partial \temperaturer}{\partial r}} = 0, &\quad
  \forall (t, r)\in [0, \infty)\times \rplus,\label{eq:stefan-r-1}\\
    &\temperaturer(t, r) = \beta(\enthalpydensityr(t, r)), &\quad
    \forall (t, r)\in [0, \infty)\times \rplus,\label{eq:stefan-r-2}\\
      &\temperaturer(0, r) = \tinit(r),&\quad \forall r \in \rplus.\label{eq:stefan-r-3}
\end{align}

On note $R_f$ la distance entre l'origine et la position du front de
solidification, définie par la relation:
\begin{equation}
\beta(\enthalpydensityr(t, R_f(t))) = \tliq.
\end{equation}
L'épaisseur de la couche de bain solidifiée $R_g$ est définie par la
relation:
\begin{equation}
R_g(t) = R_f(t) - R_p.
\end{equation}
On définit alors le temps de latence $\tlat(R_p) > 0$, le temps
nécessaire à ce que l'épaisseur de bain solidifié atteigne à nouveau
zéro:
\begin{equation}
  R_g(\tlat).
\end{equation}

Le problème formé par les équations
(\ref{eq:stefan-r-1})-(\ref{eq:stefan-r-3}) est un problème de
Stefan. Dans la section qui suit on décrit le schéma de discrétisation
utilisé, en suivant le travail de Y. Safa \cite{Safa2005}.


\subsection*{Discrétisation du problème de Stefan}
Soit
