% introduction: dans cette section, on introduit le problème de Stefan
%  pour une particule isolée dans un bain électrolytique infini
Dans cette section nous détaillons et étudions numériquement les
phénomènes physiques qui sont responsables du temps de latence
préalable au début de la dissolution des grains d'alumine dans le bain
électrolytique. Dans ce but, on considère un modèle de Stefan pour
traiter le problème de la transition de phase dans le bain
électrolytique environnant une région de bain solidifiée. Pour
approcher numériquement l'évolution de la région de transition de
phase autour de la particule de bain, on fera l'hypothèse d'une
symétrique sphérique de la solution.

% description du système: particule et bain, symétrie sphérique,
%  caractérisation des matériaux.
\subsection{Un particule de bain solide}
On considère une sphère d'électrolyte, placée à l'origine du système
de coordonnées, de température initiale $\tinj<\tliq$. Cette sphère
est donc gelée, c'est-à-dire sous forme solide. Le reste de l'espace
est occupé par le bain électrolytique à l'état liquide, au repos, et
de température initiale $\tinit > \tliq$.

On s'intéresse à l'évolution de la température au cours du
temps dans le système formé par la sphère de bain gelé et le
bain liquide environnant. En particulier, la température de la
particule étant inférieure à la température du liquidus
$\tliq$, le bain au voisinage de la surface de la particule va
commencer par geler. Puis, la température du système s'équilibrant par
diffusion thermique, cette couche de bain va peu à peu fondre. On
cherche à déterminer l'évolution au cours du temps de l'épaisseur de
cette couche de bain gelée.

Soit $R_p$ le rayon de la particule placée à l'origine du
système de coordonnées, $\electrolytehc$ la chaleur spécifique du bain et
$\electrolytetdiff$ le coefficient de diffusion thermique du bain.

Soit $f_s(\temperature)$ la fraction solide du bain électrolytique. La
fraction solide est définie par:
\begin{equation}
  f_s(\temperature) = \left\{
  \begin{array}{ll}
    0           & \text{ si } \temperature \geq \tliq,\\
    1           & \text{ si } \temperature < \tliq.
  \end{array}
  \right.
\end{equation}
Le comportement de la fonction $f_s$ au voisinage de $\tliq$ est
représenté sur le graphique de gauche de la figure
\ref{fig:solid-fraction-enthalpy}.

\begin{figure}
  \begin{center}
    \input{../media/particles/solidfraction/solidfraction.tex}
    \input{../media/particles/enthalpy/enthalpy.tex}
    \caption{Gauche: fraction solide du bain $f_s$ en fonction de la
      température. Droite: Enthalpie du bain électrolytique en
      fonction de la température.}
    \label{fig:solid-fraction-enthalpy}
  \end{center}
\end{figure}

Les propriétés thermiques du bain électrolytique sont
caractérisées par la relation entre la température $\temperature$ et
l'enthalpie par unité de masse $\enthalpy$ qui s'exprime de la manière
suivante:
\begin{equation}
  \enthalpy(\temperature) = %
    \int_0^\temperature
      \electrolytedensity\electrolytehc\intd{s} %
    + \fusionenthalpy(1 - f_s(\temperature)),
\end{equation}
où $\electrolytedensity$ est la densité du bain, $\fusionenthalpy$ est
l'enthalpie par unité de masse libérée par la transition entre les
phases solide et liquide, et $f_s$ la fraction solide du bain en
fonction de la température. Dans la suite, on fera l'hypothèse que les
paramètres $\electrolytehc$ et $\electrolytetdiff$ sont des fonctions
de la température $\temperature$, mais constants dans chaque phase,
c'est-à-dire que
\begin{equation}
  \electrolytehc(\temperature) = \left\{
  \begin{array}{ll}
    \electrolyteshc, & \text{ si } \temperature < \tliq,\\
    \electrolytelhc, & \text{ si } \temperature \geq \tliq,
  \end{array}
  \right.
  \quad\text{et}\quad
  \electrolytetdiff(\temperature) = \left\{
  \begin{array}{ll}
    \electrolytestdiff, & \text{ si } \temperature < \tliq,\\
    \electrolyteltdiff, & \text{ si } \temperature \geq \tliq,
  \end{array}
  \right.
\end{equation}
où $\electrolyteshc$, $\electrolytelhc$, $\electrolytestdiff$,
$\electrolyteltdiff$ sont des réels positifs donnés.
Ce choix se justifie par le fait que ces valeurs
varient peu sur l'intervalle de température qui nous intéresse, et
on peut se contenter de considérer une valeur moyenne sur cet intervalle.

Le graphique de droite sur la figure \ref{fig:solid-fraction-enthalpy}
représente le comportement de la fonction $\enthalpy(\temperature)$ au
voisinage de la température de transition $\tliq$. La densité de
l'électrolyte $\electrolytedensity$ est supposée constante pour tout
$\temperature$. En particulier, on suppose que la transition entre les
phases solide et liquide a lieu à volume constant.

Bien que la fonction $\enthalpy$ présente une discontinuité en
$\temperature = \tliq$, elle est strictement monotone, et on peut
définir une fonction $\beta(\enthalpy)$ telle que:
\begin{equation}
\beta(\enthalpy(\temperature)) = \temperature
\end{equation}
pour tout $\temperature$ de la manière suivante:
\begin{equation}
  \beta(\enthalpy) = \left\{
  \begin{array}{ll}
    \frac{\enthalpy}{\electrolytehc\electrolytedensity} %
      & \enthalpy < \tliq \electrolytehc\electrolytedensity,\\
    \tliq %
      & \tliq \electrolytehc\electrolytedensity < \enthalpy < \tliq \electrolytehc\electrolytedensity + \fusionenthalpy,\\
    \frac{\enthalpy - \fusionenthalpy}{\electrolytehc\electrolytedensity} %
      & \tliq\electrolytehc\electrolytedensity + \fusionenthalpy < \enthalpy.\\
  \end{array}
  \right.
\end{equation}
La fonction $\beta$ est représentée sur la figure \ref{fig:beta}. Pour
les besoins de la représentation, on a choisi ici $\tliq = 1218\si{\kelvin}$.
\begin{figure}
  \begin{center}
    \input{../media/particles/beta/beta.tex}
    \caption{Fonction $\beta(\enthalpy)$. On a noté $H_1 =
      \tliq\electrolyteshc\electrolytedensity$ et $H_2 =
      \tliq\electrolyteshc\electrolytedensity + \fusionenthalpy$, et
      $\tliq = 1218\si{\kelvin}$.}
    \label{fig:beta}
  \end{center}
\end{figure}

% modèle mathématique: définitions des domaines, front de transition
%  de phase, problème de la chaleur dans chaque phase, condition de Stefan, formulation
%  forte, conditions aux limites.
On introduit maintenant le cadre nécessaire à l'écriture d'un modèle
mathématique qui décrive l'évolution thermique de la particule et du
bain. Dans ce travail, nous choisissons de formuler le problème de la
thermique dans le système formé par la particule de bain gelé et le
bain liquide environnant sous forme enthalpique, c'est-à-dire qu'on
cherche à déterminer conjointement l'évolution au cours du temps de la
température $\temperature$ et de la densité d'enthalpie
$\enthalpydensity$ dans tout l'espace.

Soit $\Omega$ un domaine occupé par du bain électrolytique, que l'on
choisit égal à un sous-ensemble de $\mathbb R$ ou $\mathbb R^3$ selon
le contexte. On note $\temperature(t, x)$ et $\enthalpydensity(t, x)$
la température et la densité d'enthalpie de l'électrolyte au point $x
\in \Omega$ et à l'instant $t > 0$.

Soient $\Omega_1(t)\subset \Omega$ et $\Omega_2(t)\subset \Omega$
des ouverts fonctions du temps $t$, définis de la manière suivante:
\begin{equation}\label{eq:phase-domains}
  \Omega_1(t) = \cparent{x\in\Omega \mid %
                         \temperature(t, x) < \tliq},\quad %
  \Omega_2(t) = \cparent{x\in\Omega \mid \temperature(t, x) %
                         \geq \tliq}.
\end{equation}
Bien entendu, $\overline\Omega_1(t)\cup \overline\Omega_2(t) =
\overline\Omega$ et $\Omega_1(t) \cap \Omega_2(t) = \emptyset$
$\forall t > 0$. Par construction, le domaine $\Omega_1(t)$ correspond
à l'électrolyte dans la phase solide, et $\Omega_2(t)$ à l'électrolyte
dans la phase liquide.

On suppose donnée une subdivision $\Gamma_\mathrm{N}$,
$\Gamma_\mathrm{D}\subset \partial \Omega$ telle que
$\Gamma_\mathrm{N}\cap \Gamma_\mathrm{D} = \emptyset$ et
$\Gamma_\mathrm{N}\cup \Gamma_\mathrm{D} = \partial \Omega$. Sur la
partie du bord $\Gamma_\mathrm{N}$ on spécifiera des conditions au
limites de Neumann, tandis que sur la partie du bord
$\Gamma_\mathrm{D}$ on spécifiera des conditions aux limites de
Dirichlet. Finalement, on note $\Gamma(t)\subset \Omega$ la frontière
commune de $\Omega_1(t)$ et $\Omega_2(t)$:
\begin{equation}\label{eq:phase-frontier}
  \Gamma(t) = \overline \Omega_1(t)\cap\overline \Omega_2(t)\ \forall
  t > 0.
\end{equation}
L'ensemble $\Gamma(t)$ correspond à la région de transition entre les
phases solide et liquide de l'électrolyte.

La température de l'électrolyte $\temperature$ satisfait une équation
de la chaleur dans chaque phase:
\begin{align}
  &\electrolytedensity\electrolyteshc \frac{\partial
    \temperature}{\partial t} - \electrolytestdiff \Delta\temperature
  = 0, & \forall (t, x) \in [0,\infty)\times \Omega_1(t),\label{eq:heat-solid-phase}\\
    %
    &\electrolytedensity\electrolytelhc \frac{\partial
    \temperature}{\partial t} - \electrolyteltdiff \Delta\temperature
  = 0, & \forall (t, x) \in [0,\infty)\times \Omega_2(t).\label{eq:heat-liquid-phase}
\end{align}
On suppose données les fonctions
$\temperature_\mathrm{D}:\Gamma_\mathrm{D}\to\mathbb R$ et
$\temperature_\mathrm{N}:\Gamma_\mathrm{N}\to\mathbb R$. Les équation
(\ref{eq:heat-solid-phase}), (\ref{eq:heat-liquid-phase}) sont
complétées par les conditions aux limites:
\begin{align}
  &\temperature(t, x) = \tliq, %
  &\forall (t, x) \in [0,\infty)\times\Gamma(t),\label{eq:heat-transition}\\
    %
  &\temperature(t, x) = \temperature_\mathrm{D}(x), %
  &\forall (t, x) \in [0,\infty)\times\Gamma_\mathrm{D}(t),\label{eq:heat-dirichlet}\\
    %
  &\frac{\partial \temperature}{\partial \nu}(t, x) = \temperature_\mathrm{N}(x), %
  &\forall (t, x) \in [0,\infty)\times\Gamma_\mathrm{N}(t),\label{eq:heat-neumann}
\end{align}
ainsi qu'une condition initiale appropriée à $t = 0$ pour $\temperature$.

Finalement, on suppose que l'interface entre les phases liquide est
solide satisfait la condition de Stefan. On note $v_\Gamma(t, x)$ la
vitesse à l'instant $t$ d'une particule astreinte à la surface
$\Gamma(t)$ au point $x \in \Gamma(t)$. On note $[g]_s^l$ la valeur du
saut d'une quantité $g$ quelconque définie de part et d'autre de
l'interface $\Gamma$. On note $\nu: \Gamma(t)\to \mathbb R^3$ le vecteur
unité à la surface $\Gamma(t)$, dirigé vers la phase liquide, \ie,
$\Omega_2$. La condition de Stefan pour l'interface $\Gamma(t)$
s'écrit alors:
\begin{equation}
  \electrolytedensity \fusionenthalpy v_\Gamma = %
  - \left[\electrolytetdiff\nabla\temperature\right]_s^l\cdot \nu.%
  \label{eq:stefan-condition}
\end{equation}
La condition (\ref{eq:stefan-condition}) correspond à un bilan
d'énergie thermique au niveau de l'interface entre les phases.  Le
membre de gauche de la relation (\ref{eq:stefan-condition}) correspond
à la quantité d'énergie absorbée ou libérée par le déplacement du
front de solidification, tandis que le membre de droite correspond à
la somme des flux d'énergie thermique au niveau du front de
solidification.

Le problème de Stefan classique consiste à chercher une fonction
$\temperature:[0,\infty)\times\Omega\to\mathbb R$ et deux
sous-domaines $\Omega_1,\Omega_2$ qui satisfasse les équations
(\ref{eq:phase-domains}) et
(\ref{eq:heat-solid-phase})-(\ref{eq:stefan-condition}).

Le lecteur intéressé trouvera une discussion détaillée des différentes
formulations des problèmes de Stefan et de leurs analyses mathématiques
dans l'ouvrage de J. Hill \cite{HillStefanProblems}, ainsi que celui
de L. I. Rubenstein \cite{Rubenstein1971}, par exemple.

\subsection*{Formulation faible du problème de Stefan}
On prétend sans démonstration que le problème de Stefan à deux phases
formé par les équations
(\ref{eq:heat-solid-phase}),(\ref{eq:heat-liquid-phase}) et les
conditions aux limites (\ref{eq:heat-transition}),
(\ref{eq:heat-dirichlet}) et (\ref{eq:heat-neumann}) peut s'écrire
pour la densité d'enthalpie $\enthalpydensity$ sous la forme:
\begin{align}
  & \frac{\partial \enthalpydensity}{\partial t} %
  - \div\parent{ \nabla \beta(\enthalpydensity)} = 0,%
  & \forall (t, x) \in [0,\infty)\times \Omega,\label{eq:heat-one-phase}\\
  & \beta(\enthalpydensity) = \temperature_\mathrm{D}, %
  & \forall (t, x) \in [0,\infty)\times \Gamma_\mathrm{D},\label{eq:heat-one-phase-dirichlet}\\
  & \frac{\partial \beta(\enthalpydensity)}{\partial \nu} = \temperature_\mathrm{N}, %
  & \forall (t, x) \in [0,\infty)\times \Gamma_\mathrm{N}.\label{eq:heat-one-phase-neumann}
\end{align}
Ici, l'équation (\ref{eq:heat-one-phase}) et les conditions aux
limites (\ref{eq:heat-one-phase-dirichlet}),
(\ref{eq:heat-one-phase-neumann}) sont à comprendre au sens faible. Le
lecteur intéressé par une preuve de cette équivalence peut consulter
le travail de J. Hill \cite{HillStefanProblems}.\footnote{Ou plutôt le
livre rouge, sur l'étagère en bas au fond à droite de la
bibliothèque de mathématique...}

Étant donné une fonction $\enthalpydensity$ qui satisfait les
équations (\ref{eq:heat-one-phase})-(\ref{eq:heat-one-phase-neumann}),
la température $\temperature$ est définie par la relation
$\temperature(t, x) = \beta(\enthalpydensity(t, x))$ $\forall (t, x)
\in [0,\infty)\times\Omega$. Finalement les domaines $\Omega_1(t)$,
$\Omega_2(t)$ occupés par les phases solide et liquides et la région
de transition de phase $\Gamma(t)$ sont définis par les relations
(\ref{eq:phase-domains}) et (\ref{eq:phase-frontier}).

Cette formulation, qu'on qualifie d'enthalpique, présente l'avantage
de ne pas devoir résoudre directement des domaines occupés par chacune
des phases. La région de transition $\Gamma$ est obtenue par un
post-processing de la densité d'enthalpie locale. Cet aspect est
déterminant pour le choix du schéma numérique utilisé pour approcher
la solution du problème de Stefan.

% discrétisation: symétrie sphérique ou plane, discrétisation en temps,
%  discrétisation en espace.
\subsection*{Schéma de discrétisation en temps}
Pour obtenir un schéma de discrétisation en temps du problème de
Stefan, on part de la formulation enthalpique
(\ref{eq:heat-one-phase})-(\ref{eq:heat-one-phase-neumann}), et on
suit le travail de M. Paolini et al. \cite{Paolini1988}.

Soit un temps final $T > 0$, $N\in \mathbb \nstar$ le nombre de pas de
temps. Soit $\tau = T / N$ le pas de temps et $t^n = \tau n$. On note
$L_\beta$ la constante de Lipschitz de la fonction $\beta$:
\begin{equation}
  L_\beta = \sup_{\enthalpy}\abs{\beta(\enthalpy)}.
\end{equation}
Soit $\mu$ un paramètre de relaxation fixé, tel que $0 < \mu \leq
1/L_\beta$.

On note $\temperature^n(x)$ et $\enthalpydensity^n(x)$ les approximation
de $\temperature(t^n, x)$ et $\enthalpydensity(t^n, x)\ \forall x\in
\Omega$.  On suppose donnée la densité d'enthalpie initiale $h^0$, et
on pose $\temperature^0 = \beta(h^0)$. Pour $0 < n \leq N$, on résout
successivement les équations:
\begin{align}
  &\temperature^{n+1} - \frac{\tau}{\mu} \Delta \temperature^{n+1} = %
  \beta\parent{\enthalpydensity^n}, & \text{dans } \Omega,\label{eq:chernoff-heat}\\
  &\temperature^{n+1}(x) = \temperature_{\mathrm D}(t^{n+1}, x), &
  \forall x\in%
  \Gamma_\mathrm{D},\label{eq:chernoff-dirichlet}\\
  &\frac{\partial\temperature^{n+1}}{\partial \nu}(x) = %
  \temperature_\mathrm{N}(t^{n+1}, x), %
  & \forall x \in \Gamma_\mathrm{N},\label{eq:chernoff-neumann}
\end{align}
avec la correction pour la densité d'enthalpie à chaque pas de temps:
\begin{equation}\label{eq:chernoff-update}
\enthalpydensity^{n+1} = \enthalpydensity^{n} +
\mu\parent{\temperature^{n+1} - \beta(\enthalpydensity^{n})}, \quad
\text{dans } \Omega.
\end{equation}

Ce schéma numérique basé sur la formule de Chernoff
(\ref{eq:chernoff-update}) a été proposé en premier par M. Paolini
\cite{Paolini1988}. On décrit maintenant la discrétisation en
espace des équations (\ref{eq:chernoff-heat})-(\ref{eq:chernoff-neumann}).


\subsection*{Formulation du schéma de Chernoff en coordonnées sphériques}
Soit $\Phi:[0,\infty)\times[0,2\pi)\times[0,\pi)\to\mathbb R^3$ le
changement de variable de coordonnées sphériques en coordonnées
cartésiennes défini par:
\begin{equation}
  \Phi(r, \psi, \phi) = \begin{pmatrix}
    r\cos\psi\sin\phi\\
    r\sin\psi\sin\phi\\
    r\cos\phi
  \end{pmatrix}.
\end{equation}
Le jacobien de $\Phi$ est donné par:
\begin{equation}\label{eq:spherical-jacobian}
  \det J_\Phi(r, \psi,\phi) = -r^2\sin\phi.
\end{equation}
Pour toute fonction $f:\mathbb R^3\to\mathbb R$, on note la fonction
associée en coordonnées sphérique $\tilde
f:[0,\infty)\times[0,2\pi)\times[0,\pi)\to\mathbb R$ définie par:
\begin{equation}
  \tilde f(r, \psi, \phi) = \parent{f \circ \Phi}(r, \psi, \phi).
\end{equation}

On dit qu'une fonction $f:\mathbb R^3\to\mathbb R$ est à symétrie
sphérique si la fonction $\tilde f$ ne dépend que de la coordonnée
$r$, c'est-à-dire que
\begin{equation}\label{eq:spherical-symmetry}
  \frac{\partial \tilde f}{\partial \psi} = 0 %
  \quad \text{et} \quad %
  \frac{\partial \tilde f}{\partial \phi} = 0.
\end{equation}
Par abus de notation, si une fonction possède une symétrie
sphérique on omettra les coordonnées $\psi$ et $\phi$ et on
l'identifiera à une fonction définie sur le domaine $\rplus$.

Enfin, on rappelle que le gradient $\nabla f$ en coordonnées
sphérique s'écrit:
\begin{equation}\label{eq:spherical-gradient}
  \tilde\nabla \tilde f = %
  \frac{\partial \tilde f}{\partial r} \hat e_r %
  + \frac{1}{r}\frac{\partial \tilde f}{\partial \psi}\hat e_\psi %
  + \frac{1}{r\sin\psi}\frac{\partial \tilde f}{\partial \phi} \hat e_\phi,
\end{equation}
où les vecteurs orthonormés $\hat e_r$, $\hat e_\psi$ et $\hat
e_\phi$ forment la base standard associées au système de
coordonnées sphérique.

On se donne un réel $\rmax > 0$ et le domaine $\Omega = \mathcal B(0,
\rmax) \subset \mathbb R^3$ la boule ouverte de centre $0$ de rayon
$\rmax$. On suppose maintenant que les données du problème de Stefan
(\ref{eq:heat-one-phase})-(\ref{eq:heat-one-phase-neumann}), \ie, les
fonctions $\temperature_\mathrm{D}$, $\temperature_\mathrm{N}$ ainsi
que la condition initiale $\tinit$ pour la température, sont à
symétrie sphérique:
\begin{align}
  \frac{\partial \tilde \temperature_{\mathrm{D}}}{\partial \psi} = \frac{\partial \tilde \temperature_{\mathrm{D}}}{\partial \phi} = 0,\quad
  \frac{\partial \tilde \temperature_{\mathrm{N}}}{\partial \psi} = \frac{\partial \tilde \temperature_{\mathrm{N}}}{\partial \phi} = 0,\quad
  \frac{\partial \tinitr}{\partial \psi} = \frac{\partial \tinitr}{\partial \phi} = 0.
\end{align}

La subdivision du bord $\Gamma_\mathrm{D}$, $\Gamma_\mathrm{N}$ doit
également satisfaire la condition de symétrie. Par conséquent, on a
soit $\Gamma_\mathrm{D} = \partial\mathcal B(0,\rmax)$ et
$\Gamma_\mathrm{N} = \emptyset$, soit $\Gamma_\mathrm{D} = \emptyset$
et $\Gamma_\mathrm{N} = \partial\mathcal B(0,\rmax)$.

Sous ces hypothèses, on cherche une solution au problème de Stefan
(\ref{eq:heat-one-phase})-(\ref{eq:heat-one-phase-neumann}) en
utilisant le schéma de Chernoff
(\ref{eq:chernoff-heat})-(\ref{eq:chernoff-update}) qui soit également
à symétrie sphérique.

La formulation faible du système d'équations
(\ref{eq:chernoff-heat})-(\ref{eq:chernoff-neumann}) consiste $\forall
n\leq N$ à chercher une fonction $\temperature^{n+1}\in H_0^1(\Omega)$
à symétrie sphérique telle que la relation:
\begin{equation}\label{eq:weak-form-cart-coord}
  \int_\Omega \temperature^{n+1} v\,\intd{x} + \frac{\tau}{\mu} %
  \int_\Omega \nabla \temperature^{n+1} \nabla v\,\intd{x} - %
  \int_{\Gamma_\mathrm{N}} \temperature^{n+1}_\mathrm{N} v\,\intd{\sigma} %
  = \int_\Omega \beta(\enthalpydensity^n)v\,\intd{x}
\end{equation}
soit vérifiée pour toute fonction test $v\in H_0^1(\Omega)$ à symétrie
sphérique.

On réecrit l'expression (\ref{eq:weak-form-cart-coord}) en formulant
les intégrales en coordonnées sphériques. En utilisant
(\ref{eq:spherical-jacobian}), (\ref{eq:spherical-symmetry}),
(\ref{eq:spherical-gradient}) et en intégrant par rapport aux
coordonnées $\psi$ et $\phi$ on obtient:
\begin{multline}\label{eq:weak-form-spher-coord}
  4\pi\int_0^{\rmax} \tilde \temperature^{n+1}\tilde v r^2\,\intd{r}
  + \frac{4\pi\tau}{\mu}\int_0^{\rmax} \frac{\partial %
    \tilde \temperature^{n+1}}{\partial r} \frac{\partial \tilde v}{\partial %
    r}r^2\,\intd{r} \\
  - \frac{\abs{\Gamma_N}}{\rmax^2}\tilde\temperature^{n+1}_\mathrm{N}(\rmax)\tilde v(\rmax) %
  = 4\pi\int_0^{\rmax} \beta(\tilde \enthalpydensity^n)\tilde v r^2\,\intd{r},
\end{multline}
où on a noté $\abs{\Gamma_\mathrm{N}}$ la mesure de la partie du bord
$\Gamma_\mathrm{N}$. Lorsque $\Gamma_\mathrm{N} = \partial
\mathcal B(0, \rmax)$, on a $\abs{\Gamma_\mathrm{N}} = 4\pi\rmax^2$,
et, bien entendu, lorsque $\Gamma_\mathrm{N} = \emptyset$ on a
$\abs{\Gamma_\mathrm{N}} = 0$.


\subsection*{Discrétisation en espace}
On donne maintenant la discrétisation du problème faible
(\ref{eq:weak-form-spher-coord}) par une méthode de Galerkin. Soit un
réel $\delta$ tel que $0 < \delta < \rmax$, et une subdivision
$\mathcal T_\delta$ de l'intervalle $[0, \rmax]$, telle que pour tout
élément $K\in\mathcal T_\delta$ son diamètre $\diam(K)$ soit borné par
$\delta$.

On définit l'espace éléments finis
\begin{equation}
V_\delta = \cparent{v \in C^0(0, \rmax)\mid v\vert_K \in \mathbb
  P_1(K)\ \forall K\in\mathcal T_h},
\end{equation}
où $\mathbb P_1(K)$ est l'espace des polynômes de degré 1 sur
l'intervalle $K$.

Pour tout $n \leq N$, on note $\tilde \temperature_\delta^{n+1} \in
V_\delta$ une approximation de la fonction $\tilde \temperature_\delta^{n+1}$.
L'approximation de Galerkin du problème faible (\ref{eq:weak-form-spher-coord})
consiste à chercher une fonction $\tilde \temperature_\delta^{n+1} \in
V_\delta$ telle que la relation
\begin{multline}\label{eq:galerkin-weak-form-spher-coord}
  4\pi\int_0^{\rmax} \tilde \temperature_\delta^{n+1}\tilde v_\delta r^2\,\intd{r}
  + \frac{4\pi\tau}{\mu}\int_0^{\rmax} \frac{\partial %
    \tilde \temperature_\delta^{n+1}}{\partial r} \frac{\partial \tilde v_\delta}{\partial %
    r}r^2\,\intd{r} \\
  - \frac{\abs{\Gamma_N}}{\rmax^2}\tilde\temperature^{n+1}_\mathrm{N}(\rmax)\tilde v_\delta(\rmax) %
  = 4\pi\int_0^{\rmax} \beta(\tilde \enthalpydensity_\delta^n)\tilde v_\delta r^2\,\intd{r},
\end{multline}
soit vérifiée pour tout $v_\delta \in V_\delta$. Les intégrales qui
interviennent dans l'expression
(\ref{eq:galerkin-weak-form-spher-coord}) sont approchées
numériquement par une formule de quadrature de Gauss à 3 points.

\subsection*{Solution exacte}
Un certain nombre de solutions exactes au problème de Stefan
classique sont connues. On décrit à présent l'une d'elles, découverte en
premier par J. Neumann. Cette solution a l'avantage de correspondre à
une situation physique que l'on peut facilement interpréter.

On considère le problème de Stefan classique formé par les équations
(\ref{eq:heat-solid-phase})-(\ref{eq:heat-liquid-phase}), et on fixe
les données de la manière suivante. On fixe
$\electrolytehc(\temperature) = 1$ et
$\electrolytetdiff(\temperature)=1$ $\forall\temperature \in \mathbb
R$, $\electrolytedensity = 1$, $\fusionenthalpy = 1$ et $\tliq =
0$. On s'intéresse à la température dans un matériau qui occupe la
demi-droite positive, c'est-à-dire que $\Omega = \rplus$.

On choisit la condition initiale suivante pour la température
$\temperature$:
\begin{equation}
  \temperature(0, x) = 0, \quad \forall x\in\rplus,
\end{equation}
et la condition aux limites de Dirichlet sur le bord $\Gamma_\mathrm{D} =
\cparent{0}$:
\begin{equation}
  \temperature(t, 0) = -1, \quad \forall t > 0.
\end{equation}

On note $X(t) > 0$ la position du front de solidification à l'instant
$t$. Dans ces conditions, on peut montrer par un calcul algébrique que
la solution des équations
(\ref{eq:heat-solid-phase})-(\ref{eq:stefan-condition}) s'écrit:
\begin{align}
  & X(t) = \sqrt{2\gamma t},& \forall t > 0,\label{eq:neumann-sol-frontier}\\
  & \temperature(t, x) = -\gamma\int_{x / \sqrt{2\gamma
      t}}^1\exp\parent{\frac{\gamma(1 - \xi^2)}{2}}\, \intd{\xi},%
  & \forall (t, x) \in [0,\infty)\times[0,X(t)),\label{eq:neumann-sol-solid-phase}\\
    & \temperature(t, x) = 0,
    & \forall (t, x) \in [0,\infty)\times[X(t), \infty),\label{eq:neumann-sol-liquid-phase}
\end{align}
où on a noté $\gamma$ la constante réelle définie comme la solution de
l'équation transcendante:
\begin{equation}
  \gamma \int_0^1\exp\parent{\frac{\gamma(1 - \xi)^2}{2}}\,\intd{\xi}
  = 1.
\end{equation}

On trouvera les détails de la dérivation de cette solution, entre
autres, dans l'ouvrage de J. Hill \cite{HillStefanProblems}.

On calcul une approximation numérique de $\gamma$ à l'aide d'une
itération de Newton, implémentée avec le logiciel MatLAB. On obtient
\begin{equation}\label{eq:gamma}
  \gamma =\num{0.768955338463582}.
\end{equation}

La solution
(\ref{eq:neumann-sol-solid-phase})-(\ref{eq:neumann-sol-liquid-phase})
pour la température $\temperature$ s'écrit de manière explicite
en vue de son évaluation numérique sous la forme suivante:
\begin{equation}\label{eq:neumann-sol}
  \temperature(t, x) = \left\{
  \begin{array}{ll}
    -\exp\parent{\frac{\gamma}{2}}\sqrt{\frac{\pi\gamma}{2}} %
    \parent{\erf{\frac{\gamma}{2}} - \erf{\frac{x}{2\sqrt{t}}}}, %
    & \text{si } x \leq \sqrt{2\gamma t},\\
    0 %
    & \text{sinon,}
  \end{array}
  \right.
\end{equation}
où la fonction $\erf:\mathbb R\to(-1,1)$ est la fonction d'erreur
standard définie par:
\begin{equation}
  \erf(x) = \frac{1}{\sqrt \pi}\int_0^x \exp\parent{-s^2}\,\intd{s}.
\end{equation}

La solution exacte (\ref{eq:neumann-sol}) est représentée sur la
figure \ref{fig:neumann-sol} à l'instant $t = 0.5$, avec $\gamma$
donné par la relation (\ref{eq:gamma}).

\begin{figure}
  \begin{center}
    \input{../media/particles/remelt/neumann-exact.tex}
    \caption{Solution exacte de Neumann du problème de Stefan classique
      (\ref{eq:heat-solid-phase})-(\ref{eq:stefan-condition}), évaluée
      à $t = 0.5$.}
    \label{fig:neumann-sol}
  \end{center}
\end{figure}

D'un point de vue physique, la solution exacte de Neumann correspond à
la situation suivante. Un matériau, qui présente une transition entre
les phases solide et liquide à $\tliq = 0$, occupe le demi espace
$\cparent{(x_1, x_2, x_3)\in \mathbb R^3 \mid x_1 > 0}$. Ce matériau
est initialement dans l'état liquide à la température $\temperature =
\tliq$. Au temps initial, on le met en contact sur le bord
$\cparent{(x_1, x_2, x_3)\in\mathbb R^3 \mid x_1 = 0}$ avec un
réservoir thermique à température constante $\temperature = -1$. Au
cours de l'évolution temporelle, un front de transition de phase, de
liquide à solide, se propage dans le matériau. Pour des raisons de
symétrie, le front de transition reste en tout temps parallèle au
bord $\cparent{(x_1, x_2, x_3)\in\mathbb R^3 \mid x_1 = 0}$.

On utilisera cette solution exacte dans la suite pour valider le schéma de
discrétisation des équations
(\ref{eq:heat-one-phase})-(\ref{eq:heat-one-phase-neumann}), que l'on
décrit maintenant.

\subsection*{Validation numérique}
Pour valider l'implémentation du schéma numérique
(\ref{eq:chernoff-heat})-(\ref{eq:chernoff-update}), on évalue la
convergence de l'erreur entre la solution exacte définie par la
relation (\ref{eq:neumann-sol}) et $\temperature^N_\delta$,
l'approximation de la température à l'instant $T$. On fixe $\mu =
L_\beta = 1$. On choisit $\Omega = [0, 1] \subset \rplus$ et $T =
0.5$. On choisit $N$ dans l'intervalle $[100, 12800]$, et $\delta =
O(\tau)$.

\begin{figure}
  \begin{center}
    \input{../media/particles/remelt/neumann-convergence-h.tex}
    \caption{Erreur $L^2$ entre la solution exacte de Neumann et
      l'approximation numérique $\temperature^N_\delta$}
    \label{fig:neumann-convergence}
  \end{center}
\end{figure}

On constate sur la figure \ref{fig:neumann-convergence} que l'erreur
$L^2$ entre la solution exacte et l'approximation numérique converge
vers 0, avec un ordre approximatif $O(h^{3/4})$.




% application: formation et refonte de bain gele autour d'une
% particule.
\subsection*{Formation de gel autour d'une particule}
Dans cette partie, on applique la méthode numérique
(\ref{eq:chernoff-heat})-(\ref{eq:chernoff-neumann}) au calcul de
la formation de gel autour d'une particule de bain gelé.

On donne dans le tableau \ref{tab:freeze-physical-parameters} la
valeur des différents paramètres physique liés aux propriétés
thermiques du bain électrolytique.

\begin{table}
  \begin{center}
    \caption{Paramètres physiques qui interviennent dans le
      phénomène de formation de gel autour de particules.}
    \label{tab:freeze-physical-parameters}
  \begin{tabularx}{\textwidth}{@{}lllX@{}}
    \toprule
    Quantité          & Valeur     & Unité                           & Description \\
    \midrule
    $\electrolytedensity$         & \num{2130} & \si{\kg\per\cubic\meter}        & Masse volumique du bain électrolytique\\
    $\electrolyteshc$ & \num{} & \si{\joule\per\kilo\gram\per\kelvin} & Chaleur spécifique de  l'électrolyte dans la phase solide,\\
    $\electrolytelhc$ & \num{} & \si{\joule\per\kilo\gram\per\kelvin} & Chaleur spécifique de
    l'électrolyte dans la phase liquide,\\
    $\tinj$ & \num{423.15} & \si{\kelvin} & Température de la
    particule de bain gelé au moment de l'injection\\
    $\tliq$ & \num{1223.2} & \si{\kelvin} & Température du liquidus de
    l'électrolyte\\
    $\electrolytetdiff$ & \num{2.0} & \si{\joule\per\second\per\meter\per\kelvin} & Conductivité thermique de l'électrolyte \\
    \bottomrule
  \end{tabularx}
\end{center}
\end{table}

Puisqu'on s'intéresse à une particule sphérique, on fait l'hypothèse
que les données du problème présentent une symétrie sphérique. Ainsi,
on décrite l'ensemble des données en fonction de la distance à
l'origine $r$. De plus, on cherche la température $\temperature$ dans
l'électrolyte en fonction de la coordonnée $r$. Par abus de notation
et pour éviter de surcharger les notation, on réutilise les mêmes
symboles, mais explicitant la coordonnée $r$.

Soit $\tinj\in\mathbb R$ la température à laquelle se trouve les
particules d'alumine au moment de leur injection dans la cuve, et
$\telectrolyte$ la température du bain électrolytique. La condition
initiale du système formé par la particule et le bain environnant est
définie par:
\begin{equation}
  \tinit(r) = \left\{
  \begin{array}{ll}
    \tinj & \text{ si } r < R_p,\\
    \telectrolyte & \text{ si } r \geq R_p,\\
  \end{array}
  \right.\quad \forall r\in \rplus .\label{eq:initial-temperature}
\end{equation}

La figure \ref{fig:particle-initial-temperature} représente la
température initiale au voisinage de la particule. En dehors de la
particule, la température du bain est suffisante pour que celui-ci
soit dans la phase liquide. Cependant, on suppose que le bain liquide
est au repos en tout temps.

\begin{figure}[h]
  \begin{center}
    \input{../media/particles/temperature/temperature.tex}
    \caption{Température initiale du système formé par la particule de
      bain gelé placée à l'origine du système de coordonnées, et du
      bain électrolytique environnant.}
    \label{fig:particle-initial-temperature}
  \end{center}
\end{figure}

On note $R_f$ la distance entre l'origine et la position du front de
solidification, définie par la relation:
\begin{equation}
\beta(\enthalpydensityr(t, R_f(t))) = \tliq.
\end{equation}
L'épaisseur de la couche de bain solidifiée $R_g$ est définie par la
relation:
\begin{equation}
R_g(t) = R_f(t) - R_p.
\end{equation}
On définit alors le temps de latence $\tlat(R_p) > 0$, le temps
nécessaire à ce que l'épaisseur de bain solidifié atteigne à nouveau
zéro:
\begin{equation}
  R_g(\tlat).
\end{equation}

\begin{figure}
\begin{center}
  \input{../media/particles/remelt/frozen-layer-sur2-5.tex}
  \input{../media/particles/remelt/frozen-layer-sur5.tex}
  \input{../media/particles/remelt/frozen-layer-sur10.tex}
  \caption{Évolution du rayon de la sphère de bain gelé pour des
    température de surchauffe de \num{2.5}\si{\kelvin},
    \num{5}\si{\kelvin} et \num{10}\si{\kelvin} et des particules de
    rayons initiaux $r_0 = $\num{40}\si{\micro\meter},
    \num{60}\si{\micro\meter} et \num{80}\si{\micro\meter}.}
  \label{fig:freeze-radius}
\end{center}
\end{figure}
On considère une particule de bain solidifié de rayon initial $R_0 =$
\num{40}\si{\micro\meter}, \num{60}\si{\micro\meter} et
\num{80}\si{\micro\meter}. La surchauffe du bain électrolytique est
définie ici comme la différence entre la température initiale du bain
$\telectrolyte$ et la température de liquidus $\tliq$. La figure
\ref{fig:freeze-radius} présente l'évolution du rayon de la particule
pour trois surchauffes différentes du bain électrolytique environnant.

On constate que le temps nécessaire à refondre le bain gelé
diminue avec la surchauffe. De même, le temps de refonte diminue avec
la taille de la particule.

La température initiale de la particule est $T_\text{inj} =
423.15\si{\kelvin}$. Dans le cas le plus défavorable où $T -
T_\text{Liq} = 2.5\si{\kelvin}$ et $r_0 = 80\si{\micro\meter}$, le
temps nécessaire pour refondre entièrement la couche de gel est de
l'ordre de 120\si{\milli\second}.

Dans les calculs présentés ici, on a fait plusieurs hypothèses
simplificatrices. On a supposé que le fluide environnant est au
repos, ce qui n'est certainement pas le cas dans une cuve
d'électrolyse industrielle. Les forces de Lorentz et la formation de
bulles de gaz agitent les fluides, et créent des turbulences qui sont,
entre autres, de l'échelle des particules \cite{Rochat2016}. Une
agitation de l'électrolyte au voisinage d'une particule accélère le
transport de l'énergie thermique, et tend à accélérer la refonte du
bain gelé.

Pour simplifier le modèle, nous avons étudié la formation de gel
autour d'une particule sphérique de bain solidifiée. Bien entendu,
dans une cuve d'électrolyse industrielle on injecte des particules
particules d'alumine à basse température, plutôt que des particule de
bain gelé. Cependant, les propriétés thermiques des deux systèmes sont
proches, ce qui justifie cette approximation.

Lors de l'injection d'une dose de particules d'alumine, l'hypothèse
d'une particule isolée dans un bain infini au repos n'est plus
vérifiée. Les différentes particules qui constituent la dose
interagissent thermiquement entre elles par l'intermédiaire du bain
environnant. Ce comportement explique la raison pour laquelle
l'hypothèse de particules isolée mène à des temps typique de
formation et refonte de gel largement inférieurs aux valeurs
recommandées par exemple dans le travail de V. Dassylva-Raymond \cite{Dassylva2015}.

Selon la recommandation de V. Dassylva-Raymond \cite{Dassylva2015}, on
choisira, dans la suite, pour le temps de latence d'une dose de
particules injectées dans le bain une valeur uniforme $\tlat =
1\si{\second}$, indépendante du rayon.
