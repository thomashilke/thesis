Dans ce paragraphe nous détaillons les phénomènes physiques qui
motivent la nécessité de considérer un temps de latence
préalable au début de la dissolution des grains d'alumine dans le
bain électrolytique.

On considère une sphère d'électrolyte, placée à l'origine du
système de coordonnées, de température initiale $\tinj<\tliq$. Cette
sphère est donc gelée, c'est-à-dire sous forme solide. Le reste de
l'espace est occupé par le bain électrolytique au repos et de
température initiale $\tinit > \tliq$.

On s'intéresse à l'évolution de la température au cours du
temps dans le système formé par la sphère de bain gelé et le
bain liquide environnant. En particulier, la température de la
particule étant inférieure à la température du liquidus
$\tliq$, le bain au voisinage de la surface de la particule va
commencer par geler. Puis, la température du système s'équilibrant par
diffusion thermique, cette couche de bain va peu à peu fondre. On
cherche à déterminer l'évolution au cours du temps de l'épaisseur de
cette couche de bain gelée.

Soit $R_p$ le rayon de la particule placée à l'origine du
système de coordonnées, $\electrolytehc$ la chaleur spécifique du bain et
$\electrolytetdiff$ le coefficient de diffusion thermique du bain.

Soit $f_s(\temperature)$ la fraction solide du bain électrolytique. La
fraction solide est définie par:
\begin{equation}
  f_s(\temperature) = \left\{
  \begin{array}{ll}
    0           & \text{ si } \temperature > \tliq,\\
    \frac{1}{2} & \text{ si } \temperature = \tliq,\\
    1           & \text{ si } \temperature < \tliq.
  \end{array}
  \right.
\end{equation}
Le comportement de la fonction $f_s$ au voisinage de $\tliq$ est
représenté sur le graphique de gauche de la figure
\ref{fig:solid-fraction-enthalpy}.

\begin{figure}
  \begin{center}
    \input{../media/particles/solidfraction/solidfraction.tex}
    \input{../media/particles/enthalpy/enthalpy.tex}
    \caption{Gauche: fraction solide du bain $f_s$ en fonction de la
      température. Droite: Enthalpie du bain électrolytique en
      fonction de la température.}
    \label{fig:solid-fraction-enthalpy}
  \end{center}
\end{figure}

Les propriétés thermiques du bain électrolytique sont
caractérisées par la relation entre la température $\temperature$ et
l'enthalpie $\enthalpy$ qui s'exprime de la manière suivante:
\begin{equation}
  \enthalpy(\temperature) = %
    \int_0^\temperature
      \electrolytedensity\electrolytehc\intd{s} %
    + \fusionenthalpy(1 - f_s(\temperature)),
\end{equation}
où $\electrolytedensity$ est la densité du bain, $\fusionenthalpy$
est l'enthalpie de la transition entra la phase solide et liquide, et
$f_s$ la fraction solide du bain en fonction de la température. Dans
la suite, on fera l'hypothèse que les paramètres $\electrolytehc$
et $\electrolytetdiff$ sont constants dans chaque phase.
Ce choix se justifie par le fait que ces valeurs
varient peu sur l'intervalle de température qui nous intéresse, et
on peut se contenter de considérer une valeur moyenne sur cet intervalle.

Le graphique de droite sur la figure \ref{fig:solid-fraction-enthalpy}
représente le comportement de la fonction $\enthalpy(\temperature)$ au
voisinage de la température de transition $\tliq$.

Bien que la fonction $\enthalpy$ présente une discontinuité en
$\temperature = \tliq$, elle est strictement monotone, et on peut
définir une fonction $\beta(\enthalpy)$ telle que:
\begin{equation}
\beta(\enthalpy(\temperature)) = \temperature
\end{equation}
pour tout $\temperature$ de la manière suivante:
\begin{equation}
  \beta(\enthalpy) = \left\{
  \begin{array}{ll}
    \frac{\enthalpy}{\electrolytehc\electrolytedensity} %
      & \enthalpy < \tliq \electrolytehc\electrolytedensity,\\
    \tliq %
      & \tliq \electrolytehc\electrolytedensity < \enthalpy < \tliq \electrolytehc\electrolytedensity + \fusionenthalpy,\\
    \frac{\enthalpy - \fusionenthalpy}{\electrolytehc\electrolytedensity} %
      & \tliq\electrolytehc\electrolytedensity + \fusionenthalpy < \enthalpy.\\
  \end{array}
  \right.
\end{equation}
La fonction $\beta$ est représentée sur la figure \ref{fig:beta}.
\begin{figure}
  \begin{center}
    \input{../media/particles/beta/beta.tex}
    \caption{Fonction $\beta(\enthalpy)$.}
    \label{fig:beta}
  \end{center}
\end{figure}

Soit $\tinj\in\mathbb R$ la température à laquelle se trouve les
particules d'alumine au moment de leur injection dans la cuve, et
$\telectrolyte$ la température du bain électrolytique. La condition
initiale du système formé par la particule et le bain environnant est
définie par:
\begin{equation}
  \tinit(x) = \left\{
  \begin{array}{ll}
    \tinj & \text{ si } \norm{x} < R_p,\\
    \telectrolyte & \text{ si } \norm{x} \geq R_p.\\
  \end{array}
  \right.\label{eq:initial-temperature}
\end{equation}

\begin{figure}
  \begin{center}
    \input{../media/particles/temperature/temperature.tex}
    \caption{Température initiale du système formé par la
      particule de bain gelé placée à l'origine du système de
      coordonnées, et du bain électrolytique environnant.}
    \label{fig:particle-initial-temperature}
  \end{center}
\end{figure}
La figure \ref{fig:particle-initial-temperature} représente la
température initiale au voisinage de la particule. En dehors de la
particule, la température du bain est suffisante pour que celui-ci
soit dans la phase liquide. Cependant, on suppose que le bain liquide
est au repos en tout temps.

Soit $\enthalpydensity(t, x)$ la densité d'enthalpie au point $x\in\mathbb
R^3$ et au temps $t > 0$ dans le système formé par la particule de
rayon $R_p$ placée à l'origine et le bain électrolytique
occupant le reste de l'espace. La densité d'enthalpie
$\enthalpydensity$ est solution du problème suivant:
\begin{align}
  &\frac{\partial \enthalpydensity}{\partial t}(t,x) -\div %
  \parent{\electrolytetdiff\nabla \temperature(t, x)} = 0,& \quad %
  \forall (t, x) \in [0,\infty)\times \mathbb R^3,\label{eq:stefan-1}\\
  &\temperature(t, x) = \beta(\enthalpydensity(t, x)), & \quad %
  \forall (t, x) \in [0,\infty)\times \mathbb R^3,\label{eq:stefan-2}\\
    &\temperature(0, x) = \tinit(x),& \quad \forall x\in \mathbb R^3.\label{eq:stefan-3}
\end{align}
Puisque la condition initiale \ref{eq:initial-temperature} présente
une symétrie sphérique, et que le système formé par les équations
(\ref{eq:stefan-1})-(\ref{eq:stefan-3}) est invariant par symétrie
sphérique, on cherche une solution à symétrie sphérique. On définit
les fonctions $\enthalpydensityr(t, r)$ et $\temperaturer(t, r)$, la
densité d'enthalpie et la température au temps $t$ et à distance $r$
de l'origine.

Les équations (\ref{eq:stefan-1})-(\ref{eq:stefan-3}) se réécrivent en
fonction de la coordonnée $r$ de la manière suivante:
\begin{align}
  &\frac{\partial\enthalpydensityr}{\partial t}(t, r) -
  \electrolytetdiff \frac{1}{r^2}\frac{\partial}{\partial
    r}\parent{r^2\frac{\partial \temperaturer}{\partial r}} = 0, &\quad
  \forall (t, r)\in [0, \infty)\times \rplus,\label{eq:stefan-r-1}\\
    &\temperaturer(t, r) = \beta(\enthalpydensityr(t, r)), &\quad
    \forall (t, r)\in [0, \infty)\times \rplus,\label{eq:stefan-r-2}\\
      &\temperaturer(0, r) = \tinit(r),&\quad \forall r \in \rplus.\label{eq:stefan-r-3}
\end{align}
Le problème formé par les équations
(\ref{eq:stefan-r-1})-(\ref{eq:stefan-r-3}) est un problème de
Stefan. Dans la section qui suit on décrit le schéma de
discrétisation, en suivant le travail de Y. Safa \cite{Safa2009}.

\subsection*{Discrétisation du problème de Stefan}
