La densité de l'alumine $\aluminadensity$ qui constitue les particules
est de \num{3960}\si{\kilo\gram\per\cubic\meter} soit environ deux
fois supérieure à celle du bain électrolytique $\electrolytedensity$,
qui est de \num{2130}\si{\kilo\gram\per\cubic\meter}. Une particule
d'alumine placée dans un bain électrolytique animé par un écoulement
stationnaire subit ainsi l'effet de trois forces
distinctes. D'une part, cette particule est entraînée dans le fluide
par l'intermédiaire d'une force de traînée $\dragforce$. Cette force
est opposée à la vitesse relative de la particule par rapport au
fluide. D'autre part, cette particule est entrainée vers le fond de
la cuve par la force de gravité $\gravityforce$, à laquelle s'oppose
la force d'Archimède $\buoyancyforce$.

Dans cette section, nous nous proposons d'évaluer l'importance des
forces de gravité et d'Archimède devant la force de traînée. Dans ce
but, on considère une particule d'alumine sphérique de rayon initial
$r_0$ placée dans un bain électrolytique au repos. On suppose que la
température du bain $\temperature$ et la concentration d'alumine
$\concentration$ sont maintenue constantes au cours du temps. On
suppose de plus que mouvement du fluide autour de la particule
n'influence pas sa dissolution. Alors, la variation du rayon $r$ de la
particule au cours du temps est décrite par l'équation
(\ref{eq:radius-edo-cst-diss-rate}). Comme on l'a vu dans la section
\ref{sec:particle-population-dissolution}, le rayon à l'instant $t$
est donné par
\begin{equation}\label{eq:radius}
  r(t) = \sqrt{r_0^2 - 2\dissolutionrate t},
\end{equation}
où $\dissolutionrate$ est le taux de dissolution, constant au cours du
temps. Clairement pour
\begin{equation}
  T = \frac{r_0^2}{2\dissolutionrate}
\end{equation}
on obtient $r(T) = 0$, \ie, la particule est complètement dissoute
au temps $T$.

On modélise la particule d'alumine, supposée non poreuse, par un point
matériel de volume $V(t) = \frac{4}{3}\pi r^3(t)$. On travaille dans
le référentiel de la Terre, et on suppose qu'elle se déplace
verticalement dans ce référentiel. On note $x(t)\in \mathbb R$ sa
position selon un système de coordonnées vertical à l'instant $t\in
[0,T]$ et $\dot x(t) = \frac{\mathrm dx}{\mathrm dt}(t)$ sa
vitesse. La particule étant sphérique, on propose d'approximer la
force de traînée par la loi de Stokes. Si $v$ est la vitesse de la
particule par rapport au fluide, lui-même au repos dans le
référentiel, on a
\begin{equation}
\dragforce(r, v) = -6\pi\mu r v.
\end{equation}
Ici on a noté $\mu$ la viscosité dynamique de l'électrolyte. Il est
connu que cette approximation est valide pour autant que le nombre de
Reynolds de l'écoulement $\reynolds$ soit suffisamment faible, \ie,
$\reynolds \lesssim 1$. On montrera a posteriori que cette condition
est en général satisfaite pour des particules d'alumine qui sont
typiquement injectées dans le bain électrolytique.

Les forces de gravité $\gravityforce$ et d'Archimède $\buoyancyforce$
en fonction du rayon $r$ de la particule sont données respectivement
par
\begin{equation}
\gravityforce(r) = -\frac{4}{3}\pi r^3 \aluminadensity g
\quad\text{et}\quad
\buoyancyforce(r) = \frac{4}{3}\pi r^3 \electrolytedensity g
\end{equation}
où $g = 9.81$ est l'accélération de la gravité.

L'équation du mouvement de la particule s'obtient à l'aide de la
deuxième loi de Newton qui lie la somme des forces à l'impulsion $p =
m\dot x$:
\begin{equation}
\frac{\mathrm dp}{\mathrm dt}(t) = \dragforce(r(t), \dot x(t)) + \gravityforce(r(t)) + \buoyancyforce(r(t))
\end{equation}
soit
\begin{equation}\label{eq:mvt}
  \frac{\mathrm d}{\mathrm dt}\parent{\aluminadensity V(t)x(t)} =
  \frac{4}{3}\pi r^3\parent{\electrolytedensity - \aluminadensity}g -
  6\pi\electrolyteviscosity r(t)\dot x(t).
\end{equation}

En utilisant la relation (\ref{eq:radius}), nous avons
\begin{equation}\label{eq:volume-variation}
  \frac{\mathrm d}{\mathrm dt}V(t) = -4\pi \dissolutionrate r(t).
\end{equation}
En remplaçant (\ref{eq:volume-variation}) dans l'égalité
(\ref{eq:mvt}), en tenant compte du fait que $V(t) = \frac{4}{3}\pi
r^3(t)$, puis en divisant par $\pi r(t)$ nous obtenons
\begin{equation}
\aluminadensity \frac{4}{3}r^2(t)\frac{\mathrm d^2}{\mathrm dx^2}x(t)
= g\parent{\aluminadensity - \electrolytedensity}\frac{4}{3}r^2(t) -
\parent{6\electrolyteviscosity - 4\aluminadensity
  \dissolutionrate}\frac{\mathrm dx}{\mathrm dt}.
\end{equation}
En utilisant successivement (\ref{}), (\ref{}) et (\ref{}), et en
intégrant l'équation (\ref{}) sur l'intervalle de temps $[0, T]$, nous
obtenons
\begin{equation}
\parent{6\electrolyteviscosity - 4\aluminadensity\dissolutionrate}x(T)
= g\parent{\aluminadensity -
  \electrolytedensity}\frac{r_0^4}{3\dissolutionrate}-\aluminadensity\frac{4}{3}\int_0^T\parent{r_0^2
- 2\dissolutionrate t}\frac{\mathrm d^2}{\mathrm dt^2}x(t)\,\intd{t}.
\end{equation}



% Si la force d'archimede est importante devant la force de trainee,
% alors le fait que le fluide n'est pas au repos et que le gradient de
% p n'est pas strictement vertical peut avoir une importance.
%
% Ensuite, si le temps caracteristique de l'acceleration initiale de
% la particule avant d'atteindre une vitesse limite controlee par la
% force de trainee est importante, alors l'approximation de dire que
% les particules sont transportees par le fluide est fausse.
