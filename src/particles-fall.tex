La densité de l'alumine $\aluminadensity$ qui constitue les particules
est de \num{3960}\si{\kilo\gram\per\cubic\meter} soit environ deux
fois supérieure à celle du bain électrolytique $\electrolytedensity$,
qui est de \num{2130}\si{\kilo\gram\per\cubic\meter}. Une particule
d'alumine placée dans un bain électrolytique animé par un écoulement
stationnaire subit ainsi l'effet de trois forces
distinctes. D'une part, cette particule est entraînée dans le fluide
par l'intermédiaire d'une force de traînée $\dragforce$. Cette force
est opposée à la vitesse relative de la particule par rapport au
fluide. D'autre part, cette particule est entraînée vers le fond de
la cuve par la force de gravité $\gravityforce$, à laquelle s'oppose
la force d'Archimède $\buoyancyforce$.

Dans cette section, nous nous proposons d'évaluer l'importance des
forces de gravité et d'Archimède devant la force de traînée. Dans ce
but, on considère une particule d'alumine sphérique de rayon initial
$r_0$ placée dans un bain électrolytique au repos. On suppose que la
température du bain $\temperature$ et la concentration d'alumine
$\concentration$ sont maintenue constantes au cours du temps. On
suppose de plus que mouvement du fluide autour de la particule
n'influence pas sa dissolution. Alors, la variation du rayon $r$ de la
particule au cours du temps est décrite par l'équation
(\ref{eq:radius-edo-cst-diss-rate}). Comme on l'a vu dans la section
\ref{sec:particle-population-dissolution}, le rayon à l'instant $t$
est donné par
\begin{equation}\label{eq:radius}
  r(t) = \sqrt{r_0^2 - 2\dissolutionrate t},
\end{equation}
où $\dissolutionrate$ est le taux de dissolution, constant au cours du
temps. Clairement pour
\begin{equation}\label{eq:final-time}
  T = \frac{r_0^2}{2\dissolutionrate}
\end{equation}
on obtient $r(T) = 0$, \ie, la particule est complètement dissoute
au temps $T$.

On modélise la particule d'alumine, supposée non poreuse, par un point
matériel de volume $V(t) = \frac{4}{3}\pi r^3(t)$. On travaille dans
le référentiel de la Terre, et on suppose qu'elle se déplace
verticalement dans ce référentiel. On note $x(t)\in \mathbb R$ sa
position selon un système de coordonnées vertical à l'instant $t\in
[0,T]$ et $\dot x(t) = \frac{\mathrm dx}{\mathrm dt}(t)$ sa
vitesse. La particule étant sphérique, on propose d'approximer la
force de traînée par la loi de Stokes. Si $v$ est la vitesse de la
particule par rapport au fluide, lui-même au repos dans le
référentiel, on a
\begin{equation*}
\dragforce(r, v) = -6\pi\mu r v.
\end{equation*}
Ici on a noté $\mu$ la viscosité dynamique de l'électrolyte. Il est
connu que cette approximation est valide pour autant que le nombre de
Reynolds de l'écoulement $\reynolds$ soit suffisamment faible, \ie,
$\reynolds \lesssim 1$. On montrera a posteriori que cette condition
est en général satisfaite pour des particules d'alumine qui sont
typiquement injectées dans le bain électrolytique.

Les forces de gravité $\gravityforce$ et d'Archimède $\buoyancyforce$
en fonction du rayon $r$ de la particule sont données respectivement
par
\begin{equation*}
\gravityforce(r) = -\frac{4}{3}\pi r^3 \aluminadensity g
\quad\text{et}\quad
\buoyancyforce(r) = \frac{4}{3}\pi r^3 \electrolytedensity g
\end{equation*}
où $g = 9.81$ est l'accélération de la gravité.

L'équation du mouvement de la particule s'obtient à l'aide de la
deuxième loi de Newton qui lie la somme des forces à l'impulsion $p =
m\dot x$:
\begin{equation*}
\frac{\mathrm dp}{\mathrm dt}(t) = \dragforce(r(t), \dot x(t)) + \gravityforce(r(t)) + \buoyancyforce(r(t)),
\end{equation*}
soit
\begin{equation}\label{eq:mvt}
  \frac{\mathrm d}{\mathrm dt}\parent{\aluminadensity V(t)x(t)} =
  \frac{4}{3}\pi r^3(t)\parent{\electrolytedensity - \aluminadensity}g -
  6\pi\electrolyteviscosity r(t)\dot x(t).
\end{equation}
On suppose que la particule se trouve initialement en $x(0) = 0$ avec
une vitesse $\dot x(0) = v_0$.

En utilisant la relation (\ref{eq:radius}), nous avons
\begin{equation}\label{eq:volume-variation}
  \frac{\mathrm d}{\mathrm dt}V(t) = -4\pi \dissolutionrate r(t).
\end{equation}
En remplaçant (\ref{eq:volume-variation}) dans l'égalité
(\ref{eq:mvt}), en tenant compte du fait que $V(t) = \frac{4}{3}\pi
r^3(t)$, puis en divisant par $\pi r(t)$ nous obtenons
\begin{equation}\label{eq:simple-mvt}
\aluminadensity \frac{4}{3}r^2(t)\frac{\mathrm d^2}{\mathrm dx^2}x(t)
= g\parent{\aluminadensity - \electrolytedensity}\frac{4}{3}r^2(t) -
\parent{6\electrolyteviscosity - 4\aluminadensity
  \dissolutionrate}\frac{\mathrm dx}{\mathrm dt}.
\end{equation}
En utilisant (\ref{eq:radius}), (\ref{eq:radius}) et en
intégrant l'équation (\ref{eq:simple-mvt}) sur l'intervalle de temps $[0, T]$, nous
obtenons
\begin{equation}\label{eq:mvt-integration-1}
\parent{6\electrolyteviscosity - 4\aluminadensity\dissolutionrate}x(T)
= g\parent{\aluminadensity -
  \electrolytedensity}\frac{r_0^4}{3\dissolutionrate}-\aluminadensity\frac{4}{3}\int_0^T\parent{r_0^2
- 2\dissolutionrate t}\frac{\mathrm d^2}{\mathrm dt^2}x(t)\,\intd{t}.
\end{equation}
En intégrant le dernier terme de l'équation
(\ref{eq:mvt-integration-1}) par partie, nous obtenons finalement la
profondeur terminale de la particule au moment de sa dissolution complète
\begin{equation}\label{eq:particle-terminal-position}
x(T) = \frac{g\parent{\electrolytedensity -
    \aluminadensity}}{\dissolutionrate
  \parent{18\electrolyteviscosity - 4\aluminadensity
    \dissolutionrate}}r_0^4 + \frac{4 \aluminadensity
  v_0}{18\electrolyteviscosity - 4\aluminadensity \dissolutionrate} r_0^2.
\end{equation}

\begin{table}
  \begin{center}
    \caption{Paramètres physiques qui interviennent dans la chute
      d'une particule d'alumine dans un bain électrolytique.}
    \label{tab:fall-physical-parameters}
    \begin{tabularx}{\textwidth}{@{}lllX@{}}
      \toprule
      Quantité                & Valeur       & Unités                                      & Description \\
      \midrule
      $\electrolytedensity$   & \num{2130}   & \si{\kg\per\cubic\meter}                    & Masse volumique du bain électrolytique \\
      $\aluminadensity$       & \num{3960}   & \si{\kg\per\cubic\meter}                    & Masse volumique de l'oxyde d'aluminium \\
      $g$                     & \num{9.81}   & \si{\meter\per\square\second}               & Accélération de la gravité terrestre\\
      $\dissolutionrate$      & \num{0.5e-9} & \si{\square\meter\per\second}               & Taux de dissolution de l'alumine \\
      $\electrolyteviscosity$ & \num{2e-3}   & \si{\kilo\gram\per\meter\per\second}        & Viscosité dynamique du bain électrolytique \\
      \bottomrule
    \end{tabularx}
  \end{center}
\end{table}

La valeur des paramètres physiques qui correspondent à une particule
d'alumine dans un bain et qui interviennent dans l'équation
(\ref{eq:particle-terminal-position}) sont synthétisées dans le
tableau \ref{tab:fall-physical-parameters}. Contrairement à
l'hypothèse admise ici, le bain d'une cuve d'électrolyse est animé par
une forte agitation. Ces turbulences correspondent à une viscosité
équivalente $\electrolyteturbviscosity$ qui caractérise l'écoulement
moyenné au cours du temps. Dans le bain d'une installation
industrielle, cette viscosité turbulente est typiquement de l'ordre de
\num{1} \si{\kilo\gram\per\meter\per\second} ou plus, et domine donc
largement la viscosité physique du fluide. Pour cette raison nous
considérons des viscosité dans l'intervalle $[\num{2e-3}, \num{1}]$.

Les doses de particules d'alumine ne sont en général pas délicatement
déposées à la surface du bain; les particules sont lâchées depuis une
hauteur qui varie entre \num{20}\si{\centi\meter} et
\num{40}\si{\centi\meter} par rapport à la surface du bain. On
modélise cette condition à l'aide de la vitesse initiale $v_0$. La
vitesse verticale atteinte par une masse en chute libre dans le champ
de pesanteur terrestre sur une hauteur $L$ et initialement au repos
est donnée par l'expression $\sqrt{2gL}$. Par conséquent nous
considérerons des vitesses initiales $v_0$ entre \num{0} et
\num{3} \si{\meter\per\second}.

\begin{table}
  \begin{center}
    \caption{Profondeur terminale de la particule dans le bain
      électrolytique en fonction des conditions initiales $r_0$
      [\si{\micro\meter}], $v_0$ [\si{\meter\per\second}] et de la
      viscosité dynamique $\electrolyteviscosity$
      [\si{\kilo\gram\per\meter\per\second}].}
    \label{tab:fall-results}
    \begin{tabularx}{\textwidth}{@{}lXXXX@{}}
      \toprule
      & $\electrolyteviscosity = \num{2e-3}$ & $\electrolyteviscosity = \num{1e-2}$ & $\electrolyteviscosity = \num{1e-1}$ & $\electrolyteviscosity = \num{1}$ \\
      \midrule
      \input{../media/particles/fall/results.tex}
      \bottomrule
    \end{tabularx}
  \end{center}
\end{table}

Le tableau \ref{tab:fall-results} présente la profondeur
terminale de la particule d'alumine donnée par la relation
(\ref{eq:particle-terminal-position}) en fonction des conditions
initiales $r_0$, $v_0$ et la viscosité dynamique du fluide
$\electrolyteviscosity$. Lorsque $\electrolyteviscosity =
\num{2e-3}$, c'est-à-dire que la particule est placée dans un fluide
immobile, une profondeur maximale de \num{20} \si{\centi\meter}
environ est atteinte lorsque $r_0 = \num{80}$ \si{\micro\meter} et
avec une vitesse initiale $v_0 = \num{3}$ \si{\meter\per\second}. La
profondeur maximale lorsque le rayon initial est inférieur à
\num{60} \si{\micro\meter} est systématiquement inférieure à
\num{1} \si{\centi\meter}.

Si on suppose maintenant que le fluide est turbulent, la viscosité
effective de l'écoulement moyen est supérieure à la viscosité
physique du bain. Cette situation correspond aux trois dernières
colonnes à droite du tableau \ref{tab:fall-results}. Dans ce cas, on
constate que la profondeur maximale est systématiquement de l'ordre de
\num{1}\si{\milli\meter}, ou inférieure.

Le bain électrolytique d'une cuve n'est jamais au repos, en
particulier dans les canaux où le fluide est agité par les bulles de
gaz qui résultent de l'électrolyse et qui remontent à la surface du
bain. Dans la suite de ce travail, nous ferons l'hypothèse que
l'injection des doses de particules d'alumine ne perturbe pas
l'écoulement du bain. Dans ce cadre, nous négligerons l'effet de la
force de gravité sur la trajectoire des particules.

Nous concluons cette section par trois remarques qui traitent
premièrement de la validité de la loi de Stokes pour la force de traînée,
deuxièmement de l'importance de la vitesse initiale de la particule
sur la profondeur terminale de la particule, et troisièmement du temps
caractéristique pour qu'une particule soit emportée par un fluide en
mouvement laminaire.

%%\begin{remarque}
%%  (Sur le fait qu'on peut negliger le terme de reaction dans
%%  l'equation du mouvement)
%%\end{remarque}

%%\begin{remarque}
%%  Le temps caracteristique du transitoire pour atteindre la vitesse
%%  du fluide doit etre faible devant les autres temps
%%  caracteristiques pour que l'approximation de dire que les
%%  particules sont transportee par les lignes de courant du fluide
%%  est une bonne approximation.
%%\end{remarque}

\begin{remarque}
  En considérant la relation (\ref{eq:particle-terminal-position}), on
  peut évaluer l'effet de la vitesse initiale de la particule sur sa
  position terminale. Les contributions des deux termes à droite de
  l'égalité (\ref{eq:particle-terminal-position}) sont identiques
  lorsque la vitesse initiale est telle que
  \begin{equation*}
    v_0 = \frac{g}{4\dissolutionrate}\frac{\aluminadensity -
      \electrolytedensity}{\aluminadensity} r_0^2.
  \end{equation*}
  Pour une particule de rayon initial $r_0 = \num{80}$
  \si{\micro\meter}, cette vitesse initiales est de $v_0 = \num{14.5}$
  \si{\meter\per\second}, ce qui correspond à une hauteur de chute
  libre d'environ \num{10} \si{\meter}. On conclut que, dans une cuve
  d'électrolyse industrielle typique, la vitesse initiales des
  particules suffisamment dispersées qui pénètrent dans le bain est
  négligeable sur leur profondeur de pénétration.
\end{remarque}

\begin{remarque}
  Le nombre de Reynolds d'une sphère de rayon $r$ en mouvement
  rectiligne uniforme dans un fluide au repos est donné par
  \begin{equation*}
    \reynolds = \frac{2 \electrolytedensity v r}{\electrolyteviscosity}
  \end{equation*}
  où $v$ est sa vitesse relative au fluide.

  En supposant $r$ fixé, l'équation du mouvement d'une particule
  dans le fluide s'écrit
  \begin{equation}\label{eq:mvt-fixed-r}
    \aluminadensity \frac{4}{3}\pi r^3 \ddot x(t) = \frac{4}{3}\pi r^3\parent{\electrolytedensity - \aluminadensity}g -
  6\pi\electrolyteviscosity r\dot x(t).
  \end{equation}
  On obtient la vitesse limite de chute de la particule $v_L$ en
  remplaçant dans (\ref{eq:mvt-fixed-r}) $\dot x$ par $v_L$ et en posant $\ddot x
  = 0$. En résolvant pour $v_L$ on obtient
  \begin{equation}\label{eq:terminal-velocity}
    v_L = \frac{2}{9}\frac{g\parent{\electrolytedensity - \aluminadensity}}{\electrolyteviscosity}r^2.
  \end{equation}
  Le nombre de Reynolds associé à cet écoulement s'écrit comme
  \begin{equation*}
    \reynolds = \frac{4}{9}\frac{g\electrolytedensity\parent{\electrolytedensity - \aluminadensity}}{\electrolyteviscosity^2}r^3.
  \end{equation*}
  En reprenant les conditions considérées dans le tableau
  \ref{tab:fall-results} et les paramètres du tableau
  \ref{tab:fall-physical-parameters}, on obtient un nombre de Reynolds
  maximal pour $\electrolyteviscosity = \num{2e-3}$
  \si{\kilo\gram\per\meter\per\second} et $r_0 = 80$
  \si{\micro\meter}, soit $\reynolds \simeq 2.2$. Ce résultat justifie
  l'utilisation de la loi de Stokes pour modéliser la force de traînée
  de la particule.
\end{remarque}

\begin{remarque}
  L'équation du mouvement d'une particule d'alumine de rayon $r_0$
  supposé constant au cours du temps s'écrit
  \begin{equation*}
    \aluminadensity \frac{4}{3}\pi r_0^3 \ddot x(t) = -
    6\pi\electrolyteviscosity r_0 \dot x(t),
  \end{equation*}
  et on considère les conditions initiales $x(0) = 0$ et $\dot x(0) =
  v_0$. On peut intégrer cette équation exactement, et on obtient la
  vitesse de la particule au cours du temps
  \begin{equation*}
    \dot x(t) = V\exp\parent{-\frac{9\electrolyteviscosity}{2\aluminadensity r_0^2}t}.
  \end{equation*}
  La vitesse approche zéro sur une échelle de temps données par
  $\frac{2\aluminadensity r_0^2}{9\electrolyteviscosity}$,
  c'est-à-dire environ \num{2.8e-3} \si{\second} pour une particule de rayon $r_0 =
  \num{80}$ \si{\micro\meter} et dans un fluide au repos, \ie, avec $\mu
  = $ \num{2e-3}. Autrement dit, le temps nécessaire à une particule
  pour être entraînée et transportée par un fluide à la même vitesse
  que celui-ci est négligeable devant les temps caractéristiques des
  autres phénomènes qui prennent place dans la cuve et que l'on
  s'intéresse à modéliser. Cette observation permet de justifier
  l'approximation qui sera faite dans le chapitre
  \ref{chap:populations}, où on supposera que les particules suivent
  exactement les lignes de courant de l'écoulement.
\end{remarque}

\begin{figure}
 \begin{center}
    \begin{subfigure}[b]{0.49\textwidth}
      \input{../media/particles/trajectories/particle_fall.tex}
      \caption{Trajectoire des particules dans le fluide.}
      \label{fig:particle-trajectories-a}
    \end{subfigure}
    \begin{subfigure}[b]{0.49\textwidth}
      \input{../media/particles/trajectories/particle_fall_zoom.tex}
      \caption{Zoom sur la phase d'accélération initiale.}
      \label{fig:particle-trajectories-b}
    \end{subfigure}

    \caption{Trajectoires de particules de rayon initial
      $r_0 = $ \numlist{40;60;80} \si{\micro\meter}. Chaque
    particule est placée à l'origine avec une vitesse nulle
    au temps $t = 0$. Les trajectoires sont intégrées jusqu'à ce que
    les particules soient complètement dissoutes, c'est-à dire sur l'intervalle
    de temps $\left[0, \frac{r_0^2}{2\dissolutionrate}\right]$. Notez
    les différentes échelles sur les deux graphiques.}
    \label{fig:particle-trajectories}
 \end{center}
\end{figure}
