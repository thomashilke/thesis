Soit $\Omega\subset \mathbb R^3$ le domaine ouvert donné, occupé par
le bain électrolytique. Ce bain est animé par une vitesse d'écoulement
$u:\Omega\to\mathbb R^3$ stationnaire donnée et telle que $\div u = 0$
dans $\Omega$ et $u\cdot \nu = 0$ sur le bord $\partial \Omega$ de
$\Omega$, où ici $\nu$ est la normale unité. On note $\tend$ le temps
auquel nous souhaitons connaître la solution. Dans ce chapitre on note
$\concentration(t, x)$ la concentration d'alumine dissoute en
\si{\mol\per\cubic\meter} et $\temperature(t, x)$ la température dans
le bain en Kelvin à l'instant $t \in [0, T]$ et à l'endroit $x \in
\Omega$.

\paragraph{Vitesse de transport}
La force de gravité a pour effet d'entraîner les particules d'alumine
vers le fond de la cuve. Nous avons vu dans la section
\ref{sec:particle-fall} que le temps caractéristique nécessaire pour
qu'une particule typique atteigne d'une part sa vitesse stationnaire de
chute et d'autre part la vitesse de l'écoulement du bain est de
l'ordre de quelques millisecondes. La vitesse maximale de l'écoulement
stationnaire dans une cuve d'électrolyse d'aluminium étant de l'ordre
de \num{0.1} \si{\meter\per\second}, cette période transitoire s'étend
sur des distances d'environ \num{1e-4} \si{\meter}, ce qui est
largement inférieure à la résolution spatiale des grilles que l'on
utilise dans des calculs industriels. Par conséquent, nous ferons deux
hypothèses simplificatrices dans le cadre du modèle de transport et
dissolution de particules d'alumine. Premièrement, nous supposerons
que le champ de gravité a pour effet de transporter les particules
vers le fond de la cuve. On prendra comme vitesse de transport la
vitesse stationnaire de chute (\ref{eq:terminal-velocity}) d'une
particule soumise à la force de traînée de Stokes. Cette force dépend
du rayon des particules. Ce transport gravitationnel des particules, qui
a lieu dans l'ensemble du bain, est modélisé par un champ $w(r):
\Omega\times[0, \infty)\to\mathbb R^3$ défini par
\begin{equation}\label{eq:fall-velocity}
  w(r) =
  -\frac{2g\parent{\aluminadensity -
      \electrolytedensity}}{9\electrolyteviscosity} r^2 \hat e_3
\end{equation}
en vertu de (\ref{eq:terminal-velocity}). On a noté $\hat e_3$ le
vecteur unitaire vertical dirigé vers le haut. Clairement, le champ
$w$ est tel que $\div w = 0$ dans $\Omega$. Deuxièmement, nous
supposerons que les particules suivent exactement les lignes de
courant du champ de transport $u + w$ dans le domaine $\Omega$. Par
ailleurs, la concentration d'alumine dissoute $\concentration$ et la
température du bain électrolytique $\temperature$ seront transportés
par la vitesse d'écoulement $u$, et diffusés.

\paragraph{Populations de particules}
Dans une cuve industrielle, de l'alumine en poudre doit être injectée
à intervalles réguliers dans le bain. Nous modélisons ces injections
sous la forme d'une série d'évènements instantanés, successifs dans le
temps. Soit $K$ le nombre total d'injections, et soit un ensemble de
nombres réels $\tau^k$, $k = 1, 2, \dots, K$ tels que $0\leq \tau^k <
T$ $\forall k$, où les $\tau^k$ représentent les instants auxquels
surviennent chaque injection. Pour toute injection $k = 1,2, \dots,
K$, la distribution initiale de particule est notée $S^k:(x,
r)\in\Omega\times\rplus\to S^k(x, r)\in\rplus$ et est supposée
donnée. Les distributions initiales de particules $S^k$ s'expriment en
\si{\kilo\gram\per\cubic\meter\per\meter}.

On note $n_p^k(t, x, r)$, $k = 1, 2, \dots, K$ la densité en espace et
en taille de particules issues de l'injection $k$ dans le bain
électrolytique à l'instant $t \in [0, T]$. Autrement dit, la quantité
$n_p^k(t, x, r)\intd{x}\intd{r}$ représente le nombre de particules, à
l'instant $t$, dans le volume infinitésimal $\intd{x}$ autour du point
$x\in\Omega$ (entre $x$ et $x + \mathrm dx$) et dont la taille est comprise dans l'intervalle $[r,
  r+\intd{r}]$. Soit $\tlat \geq 0$ le temps de latence avant que les
particules ne commencent à se dissoudre suite à leur injection dans le
bain. Ce temps de latence a fait l'objet d'une discussion dans la
section \ref{sec:particle-freeze}. Chaque population de particules
$n_p^k$, $k = 1, 2,\dots, K$ satisfait les
équations lorsque $x\in\Omega,\ r\in\rplus$:
\begin{align}
  &n_p^k(t, x, r) = 0, & 0\leq t < \tau^k,\label{eq:population-pre-injection}\\
  &n_p^k(\tau^k, x, r) = S^k(x, r),\label{eq:population-injection}\\
  &\frac{\partial n_p^k}{\partial t} + (u(x) + w(r))\cdot \nabla n_p^k
  = 0, &\tau^k < t \leq \tau^k +
  \tlat,\label{eq:population-transport}\\
  &\frac{\partial n_p^k}{\partial t} + (u(x) + w(r))\cdot \nabla n_p^k
  + \frac{\partial}{\partial r} \parent{f(r, \concentration,
    \temperature) n_p^k} = 0, &\tau^k + \tlat < t \leq \tend,\label{eq:population-dissolution}
\end{align}
où l'opérateur $\nabla = (\partial_{x_1}, \partial_{x_2},
\partial_{x_3})^t$. La vitesse de dissolution $f$ qui intervient dans
l'équation (\ref{eq:population-dissolution}) a déjà été définie dans
la section \ref{sec:particle-dissolution} par l'expression
(\ref{eq:dissolution-velocity}). Les équations
(\ref{eq:population-transport}) et (\ref{eq:population-dissolution})
sont complétées par une condition aux limites sur le bord entrant. Plus
précisément, soit $\Gamma^-$ la partie du bord $\partial \Omega$ de $\Omega$
définie par
\begin{equation}\label{eq:np-bc}
  \Gamma^- = \cparent{x\in\partial\Omega \mid \nu \cdot  (w + u)
  < 0}.
\end{equation}
Ici, $\nu$ est le vecteur normal unitaire extérieur à la surface
$\partial \Omega$. La condition aux limites sur le bord entrant s'écrit
alors pour tout $k = 1, 2, \dots, K$ et $t > \tau^k$
\begin{equation}
n_p^k(t, x) = 0, \quad x\in\Gamma^-.
\end{equation}

La densité totale de particules au temps $t \in [0, T]$, à l'endroit
$x\in\Omega$ et fonction de $r$ est donnée par
\begin{equation}\label{eq:populations-sum}
  n_p(t, x, r) = \sum_{k=1}^K n_p^k(t, x, r).
\end{equation}

\begin{remarque}
Les quantités $n_p^k$ introduites ici sont similaires au champ $n_p$
décrit dans la section \ref{sec:particle-population-dissolution}, à
ceci près que les champs $n_p^k$ sont des fonctions du temps $t$ et du
rayon $r$ des particules, mais aussi du lieu $x$ dans le domaine
$\Omega$ occupé par le bain électrolytique.
\end{remarque}

\paragraph{Consommation de l'alumine dissoute}
La concentration d'alumine dissoute est consommée\footnote{On parle de
consommation de l'alumine dissoute lorsqu'une paire de ions \ce{Al^{3+}} issus de
la dissociation d'une molécule d'\ce{Al2O3} est réduite au niveau de
la cathode.} par la réaction
d'électrolyse. Si $I$ est le courant électrique total imposé
traversant la cuve, le débit total d'alumine dissoute consommée en
\si{\mol\per\second} est proportionnel à $I$ et s'écrit
\begin{equation}\label{eq:mass-consumption}
  \frac{I}{6\faraday}
\end{equation}
où $F = \num{96485.33}$ \si{\coulomb\per\mol} est la constante de
Faraday. Le facteur \num{6} provient du fait qu'il
faut \num{6} électrons pour réduire une molécule d'\ce{Al2O3}, et
produire deux molécules d'aluminium métallique \ce{Al}.

On suppose que la consommation d'alumine dissoute qui a lieu dans le
bain électrolytique est proportionnelle à la densité de courant
électrique locale $j:\Omega\to\mathbb R^3$. On définit alors le terme
de disparition de la concentration d'alumine $q_1$ associé à la
consommation par la réaction d'électrolyse
\begin{equation}
  q_1(x) = -\frac{I}{6\faraday} %
  \frac{\abs{j(x)}}{\displaystyle\int_\Omega\abs{j(x)}\intd{x}}
\end{equation}
de sorte à avoir la consommation totale sur $\Omega$
\begin{equation}
  \int_\Omega q_1(x)\,\intd{x} = -\frac{I}{6\faraday}.
\end{equation}

\paragraph{Dissolution des particules d'alumine}
La masse perdue par la population de particules d'alumine
$\population$ vient contribuer à la concentration d'alumine
dissoute. On définit $q_2$ le terme source de la concentration
qui représente l'apport dû à la dissolution des particules de la
manière suivante. Si $t \in [0, T]$, on note $\bar k(t)$ l'entier tel
que $\bar k(t) = \max_{1\leq k\leq K}\cparent{k \mid \tau^k + \tlat < t}$ . Alors on définit
\begin{equation}
  q_2(t, x) = -\sum_{k = 1}^{\bar
    k(t)}\frac{4\pi\aluminadensity}{[\ce{Al2O3}]}\int_{\rplus}n_p^k(t,
  x, r)f(r, \concentration(t, x), \temperature(t, x))r^2\,\intd{r}.
\end{equation}
où [\ce{Al2O3}] $ = \num{0.102}$ \si{\kilo\gram\per\mol} est la masse
molaire de l'alumine. L'ensemble des indices $k$ tels que $1\leq k
\leq \bar k(t)$ représente l'ensemble des populations de particules
$n_p^k$ qui se dissolvent dans le bain à l'instant $t\in[0, T]$.

\paragraph{Concentration d'alumine dissoute}
La concentration d'alumine dissoute $\concentration$ est transportée
dans le bain par la vitesse d'écoulement $u$, mais est de plus sujette
à une diffusion liée d'une part à l'agitation moléculaire, et d'autre
part aux turbulences de l'écoulement. Soit
$\cdiffusivity:\Omega\to\mathbb R^*_+$ la diffusivité de la
concentration d'alumine dissoute dans le bain, supposée connue. Alors
la concentration $\concentration$ doit satisfaire l'équation
\begin{equation}\label{eq:concentration}
  \frac{\partial \concentration}{\partial t}(t, x) + u(x)\cdot\nabla \concentration(t, x) - \div\parent{\cdiffusivity(x)
  \nabla \concentration(t, x)} = q_1(x) + q_2(t, x),\quad \forall
  x\in\Omega,\ t\in (0, T).
\end{equation}
Puisque qu'il ne peut y avoir de flux de masse d'alumine à travers le
bord du domaine $\Omega$, la concentration doit satisfaire la
condition aux limites de Neumann homogène
\begin{equation}\label{eq:c-bc}
  \electrolytecdiff(x)\frac{\partial \concentration}{\partial \nu}(t, x) = 0 \quad  \quad
  \forall x\in\partial \Omega,\ t\in[0, T].
\end{equation}

\paragraph{Termes sources de la température du bain}
Pour simplifier le modèle, on fait l'hypothèse que le bain
électrolytique est isolé thermiquement, c'est-à-dire qu'il n'y a pas
de flux d'énergie thermique à travers le bord du domaine occupé par le
bain.

L'énergie thermique du bain provient de trois sources distinctes.

Premièrement, les particules d'alumine sont injectées avec
une température $\tinj < \temperature$, la température locale du
bain. De l'énergie thermique est prélevée dans le bain pour rétablir
l'équilibre thermique entre celui-ci et les particules. On note $p_1(t,
x)$ la densité de puissance thermique extraite du bain pour réchauffer
les particules, que l'on définit par
\begin{equation}\label{eq:power-heating}
  p_1(t, x) = -\parent{\tinit-\tinj}\sum_{k = 1}^{K}\dirac(t -
  \tau^k)\int_{\rplus}\aluminadensity\aluminahc\frac{4}{3}\pi r^3 S^k(x, r)\,\intd{r}
\end{equation}
où $\dirac$ est une masse de Dirac et $\tinit$ est la température
initiale du bain. Dans (\ref{eq:power-heating}) et pour rappel, $\tinit$ est la
température initiale du bain électrolytique, $\aluminahc$ et
$\aluminadensity$ sont la chaleur spécifique et la densité de
l'alumine, tandis que les $S^k$ sont les distributions en espace et en
rayon des doses de poudre d'alumine qui interviennent dans le membre
de droite de (\ref{eq:population-injection}).

\begin{remarque}
  En réalité, la puissance nécessaire à réchauffer les particules
  $p_1(t, x)$ est proportionnelle à ${\temperature(t, x) -
    \tinj}$. Cependant, dans (\ref{eq:power-heating}) on utilise la
  température initiale du bain $\tinit$ à la place de
  $\temperature(t, x)$, ce qui revient à négliger l'écart entre
  $\temperature(t, x)$ et $\tinit$, qui est faible devant ${\tinit -
  \tinj}$.
\end{remarque}

Deuxièmement, la réaction de dissolution de la poudre d'alumine est
endothermique. On note $p_2(t, x)$ la densité de puissance thermique
utilisée par la réaction, que l'on définit pour tout $x\in\Omega$ et
$t\in[0, T]$ par
\begin{equation}\label{eq:p2}
p_2(t, x) = - \aluminadissolutionenthalpy q_2(t, x),
\end{equation}
où $\aluminadissolutionenthalpy$ est l'enthalpie molaire nécessaire à
la dissolution de l'alumine. On rappelle que $q_2(t, .)$ est le
débit molaire par unité de volume d'alumine dissoute à l'instant $t$.

Et troisièmement, la résistivité électrique de l'électrolyte engendre
la conversion d'une partie de l'énergie électrique en énergie
thermique par effet Joule. Cette source d'énergie thermique $p_3$ par
effet Joule dépend de la densité de courant stationnaire $j$ et
s'écrit
\begin{equation}\label{eq:p3}
p_3(x) = \frac{j\cdot j}{\conductivity}.
\end{equation}
où le nombre réel $\sigma > 0$ est la conductivité électrique du bain que l'on
suppose constante dans tout $\Omega$.

\paragraph{Température du bain}
Tout comme la concentration d'alumine, la température du bain
électrolytique $\temperature$ est transportée par la vitesse
d'écoulement $u$, et diffusée dans le bain. Cette diffusion est due à
la l'agitation moléculaire d'une part, et à aux turbulences de
l'écoulement d'autre part. Soit $\electrolytetdiff:\Omega\to\mathbb
R^*_+$ la diffusivité de la température dans le bain, supposée
donnée. Comme précédemment on a noté $\electrolytedensity$,
$\electrolytehc$ la densité et la chaleur spécifique du bain
électrolytique. Alors la température $\temperature$ doit satisfaire
l'équation
\begin{equation}\label{eq:temperature}
\frac{\partial \temperature}{\partial t}(t, x) +
u(x)\cdot\nabla\temperature - \div\parent{\electrolytetdiff(x)\nabla
\temperature(t, x)} =
\frac{1}{\electrolytedensity\electrolytehc}\sum_{i = 1}^{3}p_i(t, x),
\quad \forall x\in\Omega,\ t\in(0, T).
\end{equation}
De plus, et conformément à l'hypothèse d'isolation thermique, la température
$\temperature$ doit satisfaire une condition de Neumann homogène
\begin{equation}\label{eq:t-bc}
  \electrolytehc\frac{\partial \temperature}{\partial \nu}(t, x) = 0, \quad
  \forall\ x\in\partial \Omega,\ t\in[0, T].
\end{equation}

\paragraph{Formulation du problème}
Le problème de transport et dissolution d'alumine en fonction de la
température consiste à chercher des fonctions $n_p^k:(0,
T)\times\Omega\times\rplus\to \mathbb R$, $c:(0, T)\times\Omega\to
\mathbb R$ et $\temperature:(0, T)\times\Omega\to\mathbb R$ qui
satisfont les équations couplées (\ref{eq:population-pre-injection}) à
(\ref{eq:population-dissolution}), (\ref{eq:concentration}) et
(\ref{eq:temperature}) ainsi que les conditions aux limites
(\ref{eq:np-bc}), (\ref{eq:c-bc}) et (\ref{eq:t-bc}), auxquelles on
ajoutera des conditions initiales appropriées.

La section suivante traite de la discrétisation en temps de ce
système d'équations.
