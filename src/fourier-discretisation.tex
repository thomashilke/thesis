
Soit $h > 0$ et soit $\mathcal T_h$ une triangulation du domaine
$\Lambda$ de $\mathbb R^2$
telle que $\mathrm{diam}(K) \leq h,\ \forall K\in\mathcal T_h$. On introduit
maintenant l'espace éléments finis $V_h$ défini par:
\begin{equation}
V_h = \cparent{v \in C^0(\Lambda) \setsuchthat v|_K \in \mathbb
  P_1(K)\ \forall K\in \tau_h},
\end{equation}
ainsi que l'espace enrichi par une fonction bulle:
\begin{equation}
  B_h = \cparent{v \in C^0(\Lambda)\setsuchthat v|_{K} \in \mathbb
    P_1\oplus B_K\ \forall K\in\tau_h}
\end{equation}
où $B_K$ est engendré par la fonction
bulle
\begin{equation}
  27\lambda_1^K(x_1,x_2)\lambda_2^K(x_1,x_2)\lambda_3^K(x_1, x_2),
\end{equation}
les $\lambda_i^K$, $i = 1,2,3$ étant les trois fonctions
barycentriques de $K$.

\subsection{Discrétisation en espace par une méthode d'éléments finis
  pour le mode fondamental}\label{sec:fourier-discretisation-fundamental}
La méthode d'éléments finis correspondant à \cite{}, \cite{} consiste
à chercher $(u_{0,h}, p_{0,h})\in B_h\times(V_h\cap L_0^2(\Omega))$
telle que pour tout $(v_h, q_h)$ on ait
\begin{equation}
\int_\Lambda 2\mu\times \straintensor(u_h):\straintensor (v_h)\intd{x_1}\intd{x_2}
\end{equation}




La formulation faible du système d'équations
(\ref{eq:stokes-fourier-fund-1})-(\ref{eq:stokes-fourier-fund-2})
correspond au problème suivant. On cherche les fonctions
$(u,p)\in\parent{H_0^1(\Lambda)}^2\otimes L_0^2(\Lambda)$ telles que
pour tout $(v, q)\in \parent{H_0^1(\Lambda)}^2\otimes L_0^2(\Lambda)$
les équations suivantes soient vérifiées:
\begin{equation}\label{eq:stokes-weak}
  \begin{aligned}
    \mu \int_\Lambda \nabla u: \nabla v\,\mathrm dx_1\mathrm dx_2 +
    \int_\Lambda p\,\div(v)\,\mathrm dx_1\mathrm dx_2 &= \int_\Lambda
    f\cdot v\,\mathrm dx_1\mathrm dx_2,\\
    \int_\Lambda q\,\div(u)\,\mathrm dx_1\mathrm dx_2 &= 0.
  \end{aligned}
\end{equation}
La formulation faible (\ref{eq:stokes-weak}) est discrétisée avec
une méthode de Galerkin. Le problème (\ref{eq:stokes-weak})
discrétisé correspond à chercher les fonctions $(u_h,p_h)\in
B_{h}^2\otimes V_{0,h}$ telles que les équations suivantes soient
vérifiées pour tout $(v_h, q_h) \in B_{h}^2\otimes V_{0,h}$:
\begin{equation}\label{eq:stokes-discr-weak}
  \begin{aligned}
    \mu \int_\Lambda \nabla u_h: \nabla v_h\,\mathrm dx_1\mathrm dx_2 +
    \int_\Lambda p_h\,\div(v_h)\,\mathrm dx_1\mathrm dx_2 &= \int_\Lambda
    f\cdot v_h\,\mathrm dx_1\mathrm dx_2,\\
    \int_\Lambda q_h\,\div(u_h)\,\mathrm dx_1\mathrm dx_2 &= 0.
  \end{aligned}
\end{equation}

Lorsque la viscosité $\mu$ est un tenseur, la formulation forte du
problème à résoudre pour le mode fondamental s'écrit:
\begin{equation}
  \begin{aligned}
    -\div(2\tilde\mu\tilde E(u_0)) + \nabla p = f
  \end{aligned}
\end{equation}
où on a noté le tenseur des contraintes:
\begin{equation}
  \straintensor(u)_{i,j} = \frac{1}{2}\mu_{i,j}\parent{\frac{\partial
      u_i}{\partial x_j} + \frac{\partial u_j}{\partial x_i}}, \quad
  1\leq i,j\leq 2.
\end{equation}

La formualtion faible correspondant au problème
(\ref{eq:stokes-fourier-weak-fund}) consiste a chercher les fonctions
$(u_0,p)\in\parent{H_0^1(\Lambda)}^2\otimes L_0^2(\Lambda)$ telles que
pour tout $(v, q)\in\parent{H_0^1(\Lambda)}^2\otimes L_0^2(\Lambda)$:
\begin{equation}\label{eq:stokes-fourier-visc-weak-fund}
  \begin{aligned}
    \int_\Lambda 2\mu\otimes E(u):E(v)\mathrm dx_1\mathrm dx_2 -
    \int_\Lambda p\,\div(v)\,\mathrm dx_1\mathrm dx_2 &=
    \int_\Lambda f\cdot v\,\mathrm dx_1\mathrm dx_2,\\
    \int_\Lambda q\,\div(u_0)\,\mathrm dx_1\mathrm dx_2
    &= 0.
  \end{aligned}
\end{equation}
A nouveau, la formulation (\ref{eq:stokes-fourier-visc-weak-fund})
est discrétisée avec une méthode de Galerkin de manière
similaire: on cherche les fonctions $(u_{0,h},p_h)\in
B_{h}^2\otimes V_{0,h}$ telles que pout tout $(v_h,q_h)\in
B_{h}^2\otimes V_{0,h}$ on aie:
\begin{equation}\label{eq:stokes-fourier-visc-weak-fund}
  \begin{aligned}
    \int_\Lambda 2\mu\otimes E(u_h):E(v_h)\mathrm dx_1\mathrm dx_2 -
    \int_\Lambda p\,\div(v_h)\,\mathrm dx_1\mathrm dx_2 &=
    \int_\Lambda f\cdot v_h\,\mathrm dx_1\mathrm dx_2,\\
    \int_\Lambda q\,\div(u_{0,h})\,\mathrm dx_1\mathrm dx_2
    &= 0.
  \end{aligned}
\end{equation}

\begin{remarque}
  Il est connu que le couple d'espace éléments finis $(B_{h}, V_{0,h})$ satisfait la
  condition inf-sup (\ref{eq:inf-sup}). Ainsi les formulations
  (\ref{eq:stokes-fourier-visc-weak-fund}) et
  (\ref{eq:stokes-discr-weak}) n'ont pas besoin d'être stabilisé.
\end{remarque}


\subsection{Discrétisation en espace par une méthode d'éléments finis pour les harmoniques}\label{sec:harmonic-discr-harm}
On considère à présent la discrétisation de la formulation
faible
(\ref{eq:stokes-fourier-weak-1})-(\ref{eq:stokes-fourier-weak-5}) qui
correspond aux différentes coefficients de la décomposition de
Fourier de $u$ et $p$. A nouveau on utilise une méthode de
Galerkin. Pour $k > 0$ donné, le probème discrétisé consiste
à chercher les fonctions $(u_{1,k,h}, u_{2,k,h}, u_{3,k,h}, p^0_{k,h})\in
\parent{V_h}^3\times V_{0,h}$ telles que:
\begin{align}
  \mu \int_\Lambda \parent{\nabla u_{1,k,k}\cdot \nabla v_{1,h} %
                           + \beta^2 u_{1,k,h}v_{1,h}}\,\mathrm dx_1\mathrm dx_2 %
  - \int_\Lambda p_{k,h}^0\frac{\partial v_{1,h}}{\partial x_1}\,\mathrm dx_1\mathrm dx_2 %
  &= \int_\Lambda f_{1,k}v_{1,h}\mathrm dx_1\mathrm dx_2, \label{eq:stokes-fourier-discr-weak-1}\\
  %
  \mu \int_\Lambda \parent{\nabla u_{2,k,h}\cdot \nabla v_{2,h} %
                           + \beta^2 u_{2,k,h}v_{2,h}}\,\mathrm dx_1\mathrm dx_2 %
  - \int_\Lambda p_{k,h}^0\frac{\partial v_{2,h}}{\partial x_2}\,\mathrm dx_1\mathrm dx_2 %
  &= \int_\Lambda f_{2,k,h}v_{1,h}\mathrm dx_1\mathrm dx_2, \label{eq:stokes-fourier-discr-weak-2}\\
  %
  \mu \int_\Lambda \parent{\nabla u_{3,k,h}\cdot \nabla v_{3,h} %
                           + \beta^2 u_{3,k,h}v_{3,h}}\,\mathrm dx_1\mathrm dx_2 %
  - \int_\Lambda \parent{p_{k,h}^0 + C}\,\mathrm dx_1\mathrm dx_2 %
  &= \int_\Lambda f_{3,k,h}v_{1,h}\mathrm dx_1\mathrm dx_2, \label{eq:stokes-fourier-discr-weak-3}\\
  %
  \int_\Lambda \parent{\frac{\partial u_{1,k,h}}{\partial x_1} %
                       + \frac{\partial u_{2,k,h}}{\partial x_2}}q\,\mathrm dx_1\mathrm dx_2 %
  + \beta\int_\Lambda u_{3,k,h}q\,\mathrm dx_1\mathrm dx_2 %
  &= 0,\label{eq:stokes-fourier-discr-weak-4}\\
  %
  \int_\Lambda u_{3,k,h}\,\mathrm dx_1\mathrm dx_2 %
  &= 0. \label{eq:stokes-fourier-discr-weak-5}
\end{align}

\begin{remarque}
  Les formulations (\ref{eq:stokes-discr-weak}),
  (\ref{eq:stokes-fourier-visc-weak-fund}), font intervenir l'espace
  éléments finis à moyenne nulle $V_{0,h}$. En pratique, on
  relaxera cette contrainte, et on cherchera les fonctions $p^0_h$ et
  $p^0_{k,h}$ dans $V_{h}$, et en ajoutant à la formulation
  correspondante l'équation:
  \begin{equation}
    \int_\Lambda p^0_{k,h}\,\mathrm dx_1\mathrm dx_2 = 0
  \end{equation}
  et en ajoutant un multiplicateur de Lagrange $\lambda$. Par exemple, le
  problème (\ref{eq:stokes-discr-weak}) devient:
  \begin{equation}\label{eq:stokes-discr-weak}
    \begin{aligned}
      \mu \int_\Lambda \nabla u_h: \nabla v_h\,\mathrm dx_1\mathrm dx_2 +
      \int_\Lambda p_h\,\div(v_h)\,\mathrm dx_1\mathrm dx_2 &= \int_\Lambda
      f\cdot v_h\,\mathrm dx_1\mathrm dx_2,\\
      \int_\Lambda q_h\,\div(u_h)\,\mathrm dx_1\mathrm dx_2 + \lambda
      \int_\Lambda q_h\,\mathrm dx_1\mathrm dx_2 &= 0.\\
      \int_\Lambda p_h\,\mathrm dx_1\mathrm dx_2 &= 0
    \end{aligned}
  \end{equation}
  et on cherche $\lambda \in \mathbb R$, $u_h\in \parent{B_h}^2$ et
  $p_h\in V_h$.

  Ce choix évite de devoir construire les fonctions de bases de
  $V_{0,h}$ et garanti que les matrices correspondantes aux formes
  bilinéaires soient sparse.
\end{remarque}

\subsection{Quadrature pour les coefficients de Fourier de la force}
En général, la force $f$ qui intervient dans le membre de droite
de l'équation de Stokes (\ref{eq:stokes}) n'est pas connue de fa\c
con analytique. Dans le cas industriel que nous considérerons dans
la section \ref{sec:harmonic-application}, la force sera donnée
comme un élément de $V_h$ sur un maillage tétraédrique.

En général, il est clair que $f_{i,k} \in
L^2(\Lambda)$. Cependant, pour des raisons d'implémentations il est commode
de considérer l'interpolation de Lagrange des fonctions $f_{i,k}$
sur le maillage $\tau_h$, $\hat f_{i,k}\in V_h$:
\begin{equation}
  \hat f_{i,k} = \sum_{n = 1}^{N} f_{i,k}(x_n)\varphi_n.
\end{equation}
où les $\varphi_n$ sont les fonctions de base de $V_h$.  Il suffit
alors de calculer les valeurs de $f_{i,k}$ aux noeuds $x_n$ du maillage
$\tau_h$. Soit $N_s > 0$ le nombre de subdivisions de l'intervalle $[-h,
  h]$. Les intégrales qui intervienne dans les définitions
(\ref{eq:force-coefficient-1})-(\ref{eq:force-coefficient-3}) sont
approximée par une formule de Simpson sur la subdivision
régulière $z_0 = -h < z_1 < \dots < z_{N_{s+1}} = h$ de
l'intervalle $[-h, h]$. Le nombre de subdivisions sera choisi de
sorte à ce que l'erreur de quadrature soit de l'ordre de
l'epsilon machine.

Sauf mention contraire, on utilisera dorénavant $\hat f_{i,k}$ en
lieu et place de $f_{i,k}$ dans les membres de droite des
différentes formulations, et on omettra le circonflexe pour des
raisons de lisibilité.

%\subsection{Discrétisation du terme de Navier non-linéaire}

%\subsection{Discrétisation de la viscosité turbulente non-linéaire}

\subsection{Reconstruction de la solution}\label{sec:u-h}
On se fixe un entier $K > 0$, et on suppose données les fonctions
$u_{i,k}$, $i = 1,2$, $0 \leq k \leq K$, $u_{3,k}$, $0 < k \leq K$ et
$p_k^0$, $0 \leq k \leq K$, obtenues par l'une des méthodes
numériques décrites dans les sections
\ref{sec:harmonic-discr-fund}, \ref{sec:harmonic-discr-harm}. On
reconstruit une approximation $U_h$ de la $U$ en effectuant les sommes
(\ref{eq:fourier-coeff-def-1})-(\ref{eq:fourier-coeff-def-5}), en
tronquant les sommes à $k = K$. Finalement, on obtient une
approximation $u_h$ de la solution du problème
(\ref{eq:stokes}) en restreignant $U_h$ au domaine $\Omega$:
\begin{equation}\label{eq:u-h}
  u_h(x_1, x_2, x_3) = U_h|_\Omega(x_1, x_2,x_3)\quad (x_1, x_2,
  x_3)\in\Omega.
\end{equation}
