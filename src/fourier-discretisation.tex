Dans cette section nous décrivons les méthodes de discrétisations
proposées pour approximer le coefficient du mode fondamental de
l'écoulement $(u^0, p^0)$ et les coefficients des harmoniques $(u^k,
p^k)$.

Soit un nombre réel $h > 0$ et soit $\mathcal M_h$ une triangulation
du domaine $\Lambda$ de $\mathbb R^2$ telle que $\diam(K)\leq h$
$\forall K\in \mathcal M_h$. On introduit maintenant l'espace
éléments finis $V_h$ défini par
\begin{equation}
  V_h = \cparent{v\in C^0(\Lambda)\mid v|_|K \in \mathbb
    P_1(K)\ \forall K\in\mathcal M_h}
\end{equation}
ainsi que l'espace enrichi par une fonction bulle
\begin{equation}
  B_h = \cparent{v \in C^0(\Lambda)^2 \mid v|_K \in P_1^2\oplus
    B_K\ \forall K\in\mathcal M_h} \cap H_0^1(\Lambda)^2
\end{equation}
où $B_K$ est engendré par la fonction $27 \lambda_1^K
\lambda_2^K\lambda_3^K$, les $\lambda_i^K$, $i = 1,2,3$ étant les
trois fonctions barycentriques du triangle $K$.

\paragraph{Mode fondamental}
La méthode d'éléments finis correspondant à
(\ref{eq:fourier-weak-fund-1}), (\ref{eq:fourier-weak-fund-2})
consiste à chercher $(u^0_h, p^0_h)\in \mathcal B_h\times(V_h \cap
L^2_0(\Lambda))$ tels que, pour tout $(v_h, q_h)\in \mathcal
B_h\times(V_h \cap L^2_0(\Lambda))$ on ait
\begin{align}
  &\int_\Lambda
  2\mu\otimes\reducedstraintensor(u^0_h):\reducedstraintensor(v)
  -\int_\Lambda p^0_h\div(v_h) = \int_\Lambda f^0\cdot v_h, \label{eq:fourier-fund-discrete-1}\\
  & \int_\Lambda q_h\div u^0_h = 0.\label{eq:fourier-fund-discrete-2}
\end{align}
Il est connu que le couple d'espaces éléments finis $(B_h, V_h \cap
L^2_0(\Lambda))$ est stable, ainsi (\ref{eq:fourier-fund-discrete-1}),
(\ref{eq:fourier-fund-discrete-2}) admet une unique solution.

Le schéma (\ref{eq:fourier-fund-discrete-1}),
(\ref{eq:fourier-fund-discrete-2}) fait intervenir l'espace
éléments finis à moyenne nulle $V_h \cap L^2_0(\Lambda)$ pour le mode
fondamental de la pression $p^0$. En pratique, on exprime
explicitement la contrainte sur la pression dans le problème faible
en ajoutant l'équation
\begin{equation}
\int_\Lambda p^0_h = 0
\end{equation}
et en ajoutant un multiplicateur de Lagrange. Il est alors équivalent
de chercher $(u^0_h, p^0_h)\in \mathcal B_h\times V_h$ et
$\lambda\in\mathbb R$ tels que, pour tout $(v_h, q_h)\in \mathcal
B_h\times V_h$ on ait
\begin{align}
  &\int_\Lambda
  2\mu\otimes\reducedstraintensor(u^0_h):\reducedstraintensor(v)
  -\int_\Lambda p^0_h\div(v_h) = \int_\Lambda f^0\cdot v_h, \label{eq:fourier-fund-discrete-1-prat}\\
  & \int_\Lambda q_h\div u^0_h + \lambda \int_\Lambda q_h =
  0,\label{eq:fourier-fund-discrete-2-prat}\\
  & \int_\Lambda p_h^0 = 0.\label{eq:fourier-fund-discrete-3-prat}
\end{align}

\paragraph{Harmoniques}
On considère à présent la discrétisation de la formulation
faible (\ref{eq:fourier-weak-harm-1}),
(\ref{eq:fourier-weak-harm-2}). Pour $k > 0$ donné, le problème
discrétisé consiste à chercher les fonctions $(u^k_h, p^k_h) \in
(V_h\cap H_0^1(\Lambda))^3\times V_h$ telles que
\begin{align}
  &\int_\Lambda 2\mu\otimes
  \tilde\straintensor(u^k_h):\tilde\straintensor(v) - \int_\Lambda
  p_h^k\parent{\frac{\partial v_{1,h}}{\partial x_1}
    \frac{\partial v_{2,h}}{\partial x_2} +\beta^k p^k_h v_{3,h}} =
  \int_\Lambda f\cdot v_h,\label{eq:fourier-harm-discrete-1}\\
  &\int_\Lambda \parent{\frac{\partial u^k_{1,h}}{\partial x_1} +
    \frac{\partial u^k_{2,h}}{\partial x_2} + \beta^k u^k_{3,h}}q_h +
  \delta \sum_{K\in\mathcal M_h} h_K^2
  \int_K\parent{\parent{\beta^k}^2 p^k_h q_h + \nabla p_h^k \nabla
    q_h} = 0\label{eq:fourier-harm-discrete-2}
\end{align}
pour tout $(v_{1,h}, v_{3,h}, v_{3,h}, q_h) \in (V_h\cap
H_0^1(\Lambda))^3\times V_h$. Ici $\delta > 0$ et le terme de
stabilisation dans (\ref{eq:fourier-harm-discrete-2}) est justifié
par le fait que, d'après (\ref{eq:fourier-coeff-def-3}),
\begin{equation}
  \Delta P = \Delta p^0 + \sum_{k > 0} (\Delta p^k -
  \parent{\beta^k}p^k)\cos(\beta^k x_3).
\end{equation}


%%Soit $h > 0$ et soit $\mathcal T_h$ une triangulation du domaine
%%$\Lambda$ de $\mathbb R^2$
%%telle que $\mathrm{diam}(K) \leq h,\ \forall K\in\mathcal T_h$. On introduit
%%maintenant l'espace éléments finis $V_h$ défini par:
%%\begin{equation}
%%V_h = \cparent{v \in C^0(\Lambda) \setsuchthat v|_K \in \mathbb
%%  P_1(K)\ \forall K\in \tau_h},
%%\end{equation}
%%ainsi que l'espace enrichi par une fonction bulle:
%%\begin{equation}
%%  B_h = \cparent{v \in C^0(\Lambda)\setsuchthat v|_{K} \in \mathbb
%%    P_1\oplus B_K\ \forall K\in\tau_h}
%%\end{equation}
%%où $B_K$ est engendré par la fonction
%%bulle
%%\begin{equation}
%%  27\lambda_1^K(x_1,x_2)\lambda_2^K(x_1,x_2)\lambda_3^K(x_1, x_2),
%%\end{equation}
%%les $\lambda_i^K$, $i = 1,2,3$ étant les trois fonctions
%%barycentriques de $K$.
%%
%%\subsection{Discrétisation en espace par une méthode d'éléments finis
%%  pour le mode fondamental}\label{sec:fourier-discretisation-fundamental}
%%La méthode d'éléments finis correspondant à \cite{}, \cite{} consiste
%%à chercher $(u_{0,h}, p_{0,h})\in B_h\times(V_h\cap L_0^2(\Omega))$
%%telle que pour tout $(v_h, q_h)$ on ait
%%\begin{equation}
%%\int_\Lambda 2\mu\times \straintensor(u_h):\straintensor (v_h)\intd{x_1}\intd{x_2}
%%\end{equation}
%%
%%
%%
%%
%%La formulation faible du système d'équations
%%(\ref{eq:stokes-fourier-fund-1})-(\ref{eq:stokes-fourier-fund-2})
%%correspond au problème suivant. On cherche les fonctions
%%$(u,p)\in\parent{H_0^1(\Lambda)}^2\otimes L_0^2(\Lambda)$ telles que
%%pour tout $(v, q)\in \parent{H_0^1(\Lambda)}^2\otimes L_0^2(\Lambda)$
%%les équations suivantes soient vérifiées:
%%\begin{equation}\label{eq:stokes-weak}
%%  \begin{aligned}
%%    \mu \int_\Lambda \nabla u: \nabla v\,\mathrm dx_1\mathrm dx_2 +
%%    \int_\Lambda p\,\div(v)\,\mathrm dx_1\mathrm dx_2 &= \int_\Lambda
%%    f\cdot v\,\mathrm dx_1\mathrm dx_2,\\
%%    \int_\Lambda q\,\div(u)\,\mathrm dx_1\mathrm dx_2 &= 0.
%%  \end{aligned}
%%\end{equation}
%%La formulation faible (\ref{eq:stokes-weak}) est discrétisée avec
%%une méthode de Galerkin. Le problème (\ref{eq:stokes-weak})
%%discrétisé correspond à chercher les fonctions $(u_h,p_h)\in
%%B_{h}^2\otimes V_{0,h}$ telles que les équations suivantes soient
%%vérifiées pour tout $(v_h, q_h) \in B_{h}^2\otimes V_{0,h}$:
%%\begin{equation}\label{eq:stokes-discr-weak}
%%  \begin{aligned}
%%    \mu \int_\Lambda \nabla u_h: \nabla v_h\,\mathrm dx_1\mathrm dx_2 +
%%    \int_\Lambda p_h\,\div(v_h)\,\mathrm dx_1\mathrm dx_2 &= \int_\Lambda
%%    f\cdot v_h\,\mathrm dx_1\mathrm dx_2,\\
%%    \int_\Lambda q_h\,\div(u_h)\,\mathrm dx_1\mathrm dx_2 &= 0.
%%  \end{aligned}
%%\end{equation}
%%
%%Lorsque la viscosité $\mu$ est un tenseur, la formulation forte du
%%problème à résoudre pour le mode fondamental s'écrit:
%%\begin{equation}
%%  \begin{aligned}
%%    -\div(2\tilde\mu\tilde E(u_0)) + \nabla p = f
%%  \end{aligned}
%%\end{equation}
%%où on a noté le tenseur des contraintes:
%%\begin{equation}
%%  \straintensor(u)_{i,j} = \frac{1}{2}\mu_{i,j}\parent{\frac{\partial
%%      u_i}{\partial x_j} + \frac{\partial u_j}{\partial x_i}}, \quad
%%  1\leq i,j\leq 2.
%%\end{equation}
%%
%%La formualtion faible correspondant au problème
%%(\ref{eq:stokes-fourier-weak-fund}) consiste a chercher les fonctions
%%$(u_0,p)\in\parent{H_0^1(\Lambda)}^2\otimes L_0^2(\Lambda)$ telles que
%%pour tout $(v, q)\in\parent{H_0^1(\Lambda)}^2\otimes L_0^2(\Lambda)$:
%%\begin{equation}\label{eq:stokes-fourier-visc-weak-fund}
%%  \begin{aligned}
%%    \int_\Lambda 2\mu\otimes E(u):E(v)\mathrm dx_1\mathrm dx_2 -
%%    \int_\Lambda p\,\div(v)\,\mathrm dx_1\mathrm dx_2 &=
%%    \int_\Lambda f\cdot v\,\mathrm dx_1\mathrm dx_2,\\
%%    \int_\Lambda q\,\div(u_0)\,\mathrm dx_1\mathrm dx_2
%%    &= 0.
%%  \end{aligned}
%%\end{equation}
%%A nouveau, la formulation (\ref{eq:stokes-fourier-visc-weak-fund})
%%est discrétisée avec une méthode de Galerkin de manière
%%similaire: on cherche les fonctions $(u_{0,h},p_h)\in
%%B_{h}^2\otimes V_{0,h}$ telles que pout tout $(v_h,q_h)\in
%%B_{h}^2\otimes V_{0,h}$ on aie:
%%\begin{equation}\label{eq:stokes-fourier-visc-weak-fund}
%%  \begin{aligned}
%%    \int_\Lambda 2\mu\otimes E(u_h):E(v_h)\mathrm dx_1\mathrm dx_2 -
%%    \int_\Lambda p\,\div(v_h)\,\mathrm dx_1\mathrm dx_2 &=
%%    \int_\Lambda f\cdot v_h\,\mathrm dx_1\mathrm dx_2,\\
%%    \int_\Lambda q\,\div(u_{0,h})\,\mathrm dx_1\mathrm dx_2
%%    &= 0.
%%  \end{aligned}
%%\end{equation}
%%
%%\begin{remarque}
%%  Il est connu que le couple d'espace éléments finis $(B_{h}, V_{0,h})$ satisfait la
%%  condition inf-sup (\ref{eq:inf-sup}). Ainsi les formulations
%%  (\ref{eq:stokes-fourier-visc-weak-fund}) et
%%  (\ref{eq:stokes-discr-weak}) n'ont pas besoin d'être stabilisé.
%%\end{remarque}
%%
%%
%%\subsection{Discrétisation en espace par une méthode d'éléments finis pour les harmoniques}\label{sec:harmonic-discr-harm}
%%On considère à présent la discrétisation de la formulation
%%faible
%%(\ref{eq:stokes-fourier-weak-1})-(\ref{eq:stokes-fourier-weak-5}) qui
%%correspond aux différentes coefficients de la décomposition de
%%Fourier de $u$ et $p$. A nouveau on utilise une méthode de
%%Galerkin. Pour $k > 0$ donné, le probème discrétisé consiste
%%à chercher les fonctions $(u_{1,k,h}, u_{2,k,h}, u_{3,k,h}, p^0_{k,h})\in
%%\parent{V_h}^3\times V_{0,h}$ telles que:
%%\begin{align}
%%  \mu \int_\Lambda \parent{\nabla u_{1,k,k}\cdot \nabla v_{1,h} %
%%                           + \beta^2 u_{1,k,h}v_{1,h}}\,\mathrm dx_1\mathrm dx_2 %
%%  - \int_\Lambda p_{k,h}^0\frac{\partial v_{1,h}}{\partial x_1}\,\mathrm dx_1\mathrm dx_2 %
%%  &= \int_\Lambda f_{1,k}v_{1,h}\mathrm dx_1\mathrm dx_2, \label{eq:stokes-fourier-discr-weak-1}\\
%%  %
%%  \mu \int_\Lambda \parent{\nabla u_{2,k,h}\cdot \nabla v_{2,h} %
%%                           + \beta^2 u_{2,k,h}v_{2,h}}\,\mathrm dx_1\mathrm dx_2 %
%%  - \int_\Lambda p_{k,h}^0\frac{\partial v_{2,h}}{\partial x_2}\,\mathrm dx_1\mathrm dx_2 %
%%  &= \int_\Lambda f_{2,k,h}v_{1,h}\mathrm dx_1\mathrm dx_2, \label{eq:stokes-fourier-discr-weak-2}\\
%%  %
%%  \mu \int_\Lambda \parent{\nabla u_{3,k,h}\cdot \nabla v_{3,h} %
%%                           + \beta^2 u_{3,k,h}v_{3,h}}\,\mathrm dx_1\mathrm dx_2 %
%%  - \int_\Lambda \parent{p_{k,h}^0 + C}\,\mathrm dx_1\mathrm dx_2 %
%%  &= \int_\Lambda f_{3,k,h}v_{1,h}\mathrm dx_1\mathrm dx_2, \label{eq:stokes-fourier-discr-weak-3}\\
%%  %
%%  \int_\Lambda \parent{\frac{\partial u_{1,k,h}}{\partial x_1} %
%%                       + \frac{\partial u_{2,k,h}}{\partial x_2}}q\,\mathrm dx_1\mathrm dx_2 %
%%  + \beta\int_\Lambda u_{3,k,h}q\,\mathrm dx_1\mathrm dx_2 %
%%  &= 0,\label{eq:stokes-fourier-discr-weak-4}\\
%%  %
%%  \int_\Lambda u_{3,k,h}\,\mathrm dx_1\mathrm dx_2 %
%%  &= 0. \label{eq:stokes-fourier-discr-weak-5}
%%\end{align}
%%
%%\begin{remarque}
%%  Les formulations (\ref{eq:stokes-discr-weak}),
%%  (\ref{eq:stokes-fourier-visc-weak-fund}), font intervenir l'espace
%%  éléments finis à moyenne nulle $V_{0,h}$. En pratique, on
%%  relaxera cette contrainte, et on cherchera les fonctions $p^0_h$ et
%%  $p^0_{k,h}$ dans $V_{h}$, et en ajoutant à la formulation
%%  correspondante l'équation:
%%  \begin{equation}
%%    \int_\Lambda p^0_{k,h}\,\mathrm dx_1\mathrm dx_2 = 0
%%  \end{equation}
%%  et en ajoutant un multiplicateur de Lagrange $\lambda$. Par exemple, le
%%  problème (\ref{eq:stokes-discr-weak}) devient:
%%  \begin{equation}\label{eq:stokes-discr-weak}
%%    \begin{aligned}
%%      \mu \int_\Lambda \nabla u_h: \nabla v_h\,\mathrm dx_1\mathrm dx_2 +
%%      \int_\Lambda p_h\,\div(v_h)\,\mathrm dx_1\mathrm dx_2 &= \int_\Lambda
%%      f\cdot v_h\,\mathrm dx_1\mathrm dx_2,\\
%%      \int_\Lambda q_h\,\div(u_h)\,\mathrm dx_1\mathrm dx_2 + \lambda
%%      \int_\Lambda q_h\,\mathrm dx_1\mathrm dx_2 &= 0.\\
%%      \int_\Lambda p_h\,\mathrm dx_1\mathrm dx_2 &= 0
%%    \end{aligned}
%%  \end{equation}
%%  et on cherche $\lambda \in \mathbb R$, $u_h\in \parent{B_h}^2$ et
%%  $p_h\in V_h$.
%%
%%  Ce choix évite de devoir construire les fonctions de bases de
%%  $V_{0,h}$ et garanti que les matrices correspondantes aux formes
%%  bilinéaires soient sparse.
%%\end{remarque}

\subsection{Quadrature pour les coefficients de Fourier de la force}
En pratique, les forces $f_i$, $i = 1,2,3$ sont données aux sommets du
maillage $\mathcal M_h$. Les intégrales qui interviennent dans les
définitions (\ref{eq:f-1}), (\ref{eq:f-2}) et (\ref{eq:f-3}) sont
approchées par une formule de Simpson sur une subdivision uniforme de
l'intervalle $[0, \thickness]$. Le nombre de subdivision sera choisi
de sorte à ce que cette erreur de quadrature soit négligeable.


%%En général, la force $f$ qui intervient dans le membre de droite
%%de l'équation de Stokes (\ref{eq:stokes}) n'est pas connue de fa\c
%%con analytique. Dans le cas industriel que nous considérerons dans
%%la section \ref{sec:harmonic-application}, la force sera donnée
%%comme un élément de $V_h$ sur un maillage tétraédrique.
%%
%%En général, il est clair que $f_{i,k} \in
%%L^2(\Lambda)$. Cependant, pour des raisons d'implémentations il est commode
%%de considérer l'interpolation de Lagrange des fonctions $f_{i,k}$
%%sur le maillage $\tau_h$, $\hat f_{i,k}\in V_h$:
%%\begin{equation}
%%  \hat f_{i,k} = \sum_{n = 1}^{N} f_{i,k}(x_n)\varphi_n.
%%\end{equation}
%%où les $\varphi_n$ sont les fonctions de base de $V_h$.  Il suffit
%%alors de calculer les valeurs de $f_{i,k}$ aux noeuds $x_n$ du maillage
%%$\tau_h$. Soit $N_s > 0$ le nombre de subdivisions de l'intervalle $[-h,
%%  h]$. Les intégrales qui intervienne dans les définitions
%%(\ref{eq:force-coefficient-1})-(\ref{eq:force-coefficient-3}) sont
%%approximée par une formule de Simpson sur la subdivision
%%régulière $z_0 = -h < z_1 < \dots < z_{N_{s+1}} = h$ de
%%l'intervalle $[-h, h]$. Le nombre de subdivisions sera choisi de
%%sorte à ce que l'erreur de quadrature soit de l'ordre de
%%l'epsilon machine.
%%
%%Sauf mention contraire, on utilisera dorénavant $\hat f_{i,k}$ en
%%lieu et place de $f_{i,k}$ dans les membres de droite des
%%différentes formulations, et on omettra le circonflexe pour des
%%raisons de lisibilité.

%\subsection{Discrétisation du terme de Navier non-linéaire}

%\subsection{Discrétisation de la viscosité turbulente non-linéaire}

\paragraph{Résolution des systèmes linéaires}
Les systèmes linéaires issus des formulations
(\ref{eq:fourier-fund-discrete-1-prat})-(\ref{eq:fourier-fund-discrete-3-prat})
et(\ref{eq:fourier-harm-discrete-1}),(\ref{eq:fourier-harm-discrete-2})
sont résolu avec la méthode GMRES préconditionnée par ILU(2). Le
critère d'arrêt de l'itération de GMRES est:
\begin{equation}
  \frac{\norm{Ax^n - b}_{l^2}}{\norm{Ax^0 - b}_{l^2}} \leq \num{1e-10},
\end{equation}
où $A$ est la matrice préconditionnée associée à la forme bilinéaire
d'une formulation
variationnelle, $b$ le vecteur du membre de droite associé à la
forme linéaire, et $x^n$ les valeurs de l'approximation de la
solution $x$ aux itérations successives de l'algorithme GMRES.

\paragraph{Reconstruction de la solution}
On se fixe un entier $K > 0$, on suppose avoir calculé $u^k$, $p^k$, $k
= 1, 2,\dots, K$ solutions de (\ref{eq:fourier-fund-discrete-1-prat})-%
(\ref{eq:fourier-fund-discrete-3-prat}) et
(\ref{eq:fourier-harm-discrete-1}),
(\ref{eq:fourier-harm-discrete-2}). En vertu de
(\ref{eq:fourier-coeff-def-1})-(\ref{eq:fourier-coeff-def-3}), on note
pour tout $(x_1, x_2, x_3)\in \Omega$
\begin{align}
  &u_{i,h,K}(x_1, x_2, x_3) = u_{i,h}^0(x_1, x_2) + \sum_{k = 1}^K
  u_{i,h}^k \cos\parent{\beta^k x_3},\quad i = 1,2\label{eq:u-h-12}\\
  &u_{3,h,K}(x_1, x_2, x_3) = \sum_{k = 1}^K
  u_{3,h}^k \sin\parent{\beta^k x_3}\label{eq:u-h-3}
\end{align}
et
\begin{align}
  p_{h,K}(x_1, x_2, x_3) = p_h^0(x_1, x_2) + \sum_{k = 1}^K p_h^k(x_1,
  x_2) \cos\parent{\beta^k x_3}.
\end{align}
