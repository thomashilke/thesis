Dans cette section nous décrivons les méthodes de discrétisations
proposées pour approximer le coefficient du mode fondamental de
l'écoulement $(u^0, p^0)$ et les coefficients des harmoniques $(u^k,
p^k)$.

Soit un nombre réel $h > 0$ et soit $\mathcal M_h$ une triangulation
du domaine $\Lambda$ de $\mathbb R^2$ telle que $\diam(\meshcell)\leq h$,
$\forall \meshcell\in \mathcal M_h$. On introduit maintenant l'espace
éléments finis $V_h$ défini par
\begin{equation}
  V_h = \cparent{v\in C^0(\Lambda)\mid v|_\meshcell \in \mathbb
    P_1(\meshcell)\ \forall \meshcell\in\mathcal M_h},
\end{equation}
avec $\mathbb P_1(\meshcell)$ l'espace des polynômes de degré 1 sur le
triangle $\meshcell$. On définit aussi l'espace enrichi par une fonction bulle
\begin{equation}
  B_h = \cparent{v \in C^0(\Lambda)^2 \mid v|_\meshcell \in \mathbb P_1(\meshcell)^2\oplus
    B_\meshcell\ \forall \meshcell\in\mathcal M_h} \cap H_0^1(\Lambda)^2
\end{equation}
où $B_\meshcell$ est engendré par la fonction $27 \lambda_1^\meshcell
\lambda_2^\meshcell\lambda_3^\meshcell$, les $\lambda_i^\meshcell$, $i = 1,2,3$, étant les
trois fonctions barycentriques du triangle $\meshcell$.

\paragraph{Mode fondamental}
La méthode d'éléments finis correspondant à la discrétisation des équations
(\ref{eq:fourier-weak-fund-1}), (\ref{eq:fourier-weak-fund-2})
consiste à chercher $(u^0_h, p^0_h)\in B_h\times(V_h \cap
L^2_0(\Lambda))$ tels que, pour tout $(v_h, q_h)\in
B_h\times(V_h \cap L^2_0(\Lambda))$ on ait
\begin{align}
  &\int_\Lambda
  2\mu\otimes\reducedstraintensor(u^0_h):\reducedstraintensor(v)
  -\int_\Lambda p^0_h\div(v_h) = \int_\Lambda f^0\cdot v_h, \label{eq:fourier-fund-discrete-1}\\
  & \int_\Lambda q_h\div u^0_h = 0.\label{eq:fourier-fund-discrete-2}
\end{align}
Il est connu que le couple d'espaces éléments finis $(B_h, V_h \cap
L^2_0(\Lambda))$ est stable, ainsi (\ref{eq:fourier-fund-discrete-1}),
(\ref{eq:fourier-fund-discrete-2}) admet une unique solution \cite{Temam1977}.

Le schéma (\ref{eq:fourier-fund-discrete-1}),
(\ref{eq:fourier-fund-discrete-2}) fait intervenir l'espace
éléments finis à moyenne nulle $V_h \cap L^2_0(\Lambda)$ pour le mode
fondamental de la pression $p^0$. En pratique, on exprime
explicitement la contrainte sur la pression dans le problème faible
en ajoutant l'équation
\begin{equation}
\int_\Lambda p^0_h = 0
\end{equation}
et en ajoutant un multiplicateur de Lagrange pour la fonction test $q_h$. Il est alors équivalent
de chercher $(u^0_h, p^0_h)\in B_h\times V_h$ et
$\lambda\in\mathbb R$ tels que, pour tout $(v_h, q_h)\in
B_h\times V_h$ on ait
\begin{align}
  &\int_\Lambda
  2\mu\otimes\reducedstraintensor(u^0_h):\reducedstraintensor(v)
  -\int_\Lambda p^0_h\div(v_h) = \int_\Lambda f^0\cdot v_h, \label{eq:fourier-fund-discrete-1-prat}\\
  & \int_\Lambda q_h\div u^0_h + \lambda \int_\Lambda q_h =
  0,\label{eq:fourier-fund-discrete-2-prat}\\
  & \int_\Lambda p_h^0 = 0.\label{eq:fourier-fund-discrete-3-prat}
\end{align}

\paragraph{Harmoniques}
On considère à présent la discrétisation de la formulation
faible (\ref{eq:fourier-weak-harm-1}),
(\ref{eq:fourier-weak-harm-2}). Pour $k > 0$ donné, le problème
discrétisé consiste à chercher les fonctions $(u^k_h, p^k_h) \in
(V_h\cap H_0^1(\Lambda))^3\times V_h$ telles que
\begin{align}
  &\int_\Lambda 2\mu\otimes
  \tilde\straintensor(u^k_h):\tilde\straintensor(v) - \int_\Lambda
  p_h^k\parent{\frac{\partial v_{1,h}}{\partial x_1}
    \frac{\partial v_{2,h}}{\partial x_2} +\beta^k p^k_h v_{3,h}} =
  \int_\Lambda f\cdot v_h,\label{eq:fourier-harm-discrete-1}\\
  &\int_\Lambda \parent{\frac{\partial u^k_{1,h}}{\partial x_1} +
    \frac{\partial u^k_{2,h}}{\partial x_2} + \beta^k u^k_{3,h}}q_h +
  \frac{\delta}{\displaystyle\max_{1\leq i,j\leq 3}\mu_{i,j}} \sum_{\meshcell\in\mathcal M_h} h_\meshcell^2
  \int_\meshcell\parent{\parent{\beta^k}^2 p^k_h q_h + \nabla p_h^k \nabla
    q_h} = 0\label{eq:fourier-harm-discrete-2}
\end{align}
pour tout $(v_{1,h}, v_{3,h}, v_{3,h}, q_h) \in (V_h\cap
H_0^1(\Lambda))^3\times V_h$. Ici $\delta > 0$ et le terme de
stabilisation dans (\ref{eq:fourier-harm-discrete-2}) qui est justifié
par le fait que, d'après (\ref{eq:fourier-coeff-def-3}),
\begin{equation}
  \Delta P = \Delta p^0 + \sum_{k > 0} \parent{\Delta p^k -
  \parent{\beta^k}^2 p^k}\cos(\beta^k x_3).
\end{equation}

\paragraph{Quadrature pour les coefficients de Fourier de la force}
En pratique, les forces $f_i$, $i = 1,2,3$ sont données aux sommets du
maillage $\mathcal M_h$. Les intégrales qui interviennent dans les
définitions (\ref{eq:f-1}), (\ref{eq:f-2}) et (\ref{eq:f-3}) sont
approchées par une formule de Simpson sur une subdivision uniforme de
l'intervalle $[0, \thickness]$. Le nombre de subdivision sera choisi
de sorte à ce que cette erreur de quadrature soit négligeable par
rapport aux erreurs introduites par les discrétisations éléments
finis et la troncature de la série de Fourier.

\paragraph{Résolution des systèmes linéaires}
Les systèmes linéaires issus des formulations
(\ref{eq:fourier-fund-discrete-1-prat})-(\ref{eq:fourier-fund-discrete-3-prat})
et(\ref{eq:fourier-harm-discrete-1}),(\ref{eq:fourier-harm-discrete-2})
sont résolus avec la méthode GMRES préconditionnée par ILU(2). Le
critère d'arrêt de l'itération de GMRES est:
\begin{equation}
  \frac{\norm{P^{-1}\parent{Ax^n - b}}_{l^2}}{\norm{P^{-1}\parent{Ax^0 - b}}_{l^2}} \leq \num{1e-10},
\end{equation}
où $A$ est la matrice associée à la forme bilinéaire d'une formulation
variationnelle, $P$ le préconditionneur ILU(2) associé à la matrice
$A$, $b$ le vecteur du membre de droite associé à la forme linéaire et
$x^n$ les valeurs de l'approximation de la solution $x$ aux itérations
successives de l'algorithme GMRES.

\paragraph{Reconstruction de la solution}
On se fixe un entier $K > 0$, on suppose avoir calculé $u^k$, $p^k$, $k
= 1, 2,\dots, K$ solutions de (\ref{eq:fourier-fund-discrete-1-prat})-%
(\ref{eq:fourier-fund-discrete-3-prat}) et
(\ref{eq:fourier-harm-discrete-1}),
(\ref{eq:fourier-harm-discrete-2}). En vertu de
(\ref{eq:fourier-coeff-def-1})-(\ref{eq:fourier-coeff-def-3}), on note
pour tout $(x_1, x_2, x_3)\in \Omega$
\begin{align}
  &u_{i,h,K}(x_1, x_2, x_3) = u_{i,h}^0(x_1, x_2) + \sum_{k = 1}^K
  u_{i,h}^k \cos\parent{\beta^k x_3},\quad i = 1,2\label{eq:u-h-12}\\
  &u_{3,h,K}(x_1, x_2, x_3) = \sum_{k = 1}^K
  u_{3,h}^k \sin\parent{\beta^k x_3}\label{eq:u-h-3}
\end{align}
et
\begin{align}
  p_{h,K}(x_1, x_2, x_3) = p_h^0(x_1, x_2) + \sum_{k = 1}^K p_h^k(x_1,
  x_2) \cos\parent{\beta^k x_3}.
\end{align}
Enfin on notera $u_{h,K} = (u_{1,h,K}, u_{2,h,K}, u_{3,h,K})$.
