Ce travail se divise en deux parties. Dans la partie
\ref{part:alumina}, on considère plusieurs effets physiques liés à la
gravité et aux interactions thermiques entre les particules d'alumine
et le bain environnant, que l'on exprime sous forme de modèles
mathématiques. On propose ensuite une extension du modèle de transport
et dissolution d'alumine, introduit dans le travail de thèse de
T. Hofer \cite{Hofer2011}, pour tenir compte de l'effet de la gravité
sur la trajectoire des particules d'alumine et de l'effet de la
température du bain sur leur dissolution.  Dans la partie
\ref{part:fluid} on s'intéresse à diminuer le coût de calcul de
l'écoulement dans le bain électrolytique et le métal liquide. On
propose une méthode numérique basée sur une décomposition en
harmoniques de Fourier du problème de Stokes dans la dimension
verticale. On propose de plus une méthode pour approximer la
distribution de concentration d'alumine dans le bain électrolytique,
sans recourir au calcul d'une solution transitoire.
