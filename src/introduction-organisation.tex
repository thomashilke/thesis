La partie \ref{part:alumina} de ce travail concerne plusieurs
phénomènes physiques liés à la dissolution de particules d'alumine et
de populations de particules d'alumine dans un bain électrolytique. Le
chapitre \ref{chap:particles} traite de l'interaction entre le bain
électrolytique dans une cuve et une unique particule ou d'un ensemble
de particules indépendantes. Nous commençons dans la section
\ref{sec:particle-freeze} par étudier la solidification et
liquéfaction du bain électrolytique autour d'une particule d'alumine
de température \num{150}\si{\celsius} injectée dans un bain à
\num{950}\si{\celsius}. Puis nous considérons la modélisation de la
dissolution, c'est-à-dire de l'évolution de la taille au cours du
temps d'une particule d'alumine placée dans un bain électrolytique
dans la section \ref{sec:particle-dissolution}. Dans la section
\ref{sec:particle-population-dissolution} nous utilisons le modèle de
dissolution d'une unique particule pour décrire l'évolution d'une
population de particules qui se dissout dans le bain. Enfin, dans la
section \ref{sec:particle-fall} nous discutons du mouvement d'une
particule dans le bain soumise à l'action de la gravité, et dans
quelle mesure cet effet peut être négligé. Dans le chapitre
\ref{chap:populations}, nous utilisons les résultats établis dans le
chapitre \ref{chap:particles} pour proposer un modèle de transport et
dissolution de poudre d'alumine dans le bain d'une cuve d'électrolyse
de Hall-Héroult en fonction de la température dudit bain. Dans la
section \ref{sec:populations-model} nous décrivons en détail les
équations et les données qui interviennent dans le modèle de transport
et dissolution d'alumine, que nous discrétisons dans la section
\ref{sec:populations-discretisation}. Dans la section
\ref{sec:populations-industriel} nous appliquons le modèle proposé
dans les section \ref{sec:populations-model} et
\ref{sec:populations-discretisation} au cas de la cuve d'électrolyse
industrielle AP32, et nous discutons des résultats dans la section
\ref{sec:populations-conclusion}.

La partie \ref{part:fluid} de ce travail concerne la problématique du
calcul de l'écoulement des fluides dans une cuve d'électrolyse. Dans
la section \ref{sec:fourier-model} nous proposons de calculer un
écoulement de Stokes dans les fluides d'une cuve d'électrolyse en
décomposant les données et les inconnues du problème en séries de
Fourier selon la direction verticale. Les équations qui résultent de
cette décomposition sont discrétisées dans la section
\ref{sec:fourier-discretisation}. Dans la section
\ref{sec:fourier-validation} nous étudions les propriétés du schéma
obtenu à la section \ref{sec:fourier-discretisation} et soulignons ses
avantages par rapport à une discrétisation basée sur une méthode
d'éléments finis en trois dimensions classique. Finalement, le schéma
obtenu à la section \ref{sec:fourier-discretisation} est appliqué au
problème du calcul de l'écoulement des fluides dans la cuve
d'électrolyse AP32, et nous discutons des résultats dans la section
\ref{sec:fourier-conclusion}.
