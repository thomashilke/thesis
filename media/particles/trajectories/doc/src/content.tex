\section{Introduction}
Le code disponible dans ce package impl\'emente le mod\`ele de chute
de particule sph\'erique dans un fluide visqueux au repos. Le code est
\'ecrit dans le language \texttt{GNU Octave} et a \'et\'e test\'e avec
la version 4.2.1. Ce document d\'ecrit les \`equations qui
mod\'elisent la trajectoire verticale des particules sph\'erique dans
le fluide, et d\'ecrit la proc\'edure pour ex\'ecuter le code et
obtenir les r\'esultats.

\section{Mod\`ele math\'ematique}
La dynamique d'une particule sph\'erique dans un fluide visqueux au
repos et soumis \`a l'acc\'el\'eration de la gravit\'e est d\'ecrite
par la deuxi\`eme loi de Newton. En appliquant cette loi on obtient
une \'equation diff\'erentielle ordinaire du deuxi\`eme ordre, dont la
solution est unique \`a condition de sp\'ecifier deux conditions
initiales.

On se place dans le r\'ef\'erentiel de la Terre, et on consid\`ere un
syst\`eme de coordonn\'ees unidimensionnel vertical, orient\'e vers le
haut. On note $x(t)$ la coordonn\'ee de la particule dans ce syst\`eme
au cours du temps, $\dot x(t)$ sa vitesse et enfin $\ddot x(t)$ sont
acc\'el\'eration.

La particule sph\'erique est d\'ecrite par son rayon $r$. Le rayon de
la particule \'evolue au cours du temps selon l'expression:
\begin{equation}
  r(t) = \sqrt{r_0 - 2 K t},
\end{equation}
o\`u $r_0 > 0$ est sont rayon initial et $K$ le taux de
dissolution. On s'int\'eresse \`a la trajectoire de cette particule
pour $t \in [0, r_0/(2K)]$.

Soit $\rhoal$ la densit\'e de la particule, et $\rhoe$ celle du
fluide. La masse $m$ de la particule s'\'ecrit:
\begin{equation}
  m(r) = \frac{4}{3}\pi r^3 \rhoal.
\end{equation}

Dans ce syst\`eme de coordonn\'ee, la force de gravit\'e est donn\'ee par
\begin{equation}
  F_g(r) = -\frac{4}{3}\pi r^3 \rhoal g,
\end{equation}
la force d'Archim\`ede est donn\'ee par:
\begin{equation}
  F_A(r) = \frac{4}{3}\pi r^3 \rhoe g
\end{equation}
et on note la force de tra\^in\'ee $F_D(r, \dot x)$. L'expression de
$F_D$ est non triviale, et on se r\'ef\'erera au code (\texttt{ferguson/lib/force.m}) et au manuscrit
de th\`ese de T.Foetisch (\`a para\^itre) pour les d\'etails de la
d\'erivation et l'expression explicite. Cette force est d\'eriv\'ee
d'une expression propos\'ee par \cite{Ferguson2004}.

La deuxi\`eme loi de Newton s'\'ecrit:
\begin{equation}\label{eq:newton}
  m(r(t))\ddot x(t) = F_A(r(t)) + F_g(r(t)) + F_D(r(t), \dot x(t)),
\end{equation}
et on consid\`ere les conditions initiales:
\begin{equation}
  x(0) = 0,\quad \dot x(0) = 0.
\end{equation}

\section{Ex\'ecution et r\'esultats}
Le code de ce package (\texttt{app.m}) impl\'emente dans le language \texttt{GNU
  Octave} une m\'ethode num\'erique pour int\'egrer l'\'equation
(\ref{eq:newton}), \`a l'aide de l'int\'egrateur \texttt{ode45}, afin
de s'abstraire des questions de convergence et de garantir une
certaine erreur.

Pour ex\'ecuter ce script il faut se placer dans le dossier
\texttt{ferguson/} et de lancer:
\begin{lstlisting}[frame=single, language={},basicstyle=\ttfamily\footnotesize]
  [user@localhost ~/dev/ferguson> octave app.m
\end{lstlisting}

Le code calcule la trajectoire d'un ensemble de 5 particules dont les
rayons initiaux sont compris entre $100\si{\micro\meter}$ et
$200\si{\micro\meter}$, et stocke les resultats dans le fichier
\texttt{ASCII} \texttt{particle\_fall.dat}.

Le contenu de ce fichier peut \^etre visualis\'e \`a l'aide de
\texttt{GNUPlot}. Un script \texttt{particle\_fall.plot} est
\'egalement mis \`a disposition pour g\'en\'erer les figures qui sont
consign\'ee dans le manuscrit de th\`ese.

Toutes ces fonctionnalit\'es sont mises \`a disposition par
l'interm\'ediaire d'un fichier \texttt{Makefile}. Pour ex\'ecuter le
code et g\'en\'erer les figures, il suffit de lancer:
\begin{lstlisting}[frame=single,language={},basicstyle=\ttfamily\footnotesize]
  [user@localhost ~/dev/ferguson> make
\end{lstlisting}

